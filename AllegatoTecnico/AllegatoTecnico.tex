%generare il pdf con il comando: pdflatex main.tex
\documentclass[a4paper, oneside, openany,dvipsnames,table]{article}
\usepackage{../template/sos}
\usepackage{eurosym}
\usepackage{amssymb}% http://ctan.org/pkg/amssymb
\usepackage{pifont}% http://ctan.org/pkg/pifont
\usepackage{listings}
\definecolor{bluSOS}{RGB}{48, 84, 150}
\newcommand{\shellcmd}[1]{\\\indent\indent\texttt{\footnotesize\$ #1}\\}
%\WithSuffix\def\shellcmd*#1{\indent\indent\texttt{\footnotesize\$ #1}\\}
\newcommand{\Titolo}{Piano di Progetto}

\newcommand{\Gruppo}{Sons Of SWE}

\newcommand{\Redazione}{Panozzo Stefano \newline Favero Andrea \newline Dalla Riva Giovanni \newline Federico Caldart \newline Eleonora Thiella}

\newcommand{\ACapoRedazione}{Panozzo Stefano \newline Favero Andrea \newline Dalla Riva Giovanni \newline Federico Caldart \newline Eleonora Thiella}

\newcommand{\Verifica}{Dalla Riva Giovanni \newline Menegon Lorenzo \newline Cavallin Giovanni \newline Panozzo Stefano}

\newcommand{\Approvazione}{Cavallin Giovanni \newline Favero Andrea \newline Caldart Federico}

\newcommand{\Distribuzione}{Vardanega Tullio\newline Cardin Riccardo \newline Gruppo Sons Of SWE}

\newcommand{\Uso}{Esterno}


\newcommand{\Data}{07 Maggio 2018}

\newcommand{\NomeProgetto}{Progetto Marvin}

\newcommand{\Mail}{sonsofswe.swe@gmail.com}


\newcommand{\DescrizioneDoc}{Documento contenente il piano di progetto relativo al \NomeProgetto{} scelto dal gruppo \Gruppo.}

\newcommand{\versione}{3.0.0}




\begin{document}
\copertina{}


%%%%%%%%%%%%%%%%%%%%%%%%%%%%%%%%%%%%%%%%%%%%%%%%%%%%%%%%%%%%%%%%%%%%%%%%%%%%%%%%%%%%%%%%%%%%%%%%%%%%%%%
%SOMMARIO
\definecolor{bluSOS}{RGB}{48, 84, 150}
\definecolor{greySOS}{RGB}{209, 222, 223}
\section*{Registro delle modifiche}
{
	\rowcolors{2}{greySOS}{white}
	\renewcommand{\arraystretch}{1}
	\centering
	\begin{longtable}{| c| c | C{4cm} | c | c |}
		\hline
		\rowcolor{bluSOS}
		\textcolor{white}{\textbf{Versione}} & \textcolor{white}{\textbf{Data}} & \textcolor{white}{\textbf{Descrizione}} & \textcolor{white}{\textbf{Autore}} & \textcolor{white}{\textbf{Ruolo}}\\
		\hline
		0.0.1 & 16-03-2018 &   &  & Reponsabile\\
		\hline
		0.0.1 & 16-03-2018 & Creato lo scheletro del documento e scrittura dell' \emph{introduzione}  & Giovanni Cavallin & Reponsabile\\ 
		\hline
	\end{longtable}

}


%dvipsnames, table
%\definecolor{bluSOS}{RGB}{48, 84, 150}
\definecolor{greySOS}{RGB}{209, 222, 223}
\section*{Registro delle modifiche}
{
	\rowcolors{2}{greySOS}{white}
	\renewcommand{\arraystretch}{1}
	\centering
	\begin{longtable}{| c| c | C{4cm} | c | c |}
		\hline
		\rowcolor{bluSOS}
		\textcolor{white}{\textbf{Versione}} & \textcolor{white}{\textbf{Data}} & \textcolor{white}{\textbf{Descrizione}} & \textcolor{white}{\textbf{Autore}} & \textcolor{white}{\textbf{Ruolo}}\\
		\hline
		0.0.1 & 16-03-2018 &   &  & Reponsabile\\
		\hline
		0.0.1 & 16-03-2018 & Creato lo scheletro del documento e scrittura dell' \emph{introduzione}  & Giovanni Cavallin & Reponsabile\\ 
		\hline
	\end{longtable}

}


%dvipsnames, table
%\definecolor{bluSOS}{RGB}{48, 84, 150}
\definecolor{greySOS}{RGB}{209, 222, 223}
\section*{Registro delle modifiche}
{
	\rowcolors{2}{greySOS}{white}
	\renewcommand{\arraystretch}{1}
	\centering
	\begin{longtable}{| c| c | C{4cm} | c | c |}
		\hline
		\rowcolor{bluSOS}
		\textcolor{white}{\textbf{Versione}} & \textcolor{white}{\textbf{Data}} & \textcolor{white}{\textbf{Descrizione}} & \textcolor{white}{\textbf{Autore}} & \textcolor{white}{\textbf{Ruolo}}\\
		\hline
		0.0.1 & 16-03-2018 &   &  & Reponsabile\\
		\hline
		0.0.1 & 16-03-2018 & Creato lo scheletro del documento e scrittura dell' \emph{introduzione}  & Giovanni Cavallin & Reponsabile\\ 
		\hline
	\end{longtable}

}


%dvipsnames, table
%\input{sez/RegistroModifiche.tex}
%\newpage
%\newpage
%\newpage
\newpage
\newpage
\tableofcontents
\newpage
\listoffigures
\newpage
%%%%%%%%%%%%%%%%%%%%%%%%%%%%%%%%%%%%%%%%%%%%%%%%%%%%%%%%%%%%%%%%%%%%%%%%%%%%%%%%%%%%%%%%%%%%%%%%%%%%%%%
%PARAGRAFI
\hypersetup{linkcolor=bluSOS}
\section{Introduzione}
\subsection{Scopo del documento}
Questo documento ha lo scopo di descrivere gli \emph{attori}\ped{G} del sistema, individuare i \emph{casi d'uso}\ped{G} a partire dai \emph{requisiti}\ped{G} e fornire una visione chiara ai \progs{} sul problema da trattare. I requisiti verranno classificati in questo documento a seguito di una trattazione col proponente.

\subsection{Scopo del prodotto}
Lo scopo del prodotto è quello di realizzare un \emph{prototipo}\ped{G} di \emph{Uniweb}\ped{G} come \emph{ÐApp}\ped{G} in esecuzione su \emph{Ethereum}\ped{G}. I cinque attori principali che si rapportano con Marvin sono:
\begin{itemize}
	\item Utente non autenticato;
	\item Università;
	\item Amministratore;
	\item Professori;
	\item Studenti.
\end{itemize} 
Il portale deve quindi permettere agli studenti di accedere alle informazioni riguardanti le loro carriere universitarie, di iscriversi agli esami, di accettare o rifiutare voti e di poter vedere il loro libretto universitario.
Ai professori deve invece essere permesso di registrare i voti degli studenti.
L'università ogni anno crea una serie di corsi di laurea rivolti a studenti, dove ognuno di essi comprende un elenco di esami disponibili per anno accademico. Ogni esame ha un argomento, un numero di crediti e un professore associato. Gli studenti si iscrivono ad un corso di laurea e tramite il libretto elettronico mantengono traccia ufficiale del progresso.

\subsection{Glossario}
Nel documento Glossario i termini tecnici, gli acronimi e le abbreviazioni sono definiti in modo chiaro e conciso, in modo tale da evitare ambiguità e massimizzare la comprensione dei documenti.
\newline I vocaboli presenti in esso saranno posti in corsivo e presenteranno una "G" maiuscola a pedice.
\subsection{Riferimenti}
\subsubsection{Normativi}
\begin{itemize}
	\item \textcolor{red}\NdP
\end{itemize}

\subsubsection{Informativi}
\begin{itemize}
	\item Capitolato d'appalto C6: Marvin. Reperibile all'indirizzo:\\ 
	\href{http://www.math.unipd.it/~tullio/IS-1/2017/Progetto/C6.pdf}{http://www.math.unipd.it/~tullio/IS-1/2017/Progetto/C6.pdf};
	\item \textcolor{red}\SdF;
\end{itemize}

\section{Descrizione generale}
	\subsection{Contesto d'uso del prodotto}
	Il prodotto finale vuole essere una sorta di \emph{PoC}\ped{G} per dimostrare la fattibilità di utilizzo di tali tecnologie in quest’ambito. L’applicazione sarà un prototipo di Uniweb, quindi si colloca in un contesto universitario dove gli attori si approcciano al sistema come nell’attuale Uniweb. La differenza sta nel \emph{back-end}\ped{G} dove, invece del classico sistema \emph{client}\ped{G}/\emph{server}\ped{G}, troviamo un database distribuito che sfrutta la piattaforma Ethereum.
	
	\subsection{Caratteristiche degli utenti}
	Questo prodotto deve risultare accessibile ad un'ampia categoria di utenti senza particolari competenze. L’interfaccia dovrà quindi essere il più chiara ed intuitiva possibile. Verrà fornito anche un \MU{} con tutte le indicazioni necessarie per consentire il corretto utilizzo del prodotto.
	
	\subsection{Assunzione dipendenze}
	Per il corretto funzionamento dell’applicazione sarà necessario l’utilizzo di un browser che sia compatibile con \emph{HTML5}\ped{G}, \emph{SCSS}\ped{G} e \emph{Javascript}\ped{G}.

\newpage
%%%%%%%%%%%%%%%%%%%%%%%%%%%%%%%%%%%%%%%%%%%%%%%%%%%%%%%%%%%%%%%%%%%%%%%%%%%%%%%%%%%%%%%%%%%%%%%%%%%%%%%
\section{Requisiti di sistema}
Per l'installazione e l'utilizzo di questo software sono richiesti alcuni prerequisiti:
\begin{itemize}
	\item Browser web Google Chrome (aggiornato alla versione 60 o superiori) o Mozilla Firefox (aggiornato alla versione 50 o superiori);
	\item Plugin Metamask (aggiornato alla versione 4.7.1 o superiori) per i browser di cui sopra; \\
	\url{https://metamask.io/}
	\item Git \\
	\url{https://git-scm.com/downloads}
	\item Python aggiornato alla versione 2.7; \\
	\url{https://www.python.org/downloads/}
	\item Node package manager alla versione 6, e Node alla 8.11.2 \\
	\url{https://nodejs.org/it/}
	\item Nel caso si utilizzi Windows sarà necessario installare \textit{windows-build-tools} digitando nella powershell: 
	\begin{lstlisting}
		npm install --global --production windows-build-tools
	\end{lstlisting}

	 
\end{itemize}
\newpage
\section{Installazione ed esecuzione}
Il codice relativo alla Product Baseline lo si può trovare al seguente link:
\begin{center}
	\url{linkAllaRepo}
\end{center}
Una volta fatto il clone della repository o dopo aver scaricato lo zip, sono necessari i seguenti passi per far partire l'applicazione:
\begin{enumerate}
	\item Posizionarsi nella root della repo ed eseguire nella shell:
		\begin{lstlisting}
	npm install -g ganache-cli
	npm install -g truffle
	npm i
		\end{lstlisting}
	\item In seguito sempre nella shell:
		\begin{lstlisting}
	./startBlockchain.ps1
		\end{lstlisting}
	\item Infine è necessario eseguire:
		\begin{lstlisting}
	./loadProject.ps1
		\end{lstlisting}
\end{enumerate}

A questo punto noterai che il tuo browser predefinito ha aperto automaticamente l'homepage.
Ora dovrai connetteri a Metamask: nel tuo browser clicca sull'icona di Metamask e accetta l'informativa sulla privacy e le condizioni d'uso. Poi clicca su \textbf{Main Network} e scegli \textbf{Custom RPC}, inserisci nel primo form \textbf{http://localhost:9545} come in Figura~\ref{fig:metamask1} e clicca su  \textbf{Save}.
%Now you have to connect MetaMask: in your browser click on the MetaMask icon and accept the Privacy Notice and the Terms of use. Then click on \textbf{Main Network} and choose \textbf{Custom RPC}, type in the first form \textbf{http://localhost:9545} as in Figure~\ref{fig:metamask1} and click \textbf{Save}.

\begin{figure}[h]
\centering
\includegraphics[height=3in]{./img/settings.png}
\caption{Setta l'RPC inserendo \textbf{http://localhost:9545}}
\label{fig:metamask1}
\end{figure}

Ora, come in Figura~\ref{fig:metamask2}, clicca su \textbf{Import Existing DEN} e (vedi Figura~\ref{fig:metamask3}) inserisci la frase \textbf{candy maple cake sugar pudding cream honey rich smooth crumble sweet treat} e la password che vuoi usare per il tuo account.


\begin{figure}[h]
\centering
\includegraphics[height=1.8in]{./img/importa.png}
\caption{Clicca su \textbf{Import Existing DEN}}
\label{fig:metamask2}
\end{figure}

\begin{figure}[h]
\centering
\includegraphics[height=3in]{./img/stringa_psw.png}
\caption{Inserisci la seed phrase e la password che vuoi usare}
\label{fig:metamask3}
\end{figure}
	

\newpage
\section{Confronto con il Poc}
In precedenza è stato realizzata una \emph{Proof of Concept} in cui si è data dimostrazione delle tecnologie che si sarebbero dovute utilizzare per la realizzazione del progetto Marvin.
Tale produzione è disponibile all'indirizzo: \url{https://github.com/SOS-SonsOfSwe/Marvin-PoC}.
Durante la ricerca di strumenti adatti allo scopo il team ha deciso inizialmente di basare la prima produzione del prodotto sul framework \emph{Truffle}, dal momento che esso presentava molte delle tecnologie di interesse.
Con uno studio più approfondito del pacchetto il gruppo si è reso conto che esso era una buona base di partenza per la realizzazione di un prodotto più complesso, sebbene necessitasse di alcune correzioni strutturali. Qui di seguito si elencano i limiti riscontrati e le soluzioni adottate:
\newline
\newline

\begin{table}[hp]
	\centering
\begin{tabular}{|p{6cm}|p{6cm}|}
	\hline
	\textbf{Proof of Concept} & \textbf{Product Baseline} \\
	
	\hline Architettura ben strutturata ma male incapsulata
	&
	Organizzazione migliore della struttura del pacchetto
	a fronte di una conoscenza più approfondita del design pattern MVVM e di altri su cui il framework si basa \\ 
	
	\hline Interfaccia povera e strutturata in maniera confusionaria, non chiara e  difficilmente intuibile 
	&
	Ristrutturazione completa e incremento dell'interfaccia utente, ora provvista di tutte le parti atte alla soddisfazione della maggior parte dei requisiti opzionali e obbligatori \\ 
	
	\hline Database solidity estremamente scarno e con poche funzionalità di sicurezza e di ottimizzazione
	&
	Strutturazione avanzata del backend di solidity, desisamente più ottimizzato e performante; implementazione già in fase di completamento\\
	
	\hline
\end{tabular}
\caption{Confronto tra Proof of Concept e Product Baseline}
\end{table}
\newpage
\section{Architettura del prodotto}
Dopo un approfondito studio il team ha optato per l'utilizzo del design pattern \textbf{Model View View-Model}, che prevede tre macrosezioni:
	\begin{itemize}
		\item \textbf{Model}: rappresenta i dati contenuti nel sito, ma non i comportamenti o i servizi che manipolano l'informazione. Non è responsabile della renderizzazione;
		\item \textbf{View}: si occupa di rappresentare le informazioni contenute nel sito ed è quello con cui l'utente interagisce; la view in questo design pattern è attiva, al contrario di quello che succede nell'MVC, questo perché contiene comportamenti, eventi e riferimenti a stati del sito, che quindi presuppongo una conoscenza della logica che sta dietro ai dati;
		\item \textbf{Viewmodel}: fornisce i dati dal Model in una forma in cui la View può usufruirne. Si occupa inoltre della logica della vista e di mantenersi costantemente sincronizzato con la View.
	\end{itemize}
Per poter apprezzare questa suddivisione nel pacchetto del team viene riportato qui di seguito il diagramma generale dei package. Si proseguirà successivamente alla descrizione delle implementazioni di ogni sezione in relazione all'architettura della Product Baseline, illustrandone i rispettivi design pattern utilizzati.
\\Per facilitare la comprensione della struttura dei diagrammi delle classi, il team ha deciso di raggruppare tutti i \emph{Containers}, i \emph{Components} e le \emph{actions} nei loro rispettivi packages. Infatti il comportamento di ogni classe di ogni package è modulare:
	\begin{itemize}
		\item Tutti i \emph{Components} importano la classe \emph{React.Component}
		\item Tutti i \emph{Containers} importano il rispettivo \emph{Component}, la rispettiva \emph{Action} e il pacchetto \emph{react-redux}.
		\item Tutte le \emph{actions} importano le \emph{costants} che azionano il \emph{reducer}, il pacchetto \emph{react-router}, lo \emph{store}, il pacchetto \emph{truffle-contract} e la classe \emph{IpfsUtils}
	\end{itemize}
I diagrammi presentati in questo documento illustrano la struttura sinteticamente; la loro versione integrale si può trovare nella cartella Diagrammi.


\begin{figure}[h]
	\centering
	\includegraphics[height=3in]{./Diagrammi/FrameworkPackageGenerale.pdf}
	\caption{Diagramma generale dei package}
	\label{}
\end{figure}

	
	\subsection{View}
		\subsubsection{Design patterns}
		Progettando questa sezione ci si è resi conto che si sarebbero potute generare delle classi \emph{dumb}, ovvero slegate dalla logica del sistema e che si sarebbero dovute occupare solamente della rappresentazione delle informazioni. Per ottenere questo si è ricorso all'utilizzo del seguente design pattern:
			\begin{itemize}
				\item \textbf{Decorator}: sfruttando il macro-package \emph{Container}, che poi si occupa di collegare il componente allo stato di \emph{redux}, si possono decorare tutte le componenti ivi presenti con dei componenti react puri, ai quali vengono passate eventualmente le informazioni e le funzioni a disposizione attraverso un accesso al \emph{this.props}. In questa maniera si ha la completa separazione tra rappresentazione e dati - cosa auspicata dal framework MVVM - e si facilitano modifiche future.
			\end{itemize}
		
		\subsubsection{Diagramma delle classi}
		In Figura~\ref{fig:DiagrammaView} è riportato il diagramma delle classi relativo a questa sezione con evidenziato la posizione del design pattern utilizzato.
	
%	{\textcolor{red}{INSERIRE DIAGRAMMA CON VISTA SU VIEW}}

	
	\subsection{ViewModel}
		\subsubsection{Design patterns}
		Per la sezione riguardante il ViewModel, in Figura~\ref{fig:DiagrammaViewModel} si è ricorso all'utilizzo dei seguenti design pattern:
			\begin{itemize}
				\item \textbf{Observer}: descrizione per quanto riguarda il comportamento di trigger-client che opera Redux sullo stato di React;
				\item \textbf{Façade}: per generare un'interfaccia astratta dalle implementazioni delle classi che poi si vanno ad utilizzare;
				\item \textbf{Adapter}: per interagire col model si usa un adaptator offerto dal framework truffle che si occupa di {\textcolor{red}{fare-un-sacco-di-cose-belle-da-chiedere-a-Stefano}};
				\item \textbf{Command}: come redux ordina a react di re-renderizzare le pagine al cambiamento dello state dovuto ad un dispatch di un'azione.
			\end{itemize}
				\begin{figure}[!h]
		\centering
			\includegraphics[height=7in]{./Diagrammi/DiagrammaView.pdf}
		\caption{Diagramma delle classi del componente View}
		\label{fig:DiagrammaView}
	\end{figure}
	
		\subsubsection{Diagramma delle classi}
		In seguito è riportato il diagramma delle classi relativo a questa sezione con evidenziato la posizione del design pattern utilizzato.
	
	\clearpage
	\begin{figure}[hp]
		\centering
			\includegraphics[height=9in]{./Diagrammi/DiagrammaModelView.pdf}
		\caption{Diagramma delle classi del componente ViewModel}
		\label{fig:DiagrammaViewModel}
	\end{figure}
	\clearpage	
	
\subsection{Model}
\subsubsection{Architettura}
		Per quanto riguarda la macrosezione relativa alla componente Model, composta dai contracts salvati su blockchain ed IPFS, si sono effettuate le seguenti scelte progettuali:
		\begin{itemize}
			\item \textbf{Modularità}: dividendo la parte logica e la struttura dei dati, in caso di aggiornamento della parte logica e quindi di un nuovo deploy dei contracts interessati, si ottiene la persistenza dei dati e un minor costo per il nuovo deploy. Infatti, se i dati e la logica fossero salvati nello stesso contracts, ad un nuovo deploy, i dati non sarebbero più consistenti, essendo contenuti in un altro contratto;
			\item \textbf{Istanze di contracts}: ogni contracts ha un'unica istanza: se i contracts contengono oggetti con molteplicità, essi saranno strutturati in una struct e si salveranno in un mapping con una key che li identifica ed una value che punta all'istanza della struct interessata. Si salvano, inoltre, per ogni mapping, un array contenente le key dei mapping. Questo perchè in Solidity (il linguaggio utilizzato per programmare i contracts), non è possibile in alcun modo recuperare quali key sono già salvate nel mapping e quindi si utilizzano gli array collegati per iterare sui dati;
			\item \textbf{Denormalizzazione}: come per altri database di tipo NoSQL e contrariamente ai database relazionali, si deve mantenere un compromesso fra ridondanza dei dati e costo del salvataggio maggiore. Infatti, per ottenere delle query più semplici e veloci, è necessario mantenere la ridondanza dei campi dati utilizzati più spesso. Questo si traduce in una maggior complessità per la modifica e l'eliminazione dei dati, ma si ottengono vantaggi significativi in caso di iterazioni sui dati non dovendo effettuare chiamate a contracts quando una transazione lo richiede e quindi richiedendo un costo minore;
			\item \textbf{IPFS}: il costo del salvataggio dei dati su blockchain Ethereum è molto elevato. Per questo motivo, alcuni dati sono salvati off-chain su un altro sistema distribuito, ovvero IPFS. La scelta di salvare un dato su blockchain Ethereum o su IPFS segue quindi queste regole:
			\begin{enumerate}
				\item I dati utilizzati come identificativo per le query sono salvati su blockchain
				per evitare la latenza dovuta alla comunicazione fra backend-frontend-IPFS;
				\item I dati soggetti a problemi di concorrenza sono salvati su blockchain: in questo modo
				non serve porsi il problema dell'accesso concorrenziale per la modifica del dato da parte
				di molteplici utenti in contemporanea;
				\item Tutti i dati considerati "accessori" possono essere salvati su IPFS.
			\end{enumerate}
		\end{itemize}
		\subsubsection{Diagramma delle classi}
		In seguito è riportato il diagramma delle classi relativo a questa sezione con evidenziato la posizione del design pattern utilizzato.
	
\clearpage	
	\begin{figure}[h]
		\centering
			\includegraphics[angle=90, origin=c, height=7in]{./Diagrammi/DiagrammaModel.pdf}
		\caption{Diagramma delle classi del componente Model}
		\label{}
	\end{figure}
		\clearpage
	
	\subsection{Diagrammi di sequenza}
	Come per i diagrammi delle classi, sono stati redatti i seguenti diagrammi di sequenza, disponibili e meglio visualizzabili nella cartella Diagrammi;
	
		\subsubsection{Login}
		Il diagramma di sequenza riportato qui di seguito raffigura il processo di login, durante il quale l'utente che vuole accedere può essere autenticato dal sistema.
		
		\begin{figure}[h]
			\centering
				\includegraphics[height=5in]{./Diagrammi/DiagrammaSequenzaLogin2.pdf}
			\caption{Diagramma di sequenza del processo di login}
			\label{}
		\end{figure}
		
		\subsubsection{Inserimento di un utente}
		Il diagramma di sequenza in Figura~\ref{fig:SeqInserimentoUtente} rappresenta l'azione di inserimento di un utente nel sistema.
		\begin{figure}[h]
			\centering
				\includegraphics[height=4.5in]{./Diagrammi/DiagrammaSequenzaInsertUser.pdf}
			\caption{Diagramma di sequenza del processo di inserimento di un utente}
			\label{fig:SeqInserimentoUtente}
		\end{figure}
		
		\subsubsection{Inserimento di un anno accademico}
		Il diagramma di sequenza riportato in Figura~\ref{fig:SeqInsertYear} raffigura il processo di inserimento di un nuovo anno accademico nel sistema.
		
		\begin{figure}[h]
			\centering
				\includegraphics[height=4in]{./Diagrammi/DiagrammaSequenzaInsertAcademicYear.pdf}
			\caption{Diagramma di sequenza del processo di inserimento di un anno accademico}
			\label{fig:SeqInsertYear}
		\end{figure}
		\clearpage
		
	
	
	
	
	
\newpage
\section{Requisiti soddisfatti}
\subsection{Tabella del soddisfacimento dei requisiti}
\begin{table}[hp]
\centering
\begin{tabular}{|c|c|c|}
\hline
ID requisito & Soddisfacimento nell'architettura & Soddisfacimento nel codice \\ \hline
R0F1         & SODDISFATTO                       & SODDISFATTO                \\ \hline
R0F2         & SODDISFATTO                       & SODDISFATTO                \\ \hline
R0F3         & SODDISFATTO                       & SODDISFATTO                \\ \hline
R0F4        & SODDISFATTO                       & SODDISFATTO                \\ \hline
R0F5        & SODDISFATTO                       & SODDISFATTO                \\ \hline
R0F6        & SODDISFATTO                       & SODDISFATTO                \\ \hline
R2F7        & SODDISFATTO                       & NON SODDISFATTO                \\ \hline
R2F8        & SODDISFATTO                       & NON SODDISFATTO                \\ \hline
R2F9        & SODDISFATTO                       & NON SODDISFATTO                \\ \hline
R2F10        & SODDISFATTO                       & NON SODDISFATTO                \\ \hline
R2F11        & SODDISFATTO                       & NON SODDISFATTO                \\ \hline
R2F13        & SODDISFATTO                       & NON SODDISFATTO                \\ \hline
R2F14        & SODDISFATTO                       & NON SODDISFATTO                \\ \hline
R0F15        & SODDISFATTO                       & SODDISFATTO                \\ \hline
R0F16        & SODDISFATTO                       & SODDISFATTO                \\ \hline
R0F17        & SODDISFATTO                       & SODDISFATTO                \\ \hline
R0F18        & SODDISFATTO                       & SODDISFATTO                \\ \hline
R0F19        & SODDISFATTO                       & NON SODDISFATTO                \\ \hline
R2F20        & SODDISFATTO                       & NON SODDISFATTO                \\ \hline
R2F21        & SODDISFATTO                       & NON SODDISFATTO                \\ \hline
R2F22        & SODDISFATTO                       & NON SODDISFATTO                \\ \hline
R0F23        & SODDISFATTO                       & SODDISFATTO                \\ \hline
R0F24        & SODDISFATTO                       & SODDISFATTO                \\ \hline
R2F25        & SODDISFATTO                       & NON SODDISFATTO                \\ \hline
R0F26        & SODDISFATTO                       & SODDISFATTO                \\ 
\hline
\end{tabular}
\caption{Requisiti soddisfatti}
\end{table}
\clearpage

\subsection{Grafici sui requisiti soddisfatti}
\begin{figure}[hp]
\centering
\includegraphics[height=7cm]{img/RequisitiSoddisfatti.png}\\
\caption{Requisiti soddisfatti}
\end{figure}

\begin{figure}[hp]
\centering
\includegraphics[height=7cm]{img/RequisitiObbligatoriSoddisfatti.png}\\
\caption{Requisiti obbligatori soddisfatti}
\end{figure}


\end{document}


