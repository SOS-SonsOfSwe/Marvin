\section{Introduzione}
\subsection{Scopo del documento}
Questo documento ha lo scopo di descrivere gli \emph{attori}\ped{G} del sistema, individuare i \emph{casi d'uso}\ped{G} a partire dai \emph{requisiti}\ped{G} e fornire una visione chiara ai \progs{} sul problema da trattare. I requisiti verranno classificati in questo documento a seguito di una trattazione col proponente.

\subsection{Scopo del prodotto}
Lo scopo del prodotto è quello di realizzare un \emph{prototipo}\ped{G} di \emph{Uniweb}\ped{G} come \emph{ÐApp}\ped{G} in esecuzione su \emph{Ethereum}\ped{G}. I cinque attori principali che si rapportano con Marvin sono:
\begin{itemize}
	\item Utente non autenticato;
	\item Università;
	\item Amministratore;
	\item Professori;
	\item Studenti.
\end{itemize} 
Il portale deve quindi permettere agli studenti di accedere alle informazioni riguardanti le loro carriere universitarie, di iscriversi agli esami, di accettare o rifiutare voti e di poter vedere il loro libretto universitario.
Ai professori deve invece essere permesso di registrare i voti degli studenti.
L'università ogni anno crea una serie di corsi di laurea rivolti a studenti, dove ognuno di essi comprende un elenco di esami disponibili per anno accademico. Ogni esame ha un argomento, un numero di crediti e un professore associato. Gli studenti si iscrivono ad un corso di laurea e tramite il libretto elettronico mantengono traccia ufficiale del progresso.

\subsection{Glossario}
Nel documento Glossario i termini tecnici, gli acronimi e le abbreviazioni sono definiti in modo chiaro e conciso, in modo tale da evitare ambiguità e massimizzare la comprensione dei documenti.
\newline I vocaboli presenti in esso saranno posti in corsivo e presenteranno una "G" maiuscola a pedice.
\subsection{Riferimenti}
\subsubsection{Normativi}
\begin{itemize}
	\item \textcolor{red}\NdP
\end{itemize}

\subsubsection{Informativi}
\begin{itemize}
	\item Capitolato d'appalto C6: Marvin. Reperibile all'indirizzo:\\ 
	\href{http://www.math.unipd.it/~tullio/IS-1/2017/Progetto/C6.pdf}{http://www.math.unipd.it/~tullio/IS-1/2017/Progetto/C6.pdf};
	\item \textcolor{red}\SdF;
\end{itemize}

\section{Descrizione generale}
	\subsection{Contesto d'uso del prodotto}
	Il prodotto finale vuole essere una sorta di \emph{PoC}\ped{G} per dimostrare la fattibilità di utilizzo di tali tecnologie in quest’ambito. L’applicazione sarà un prototipo di Uniweb, quindi si colloca in un contesto universitario dove gli attori si approcciano al sistema come nell’attuale Uniweb. La differenza sta nel \emph{back-end}\ped{G} dove, invece del classico sistema \emph{client}\ped{G}/\emph{server}\ped{G}, troviamo un database distribuito che sfrutta la piattaforma Ethereum.
	
	\subsection{Caratteristiche degli utenti}
	Questo prodotto deve risultare accessibile ad un'ampia categoria di utenti senza particolari competenze. L’interfaccia dovrà quindi essere il più chiara ed intuitiva possibile. Verrà fornito anche un \MU{} con tutte le indicazioni necessarie per consentire il corretto utilizzo del prodotto.
	
	\subsection{Assunzione dipendenze}
	Per il corretto funzionamento dell’applicazione sarà necessario l’utilizzo di un browser che sia compatibile con \emph{HTML5}\ped{G}, \emph{SCSS}\ped{G} e \emph{Javascript}\ped{G}.
