%generare il pdf con il comando: pdflatex main.tex
\documentclass[a4paper, oneside, openany, dvipsnames, table]{article}
\usepackage{../template/sos}
\usepackage{eurosym}
\usepackage{float}
\definecolor{bluSOS}{RGB}{48, 84, 150}
\definecolor{greySOS}{RGB}{209, 222, 223}
\newcommand{\Titolo}{Piano di Progetto}

\newcommand{\Gruppo}{Sons Of SWE}

\newcommand{\Redazione}{Panozzo Stefano \newline Favero Andrea \newline Dalla Riva Giovanni \newline Federico Caldart \newline Eleonora Thiella}

\newcommand{\ACapoRedazione}{Panozzo Stefano \newline Favero Andrea \newline Dalla Riva Giovanni \newline Federico Caldart \newline Eleonora Thiella}

\newcommand{\Verifica}{Dalla Riva Giovanni \newline Menegon Lorenzo \newline Cavallin Giovanni \newline Panozzo Stefano}

\newcommand{\Approvazione}{Cavallin Giovanni \newline Favero Andrea \newline Caldart Federico}

\newcommand{\Distribuzione}{Vardanega Tullio\newline Cardin Riccardo \newline Gruppo Sons Of SWE}

\newcommand{\Uso}{Esterno}


\newcommand{\Data}{07 Maggio 2018}

\newcommand{\NomeProgetto}{Progetto Marvin}

\newcommand{\Mail}{sonsofswe.swe@gmail.com}


\newcommand{\DescrizioneDoc}{Documento contenente il piano di progetto relativo al \NomeProgetto{} scelto dal gruppo \Gruppo.}

\newcommand{\versione}{3.0.0}




\begin{document}
\copertina{}
%%%%%%%%%%%%%%%%%%%%%%%%%%%%%%%%%%%%%%%%%%%%%%%%%%%%%%%%%%%%%%%%%%%%%%%%%%%%%%%%%%%%%%%%%%%%%%%%%%%%%%%
%SOMMARIO
\definecolor{bluSOS}{RGB}{48, 84, 150}
\definecolor{greySOS}{RGB}{209, 222, 223}
\section*{Registro delle modifiche}
{
	\rowcolors{2}{greySOS}{white}
	\renewcommand{\arraystretch}{1}
	\centering
	\begin{longtable}{| c| c | C{4cm} | c | c |}
		\hline
		\rowcolor{bluSOS}
		\textcolor{white}{\textbf{Versione}} & \textcolor{white}{\textbf{Data}} & \textcolor{white}{\textbf{Descrizione}} & \textcolor{white}{\textbf{Autore}} & \textcolor{white}{\textbf{Ruolo}}\\
		\hline
		0.0.1 & 16-03-2018 &   &  & Reponsabile\\
		\hline
		0.0.1 & 16-03-2018 & Creato lo scheletro del documento e scrittura dell' \emph{introduzione}  & Giovanni Cavallin & Reponsabile\\ 
		\hline
	\end{longtable}

}


%dvipsnames, table
%\definecolor{bluSOS}{RGB}{48, 84, 150}
\definecolor{greySOS}{RGB}{209, 222, 223}
\section*{Registro delle modifiche}
{
	\rowcolors{2}{greySOS}{white}
	\renewcommand{\arraystretch}{1}
	\centering
	\begin{longtable}{| c| c | C{4cm} | c | c |}
		\hline
		\rowcolor{bluSOS}
		\textcolor{white}{\textbf{Versione}} & \textcolor{white}{\textbf{Data}} & \textcolor{white}{\textbf{Descrizione}} & \textcolor{white}{\textbf{Autore}} & \textcolor{white}{\textbf{Ruolo}}\\
		\hline
		0.0.1 & 16-03-2018 &   &  & Reponsabile\\
		\hline
		0.0.1 & 16-03-2018 & Creato lo scheletro del documento e scrittura dell' \emph{introduzione}  & Giovanni Cavallin & Reponsabile\\ 
		\hline
	\end{longtable}

}


%dvipsnames, table
%\definecolor{bluSOS}{RGB}{48, 84, 150}
\definecolor{greySOS}{RGB}{209, 222, 223}
\section*{Registro delle modifiche}
{
	\rowcolors{2}{greySOS}{white}
	\renewcommand{\arraystretch}{1}
	\centering
	\begin{longtable}{| c| c | C{4cm} | c | c |}
		\hline
		\rowcolor{bluSOS}
		\textcolor{white}{\textbf{Versione}} & \textcolor{white}{\textbf{Data}} & \textcolor{white}{\textbf{Descrizione}} & \textcolor{white}{\textbf{Autore}} & \textcolor{white}{\textbf{Ruolo}}\\
		\hline
		0.0.1 & 16-03-2018 &   &  & Reponsabile\\
		\hline
		0.0.1 & 16-03-2018 & Creato lo scheletro del documento e scrittura dell' \emph{introduzione}  & Giovanni Cavallin & Reponsabile\\ 
		\hline
	\end{longtable}

}


%dvipsnames, table
%\input{sez/RegistroModifiche.tex}
%\newpage
%\newpage
%\newpage
\newpage
\tableofcontents
\newpage
%%%%%%%%%%%%%%%%%%%%%%%%%%%%%%%%%%%%%%%%%%%%%%%%%%%%%%%%%%%%%%%%%%%%%%%%%%%%%%%%%%%%%%%%%%%%%%%%%%%%%%%
%PARAGRAFI
\hypersetup{linkcolor=bluSOS}
\section{Introduzione}
\subsection{Scopo del documento}
Questo documento ha lo scopo di descrivere gli \emph{attori}\ped{G} del sistema, individuare i \emph{casi d'uso}\ped{G} a partire dai \emph{requisiti}\ped{G} e fornire una visione chiara ai \progs{} sul problema da trattare. I requisiti verranno classificati in questo documento a seguito di una trattazione col proponente.

\subsection{Scopo del prodotto}
Lo scopo del prodotto è quello di realizzare un \emph{prototipo}\ped{G} di \emph{Uniweb}\ped{G} come \emph{ÐApp}\ped{G} in esecuzione su \emph{Ethereum}\ped{G}. I cinque attori principali che si rapportano con Marvin sono:
\begin{itemize}
	\item Utente non autenticato;
	\item Università;
	\item Amministratore;
	\item Professori;
	\item Studenti.
\end{itemize} 
Il portale deve quindi permettere agli studenti di accedere alle informazioni riguardanti le loro carriere universitarie, di iscriversi agli esami, di accettare o rifiutare voti e di poter vedere il loro libretto universitario.
Ai professori deve invece essere permesso di registrare i voti degli studenti.
L'università ogni anno crea una serie di corsi di laurea rivolti a studenti, dove ognuno di essi comprende un elenco di esami disponibili per anno accademico. Ogni esame ha un argomento, un numero di crediti e un professore associato. Gli studenti si iscrivono ad un corso di laurea e tramite il libretto elettronico mantengono traccia ufficiale del progresso.

\subsection{Glossario}
Nel documento Glossario i termini tecnici, gli acronimi e le abbreviazioni sono definiti in modo chiaro e conciso, in modo tale da evitare ambiguità e massimizzare la comprensione dei documenti.
\newline I vocaboli presenti in esso saranno posti in corsivo e presenteranno una "G" maiuscola a pedice.
\subsection{Riferimenti}
\subsubsection{Normativi}
\begin{itemize}
	\item \textcolor{red}\NdP
\end{itemize}

\subsubsection{Informativi}
\begin{itemize}
	\item Capitolato d'appalto C6: Marvin. Reperibile all'indirizzo:\\ 
	\href{http://www.math.unipd.it/~tullio/IS-1/2017/Progetto/C6.pdf}{http://www.math.unipd.it/~tullio/IS-1/2017/Progetto/C6.pdf};
	\item \textcolor{red}\SdF;
\end{itemize}

\section{Descrizione generale}
	\subsection{Contesto d'uso del prodotto}
	Il prodotto finale vuole essere una sorta di \emph{PoC}\ped{G} per dimostrare la fattibilità di utilizzo di tali tecnologie in quest’ambito. L’applicazione sarà un prototipo di Uniweb, quindi si colloca in un contesto universitario dove gli attori si approcciano al sistema come nell’attuale Uniweb. La differenza sta nel \emph{back-end}\ped{G} dove, invece del classico sistema \emph{client}\ped{G}/\emph{server}\ped{G}, troviamo un database distribuito che sfrutta la piattaforma Ethereum.
	
	\subsection{Caratteristiche degli utenti}
	Questo prodotto deve risultare accessibile ad un'ampia categoria di utenti senza particolari competenze. L’interfaccia dovrà quindi essere il più chiara ed intuitiva possibile. Verrà fornito anche un \MU{} con tutte le indicazioni necessarie per consentire il corretto utilizzo del prodotto.
	
	\subsection{Assunzione dipendenze}
	Per il corretto funzionamento dell’applicazione sarà necessario l’utilizzo di un browser che sia compatibile con \emph{HTML5}\ped{G}, \emph{SCSS}\ped{G} e \emph{Javascript}\ped{G}.

\newpage

\section{Analisi dei rischi}

Al fine di ridurre al minimo i possibili ritardi sulla pianificazione e migliorare la qualità del progetto, vengono di seguito analizzati i rischi che potrebbero insorgere nel corso dello sviluppo.\\

È possibile prendere visione dei rischi che si sono effettivamente riscontrati durante i periodi di progetto nell'\hyperref[RiscontroRischi]{Appendice A}.\\
Di seguito la tabella con i rischi individuati, divisi a seconda del livello di appartenenza:

\subsection{Livello tecnologico}
\begin{table}[h!]
	\centerline{\includegraphics[scale=0.55]{img/Rischi/LivelloTecnologico.jpg}}
	\caption{Tabella dei rischi: Livello Tecnologico}
\end{table}
\clearpage

\subsection{Livello personale}
\begin{table}[h!]
	\centerline{\includegraphics[scale=0.55]{img/Rischi/LivelloPersonale.jpg}}
	\caption{Tabella dei rischi: Livello Personale}
\end{table}
\clearpage

\subsection{Livello organizzativo}
\begin{table}[h!]
	\centerline{\includegraphics[scale=0.55]{img/Rischi/LivelloOrganizzativo.jpg}}
	\caption{Tabella dei rischi: Livello Organizzativo}
\end{table}
\clearpage

\subsection{Strumenti}
\begin{table}[h!]
	\centerline{\includegraphics[scale=0.55]{img/Rischi/Strumenti.jpg}}
	\caption{Tabella dei rischi: Strumenti}
\end{table}

\subsection{Requisiti}
\begin{table}[h!]
	\centerline{\includegraphics[scale=0.55]{img/Rischi/Requisiti.jpg}}
	\caption{Tabella dei rischi: Requisiti}
\end{table}
\clearpage
\newpage
\section{Pianificazione}

Per migliorare lo sviluppo del progetto e rispettare le scadenze elencate al punto \hyperref[Scadenze]{1.6} di questo documento, lo sviluppo è stato suddiviso in sei periodi, ripartiti a loro volta in due macro periodi indicanti il primo, un periodo di investimento a carico del gruppo \Gruppo{} ed il secondo, facente parte del preventivo a carico dell'azienda proponente.

Periodo di investimento:
\begin{itemize}
\item \textbf{Analisi dei requisiti};
\item \textbf{Analisi dei requisiti in dettaglio}.
\end{itemize}
Periodo rendicontabile:
\begin{itemize}
\item \textbf{Prototipazione};
\item \textbf{Prototipazione in dettaglio};
\item \textbf{Progettazione finale e Codifica};
\item \textbf{Codifica in dettaglio, Validazione e Collaudo}.
\end{itemize}
Di seguito sono analizzati in dettaglio i periodi sopracitati e per ognuno di essi viene riportato il diagramma di Gantt; il diagramma riporta le milestone e le date di inizio e fine di ciascuna attività.

\subsection{Analisi dei requisiti}
\textbf{Periodo}: dal 2018-03-03 al 2018-04-13 (RR)\\

Questo periodo comincia con la creazione del gruppo e si conclude con la consegna dei documenti per accedere alla \RR{}.\\
I documenti stilati e successivamente verificati durante questo periodo sono:
\begin{itemize}
\item \textbf{Norme di Progetto}: questo è il primo documento redatto in ordine cronologico poiché norma lo svolgimento di tutte le attività del gruppo \Gruppo{}; esso è indipendente dal capitolato scelto;
\item \textbf{Studio di fattibilità}: in questo documento vengono analizzati tutti i capitolati proposti. Per ognuno viene analizzato il dominio applicativo e tecnologico, valutandone i fattori positivi e negativi. È un’attività critica perché definisce il progetto sul quale il gruppo andrà a lavorare e blocca la stesura del documento di \AdR{};
\item \textbf{Piano di Progetto}: vengono pianificate tutte le attività necessarie allo svolgimento del progetto ed assegnate alle risorse disponibili, distribuendo il carico di lavoro in maniera uniforme;
\item \textbf{Piano di Qualifica}: vengono definiti gli standard qualitativi e le metriche da utilizzare per verifica e validazione;
\item \textbf{Analisi dei Requisiti}: viene effettuata l’analisi approfondita del capitolato scelto con lo \SdF{} e vengono identificati i requisiti obbligatori e facoltativi;
\item \textbf{Glossario}: contiene la definizione di alcuni termini utilizzati nei vari documenti, al fine di eliminare ogni possibile ambiguità di significato;
\item \textbf{Lettera di Presentazione}: documento che dichiara l’interesse del gruppo a partecipare alla gara d’appalto.
\end{itemize}

\begin{figure}[h!]
	\centerline{\includegraphics[scale=0.35]{img/DiagrammiGantt/AnalisiRequisiti.jpg}}
	\caption{Diagramma di Gantt: Analisi dei requisiti}
	\label{fig:gantt_ana_req}
\end{figure}
\clearpage

\subsection{Analisi dei Requisiti in dettaglio}
\textbf{Periodo}: dal 2018-04-14 (RR) al 2018-04-26 (Milestone Interna)\\

In attesa dell'esito della \RR{}, i membri del gruppo si impegnano a colmare le proprie lacune tecnologiche necessarie allo svolgimento del progetto. Parallelamente, si mira a consolidare ed ampliare i requisiti richiesti dal sistema e a migliorare il documento di \AdR{} attuando le correzioni in base all’esito della \RR{}; vengono inoltre corretti e verificati anche gli altri documenti. Il termine fissato per la conclusione di questa fase corrisponde ad una milestone interna.

\begin{figure}[h!]
	\centerline{\includegraphics[scale=0.55]{img/DiagrammiGantt/AnalisiRequisitiDettaglio.jpg}}
	\caption{Diagramma di Gantt: Analisi dei requisiti in Dettaglio}
	\label{fig:gantt_ana_req_dett}
\end{figure}
\clearpage

\subsection{Prototipazione}
\textbf{Periodo}: dal 2018-04-27 (Milestone Interna) al 2018-05-07 (RP)\\

Questo periodo è caratterizzato dalla realizzazione di un prototipo utilizzando le tecnologie necessarie, sulla base delle scelte del gruppo \Gruppo{} e delle richieste del proponente.
Lo scopo è quello di comprendere pienamente il dominio tecnologico del progetto e realizzare i casi d'uso ritenuti più importanti e significativi per la buona riuscita del prodotto finale, realizzando così una \emph{Technology Baseline}\ped{G}. \\ 
Come attività di supporto si incrementano i documenti già redatti nei periodi precedenti.\\
Per semplicità si considera come periodo di consegna e presentazione della Technology Baseline la data di \RP{}; sarà poi specificata una scadenza più precisa in fase di sviluppo.

\begin{figure}[h!]
	\centerline{\includegraphics[scale=0.55]{img/DiagrammiGantt/Prototipazione.jpg}}
	\caption{Diagramma di Gantt: Prototipazione}
	\label{fig:gantt_prot}
\end{figure}
\clearpage

\subsection{Prototipazione in Dettaglio}
\textbf{Periodo}: dal 2018-05-08 (RP) al 2018-05-16 (Milestone Interna)\\

In attesa dell'esito della \RP{}, si incrementa la \emph{PoC}\ped{G} finora realizzata, migliorando i requisiti già interessati. In seguito verranno effettuate le dovute correzioni a questa e ai documenti in base all'esito della \RP{}. La conclusione di questo periodo corrisponde ad una milestone interna.

\begin{figure}[h!]
	\centerline{\includegraphics[scale=0.55]{img/DiagrammiGantt/PrototipazioneDettaglio.jpg}}
	\caption{Diagramma di Gantt: Prototipazione in dettaglio}
	\label{fig:gantt_prot_dett}
\end{figure}
\clearpage

\subsection{Progettazione finale e Codifica}
\textbf{Periodo}: dal 2018-05-17 (Milestone Interna) al 2018-06-08 (RQ)\\

Si procede con la progettazione in dettaglio dell'architettura di sistema, comprensiva di design pattern implementati e diagrammi delle classi e di sequenza; il risultato di questo insieme di attività comporrà la \emph{ProductBaseline}\ped{G} del prodotto. Una volta definita, si procede con la fase di codifica per realizzare i requisiti obbligatori stabiliti in precedenza, riutilizzando ed ampliando il codice prodotto durante la fase di Prototipazione e Prototipazione in Dettaglio. \\
Come attività di supporto si incrementano alcuni documenti già redatti nei periodi precedenti e si stilano due nuovi documenti:
\begin{itemize}
	\item \textbf{Manuale Utente}: contiene indicazioni utili all'utente finale per l'utilizzo del prodotto
	\item \textbf{Manuale Sviluppatore}: contiene indicazioni utili allo sviluppatore per la manutenzione e l'incremento del prodotto
\end{itemize}
Per semplicità si considera come periodo di consegna e presentazione della Product Baseline la data di \RQ{}; sarà poi specificata una scadenza più precisa in fase di sviluppo.

\begin{figure}[h!]
	\centerline{\includegraphics[scale=0.5]{img/DiagrammiGantt/ProgettazioneFinaleCodifica.jpg}}
	\caption{Diagramma di Gantt: Progettazione finale e Codifica}
	\label{fig:gantt_prog_fin_cod}
\end{figure}
\clearpage

\subsection{Codifica in Dettaglio, Validazione e Collaudo}
\textbf{Periodo}: dal 2018-06-15 (RQ) al 2018-07-15 (RA)\\

Durante quest'ultimo periodo di sviluppo, si incrementano i documenti redatti finora e si effettuano gli ultimi incrementi di codifica; a seguito di queste attività si procede con la verifica del sistema completo e il collaudo di esso.

\begin{figure}[h!]
	\centerline{\includegraphics[scale=0.45]{img/DiagrammiGantt/CodificaValidazioneCollaudo.jpg}}
	\caption{Diagramma di Gantt: Codifica in dettaglio, Validazione e Collaudo}
	\label{fig:gantt_cod_valid_coll}
\end{figure}
\clearpage

\newpage
\section{Preventivo}

Per il preventivo si tiene conto che i primi due periodi sono considerati di investimento e non a carico del committente, per cui le ore rendicontate durante questi periodi non saranno conteggiate nelle ore totali da retribuire.
La suddivisione oraria viene fatta tenendo conto delle seguenti regole:
\begin{itemize}
	\item ogni membro del gruppo dovrà sostenere una mole di lavoro comparabile, quindi il totale delle ore dovrà essere equamente distribuito tra i membri;
	\item ogni membro del gruppo dovrà ricoprire ogni ruolo almeno una volta durante il ciclo di sviluppo del prodotto;
	\item in nessun caso si dovrà verificare un conflitto di interessi in cui un Verificatore debba	controllare il proprio lavoro.
\end{itemize}
Le sigle utilizzate per i vari ruoli saranno:
\begin{itemize}
	\item Re: Responsabile;
	\item Ad: Amministratore;
	\item An: Analista;
	\item Pr: Progettista;
	\item Pg: Programmatore;
	\item Ve: Verificatore.
\end{itemize}
Per facilitare la lettura delle tabelle e dei grafici si è deciso che, nel caso in cui un valore sia pari a 0, questo verrà omesso lasciando la cella o il corrispettivo vuoto.

\subsection{Analisi dei requisiti}
\subsubsection{Prospetto orario}
\subsubsection{Prospetto economico}

\subsection{Analisi dei requisiti in dettaglio}
\subsubsection{Prospetto orario}
\subsubsection{Prospetto economico}

\subsection{Prototipazione}
\subsubsection{Prospetto orario}
\subsubsection{Prospetto economico}

\subsection{Prototipazione in dettaglio}
\subsubsection{Prospetto orario}
\subsubsection{Prospetto economico}

\subsection{Progettazione finale e Codifica}
\subsubsection{Prospetto orario}
\subsubsection{Prospetto economico}

\subsection{Codifica in Dettaglio, Validazione e Collaudo}
\subsubsection{Prospetto orario}
\subsubsection{Prospetto economico}
\newpage
\section{Consuntivo e Preventivo a finire}

In questa sezione viene effettuato un confronto fra le ore ed il costo preventivati con quelli riscontrati effettivamente durante i periodi del ciclo di sviluppo.
\subsection{Analisi dei requisiti}
\subsubsection{Consuntivo}
Di seguito è riportata la tabella riassuntiva per il consuntivo del primo periodo
\begin{figure}[h!]
	\centerline{\includegraphics[scale=0.4]{img/Preventivo/AnalisiRequisiti.Consuntivo.jpg}}
	\caption{Consuntivo: Analisi dei requisiti}
\end{figure}
\subsection{Conclusioni}

%%%%%%%%%%%%%%%%%%%%%%%%%%%%%%%%%%%%%%%%%%%%%%%%%%%%%%%%%%%%%%%%%%%%%%%%%%%%%%%%%%%%%%%%%%%%%%%%%%%%%%%
\appendix 
\newpage

\section{Attualizzazione dei rischi} \label{RiscontroRischi}

Vengono di seguito analizzati i problemi riscontrati durante lo svolgimento del progetto, suddivisi per ogni periodo; ad ogni occorrenza rilevata corrisponderà una breve descrizione del problema e le contromisure adottate per la sua mitigazione. È inoltre possibile prendere visione di come i problemi qui analizzati abbiano influito sul consuntivo di periodo sul preventivo finale nell'\hyperref[ConsuntivoPeriodo]{Appendice B}.\\

\subsection{Analisi dei requisiti}
\begin{table}[h!]
	\centerline{\includegraphics[scale=0.55]{img/Rischi/RiscontroProblemi-AnalisiRequisiti.jpg}}
	\caption{Riscontro problemi: Analisi dei Requisiti}
\end{table}
\clearpage

\subsection{Analisi dei requisiti in Dettaglio} \label{RiscontroAnalisiDettaglio}
\begin{table}[h!]
	\centerline{\includegraphics[scale=0.55]{img/Rischi/RiscontroProblemi-AnalisiRequisitiDettaglio.jpg}}
	\caption{Riscontro problemi: Analisi dei Requisiti in Dettaglio}
\end{table}

\subsection{Prototipazione}
\begin{table}[h!]
	\centerline{\includegraphics[scale=0.55]{img/Rischi/RiscontroProblemi-Prototipazione.jpg}}
	\caption{Riscontro problemi: Prototipazione}
\end{table}

\subsection{Prototipazione in Dettaglio} \label{RiscontroPrototipazioneDettaglio}
\begin{table}[h!]
	\centerline{\includegraphics[scale=0.55]{img/Rischi/RiscontroProblemi-PrototipazioneDettaglio.jpg}}
	\caption{Riscontro problemi: Prototipazione in Dettaglio}
\end{table}

\subsection{Progettazione finale e Codifica}
Durante questo periodo non si sono verificati problemi che abbiano influito particolarmente sul perseguimento delle attività pianificate.

\pagebreak
\newpage
\renewcommand{\arraystretch}{1}

\section{Organigramma}

\subsection{Redazione}
\begin{longtable}{ c  c  c }
	\rowcolor{bluSOS}
	\textcolor{white}{\textbf{Nominativo}} & \textcolor{white}{\textbf{Data di redazione}} & \textcolor{white}{\textbf{Firma}}\\
	
	Eleonora Thiella & 2018-06-05 & \includegraphics[height=0.5cm]{img/Firme/EleonoraThiella.png}\\
	\caption{Componente per la redazione}  \\
\end{longtable}

\subsection{Approvazione}
\rowcolors{2}{greySOS}{white}
\begin{longtable}{ c  c  c }
	\rowcolor{bluSOS}
	\textcolor{white}{\textbf{Nominativo}} & \textcolor{white}{\textbf{Data di approvazione}} & \textcolor{white}{\textbf{Firma}}\\
	
	Eleonora Thiella & 2018-06-05 & \includegraphics[height=0.5cm]{img/Firme/EleonoraThiella.png}  \\
	Tullio Vardanega & & \\
	\rowcolor{white}\caption{Componenti per l'approvazione}\\
\end{longtable}

\subsection{Accettazione dei componenti}
\rowcolors{2}{greySOS}{white}
%\cmidrule(r){1-1}\cmidrule(lr){2-2}\cmidrule(l){3-3}
% \raisebox{-\totalheight}{\includegraphics[width=0.3\textwidth, height=60mm]{images/myLboro.png}}

\begin{longtable}{ c  c  c }
	\rowcolor{bluSOS}
	\textcolor{white}{\textbf{Nominativo}} & \textcolor{white}{\textbf{Data di accettazione}} & \textcolor{white}{\textbf{Firma}}\\
	%\includegraphics[height=2.5cm]{immagine} aggiustare l'altezza
	Andrea Favero & 2018-06-05 & \includegraphics[height=0.5cm]{img/Firme/AndreaFavero.png} \\
	
	Eleonora Thiella & 2018-06-05 & \includegraphics[height=0.5cm]{img/Firme/EleonoraThiella.png} \\
	
	Federico Caldart & 2018-06-05 & \includegraphics[height=0.5cm]{img/Firme/FedericoCaldart.png} \\
	
	Giovanni Cavallin & 2018-06-05 & \includegraphics[height=0.5cm]{img/Firme/GiovanniCavallin.png} \\
	
	Giovanni Dalla Riva & 2018-06-05 & \includegraphics[height=0.5cm]{img/Firme/GiovanniDallaRiva.png} \\
	
	Lorenzo Menegon & 2018-06-05 & \includegraphics[height=0.5cm]{img/Firme/LorenzoMenegon.png} \\
	
	Stefano Panozzo & 2018-06-05 & \includegraphics[height=0.5cm]{img/Firme/StefanoPanozzo.png} \\
	\rowcolor{white}\caption{Accettazione dei componenti}\\
\end{longtable}


\subsection{Componenti}
\rowcolors{2}{greySOS}{white}
\begin{longtable}{ c  c  c  C{4cm} }
	\rowcolor{bluSOS}
	\textcolor{white}{\textbf{Nominativo}} & \textcolor{white}{\textbf{Matricola}} & \textcolor{white}{\textbf{Indirizzo di posta elettronica}} & \textcolor{white}{\textbf{Ruoli previsti}}\\
	
	Andrea Favero & 1125545 & andrea.favero.8@studenti.unipd.it & \RdP, \prog, \progr, \ver \\
	
	Eleonora Thiella & 1099980 & eleonora.thiella@studenti.unipd.it & \RdP, \adm, \prog, \progr, \ver \\
	
	Federico Caldart & 1097005 & federico.caldart.1@studenti.unipd.it & \prog, \progr, \ver \\
	
	Giovanni Cavallin & 1148957 & giovanni.cavallin.1@studenti.unipd.it & \prog, \progr, \ver\\
	
	Giovanni Dalla Riva & 1102075 & giovanni.dallariva@studenti.unipd.it & \prog, \progr, \ver \\
	
	Lorenzo Menegon & 1097596 & lorenzo.menegon@studenti.unipd.it & \RdP, \prog, \progr, \ver \\
	
	Stefano Panozzo & 1097068 & stefano.panozzo.1@studenti.unipd.it & \prog, \progr, \ver \\
	\rowcolor{white}\caption{Elenco dei componenti}\\
\end{longtable}
\pagebreak

\end{document}