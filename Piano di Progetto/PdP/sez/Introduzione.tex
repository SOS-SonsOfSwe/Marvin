\newpage
\section{Introduzione}
\subsection{Scopo del documento}
Questo documento contiene la pianificazione delle attività che saranno svolte dai membri del gruppo \Gruppo{} per realizzare il \NomeProgetto. In particolare, questo documento contiene informazioni riguardanti:

\begin{itemize}
	\item L'analisi ed il trattamento dei rischi;
	\item La pianificazione delle attività per l'intera durata del progetto;
	\item Il preventivo delle risorse necessarie allo svolgimento del progetto;
	\item Il \emph{consuntivo}\ped{G} delle attività finora svolte.
\end{itemize}

\subsection{Scopo del prodotto}
Lo scopo del prodotto è quello di realizzare un \emph{prototipo}\ped{G} di Uniweb come una \emph{\DJ App}\ped{G} in esecuzione su \emph{Ethereum}\ped{G}. I cinque attori principali che si rapportano con Marvin sono:
\begin{itemize}
	\item Utente non autenticato; 
	\item Università;
	\item Amministratore;
	\item Professore;
	\item Studente.
\end{itemize} 
Il portale deve quindi permettere agli studenti di accedere alle informazioni riguardanti le loro carriere universitarie, di iscriversi agli esami, di accettare o rifiutare voti e di poter vedere il loro libretto universitario.
Ai professori deve invece essere permesso di registrare i voti degli studenti.
L'università ogni anno crea una serie di corsi di laurea rivolti a studenti, dove ognuno di essi comprende un elenco di esami disponibili per anno accademico. Ogni esame ha un argomento, un numero di crediti e un professore associato. Gli studenti si iscrivono ad un corso di laurea e tramite il libretto elettronico mantengono traccia ufficiale del progresso.

\subsection{Glossario}
Nel documento \G{} i termini tecnici, gli acronimi e le abbreviazioni sono definiti in modo chiaro e conciso, in modo tale da evitare ambiguità e massimizzare la comprensione dei documenti.
\newline I vocaboli presenti in esso saranno posti in corsivo e presenteranno una "G" maiuscola a pedice.

\subsection{Riferimenti}
\subsubsection{Normativi}
\begin{itemize}
	\item \textbf{\NdP{}};
	\item \textbf{Capitolato d'appalto C6: \NomeProgetto}:\\
	\url{http://www.math.unipd.it/~tullio/IS-1/2017/Progetto/C6.pdf};
	\item \textbf{Regolamento del progetto didattico}:\\
	\url{http://www.math.unipd.it/~tullio/IS-1/2017/Dispense/P01.pdf};
	\item \textbf{Vincoli di organigramma e dettagli tecnico-economici}:\\
	\url{http://www.math.unipd.it/~tullio/IS-1/2017/Progetto/RO.html}.
\end{itemize}
\subsubsection{Informativi}
\begin{itemize}
	\item \textbf{\SdF{}};
	\item \textbf{\AdR{}};
	\item \textbf{Software Engineering (10th edition) - Ian Sommerville}:
	\begin{itemize}
		\item Chapter 2: Software processes;
		\item Chapter 22: Project management;
		\item Chapter 23: Project Planning.
	\end{itemize}
	\item \textbf{Slides del corso di Ingegneria del Software}:\\
	\url{http://www.math.unipd.it/~tullio/IS-1/2017/}.
\end{itemize}

\subsection{Modello di sviluppo}
Il modello di sviluppo scelto per il progetto è quello \emph{incrementale}\ped{G}.\\
Durante i primi periodi, grazie ad un'analisi del capitolato e la comunicazione con il proponente, si fissano i requisiti che il sistema dovrà soddisfare e quali invece sono considerati desiderabili ed opzionali. Questo permette di individuare quali dei requisiti hanno una maggior importanza strategica; pertanto questi verranno soddisfatti per primi, mentre gli altri saranno adempiti successivamente.\\
Il modello infatti prevede rilasci multipli successivi, dunque è possibile sottoporre al proponente un prototipo con le funzionalità di primaria importanza nel minor tempo possibile, così da permettere una valutazione in corso d’opera del lavoro svolto. Partendo da questo prototipo sarà poi possibile effettuare un incremento delle funzionalità ed un consolidamento di quelle già presenti.\\
Per ottenere ciò nel modo più \emph{efficiente}\ped{G} ed \emph{efficace}\ped{G} possibile, si prevede la scomposizione dello sviluppo in periodi, al termine dei quali è prevista una \emph{milestone}\ped{G} (interna o esterna). In questo modo le risorse vengono concentrate in un numero limitato di attività e sottoattività parallele, ottenendo come risultato una loro migliore gestione e verifica.
Questo permette un maggiore controllo sulle tempistiche e sui costi in quanto ogni sottoinsieme deve essere precedentemente pianificato; ciò riduce inoltre il rischio di ritardi.\\
Infine, per ogni attività si prevede uno \emph{slack time}\ped{G}, in modo da mitigare eventuali ritardi causati da fattori non prevedibili durante la pianificazione.

\subsection{Scadenze}\label{Scadenze}
Il gruppo \Gruppo{} ha deciso di rispettare le seguenti date di revisione:
\begin{itemize}
	\item \textbf{Revisione dei Requisiti (RR)}: 2018-04-23;
	\item \textbf{Revisione di Progettazione (RP)}: 2018-05-14;
	\item \textbf{Revisione di Qualifica (RQ)}: 2018-06-15;
	\item \textbf{Revisione di Accettazione (RA)}: 2018-07-16.
\end{itemize}
