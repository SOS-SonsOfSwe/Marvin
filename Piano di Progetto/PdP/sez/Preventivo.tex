\newpage
\section{Preventivo}

Per il preventivo si tiene conto che i primi due periodi sono considerati di investimento e non a carico del committente, per cui le ore rendicontate durante questi periodi non saranno conteggiate nelle ore totali da retribuire.
La suddivisione oraria viene fatta tenendo conto delle seguenti regole:
\begin{itemize}
	\item ogni membro del gruppo dovrà sostenere una mole di lavoro comparabile, quindi il totale delle ore dovrà essere equamente distribuito tra i membri;
	\item ogni membro del gruppo dovrà ricoprire ogni ruolo almeno una volta durante il ciclo di sviluppo del prodotto;
	\item in nessun caso si dovrà verificare un conflitto di interessi in cui un Verificatore debba	controllare il proprio lavoro.
\end{itemize}
Le sigle utilizzate per i vari ruoli saranno:
\begin{itemize}
	\item Re: Responsabile;
	\item Ad: Amministratore;
	\item An: Analista;
	\item Pr: Progettista;
	\item Pg: Programmatore;
	\item Ve: Verificatore.
\end{itemize}
Per facilitare la lettura delle tabelle e dei grafici si è deciso che, nel caso in cui un valore sia pari a 0, questo verrà omesso lasciando la cella o il corrispettivo vuoto.

\subsection{Analisi dei requisiti}
\subsubsection{Prospetto orario}
\subsubsection{Prospetto economico}

\subsection{Analisi dei requisiti in dettaglio}
\subsubsection{Prospetto orario}
\subsubsection{Prospetto economico}

\subsection{Prototipazione}
\subsubsection{Prospetto orario}
\subsubsection{Prospetto economico}

\subsection{Prototipazione in dettaglio}
\subsubsection{Prospetto orario}
\subsubsection{Prospetto economico}

\subsection{Progettazione finale e Codifica}
\subsubsection{Prospetto orario}
\subsubsection{Prospetto economico}

\subsection{Codifica in Dettaglio, Validazione e Collaudo}
\subsubsection{Prospetto orario}
\subsubsection{Prospetto economico}