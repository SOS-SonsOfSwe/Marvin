\section{Confronto con la Poc}
In precedenza è stato realizzata una \emph{Proof of Concept} in cui si è data dimostrazione delle tecnologie che si sarebbero dovute utilizzare per la realizzazione del progetto Marvin.
Tale produzione è disponibile all'indirizzo: \url{https://github.com/SOS-SonsOfSwe/Marvin-PoC}.
Durante la ricerca di strumenti adatti allo scopo il team ha deciso inizialmente di basare la prima produzione del prodotto sul framework \emph{Truffle}, dal momento che esso presentava molte delle tecnologie di interesse.
Con uno studio più approfondito del pacchetto il gruppo si è reso conto che esso era una buona base di partenza per la realizzazione di un prodotto più complesso, sebbene necessitasse di alcune correzioni strutturali. Qui di seguito si elencano i limiti riscontrati e le soluzioni adottate:
\newline
\newline

\begin{table}[hp]
	\centering
\begin{tabular}{|p{6cm}|p{6cm}|}
	\hline
	\textbf{Proof of Concept} & \textbf{Product Baseline} \\
	
	\hline Architettura ben strutturata ma male incapsulata
	&
	Organizzazione migliore della struttura del pacchetto
	a fronte di una conoscenza più approfondita del design pattern MVVM e di altri su cui il framework si basa \\ 
	
	\hline Interfaccia povera e strutturata in maniera confusionaria, non chiara e  difficilmente intuibile 
	&
	Ristrutturazione completa e incremento dell'interfaccia utente, ora provvista di tutte le parti atte alla soddisfazione della maggior parte dei requisiti opzionali e obbligatori \\ 
	
	\hline Database solidity estremamente scarno e con poche funzionalità di sicurezza e di ottimizzazione
	&
	Strutturazione avanzata del backend di solidity, desisamente più ottimizzato e performante; implementazione già in fase di completamento\\
	
	\hline
\end{tabular}
\caption{Confronto tra Proof of Concept e Product Baseline}
\end{table}