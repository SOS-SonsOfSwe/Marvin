\section{Introduction}
This is the developer manual of \textbf{Marvin}, a \DH app ran on the EVM, that shapes a subset of \href{www.uniweb.unipd.it}{Uniweb} functionalities. Uniweb is the University of Padua's informative system.
It allows students to keep track of their academic carrier. Professors use Uniweb to see the students registered to their exams and see the exams to which they have been assigned.
\\This manual is intended for programmers wishing to customize or extend \textbf{Marvin}. Those are expected to know and understand the \emph{React-redux} as well as the \emph{truffle} frameworks. In addition, the knowledge of the \emph{solidity} language is mandatory to understand the database on which this application is based. A basic understanding of the Sass preprocessor CSS, while not required, is a plus.

\subsection{What is in the Manual}
This manual will cover most of the aspects of custom development and maintainment of Marvin.
In particular it will be divided into the two main sections of \emph{Frontend} and \emph{backend}. So if the reader is intended to update or modify the \verb|solidity| database part should skip all the \emph{Frontend} one, instead if the purpose is to modify how it is rendered to screen, then he should go the first one.

\subsection{First steps and getting known}
In very few step the reader will be able to download and modify the Marvin package. 
\\First of all, clone, download or fork the repository at \url{https://github.com/SOS-SonsOfSwe/Marvin-SoS.git}. The folder will have many subfolders:
\begin{itemize}
	\item \verb|api| contains all the \emph{adapters} to the external API such as \verb|IPFS| and \verb|web3|;
	\item \verb|contracts| and \verb|migration| contain all the \verb|solidity| \emph{contracts} which are responsible for the \emph{backend} section and the instruction for the \emph{truffle} framework to migrate them to the local blockchain;
	\item \verb|public| contains only media files which can be accessible as external source;
	\item \verb|scripts| and \verb|webpack| are responsible for getting the application start, test and build. If the reader is not confident with the settings of the \emph{truffle} framework and the \emph{webpack} package it's recommended not to modify those files;
	\item \verb|src| contains all the folders for the \emph{Frontend} part;
\end{itemize}

If the reader wants to see how the application works it is suggested to read the user manual, which in the very first lines explains how to make it start.
In the next section the manual will discuss about the sections we spoke above. For any suggestion please feel free to contact us or open an issue on the GitHub portal. 


\begin{comment}
\begin{labeling}{alligator}
	\item Frontend
	\begin{itemize}
		\item View
		\item ViewModel
		\item Model
	\end{itemize}
\item Backend
\begin{itemize}
	\item \textcolor{red}{PER STEFANO}
	\item \textcolor{red}{PER STEFANO}
	\item \textcolor{red}{PER STEFANO}
\end{itemize}
\end{labeling}




More precisely:
\begin{labeling}{alligator}
\item A student will be able to:
\begin{itemize}
\item Accept or refuse a mark;%Si può sostituire con grade
\item Visualize its booklet containing the marks;
\item Subscribe to an exam.
\end{itemize}
\item A professor will be able to:
\begin{itemize}
\item Visualize the exams to which it has been assigned;
\item See the students registered to his/her exams.
\end{itemize}
\item The university and the administrator will be able to add, eliminate and modify:
\begin{itemize}
\item Academic years;
\item Degree courses;
\item Didactic activities;
\item Exams;
\item Users.
\end{itemize}
\end{labeling}

To use \project{} you need to install \emph{\href{https://metamask.io/}{MetaMask}}\subscript{G} to directly execute the \DH app on your browser.

All the other users have to be logged through Metamask.
\end{comment}