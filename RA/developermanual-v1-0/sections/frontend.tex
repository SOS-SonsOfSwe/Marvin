\section{FrontEnd}
In this section the manual talks about how to program some features that will make the user more comfortable with the package. By explaining how to do modify or add some functionalities, the reader will be guided into the logic of the application.
Since the package has been developed using the \emph{Model View View-Model} design pattern, it is recommended to follow those guide lines, as they will help to keep the package clean and efficient. In every section there will be a quick explanation of what the user is going to do and the steps that are needed for doing the tasks.
~\\~\\The guide will explain how to:
\begin{itemize}
	\item Modify or add a \textit{dumb} page whose functionality is to render some static information on the screen. This concerns the \emph{View} section;
	\item Modify or add a page in which the user wants to render some \verb|store| information by accessing its \verb|state|, maybe paying attention to which kind of visitor the user wants to render the page to. This concerns the \emph{View-Model} section as the information are already stored into the global \verb|state| application;
	\item Modify or add a page in which the user can update the \verb|store| information by making the user interacting both with interface and database. This concerns the \emph{Model} section.
\end{itemize}

\subsection{Modify or add a \emph{dumb} page}
The the \emph{dumb} pages - better known as \verb|components| - extend the \verb|React| component abstract class. The only purpose of this kind of classes is to render static information on the screen or to make the user do some in-\verb|component| tasks such as clicking a button to increase a counter.
\\Some tips: 
\begin{itemize}
	\item Each class is correlated by a \verb|constructor| to which some \verb|props| are passed by the parent \verb|component|. It is possible to define a \verb|state| with which the reader can manage the data; those are not meant to be used outside the component;
	\item The class should be correlated by a \emph{SassCSS} style sheet to make it looks more personal. The file should stay along with the file of the component;
	\item All the components must stay into the \verb|src/components/| folder under the respective field. So if the reader wants to add a component that will be rendered, for example:
	\begin{itemize}
		\item along the \emph{navigation bar}, the component should stay under the \verb|App| folder;
		\item if it is for some specific user, under \verb|src/components/Profile|/\verb|<userType>|;
		\item if it is in common with all the users, under \verb|src/Profile|;
		\item if it is not user specific, under \verb|src/components|.
	\end{itemize}  
	It is important to say that all the related files should be grouped together into the same sub folder. This is a \href{https://medium.freecodecamp.org/scaling-your-redux-app-with-ducks-6115955638be}{Duck} reference;
	\item To render a component the application must be aware of it. So the \verb|src/index.js| must be updated accordingly;
	
\end{itemize}
To add a new component - like a \verb|BetterHome| homepage for the application - it is suggested to follow those steps:
\begin{enumerate}
	\item Create a valid \verb|React| component under \verb|src/components/BetterHome/|;
	\item Add a valid \verb|React| button under \verb|src/components/App| that must be imported into the \verb|src/components/App/NavButton| so the component will be accessible from the navigation bar;
	\item Open \verb|src/index.js|, import the \verb|BetterHome| component;
	\item Under \verb|ReactDOM.render()| find the right place in which the component should stay and give it a \verb|path| name. For example:
	\begin{lstlisting}
	<Route path="/" component={App}>
	    <IndexRoute component={UserDataFetching(BetterHome)} />
	       {/* <IndexRoute component={UserDataFetching(Home)} /> */}
	         <Route path="insert-user" component={AdminIsAuthenticated(InsertUser)} />
	               <Route path="signup" component={UserIsNotAuthenticated(SignUp)} />
	\end{lstlisting}
	The "..."\verb|IsAuthenticated()| are specific gateway used to show the component only for some type of users. The button we have created before should be linked to that path.
\end{enumerate}

Now the application knows the new components and it is clickable in the navigation bar button. When the navigation button is clicked an history-\verb|action| is triggered, the \verb|state| is updated; by doing so the \verb|ReactDOM.render()| method of \verb|index.js| is updated and loads the page. The \verb|React| framework does it.

\subsection{Modify or add a \emph{not-so-dumb} page}
This chapter will introduce the reader into adding a \verb|container|, which is responsible for \emph{passing} a \verb|props| to the previous defined \verb|component| to make it more clever.
\\Some tips:
\begin{itemize}
	\item The Marvin package owns a \emph{Redux} \verb|/src/store.js| in which the user can redefine the behavior of the application. In this file are composed some specific methods for managing the \emph{history} of the application, the kind of \verb|dispatch|es it will have to manage and the \verb|action|s that the \verb|reducer|s will have to trigger;
	\item Every \verb|container| should stay under the \verb|src/containers| sub folder following the rules we described above;
	\item The \verb|container| is a special javascript class which is then \emph{decorated} by a component. In this class we can \emph{connect} a component to the \verb|store| state so it has access to the information that are kept into it.
\end{itemize} 
This example shows the \verb|AcademicYearsContainer.js| container, which imports the \verb|AcademicYears| component:
\begin{lstlisting}
import { connect } from 'react-redux'
import AcademicYears from '.../components/.../AcademicYears'
import { readAcademicYearsFromDatabase } from '.../redux/.../readAcademicYears'
const mapStateToProps = (state, ownProps) => {
	return {
		academicYears: state.admin.academicYears.payload,
		loading: state.admin.academicYears.loading,
		success: state.admin.academicYears.success,
		empty: state.admin.academicYears.empty,
		justDeleted: state.admin.academicYears.justDeleted
	}
}
const mapDispatchToProps = {
	readAcademicData: readAcademicYearsFromDatabase
}

const AcademicYearsContainer = connect(
	mapStateToProps,
	mapDispatchToProps
)(AcademicYears)

// this export is exporting a valid react class because of the importing above
export default AcademicYearsContainer
\end{lstlisting}

To make the \verb|BetterHome| component render some user's information, its own name for example, it is suggested to follow those steps:
\begin{enumerate}
	\item Create a \verb|/src/containers/BetterHome/BetterHomeContainer.js| file accordingly to the rules described above. In this class there will be the import of the \verb|BetterHome| component. Then, use the \verb|connect| statement from \verb|react-redux| to connect the \verb|BetterHome| \verb|state| to the \verb|store| \verb|state.user.data|; this data is now accessible from the components via \verb|this.props.<dataNameInContainer>|;
	\item The \verb|container| class is now a \emph{React} class, so it can be exported as an enriched component;
	\item The \verb|BetterHome| component should be updated to render the name of the user which is logged in;
	\item The import inside \verb|src/index.js| must be modified to make the application know that our component is connected to the store. It is enough to switch import from the \verb|component| to \verb|container| class.
\end{enumerate}

The \emph{dumb} component is now connected to the \verb|store| \verb|state|. Every time the information will change, the \verb|BetterHome| component will update accordingly.
\\Instead of managing all the authorization, the SoS team choosed to use the "..."\verb|IsAuthenticated()| methods with the help of the \verb|react-auth-wrapper| package. This tool is natively linked to the \emph{Redux} store and decouples the authentication from the components. The \verb|src/authentication/wrapper.js| file can read the store information and create some \verb|predicate|s in order to:
\begin{itemize}
	\item Take a \emph{React} component as parameter and decide to render it or not;
	\item Dispatch redirections in case of success/failure of the predicate.
\end{itemize}
This example shows how to limit the access only to the administrators:
\begin{lstlisting}
export const AdminIsAuthenticated = connectedReduxRedirect({
	redirectPath: '/',
	authenticatedSelector: state => state.user.data !== null && state.user.isAdmin,
	redirectAction: routerActions.replace,
	wrapperDisplayName: 'AdminIsAuthenticated'
})
\end{lstlisting}
For further explanations please refer to the \href{https://github.com/mjrussell/redux-auth-wrapper}{GitHub} page.


\subsection{Modify or add a \emph{very clever} page}
\label{IPFS}
To add some specific behavior to the component - like connecting to a server - another step is required.
\\Some tips:
\begin{itemize}
	\item The data are stored into two different kind of database: the \emph{blockchain} and the \emph{IPFS} ones. The description of what they are and which one is to be preferred for storing data is in the \emph{backend} part. The \emph{IPFS} stores some information, it retrieves an \emph{hash code} which is then written on the \emph{blockchain};
	\item This application is a \verb|react-redux| one, so knowing the reducer is important. The reducer is a javascript file which is responsible to update the state information on the base of the actions that are dispatched. A user can dispatch an action: this will update the store state and trigger the application to re-render. Every reducer should stay under \verb|/src/redux/reducers|;
	\\The example above shows the action which is responsible for passing the data academic years data from the server to the container/component:
	\begin{lstlisting}
	const academicYearsReducer = (state = initialState, action) => {
	if(action.request === adminCostants.ACADEMIC_YEARS) {
	switch(action.type) {
		default: {
			return state
		}
		case userCostants.FETCH_DATA_SUCCESS:
		{
		...state,
		payload: action.payload.load,
		success: true,
		empty: false,
		justDeleted: false,
		loading: false
		}
	}
	...
	\end{lstlisting}
	\item Even if there is no need to understand what \emph{Web3} is as the settings should not be modifyed, some skilled developers would find interesting how to change the communication between the javascript part, the Metamask plugin, the \emph{blockchain} network and the \emph{solidity} contracts. \emph{Truffle} uses this package to make a comfortable adapter to the \emph{backend}, retrieving a \emph{.json} instance of the \emph{solidity} contract and making it available for interacting with javascript;
	\item As the application makes many asynchronous calls to the Solidity contracts it is suggested to use \emph{promises} instead of \emph{callbacks} so this will keep the state coherent. If the application has to call a callback then it should be transformed into a promise.
\end{itemize}

It is suggested to follow those steps:
\\First of there must be a clear idea of what has to be added to the application. For adding a user's phone number to the previous component, for example, it is necessary to:
\begin{enumerate}
	\item Read the information that are already present on the \emph{blockchain} and \emph{IPFS}, so the application will be aware of the reading state;
	\item Keep it temporarly to make some modifications;
	\item Add the information into the \emph{IPFS} network and retrieve the hash, making the application know about the adding state;
	\item Update the hash information on the \emph{blockchain}.
\end{enumerate}

Go down the Rabbit's Hole:
\begin{enumerate}
	\item First of all the BetterHome page has to be modified to accept a number insertion and a Save button, which will be connected to the store later;
	\item As to the \verb|src/redux/reducers| the team chose to make:
	\begin{itemize}
		\item \verb|userReducer| for managing the login, logout and signup of all the users;
		\item \verb|<userType>Reducer| for managing all the actions that a particular type of user shall do (i.e adding an Academic Year for administrators)
		\item \verb|ipfsReducer| to maintain the reading and writing on/from \emph{IPFS};
		\item \verb|web3Reducer| to make the application know about the web3 connection state;
		\item Other reducer that are specific for every action that has to be implemented. The reducer pattern is shared between the reducers.
	\end{itemize}
	To use the reducer in a more familiar way there are some standard dispatched that can be used effortless. They can be found at \verb|src/redux/actions/standardDispatches/|;
	\item An action inside a \verb|/src/redux/action/| folder has to be created accordingly to the wanted behavior. It is important to use the right dispatch to make the Marvin store know about how the readings and writing are going on;
	\item To make the application be aware of the action the BetterHome container must be modified accordingly using the \verb|mapDispatchToProps| function offered by the \verb|react-redux| framework;
	\item The Save button must trigger the action. It has to be connected to the container function which will be accessible from \verb|this.props.<functionName>|.
\end{enumerate}

~\\
~\\
This guide is not finished yet, indeed this is a beta: as the developer will see there are some hide-and-seek tricks that makes this package faster and more efficient.
