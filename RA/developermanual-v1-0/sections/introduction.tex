\section{Introduction}
This is the developer manual of \textbf{Marvin}, a \DH app ran on the EVM, that shapes a subset of \href{www.uniweb.unipd.it}{Uniweb} functionalities and based on the \emph{Truffle} framework. Uniweb is the University of Padua's informative system.
It allows students to keep track of their academic carrier. Professors use Uniweb to see the lists of students that are registered to their exams and, see the exams to which they have been assigned.
\\This manual is intended for programmers wishing to customize or extend \textbf{Marvin}. Those are expected to know and understand the \emph{React-redux} as well as the \emph{truffle} frameworks. In addition, the knowledge of the \emph{solidity} language is mandatory to understand the database on which the application is based. A basic understanding of the Sass preprocessor CSS, while not required, is a plus.

\subsection{What is in the Manual}
This manual covers  most of the aspects of custom development and maintenance of Marvin.
In particular it is divided into two main sections: \emph{Frontend}  and \emph{Backend}. So if the reader wants to update or modify the \verb|solidity| database part, he should skip all the \emph{Frontend} part, instead if his purpose is to modify how the application is rendered on screen, then he should go to the first one.

\begin{comment}
\begin{labeling}{alligator}
	\item Frontend
	\begin{itemize}
		\item View
		\item ViewModel
		\item Model
	\end{itemize}
\item Backend
\begin{itemize}
	\item \textcolor{red}{PER STEFANO}
	\item \textcolor{red}{PER STEFANO}
	\item \textcolor{red}{PER STEFANO}
\end{itemize}
\end{labeling}




More precisely:
\begin{labeling}{alligator}
\item A student will be able to:
\begin{itemize}
\item Accept or refuse a mark;%Si può sostituire con grade
\item Visualize its booklet containing the marks;
\item Subscribe to an exam.
\end{itemize}
\item A professor will be able to:
\begin{itemize}
\item Visualize the exams to which it has been assigned;
\item See the students registered to his/her exams.
\end{itemize}
\item The university and the administrator will be able to add, eliminate and modify:
\begin{itemize}
\item Academic years;
\item Degree courses;
\item Didactic activities;
\item Exams;
\item Users.
\end{itemize}
\end{labeling}

To use \project{} you need to install \emph{\href{https://metamask.io/}{MetaMask}}\subscript{G} to directly execute the \DH app on your browser.

All the other users have to be logged through Metamask.
\end{comment}