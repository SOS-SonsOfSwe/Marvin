\newpage
\section{Processi primari}
\subsection{Processo di fornitura}

\subsubsection{Studio di Fattibilità}
Si tratta del documento in cui vengono analizzati tutti i capitolati, valutandone pregi e difetti, allo scopo di scegliere quello di maggiore affinità per il gruppo.
Dopo che il \RdP{} (il cui ruolo è descritto nel paragrafo \ref{responsabile_progetto}) avrà riunito il team e discusso con esso di tutti i capitolati, gli \anas{} avranno il compito di stilare lo \SdF{} seguendo le considerazioni emerse.\\
Lo \SdF{} sarà organizzato come segue:
\begin{itemize}
	\item \textbf{Informazioni sul capitolato}:
	Vengono ricordati il nome del \emph{progetto}\ped{G}, il \textit{proponente}\ped{G} e i \textit{committenti}\ped{G};
	\item \textbf{Descrizione}:
	Si riassume lo scopo del capitolato;
	\item \textbf{Dominio applicativo}:
	Si specifica il settore di utilizzo del prodotto finale;
	\item \textbf{Dominio tecnologico}:
	Si elencano le tecnologie che dovranno essere utilizzate nello sviluppo del capitolato;
	\item \textbf{Aspetti positivi}:
	Si elencano i motivi che il gruppo potrebbe considerare vantaggiosi;
	\item \textbf{Aspetti negativi}:
	Si elencano le potenziali criticità che il gruppo dovrà tenere in considerazione nella scelta finale;
	\item \textbf{Valutazione finale}:
	Si analizzano gli aspetti positivi e quelli negativi riscontrati e si motiva l'eventuale approvazione o esclusione.
\end{itemize}

\subsubsection{Rapporti con il proponente}
Una volta scelto il capitolato, si intende instaurare un rapporto quanto più costante e profittevole con la \proponente{} allo scopo di:
\begin{itemize}
	\item Stabilire un accordo in merito allo sviluppo, al mantenimento, al funzionamento e alla consegna del prodotto;
	\item Realizzare un prodotto che soddisfi totalmente i requisiti obbligatori concordati e quanto più possibile quelli desiderabili;
	\item Stimare i costi;
	\item Concordare la qualifica del prodotto.
\end{itemize}

\subsubsection{Collaudo e consegna del prodotto}
Una volta terminate le fasi di sviluppo, verifica e validazione si effettuerà il collaudo al fine di dimostrare che tutti i requisiti obbligatori e, possibilmente, anche alcuni dei requisiti opzionali siano stati soddisfatti. 
\\In questa fase inoltre si dovrà dimostrare che l'esecuzione di tutti i test di validazione abbia dato un'esito positivo.
\\Il team consegnerà in ultima il prodotto finale su un supporto fisico ai committenti \committenti.

\subsection{Processo di sviluppo}
\subsubsection{Scopo del processo}
Contiene tutte le attività svolte dal team, al fine di produrre il software finale.
\subsubsection{Aspettative del processo}
Le aspettative previste dal team per una efficace implementazione del processo sono le seguenti:
\begin{itemize}
	\item realizzare un prodotto finale che soddisfi i testi di validazione e verifica, secondo quanto concordato con il proponente;
	\item fissare gli obiettivi di sviluppo;
	\item normare i vincoli tecnologici per lo sviluppo.
\end{itemize}
\subsubsection{Descrizione}
Secondo lo standard ISO/IEC 12207, il processo di sviluppo si divide nelle seguenti attività:
\begin{itemize}
	\item Analisi dei requisiti;
	\item Progettazione;
	\item Codifica.
\end{itemize}
\subsubsection{Attività}
\paragraph{Analisi dei Requisiti}
\subparagraph{Scopo}:
	Lo scopo dell'\AdR consiste nell'individuare i requisiti del progetto tramite le specifiche del capitolato, tramite le comunicazioni con il proponente e tramite le valutazioni effettuate durante le riunioni interne del gruppo. 
\subparagraph{Descrizione}:
	Il risultato di tale attività dovrà essere stilato dagli \anas, contenente la lista dei requisiti e dei casi d'uso. Essi dovranno inoltre fornire un diagramma conforme
	allo standard UML per ogni caso d’uso. Tale documento ha il fine di comprendere maggiormente le scelte di progettazione effettuate.
\subparagraph{Classificazione dei Requisiti}:
	i requisiti saranno classificati secondo la seguente codifica:
		\begin{center}
			R[Importanza][Tipo][Codice]
		\end{center}
	dove:
	\begin{itemize}
		\item \textbf{Importanza} può assumere questi valori:
			\begin{itemize}
				\item F: indica un requisito funzionale;
				\item Q: indica un requisito di qualità;
				\item P: indica un requisito prestazionale;
				\item V: indica un requisito di vincolo.
			\end{itemize}
		\item \textbf{Tipo} può assumere questi valori:
			\begin{itemize}
				\item O: indica un requisito obbligatorio;
				\item D: indica un requisito desiderabile;
				\item F: indica un requisito facoltativo.
			\end{itemize}
		\item \textbf{Codice} indica il codice identificativo del requisito, è univoco e deve essere indicato in
		forma gerarchica.
	\end{itemize}
	Per ogni requisito si dovrà inoltre indicare una breve descrizione e la fonte, che può essere una tra le seguenti:
		\begin{itemize}
		\item Capitolato: deriva direttamente dal testo del capitolato;
		\item Verbale: deriva da un incontro verbalizzato;
		\item Interno: deriva da discussioni interne al team.
		\end{itemize}
	
\subparagraph{Classificazione dei casi d'uso}: i casi d’uso saranno classificati secondo la seguente codifica:
		\begin{center}
			UC[Codice padre].[Codice identificativo]
		\end{center}
	dove:
		\begin{itemize}
			\item Codice padre: indica il codice del caso d’uso padre di quello in esame, se non è identificabile è da omettere;
			\item Codice identificativo: codice univoco e progressivo del caso d’uso in esame.
		\end{itemize}
	Per ogni caso d’uso saranno inoltre identificate le seguenti informazioni:
		\begin{itemize}
			\item \textbf{Attori}: indica gli attori coinvolti nel caso d’uso;
			\item \textbf{Descrizione}: chiara, precisa e concisa descrizione del caso d’uso;
			\item \textbf{Precondizione}: indica la situazione che deve essere vera prima dell’esecuzione del caso d’uso;
			\item \textbf{Flusso principale degli eventi}: descrizione composta dal flusso dei casi d’uso figli;
			\item \textbf{Postcondizione}: indica la situazione che deve essere vera dopo l’esecuzione del caso d’uso;
			\item \textbf{Estensioni}: indica quali sono tutte le estensioni, se presenti;
			\item \textbf{Generalizzazioni}: indica quali sono tutte le generalizzazioni, se presenti.
		\end{itemize}
	
\paragraph{Progettazione}
\subparagraph{Scopo}
	~\\L'attività di progettazione deve precedere la produzione del software ed ha il compito di soddisfare le peculiarità ed i bisogni individuati durante l'attività di \emph{analisi dei requisiti}, e di realizzare al meglio i requisiti di qualità imposti dal committente.
	L’attività di progettazione si impone inoltre di generare una soluzione soddisfacente per tutti gli individui che fanno parte del progetto. Una volta compresi pienamente i requisiti del problema e approfondita la progettazione in parti abbastanza semplici, si forniranno le istruzioni necessarie ai \progs{} per sviluppare il prodotto finito.
	~\\I \progs{} (il cui ruolo è descritto nel paragrafo \ref{progettista}) dovranno delinerare i requisiti utili alla documentazione specifica e determinare le linee guida da seguire in funzione dei requisiti individuati dall'analisi.
	
	\subparagraph{Progettazione architetturale}
	La progettazione permette di:
	\begin{itemize}
		\item Costruire l'architettura logica del prodotto;
		\item Ottimizzare l'utilizzo delle risorse;
		\item Garantire la qualità prefissata del prodotto;
		\item Organizzare le varie parti del progetto in modo da ottenere componenti singole e facili da implementare durante la successiva fase di codifica.
	\end{itemize}
	
	La progettazione si colloca nel periodo di \emph{Progettazione e codifica} all'interno del \PdP{}; periodo all'interno del quale vengono individuati i design pattern e realizzata l'architettura del prodotto tramite l'utilizzo dei diagrammi delle classi e di sequenza. Il lavoro complessivo ottenuto in questo arco di tempo rappresenta la \textbf{Product Baseline}. Una volta sviluppata l'architettura generale del software dovranno inoltre essere documentati i seguenti task:
	\begin{itemize}
		\item Suddividere il prodotto in componenti;
		\item Definire il ruolo dei membri;
		\item Definire le interazioni tra i componenti;
		\item Assicurarsi che ogni requisito sia soddisfatto da almeno un componente;
		\item Realizzare e documentare i test d'integrazione, al fine di verificare il corretto funzionamento di più componenti tra loro;
		\item 
	\end{itemize}


	\subparagraph{Linee guida per la progettazione}
	Dopo aver completato l'\AdR{} i \progs{} dovranno sottostare alle seguenti linee guida per lo sviluppo dell'architettura logica del sistema:
	\begin{itemize}
		\item Si dovrà puntare ad una progettazione chiara e di immediata comprensione;
		\item Le componenti progettate dovranno essere quanto più riutilizzabili e manutenibili;
		\item La complessità non dovrà mai essere intrattabile;
		\item I \progs{} dovranno rientrare nei costi e nelle risorse disponibili;
		\item I \progs{} dovranno descrivere i \emph{design pattern}\ped{G} che intendono utilizzare per la realizzazione dell'architettura, fornendone una breve descrizione e un diagramma;
		\item Si dovrà evitare l'utilizzo, ove possibile di classi astratte;
		\item Si dovrà evitare package vuoti, classi non utilizzate, e parametri senza tipo.
	\end{itemize}

	\subparagraph{UML}
	~\\Le tipologie di diagrammi \emph{UML}\ped{G} che verranno adoperate per analizzare, descrivere e specificare le scelte progettuali adottate saranno:
	\begin{itemize}
		\item \textbf{Diagrammi di classe}: illustrano una collezione di elementi che rappresenta un modello di classi e tipi, con i rispettivi contenuti e relazioni;
		\item \textbf{Diagrammi di } \emph{\textbf{package}}\ped{G}: documentano le dipendenze tra le classi ed è utile per controllare la complessità strutturale in sistemi medio-grandi;
		\item \textbf{Diagrammi di attività}: modellano un processo e organizzano più entità in un sistema di azioni secondo un determinato flusso. I diagrammi delle attività sono un tipo particolare di \emph{diagramma di stato}\ped{G} che identifica la variazione di stato al verificarsi di alcune condizioni legate ad una o più entità;
		\item \textbf{Diagrammi di sequenza}: descrivono la collaborazione di un gruppo di oggetti che devono implementare collettivamente un comportamento. Sono diagrammi molto semplici, ma che permettono di capire se l'architettura creata viene eseguita.
	\end{itemize}

	\subparagraph{Test di integrazione}
	Per quanto riguarda l’attività di test, i \progs{} devono preoccuparsi di definire delle classi di verifica per accertare il funzionamento dei componenti. In alcuni casi non è quindi possibile fornire ai \vers{} degli strumenti automatici per l'esecuzione dei test. La progettazione delle classi per i test necessita di essere svolta senza sovrapporre lo stesso ruolo; chi crea una classe non può essere incaricato anche di testarla.
	
	
	\subparagraph{Requisiti per i progettisti}
	I \progs{} sono responsabili delle attività di progettazione. Per un corretto svolgersi del loro lavoro, sono tenuti ad acquisire le seguenti conoscenze:
	\begin{itemize}
		\item Buona conoscenza generale del processo di sviluppo software;
		\item Estesa visione generale delle scelte progettuali da attuare in situazioni particolarmente problematiche;
		\item Discreta inventiva per trovare possibili soluzioni progettuali anche nel caso in cui non vi siano casi metodologici esplicitamente applicabili;
		\item Capacità di individuare le scelte più opportune e coerenti alle esigenze del team.
	\end{itemize}

	\subparagraph{Obiettivi di progettazione}
	La progettazione ha lo scopo di conseguire i seguenti obiettivi:
	\begin{itemize}
		\item  Progettare un software con caratteristiche qualitative pari a quelle descritte nella fase di analisi dei requisiti;
		\item  Progettare un software affinchè sia possibile effettuare modifiche nel caso in cui sia necessario, senza avere la necessità di mettere in discussione l'intera struttura del software;
		\item  Progettare un software che sia modulare, ossia che ogni parte sia chiara e distinta ed abbia una determinata funzione;
		\item Soddisfare i requisiti fissati dal committente;
		\item Costruire un software robusto, ed in grado di gestire malfunzionamenti di fronte a situazioni critiche improvvise;
		\item Ridurre ove sia possibile gli sprechi di tempo e spazio, ragionando in termini di efficienza;
		\item Progettare un software sicuro per quanto riguarda possibili intrusioni e pericoli esterni;
		\item Rispettare il principio dell'incapsulazione, tramite un corretto utilizzo del \emph{data hiding}\ped{G}.
	\end{itemize}
	
	
	\paragraph{Codifica}
	In questa fase i \progrs{} (il cui ruolo è descritto nel paragrafo \ref{programmatore}), seguendo le norme delineate nella progettazione, devono realizzare il passaggio dalla fase di pianificazione all'effettiva realizzazione del prodotto.
	Le norme qui presenti serviranno come strumento per realizzare un codice uniforme e di alta qualità. Inoltre, per mantenerne la manutenibilità, dovrà essere realizzato in inglese.
	I \progrs{} si dovranno attenere agli standard di codifica qui di seguito elencati.
	\subparagraph{Indentazione}
	I blocchi di codice prodotti dovranno essere indentati secondo le seguenti norme:
	\begin{itemize}
		\item \emph{Annidamento}: I blocchi annidati dovranno essere correttamente indentati, usando per ciascun livello di profondità un \emph{Tab} (equivalente a quattro spazi) per allineare correttamente le sezioni di codice;
		\item \emph{Capo linea}: Per andare a capo linea di una riga di codice è opportuno utilizzare due volte un \emph{Tab} (equivalente a otto spazi), per allineare correttamente le sezioni di codice;
		\item \emph{Profondità}: Nel caso in cui ci siano blocchi di codice annidati tra loro, si dovrà sviluppare il codice sul livello relativo al proprio blocco, tenendo conto delle regole precedentemente elencate.
		\item \emph{Parentesizzazione}: I blocchi di codice andranno sempre chiusi all'interno di parentesi graffe, anche nel caso in cui il linguaggio non lo richieda esplicitamente, come per esempio nel caso dei controlli condizionali \emph{if-else};
		\item \emph{Parentesizzazione 2}: Le parentesi graffe andranno collocate sulla stessa riga in cui inizia il blocco di codice.
	\end{itemize}
	\subparagraph{Nomi}
	\begin{itemize}
		\item Ogni elemento (classe, metodo, e variabile) deve avere un nome rappresentativo e pertinente alla funzione da esso svolta;
		\item Il nome di una classe deve sempre iniziare con la lettera maisucola;
		\item Si dovrà utilizzare la notazione \emph{CamelCase}, ovvero la concatenazione di più parole, ognuna delle quali con lettera iniziale maiuscola. In caso di metodi e variabili la prima lettera dovrà essere minuscola, mentre per le classi maiuscola;
		\item Si potranno utilizzare singole lettere esclusivamente per identificare gli indici dei cicli;
		\item Saranno da evitare notazioni troppo simili tra di loro in significato e denominazione;
		\item Si dovranno evitare errori di ortografia.
	\end{itemize}
	\subparagraph{Commenti}
	~\\Sarà necessario che il codice contenga dei commenti esplicativi per facilitarne la comprensione.
	\newline In particolare, si dovranno seguire le seguenti linee guida:
	
	\begin{itemize}
		\item Quando si modifica codice già esistente, vanno aggiornati i relativi commenti;
		\item Evitare di aggiungere commenti sulla stessa riga di codice; la continua presenza di codice misto a testo può creare confusione e difficoltà di lettura;
		\item Evitare di scrivere commenti privi di senso logico, al fine di separare righe di codice; nel caso in cui insorgesse questa necessità si dovrà utilizzare spazi vuoti;
		\item Il fine dei commenti deve essere di chiarire il codice scritto; è quindi opportuno utilizzare frasi semplici, di senso compiuto e con una corretta punteggiatura;
		\item E' opportuno commentare tutte le parti di codice non chiare o particolarmente complicate;
		\item E' opportuno utilizzare i commenti nel caso in cui una parte di codice venga corretta in seguito a degli errori, al fine di tenere presenti errori frequenti e le possibili soluzioni da applicare.
	\end{itemize}
	
	
	\subparagraph{Formattazione}
	La formattazione serve a facilitare la comprensione dell'organizzazione logica del codice. Per consentire agli sviluppatori di decifrare facilmente il codice si dovranno seguire le seguenti regole:
	\begin{itemize} 
		\item Il rientro predefinito dovrà essere di un \emph{Tab} per allineare le sezioni di codice;
		\item Prima e dopo ogni operatore dovrà esserci uno spazio;
		\item I blocchi saranno tra loro spaziati per una maggiore comprensione;
		\item Utilizzare spazi vuoti per definire la struttura del codice; in questo modo sarà possibile ottenere paragrafi separati tra loro più facilmente comprensibili;
%		\item Il codice sorgente dovrà essere quanto più suddiviso in file e dovrà essere raggruppato in sottocartelle che dovranno rispecchiare il pattern utilizzato.
%		\item \textcolor{red}{Se ce ne sono altre le inseriamo qui. Da pensare al pattern, }
	\end{itemize}

	\subparagraph{Visibilità}
	Le variabili vanno dichiarate nel blocco più interno che comprenda l'utilizzo della variabile stessa, al fine di ottenere codice di qualità migliore. E' quindi fortemente sconsigliato l'utilizzo di variabili globali.
	
	\subparagraph{Inizializzazione}
	Le variabili devono sempre essere inizializzate quanto prima possibile. E' consigliato inizializzare una variabile ogni qual volta venga dichiarata. Nel caso in cui non si possa effettuare questa operazione, è consigliabile posticipare la dichiarazione al momento in cui si potrà effettuare.
	
	\subparagraph{Codice esterno}
	Nel caso in cui si decida di utilizzare codice esterno sviluppato da terze parti, è consigliabile utilizzare lo stile e le norme di codifica degli autori di quel codice, al fine di non creare ambiguità stilistiche e facilitarne la comprensione.
	
	
\subsubsection{Strumenti}
Gli strumenti utilizzati durante la fase dei processi primari sono:
\begin{itemize}
	\item \textbf{TexStudio}
	\newline Il gruppo ha scelto \emph{TexStudio} come editor multipiattaforma per comporre i documenti in \emph{\LaTeX}\ped{G};
	\item \emph{\textbf{\emph{SWEgo}}}\ped{G} \footnote{\href{https://www.swego.it/}{https://www.swego.it/}}
	~\\ Il gruppo ha scelto \emph{SWEgo} come database per la gestione dei casi d'uso, dei requisiti e del loro tracciamento per il documento \AdR{}. Dopo un'attenta analisi preliminare dello strumento il team ha deciso di utilizzarlo nonostante alcuni problemi rilevati nella generazione del codice \LaTeX{} perché, permette una stesura standardizzata e sempre aggiornata di alcuni capitoli importanti del documento, a fronte di correzioni minori e applicabili in maniera seriale e regolamentata;
	\item \textbf{Astah}\ped{G}
	~\\Il gruppo ha scelto l'utilizzo di \emph{Astah} per la generazione dei diagrammi dei casi d'uso dell'\AdR{};
	\item \textbf{Microsoft Excel 365}
	\newline Il gruppo ha scelto di utilizzare \emph{Microsoft Excel 365}\ped{G} per la generazione di grafici e di tabelle da poter inserire all'interno del \PdQ{} e \PdP{} perché di immediata realizzazione e visualizzazione;
	\item \textbf{Instagantt} \footnote{\href{https://instagantt.com/}{https://instagantt.com/}}
	~\\Instagantt è un servizio web fortemente legato ad Asana (vedi descrizione negli strumenti dei \emph{processi organizzativi}), strumento con il quale si integra perfettamente, nonostante sia disponibile anche una versione \emph{standalone}\ped{G}. Esso permette di creare diagrammi di Gantt e gestire in maniera semplificata la timeline e la struttura di un progetto.
	\item \textbf{Visual Studio Code} \footnote{\href{https://code.visualstudio.com/}{https://code.visualstudio.com/}} 
	~\\Visual Studio Code è un editor di testo gratuito e multipiattaforma che supporta operazioni di debugging e di versionamento. Permette inoltre di installare vari pacchetti che consentono di personalizzarne l'usabilità e di supportare nuove tecnologie o linguaggi di programmazione. Il gruppo lo ha scelto come editor di testo per scrivere il codice in React e Redux.
	\item \textbf{Truffle} \footnote{\href{http://truffleframework.com/}{http://truffleframework.com/}}
	~\\Truffle è un ambiente di sviluppo e testing per Ethereum. Il gruppo ha deciso di utilizzarlo perché fornisce un aiuto nello sviluppare, pubblicare e testare gli smart contract.
	\item \textbf{Ganache} \footnote{\href{http://truffleframework.com/ganache/}{http://truffleframework.com/ganache/}}
	~\\Ganache è uno strumento legato a Truffle che consente di avere una blockchain virtuale locale utilizzabile per sviluppare contratti, applicazioni ed effettuare dei test. Permette anche di generare degli account di testing  utilizzabili  durante lo sviluppo.
	\item \textbf{Npm} \footnote{\href{https://www.npmjs.com/}{https://www.npmjs.com/}}
	~\\npm è un gestore di pacchetti per il linguaggio Javascript.
	
	\item \textbf{Remix}
	~\\Remix è l'ambiente di sviluppo integrato ufficiale di Ethereum per il linguaggio di programmazione di smart contract \textit{solidity}\ped{G}. Viene eseguito all'interno di un browser web, online all'indirizzo  \url{https://remix.ethereum.org/} oppure in locale scaricando i relativi file sulla propria macchina.% Il gruppo ha deciso di utilizzarlo in locale perché c'è il vantaggio di poter comunicare con un nodo client di ethereum eseguito in locale.
\end{itemize}	

