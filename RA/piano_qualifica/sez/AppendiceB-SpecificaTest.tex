\pagebreak
\section{Specifica dei test}
Questa sezione definirà i test che verranno implementati ed eseguiti dal team al fine di garantire, in seguito al loro superamento, la creazione di software di alta qualità che soddisfi le richieste ed aspettative del proponente. 
Ogni test viene riconosciuto tramite un identificativo univoco, la cui sintassi è descritta nelle \NdP (§3.2.2) e può assumere uno tra questi stati: \Ts, \Ti, \Ts. 
Dato che il progetto è ancora alla fase iniziale, si presume che questa sezione sarà soggetta a modifiche basate sulle esigenze ed i problemi che si incontreranno durante lo sviluppo.

\subsection{Test di validazione}
Questi test verranno utilizzati durante la fase di collaudo finale, in presenza di committente e proponente, per valutare se il prodotto è conforme a quanto specificato nel contratto.

\newcolumntype{H}{>{\centering\arraybackslash}m{7cm}}
\normalsize
\begin{longtable}{|c|H|c|}
	\hline
	\textbf{Codice test} & \textbf{Descrizione} & \textbf{Stato}\\
	\hline
	\endhead
	
	\hline
	TV0F1A & Viene verificato che un utente non autenticato visualizzi la pagina di registrazione. & \Ts \\
	\hline
	TV0F1B & Viene verificato l'inserimento nel campo realtivo al nome. & \Ts \\
	\hline
	TV0F1C & Viene verificato l'inserimento nel campo realtivo al cognome. & \Ts \\
	\hline
	TV0F1D & Viene verificato l'inserimento nel campo realtivo all'e-mail. & \Ts \\
	\hline
	TV0F1E & Viene verificato l'inserimento nel campo realtivo al codice fiscale. & \Ts \\
	\hline
	TV0F1F & Viene verificato l'inserimento nel campo realtivo al codice univoco. & \Ts \\
	\hline
	TV0F1G & Viene verificato l'inserimento nel campo realtivo al matricola. & \Ts \\
	\hline
	TV0F1H & Viene verificato che \emph{metamask}\ped{G} sia attivato correttamente nel caso in cui un utente non autenticato voglia registrarsi. & \Ts \\
	\hline
	TV0F1I & Viene verificato che venga confermata la registrazione. & \Ts \\
	\hline
	TV0F1L & Viene verificato che la creazione sia andata a buon fine effettuando un login. & \Ts \\
	\hline
	TV2F1.8 & Viene verificata la visualizzazione di un messaggio d'errore qualora l'utente non autenticato  riempia i campi dato relativi alla registrazione con informazioni già presenti, non valide o nulle. & \Ts \\
	\hline
	TV0F2A & Viene verificato che \emph{metamask}\ped{G} sia attivato correttamente nel caso in cui un utente non autenticato voglia autenticarsi dopo essersi registrato. & \Ts \\
	\hline
	TV0F2B & Viene verificato che, nel caso in cui l'utente non autenticato clicchi il pulsante di login, il sistema segnali la corretta autenticazione. & \Ts \\
	\hline
	TV0F3A & Viene verificato che l'utente amministratore o l'utente università siano correttamente autenticati rispettivamente come amministratore o come università. & \Ts \\ 
	\hline
	TV0F3B & Viene verificata la visualizzazione della pagina per l'inserimento di un anno accademico. & \Ts \\ 
	\hline
	TV0F3C & Viene verificato l'inserimento dell'inizio dell'anno accademico. & \Ts \\ 
	\hline
	TV0F3D & Viene verificato l'inserimento della fine dell'anno accademico. & \Ts \\ 
	\hline
	TV0F3E & Viene verificato l'inserimento della nome dell'anno accademico. & \Ts \\ 
	\hline
	TV0F3F & Viene verificata la conferma dell'aggiunta dell'anno accademico. & \Ts \\ 
	\hline
	TV0F3G & Viene verificato che l'aggiunta dell'anno accademico sia andata a buon fine. & \Ts \\ 
	\hline
	TV2F3.5 & Viene verificata la visualizzazione di un messaggio d'errore qualora l'utente amministratore o università riempiano i campi dato relativi all'inserimento di un anno accademico con informazioni errate o nulle. & \Ts \\
	\hline
	TV0F4A & Viene verificata la visualizzazione della pagina per l'inserimento di un corso di laurea. & \Ts \\
	\hline
	TV0F4B & Viene verificato l'inserimento del codice del corso di laurea. & \Ts \\
	\hline
	TV0F4C & Viene verificato l'inserimento del nome del corso di laurea. & \Ts \\
	\hline
	TV0F4D & Viene verificato l'inserimento della descrizione del corso di laurea. & \Ts \\
	\hline
	TV0F4E & Viene verificato l'inserimento della tipologia del corso di laurea. & \Ts \\
	\hline
	TV0F4G & Viene verificata la conferma dell'aggiunta di un corso di laurea. & \Ts \\
	\hline
	TV0F4H & Viene verificato che l'aggiunta del corso di laurea sia andata a buon fine. & \Ts \\
	\hline
	TV2F4.7 & Viene verificata la visualizzazione di un messaggio d'errore qualora l'utente amministratore o università riempiano i campi dato relativi all'inserimento di un corso di laurea con informazioni errate o nulle. & \Ts \\
	\hline
	TV0F5A & Viene verificata la visualizzazione della pagina per l'inserimento di un'attività didattica. & \Ts \\
	\hline
	TV0F5B & Viene verificato l'inserimento del codice dell'attività didattica. & \Ts \\
	\hline
	TV0F5C & Viene verificato l'inserimento del nome dell'attività didattica. & \Ts \\
	\hline
	TV0F5D & Viene verificato l'inserimento della descrizione dell'attività didattica. & \Ts \\
	\hline
	TV0F5E & Viene verificato l'inserimento del professore associato all'attività didattica. & \Ts \\
	\hline
	TV0F5F & Viene verificato l'inserimento dei crediti dell'attività didattica. & \Ts \\
	\hline
	TV0F5G & Viene verificato l'inserimento del periodo dell'attività didattica. & \Ts \\
	\hline
	TV0F5H & Viene verificata la conferma dell'aggiunta dell'attività didattica. & \Ts \\
	\hline
	TV0F5I & Viene verificato che l'aggiunta dell'attività didattica sia andata a buon fine. & \Ts \\
	\hline
	TV2F5.9 & Viene verificata la visualizzazione di un messaggio d'errore qualora l'utente amministratore o università riempiano i campi dato relativi all'inserimento di un'attività didattica con informazioni errate o nulle. & \Ts \\
	\hline
	TV0F6A & Viene verificata la visualizzazione della pagina per l'inserimento di un esame. & \Ts \\
	\hline
	TV0F6B & Viene verificato l'inserimento del codice dell'esame. & \Ts \\
	\hline
	TV0F6C & Viene verificato l'inserimento della descrizione dell'esame. & \Ts \\
	\hline
	TV0F6D & Viene verificato l'inserimento dell'intervallo di prenotazione dell'esame. & \Ts \\
	\hline
	TV0F6E & Viene verificato l'inserimento della data dell'esame. & \Ts \\
	\hline
	TV0F6F & Viene verificato l'inserimento della tipologia dell'esame. & \Ts \\
	\hline
	TV0F6G & Viene verificato l'inserimento del luogo dell'esame. & \Ts \\
	\hline
	TV0F6H & Viene verificato l'inserimento del professore associato all'esame. & \Ts \\
	\hline
	TV0F6I & Viene verificata la conferma dell'aggiunta dell'esame. & \Ts \\
	\hline
	TV0F6L & Viene verificato che l'aggiunta dell'esame sia andata a buon fine. & \Ts \\
	\hline
	TV2F6.9 & Viene verificata la visualizzazione di un messaggio d'errore qualora l'utente amministratore o università riempiano i campi dato relativi all'inserimento di un'esame con informazioni errate o nulle. & \Ts \\
	\hline
	TV2F11A & Viene verificata la visualizzazione della pagina relativa alla cancellazione di un anno accademico. & \Ts \\
	\hline
	TV2F11B & Viene verificata l'effettiva eleminazione dell'anno accademico qualora l'utente amministratore o l'utente università abbia cliccato sul pulsante di eliminazione. & \Ts \\
	\hline
	TV2F12A & Viene verificata la visualizzazione della pagina relativa alla cancellazione di un corso di laurea. & \Ts \\
	\hline
	TV2F12B & Viene verificata l'effettiva eleminazione del corso di laurea qualora l'utente amministratore o l'utente università abbia cliccato sul pulsante di eliminazione. & \Ts \\
	\hline
	TV2F13A & Viene verificata la visualizzazione della pagina relativa alla cancellazione di un'attività didattica. & \Ts \\
	\hline
	TV2F13B & Viene verificata l'effettiva eleminazione dell'attività didattica qualora l'utente amministratore o l'utente università abbia cliccato sul pulsante di eliminazione. & \Ts \\
	\hline
	TV2F14A & Viene verificata la visualizzazione della pagina relativa alla cancellazione di un'esame. & \Ts \\
	\hline
	TV2F14B & Viene verificata l'effettiva eleminazione dell'esame qualora l'utente amministratore o l'utente università abbia cliccato sul pulsante di eliminazione. & \Ts \\
	\hline
	TV2F16A & Viene verificato che l'utente studente sia correttamente autenticato come studente.  & \Ts \\
	\hline
	TV2F16B & Viene verificata la visualizzazione della pagina della lista di esami associati allo studente.  & \Ts \\
	\hline
	TV2F16C & Viene verificata la conferma dell'iscrizione all'esame da parte dello studente. & \Ts \\
	\hline
	TV2F16D & Viene verificato che l'iscrizione all'esame da parte dello studente sia andata a buon fine. & \Ts \\
	\hline
	TV0F17A & Viene verificato che l'utente professore sia correttamente autenticato come professore.  & \Ts \\
	\hline
	TV0F17B & Viene verificata la visualizzazione della pagina della lista di esami associati al professore.  & \Ts \\
	\hline
	TV0F18 & Viene verificata la visualizzazione della pagina della lista di studenti iscritti ad un esame associato al professore.  & \Ts \\
	\hline
	TV0F23A & Viene verificato l'inserimento di un voto ad uno studente da parte di un professore. & \Ts \\
	\hline
	TV0F23B & Viene verificato che l'inserimento di un voto ad uno studente da parte di un professore sia andato a buon fine. & \Ts \\
	\hline
	TV2F23.3 & Viene verificata la visualizzazione di un messaggio d'errore qualora un professore inserisca un voto in un formato errato. & \Ts \\
	\hline
	TV0F24A & Viene verificata la visualizzazione della pagina relativa all'aggiunta di un nuovo utente da parte di un utente amministratore o di un utente università.  & \Ts \\
	\hline
	TV0F24B & Viene verificato l'inserimento del codice fiscale dell'utente che l'utente amministratore o l'utente università vuole inserire.  & \Ts \\
	\hline
	TV0F24C & Viene verificato l'inserimento del codice univoco dell'utente che l'utente amministratore o l'utente università vuole inserire.  & \Ts \\
	\hline
	TV0F24D & Viene verificata la selezione della tipologia dell'utente che l'utente amministratore o l'utente università vuole inserire.  & \Ts \\
	\hline
	TV0F24E & Viene verificata la selezione dell'anno accademico relativo all'utente che l'utente amministratore o l'utente università vuole inserire.  & \Ts \\
	\hline
	TV0F24F & Viene verificata la selezione del corso di laurea relativa all'utente che l'utente amministratore o l'utente università vuole inserire.  & \Ts \\
	\hline
	TV0F24G & Viene verificata la conferma dell'aggiunta di un nuovo utente da parte di un utente amministratore o di un utente università. & \Ts \\
	\hline
	TV0F24H & Viene verificato che l'aggiunta di un utente da parte di un utente amministratore o di un utente università sia andata a buon fine. & \Ts \\
	\hline
	TV2F24.6 & Viene verificata la visualizzazione di un messaggio d'errore qualora l'utente amministratore o università riempiano i campi dato relativi all'aggiunta di un nuovo utente con informazioni errate o nulle. & \Ts \\
	\hline
	TV0F25A & Viene verificata l'eliminazione di un utente studente da parte dell'utente amministratore o dell'utente università.	& \Ts \\
	\hline
	TV0F25B & Viene verificata l'eliminazione di un utente professore da parte dell'utente amministratore o dell'utente università.	& \Ts \\
	\hline
	TV0F25C & Viene verificata l'eliminazione di un utente amministratore da parte dell'utente università.	& \Ts \\
	\hline
	TV0F26A & Viene verificata la visualizzazione della pagina con la lista di esami a cui lo studente è registrato. & \Ts \\
	\hline
	TV0F26B & Viene verificata la visualizzazione della pagina del libretto universitario da parte dello studente. & \Ts \\
	\hline
	\caption[Test di validazione]{Test di validazione}
	\label{tabella:TV}
\end{longtable}

	\subsubsection{Tracciamento Test di validazione - Requisiti}
\normalsize
\begin{longtable}{|c|c|}
	\hline
	\textbf{Codice test} & \textbf{Codice requisito} \\
	\hline
	\endhead
	TV0F1A & R0F1\\
	\hline
	TV0F1B & R0F1\\
	\hline
	TV0F1C & R0F1\\
	\hline
	TV0F1D & R0F1\\
	\hline
	TV0F1E & R0F1\\
	\hline
	TV0F1F & R0F1\\
	\hline
	TV0F1G & R0F1\\
	\hline
	TV0F1H & R0F1\\
	\hline
	TV0F1I & R0F1\\
	\hline
	TV0F1L & R0F1\\
	\hline
	TV2F1.8 & R2F1.8\\
	\hline
	TV0F2A & R0F2\\
	\hline
	TV0F2B & R0F2\\
	\hline
	TV0F3A & R0F3\\
	\hline
	TV0F3B & R0F3\\
	\hline
	TV0F3C & R0F3\\
	\hline
	TV0F3D & R0F3\\
	\hline
	TV0F3E & R0F3\\
	\hline
	TV0F3F & R0F3\\
	\hline
	TV0F3G & R0F3\\
	\hline
	TV2F3.5 & R2F3.5\\
	\hline
	TV0F4A & R0F4\\
	\hline
	TV0F4B & R0F4\\
	\hline
	TV0F4C & R0F4\\
	\hline
	TV0F4D & R0F4\\
	\hline
	TV0F4E & R0F4\\
	\hline
	TV0F4F & R0F4\\
	\hline
	TV0F4G & R0F4\\
	\hline
	TV0F4H & R0F4\\
	\hline
	TV2F4.7 & R2F4.7\\
	\hline
	TV0F5A & R0F5\\
	\hline
	TV0F5B & R0F5\\
	\hline
	TV0F5C & R0F5\\
	\hline
	TV0F5D & R0F5\\
	\hline
	TV0F5E & R0F5\\
	\hline
	TV0F5F & R0F5\\
	\hline
	TV0F5G & R0F5\\
	\hline
	TV0F5H & R0F5\\
	\hline
	TV0F5I & R0F5\\
	\hline
	TV2F5.9 & R2F5.9\\
	\hline
	TV0F6A & R0F6\\
	\hline
	TV0F6B & R0F6\\
	\hline
	TV0F6C & R0F6\\
	\hline
	TV0F6D & R0F6\\
	\hline
	TV0F6E & R0F6\\
	\hline
	TV0F6F & R0F6\\
	\hline
	TV0F6G & R0F6\\
	\hline
	TV0F6H & R0F6\\
	\hline
	TV0F6I & R0F6\\
	\hline
	TV0F6L & R0F6\\
	\hline
	TV2F6.9 & R2F6.9\\
	\hline
	TV2F11A & R2F11\\
	\hline
	TV2F11B & R2F11\\
	\hline
	TV2F12A & R2F12\\
	\hline
	TV2F12B & R2F12\\
	\hline
	TV2F13A & R2F13\\
	\hline
	TV2F13B & R2F13\\
	\hline
	TV2F14A & R2F14\\
	\hline
	TV2F14B & R2F14\\
	\hline
	TV2F16A & R2F16\\
	\hline
	TV2F16B & R2F16\\
	\hline
	TV2F16C & R2F16\\
	\hline
	TV2F16D & R2F16\\
	\hline
	TV0F17A & R0F17\\
	\hline
	TV0F17B & R0F17\\
	\hline
	TV0F18 & R0F18\\
	\hline
	TV0F23A & R0F23\\
	\hline
	TV0F23B & R0F23\\
	\hline
	TV2F23.3 & R2F23.3\\
	\hline
	TV0F24A & R0F24\\
	\hline
	TV0F24B & R0F24\\
	\hline
	TV0F24C & R0F24\\
	\hline
	TV0F24D & R0F24\\
	\hline
	TV0F24E & R0F24\\
	\hline
	TV0F24F & R0F24\\
	\hline
	TV0F24G & R0F24\\
	\hline
	TV0F24H & R0F24\\
	\hline
	TV2F24.6 & R2F24.6\\
	\hline
	TV0F25A & R0F25\\
	\hline
	TV0F25B & R0F25\\
	\hline
	TV0F25C & R0F25\\
	\hline
	TV0F26A & R0F26\\
	\hline
	TV0F26B & R0F26\\
	\hline
	\caption[Tracciamento test di validazione - requisiti]{Tracciamento test di validazione - requisiti}
\end{longtable}
\clearpage
	
\subsection{Test di sistema}
Questi test servono per verificare il comportamento dinamico complessivo dell'intero sistema in riferimento ai requisiti dichiarati nel documento \AdR, come attività di controllo interna svolta dal fornitore.

\newcolumntype{H}{>{\centering\arraybackslash}m{7cm}}
\normalsize
\begin{longtable}{|c|H|c|}
	\hline
	\textbf{Codice test} & \textbf{Descrizione} & \textbf{Stato}\\
	\hline
	TS0F1 & Viene verificato che il sistema permetta la creazione di un account permettendo di usufruire delle funzionalità del sistema, limitate dal tipo di account creato.& \Ts \\
	\hline
    TS0F2 & Viene verificato che il sistema permetta ad ogni utente correttamente registrato di autenticarsi automaticamente. & \Ts \\
	\hline
	TS0F3 &Viene verificato che il sistema permetta ad ogni utente amministratore o università di aggiungere un nuovo anno accademico al sistema. & \Ts \\ 
	\hline
	TS0F4 & Viene verificato che il sistema permetta ad ogni utente amministratore o università di aggiungere un nuovo corso di laurea al sistema. & \Ts \\
	\hline
	TS0F5 & Viene verificato che il sistema permetta ad ogni utente amministratore o università di aggiungere una nuova attività didattica al sistema. & \Ts \\
	\hline
	TS0F6 &  Viene verificato che il sistema permetta ad ogni utente amministratore o università di aggiungere un nuovo esame al sistema. & \Ts \\
	\hline
	TS2F7 & Viene verificato che il sistema permetta ad ogni utente amministratore o università di modificare un anno accademico già presente nel sistema. & \Ts \\ 
	\hline
	TS2F8 & Viene verificato che il sistema permetta ad ogni utente amministratore o università di modificare un corso di laurea già presente nel sistema.  & \Ts \\
	\hline
	TS2F9 & Viene verificato che il sistema permetta ad ogni utente amministratore o università di modificare un'attività didattica già presente nel sistema. & \Ts \\
	\hline
	TS2F10 & Viene verificato che il sistema permetta ad ogni utente amministratore o università di modificare un esame già presente nel sistema. & \Ts \\
	\hline
	TS2F11 & Viene verificato che il sistema permetta ad ogni utente amministratore o università di eliminare un anno accademico già presente nel sistema. & \Ts \\
	\hline
	TS2F12 & Viene verificato che il sistema permetta ad ogni utente amministratore o università di eliminare un corso di laurea già presente nel sistema. & \Ts \\
	\hline
	TS2F13 & Viene verificato che il sistema permetta ad ogni utente amministratore o università di eliminare un'attività didattica già presente nel sistema. & \Ts \\
	\hline
	TS2F14 & Viene verificato che il sistema permetta ad ogni utente amministratore o università di eliminare un esame già presente nel sistema. & \Ts \\
	\hline
	TS2F15 &Viene verificato che il sistema permetta ad ogni utente amministratore o università di inserire un nuovo professore associato ad un'attività didattica. & \Ts \\
	\hline
	TS2F16 & Viene verificato che il sistema permetta ad ogni utente studente di iscriversi ad un esame.& \Ts \\
	\hline
	TS0F17 & Viene verificato che il sistema permetta ad ogni utente professore di visualizzare la lista degli esami a lui associati. & \Ts \\
	\hline
	TS0F18 & Viene verificato che il sistema permetta ad ogni utente professore di visualizzare la lista degli studenti iscritti ad uno degli esami a lui associati. & \Ts \\
	\hline
	TS0F19 & Viene verificato che il sistema visualizzi per ogni operazione il costo associato. & \Ts \\
	\hline
	TS0F20 & Viene verificato che il sistema permetta ad ogni utente studente di eliminare l'iscrizione ad un esame. & \Ts \\
	\hline
	TS0F21 & Viene verificato che il sistema permetta ad ogni utente studente di rifiutare il risultato di un esame e che se il sistema accetti automaticamente il voto se esso non è stato rifiutato entro 8 giorni. & \Ts \\
	\hline
	TS0F22 & Viene verificato che il sistema permetta ad ogni utente professore di modificare gli esami a lui associati. & \Ts \\
	\hline
	TS0F23 & Viene verificato che il sistema permetta ad ogni utente professore di registrare il voto relativo ad un esame di uno studente. & \Ts \\
	\hline
	TS0F24 & Viene verificato che il sistema permetta ad ogni utente amministratore o università di creare un nuovo utente nel sistema.& \Ts \\
    \hline
	TS0F25 & Viene verificato che il sistema permetta ad ogni utente amministratore o università di rimuovere un utente dal sistema. & \Ts \\
	\hline
	TS0F26 & Viene verificato che il sistema permetta ad ogni utente studente di visualizzare il proprio libretto. & \Ts \\
	\hline
	TS &  Viene verificato che il sistema offra correttamente tutte le sue funzionalità con la versione 6.0.286 o superiore di Google Chrome, utilizzando metamask versione 4.6 o superiore.  & \Ts \\
	\hline
	TS &   Viene verificato che il sistema offra correttamente tutte le sue funzionalità con la versione 50 o superiore di Mozilla Firefox, utilizzando metamask versione 4.5 o superiore. & \Ts \\
	\hline
	TS &   Viene verificato che il sistema offra correttamente tutte le sue funzionalità con la versione 52 o superiore di Opera, utilizzando metamask versione 3.13.4 o superiore. & \Ts \\
	\hline
	\caption[Test di sistema]{Test di sistema}
\end{longtable}
\clearpage

\subsubsection{Tracciamento Test di sistema - Requisiti}
\normalsize
\begin{longtable}{|c|c|}
	\hline
	\textbf{Codice test} & \textbf{Codice requisito} \\
	\hline
	\endhead
	TS0F1 & R0F1\\
	\hline
	TS0F2 & R0F2\\
	\hline
	TS0F3 & R0F3\\
	\hline
	TS0F4 & R0F4\\
	\hline
	TS0F5 & R0F5\\
	\hline
	TS0F6 & R0F6\\
	\hline
	TS2F7 & R2F7\\
	\hline
	TS2F8 & R2F8\\
	\hline
	TS2F9 & R2F9\\
	\hline
	TS2F10 & R02F10\\
	\hline
	TS2F11 & R2F11\\
	\hline
	TS2F12 & R2F12\\
	\hline
	TS2F13 & R2F13\\
	\hline
	TS2F14 & R2F14\\
	\hline
	TS2F15 & R2F15\\
	\hline
	TS2F16 & R2F16\\
	\hline
	TS0F17 & R0F17\\
	\hline
	TS0F18 & R0F18\\
	\hline
	TS0F19 & R0F19\\
	\hline
	TS0F20 & R0F20\\
	\hline
	TS0F21 & R0F21\\
	\hline
	TS0F22 & R0F22\\
	\hline
	TS0F23 & R0F23\\
	\hline
	TS0F24 & R0F24\\
	\hline
	TS0F25 & R0F25\\
	\hline
	TS0F26 & R0F26\\
	\hline
	TS0F27 & R0F27\\
	\hline
	TS0F28 & R0F28\\
	\hline
	\caption[Tracciamento test di sistema - requisiti]{Tracciamento test di sistema - requisiti}
\end{longtable}
\clearpage

\subsection{Test di integrazione}
Questi test verificano il corretto comportamento di ogni singola componente e le relazioni con il resto del sistema.

\newcolumntype{H}{>{\centering\arraybackslash}m{7cm}}
\normalsize
\begin{longtable}{|c|H|c|}
	\hline
	\textbf{Codice test} & \textbf{Descrizione} & \textbf{Stato}\\
	\hline
	TI1 & Viene verificato che l’applicazione Web gestisca correttamente il front end del prodotto e le sue interazioni con il back end. & \Ts \\
	\hline
	TI2 & Viene verificato che i contratti Solidity si integrino correttamente con la componente web3. & \Ts \\ 
	\hline
	TI3 & Viene verificato che le componenti \emph{action}, \emph{store} e \emph{reducers} di Redux comunichino correttamente tra loro.  & \Ts \\
	\hline
	TI4 & Viene verificato che le componenti visive di React si integrino correttamente con la gestione degli \emph{state} di Redux. & \Ts \\
	\hline
	TI5 & Viene verificato che l'immissione dei dati nel front end corrisponda ad una loro effettiva scrittura nel database del back end. & \Ts \\
	\hline
	TI6 &Viene verificato che l'applicativo web mostri correttamente i dati prelevati dal back end. & \Ts \\ 
	\hline
		\caption[Test di integrazione]{Test di integrazione}
\end{longtable}
\clearpage

\subsection{Test di unità}
Questi test servono per verificare il corretto funzionamento delle singole parti del sistema.

\newcolumntype{H}{>{\centering\arraybackslash}m{7cm}}
\normalsize
\begin{longtable}{|c|H|c|}
	\hline
	\textbf{Codice test} & \textbf{Descrizione} & \textbf{Stato}\\
	\hline
	TU1 & Viene verificato che il metodo \emph{LogoutUser()} faccia uscire l'utente dal sistema e faccia il dispatch dello stato dell'operazione. & \Ts \\
	\hline
	TU2 & Viene verificato che il metodo \emph{OnInsertUserFormSubmit(string,string,int)} inserisca un utente nel sistema e faccia il dispatch dello stato dell'operazione. & \Ts \\
	\hline
	TU3 &Viene verificato che dopo aver aggiunto un utente al sistema, esso sia effettivamente salvato nella blockchain assieme al suo codice univoco e codice fiscale. & \Ts \\ 
	\hline
	TU4 & Viene verificato che, dopo che un utente ha eseguito la registrazione tramite il codice univoco fornito dall'università ed il suo codice fiscale, immettendo i propri dati personali, esso risulti effettivamente registrato con i propri dati salvati. & \Ts \\
	\hline
	TU5 & Viene verificato che l'inserimento di un nuovo anno accademico sia effettivamente salvato con i dati relativi specificati. & \Ts \\
	\hline
	TU6 &  Viene verificato che l'inserimento di un nuovo corso di laurea sia effettivamente salvato con i dati relativi specificati. & \Ts \\
	\hline
	TU7 & Viene verificato che l'inserimento di un nuovo esame sia effettivamente salvato con i dati relativi specificati. & \Ts \\ 
	\hline
	TU8 & Viene verificato che l'associazione di un esame con un professore comporti effettivamente un collegamento tra le due parti.  & \Ts \\
	\hline
	TU9 & Viene verificato che il metodo \emph{RemoveAdmic(admin)} rimuova correttamente le informazioni e faccia il dispatch dello stato dell'operazione. & \Ts \\
	\hline
	TU10 & Viene verificato che il metodo \emph{RemoveStudent(student)} rimuova correttamente le informazioni e faccia il dispatch dello stato dell'operazione. & \Ts \\
	\hline
	TU11 & Viene verificato che il metodo \emph{ViewStudentList(exam)} visualizzi correttamente le informazioni e faccia il dispatch dello stato dell'operazione. & \Ts \\
	\hline
	TU12 & Viene verificato che il metodo \emph{LoginUser()} autentichi correttamente l'utente nel sistema e faccia il dispatch dello stato dell'operazione. & \Ts \\
	\hline
	TU13 & Viene verificato che il metodo \emph{render()} del component \emph{Help} comporti la corretta renderizzazione del componente. & \Ts \\
	\hline
	TU14 & Viene verificato che il metodo \emph{render()} del component \emph{LoginButoon} comporti la corretta renderizzazione del componente. & \Ts \\
	\hline
	TU15 & Viene verificato che il metodo \emph{render()} del component \emph{LogoutButton} comporti la corretta renderizzazione del componente. & \Ts \\
	\hline
	TU16 & Viene verificato che il metodo \emph{render()} del component \emph{NavButton} comporti la corretta renderizzazione del componente. & \Ts \\
	\hline
	TU17 & Viene verificato che il metodo \emph{render()} del component \emph{Home} comporti la corretta renderizzazione del componente. & \Ts \\
	\hline
	TU18 & Viene verificato che il metodo \emph{render()} del component \emph{App} comporti la corretta renderizzazione del componente. & \Ts \\
	\hline
	TU19 & Viene verificato che il metodo \emph{render()} del component \emph{InsertUserForm} comporti la corretta renderizzazione del componente. & \Ts \\
	\hline
	TU20 & Viene verificato che il metodo \emph{render()} del component \emph{InsertUser} comporti la corretta renderizzazione del componente. & \Ts \\
	\hline
	TU21 & Viene verificato che il metodo \emph{render()} del component \emph{EmptyData} comporti la corretta renderizzazione del componente. & \Ts \\
	\hline
	TU22 & Viene verificato che il metodo \emph{render()} del component \emph{DeleteAdministrator} comporti la corretta renderizzazione del componente. & \Ts \\
	\hline
	TU23 & Viene verificato che il metodo \emph{render()} del component \emph{Administrators} comporti la corretta renderizzazione del componente. & \Ts \\
	\hline
	TU24 & Viene verificato che il metodo \emph{render()} del component \emph{LoadingData} comporti la corretta renderizzazione del componente. & \Ts \\
	\hline
	TU25 & Viene verificato che il metodo \emph{render()} del component \emph{LoadingUser} comporti la corretta renderizzazione del componente. & \Ts \\
	\hline
	TU26 & Viene verificato che il metodo \emph{render()} del component \emph{NotFound} comporti la corretta renderizzazione del componente. & \Ts \\
	\hline
	TU27 & Viene verificato che il metodo \emph{render()} del component \emph{ModifyAcademicYear} comporti la corretta renderizzazione del componente. & \Ti \\
	\hline
	TU28 & Viene verificato che il metodo \emph{render()} del component \emph{InsertAcademicYear} comporti la corretta renderizzazione del componente. & \Ts \\
	\hline
	TU29 & Viene verificato che il metodo \emph{render()} del component \emph{DeleteAcademicYear} comporti la corretta renderizzazione del componente. & \Ti \\
	\hline
	TU30 & Viene verificato che il metodo \emph{render()} del component \emph{AcademicYears} comporti la corretta renderizzazione del componente. & \Ts \\
	\hline
	TU31 & Viene verificato che il metodo \emph{render()} del component \emph{ModifyDegreeCourse} comporti la corretta renderizzazione del componente. & \Ti \\
	\hline
	TU32 & Viene verificato che il metodo \emph{render()} del component \emph{InsertDegreeCourse} comporti la corretta renderizzazione del componente. & \Ts \\
	\hline
	TU33 & Viene verificato che il metodo \emph{render()} del component \emph{DeleteDegreeCourse} comporti la corretta renderizzazione del componente. & \Ti \\
	\hline
	TU34 & Viene verificato che il metodo \emph{render()} del component \emph{DegreeCourses} comporti la corretta renderizzazione del componente. & \Ts \\
	\hline
	TU35 & Viene verificato che il metodo \emph{render()} del component \emph{ModifyDidacticActivity} comporti la corretta renderizzazione del componente. & \Ti \\
	\hline
	TU36 & Viene verificato che il metodo \emph{render()} del component \emph{InsertDidacticActivity} comporti la corretta renderizzazione del componente. & \Ts \\
	\hline
	TU37 & Viene verificato che il metodo \emph{render()} del component \emph{DeleteDidacticActivity} comporti la corretta renderizzazione del componente. & \Ti \\
	\hline
	TU38 & Viene verificato che il metodo \emph{render()} del component \emph{DidacticActivities} comporti la corretta renderizzazione del componente. & \Ts \\
	\hline
	TU39 & Viene verificato che il metodo \emph{render()} del component \emph{InsertExam} comporti la corretta renderizzazione del componente. & \Ts \\
	\hline
	TU40 & Viene verificato che il metodo \emph{render()} del component \emph{DeleteProfessor} comporti la corretta renderizzazione del componente. & \Ti \\
	\hline
	TU41 & Viene verificato che il metodo \emph{render()} del component \emph{Professors} comporti la corretta renderizzazione del componente. & \Ts \\
	\hline
	TU42 & Viene verificato che il metodo \emph{render()} del component \emph{DeleteStudent} comporti la corretta renderizzazione del componente. & \Ts \\
	\hline
	TU43 & Viene verificato che il metodo \emph{render()} del component \emph{Students} comporti la corretta renderizzazione del componente. & \Ts \\
	\hline
	TU44 & Viene verificato che il metodo \emph{render()} del component \emph{ExamList} comporti la corretta renderizzazione del componente. & \Ts \\
	\hline
	TU45 & Viene verificato che il metodo \emph{render()} del component \emph{ExamPage} comporti la corretta renderizzazione del componente. & \Ts \\
	\hline
	TU46 & Viene verificato che il metodo \emph{render()} del component \emph{ExamsProfessorList} comporti la corretta renderizzazione del componente. & \Ts \\
	\hline
	TU47 & Viene verificato che il metodo \emph{render()} del component \emph{RegisteredStudentsList} comporti la corretta renderizzazione del componente. & \Ts \\
	\hline
	TU48 & Viene verificato che il metodo \emph{render()} del component \emph{ExamsStudentList} comporti la corretta renderizzazione del componente. & \Ts \\
	\hline
	TU49 & Viene verificato che il metodo \emph{render()} del component \emph{SchoolRecords} comporti la corretta renderizzazione del componente. & \Ts \\
	\hline
	TU50 & Viene verificato che il metodo \emph{render()} del component \emph{Profile} comporti la corretta renderizzazione del componente. & \Ts \\
	\hline
	TU51 & Viene verificato che il metodo \emph{render()} del component \emph{SubNavbar} comporti la corretta renderizzazione del componente. & \Ts \\
	\hline
	TU52 & Viene verificato che il metodo \emph{render()} del component \emph{SubNavButton} comporti la corretta renderizzazione del componente. & \Ts \\
	\hline
	TU53 & Viene verificato che il metodo \emph{render()} del component \emph{SignUp} comporti la corretta renderizzazione del componente. & \Ts \\
	\hline
	TU54 & Viene verificato che il metodo \emph{render()} del component \emph{SignUpForm} comporti la corretta renderizzazione del componente. & \Ts \\
	\hline
	TU55 & Viene verificato che il metodo \emph{ReadAcademicData()} legga correttamente le informazioni e faccia il dispatch dello stato dell'operazione. & \Ts \\
	\hline
	TU56 & Viene verificato che il metodo \emph{ReadDidacticActivities(int,int)} legga correttamente le informazioni e faccia il dispatch dello stato dell'operazione. & \Ts \\
	\hline
	TU57 & Viene verificato che il metodo \emph{ReadDegreeCourses(int)} legga correttamente le informazioni e faccia il dispatch dello stato dell'operazione. & \Ts \\
	\hline
	TU58 & Viene verificato che il metodo \emph{ReadExams(int,int,int)} legga correttamente le informazioni e faccia il dispatch dello stato dell'operazione. & \Ts \\
	\hline
	TU59 & Viene verificato che il metodo \emph{ReadProfessorList()} legga correttamente le informazioni e faccia il dispatch dello stato dell'operazione. & \Ts \\
	\hline
	TU60 & Viene verificato che il metodo \emph{ReadAdminList()} legga correttamente le informazioni e faccia il dispatch dello stato dell'operazione. & \Ts \\
	\hline
	TU61 & Viene verificato che il metodo \emph{ReadStudentsList()} legga correttamente le informazioni e faccia il dispatch dello stato dell'operazione. & \Ts \\
	\hline
	TU62 & Viene verificato che il metodo \emph{ReadExam(int,int,int)} legga correttamente le informazioni e faccia il dispatch dello stato dell'operazione. & \Ts \\
	\hline
	TU63 & Viene verificato che il metodo \emph{ReadExamList(professor)} legga correttamente le informazioni e faccia il dispatch dello stato dell'operazione. & \Ts \\
	\hline
	TU64 & Viene verificato che il metodo \emph{ReadExamList(student)} legga correttamente le informazioni e faccia il dispatch dello stato dell'operazione. & \Ts \\
	\hline
	TU65 & Viene verificato che il metodo \emph{ModifyAcademicData()} modifichi correttamente le informazioni e faccia il dispatch dello stato dell'operazione. & \Ts \\
	\hline
	TU66 & Viene verificato che il metodo \emph{ModifyDidacticActivity(int,int,int)} modifichi correttamente le informazioni e faccia il dispatch dello stato dell'operazione. & \Ts \\
	\hline
	TU67 & Viene verificato che il metodo \emph{ModifyDegreeCourse(int,int)} modifichi correttamente le informazioni e faccia il dispatch dell'operazione. & \Ts \\
	\hline
	TU68 & Viene verificato che il metodo \emph{ModifyExam(int,int,int,int)} modifichi correttamente le informazioni e faccia il dispatch dell'operazione. & \Ts \\
	\hline
	TU69 & Viene verificato che il metodo \emph{OnSignUpFormSubmit(string,string,string,string,int)} inserisca un utente nel sistema registrandone i dati nel sistema e faccia il dispatch dello stato dell'operazione. & \Ts \\
	\hline
	TU70 & Viene verificato che il metodo \emph{AddAcademicYear(int)} aggiunga correttamente le informazioni e faccia il dispatch dello stato dell'operazione. & \Ts \\
	\hline
	TU71 & Viene verificato che il metodo \emph{AddDidacticActivity(int,int,int)} aggiunga correttamente le informazioni e faccia il dispatch dello stato dell'operazione. & \Ts \\
	\hline
	TU72 & Viene verificato che il metodo \emph{AddDegreeCourse(int,int)} aggiunga correttamente le informazioni e faccia il dispatch dello stato dell'operazione. & \Ts \\
	\hline
	TU73 & Viene verificato che il metodo \emph{AddExam(int,int,int,int)} aggiunga correttamente le informazioni e faccia il dispatch dello stato dell'operazione. & \Ts \\
	\hline
	TU74 & Viene verificato che il metodo \emph{RemoveAcademicYear(int)} rimuova correttamente le informazioni e faccia il dispatch dello stato dell'operazione. & \Ts \\
	\hline
	TU75 & Viene verificato che il metodo \emph{RemoveDidacticActivity(int,int,int)} rimuova correttamente le informazioni e faccia il dispatch dello stato dell'operazione. & \Ts \\
	\hline
	TU76 & Viene verificato che il metodo \emph{RemoveDegreeCourse(int,int)} rimuova correttamente le informazioni e faccia il dispatch dello stato dell'operazione. & \Ts \\
	\hline
	TU77 & Viene verificato che il metodo \emph{RemoveExam(int,int,int,int)} rimuova correttamente le informazioni e faccia il dispatch dello stato dell'operazione. & \Ts \\
	\hline
	TU78 & Viene verificato che il metodo \emph{RemoveProfessor(admin)} rimuova correttamente le informazioni e faccia il dispatch dello stato dell'operazione. & \Ts \\
	\hline

	\caption[Test di sistema]{Test di sistema}
\end{longtable}

\subsubsection{Tracciamento Test di unità - Metodi}
\normalsize
\begin{longtable}{|>{\centering\arraybackslash}p{2cm}| p{15cm}|}
	\hline
	\textbf{Codice test} & \textbf{Classe:metodo} \\
	\hline
	\endhead
	TU1 & $InsertUserFormAction:OnInsertUserFormSubmit(string,string,int)$ \\
	\hline
	TU2 & $LogoutButtonAction:LogoutUser()$ \\
	\hline
	TU3 & $Admin:addUser(bytes32 \textunderscore fiscalCode, bytes10 \textunderscore uniCode, uint8 \textunderscore userType)$\newline$UserData:getUsersUniCode(bytes32 \textunderscore fiscalCode)$\newline$UserData:getUsersUserType(bytes32 \textunderscore fiscalCode)$
	\\
	\hline

	TU4 & $UserLogic:signUp(bytes32 \textunderscore fiscalCode, bytes10 \textunderscore uniCode, bytes32 \textunderscore hashData)$\newline$UserData:getRegUsersUniCode(address \textunderscore address) $\newline$UserData:getRegUsersFiscalCode(address \textunderscore address)$\newline$UserData:getRegUsersUserType(address \textunderscore address) $\newline$UserData:getRegUsersBadgeNumber(address \textunderscore address)$\newline$UserData:getRegUsersHashData(address \textunderscore address)$\\
	\hline
	TU5 & $Admin:addNewDegree(bytes10 \textunderscore degreeUniCode, bytes4 \textunderscore year, bytes32 \textunderscore hashData) $\newline$DegreeData:isDegree(bytes10 \textunderscore degreeUniCode) $\newline$DegreeData:getDegreeCourses(bytes10 \textunderscore degreeUniCode)$\\
	\hline
	TU6 &$Admin:addNewExam(bytes10 \textunderscore courseUniCode, bytes10 \textunderscore examUniCode, uint32 \textunderscore examTeacher,$\newline$ bytes32 \textunderscore examHashData) $\newline$ExamData:examExist(bytes10 \textunderscore examUniCode)$\newline$ExamData:getExamTeacher(bytes10 \textunderscore examUniCode) $\newline$ExamData:getAllIdentifiers()$\\
	\hline
	TU7 & $Teacher:myExams()$\\
	\hline
	TU8 &$ Teacher:registerResult(bytes10 \textunderscore examUniCode, uint32 \textunderscore studentBadgeNumber, bytes2 \textunderscore result)$\\
	\hline
	TU9 & $DeleteAdministrator:RemoveAdmin(admin)$\\
	\hline
	TU10 & $DeleteStudent:RemoveStudent(steudent)$\\
	\hline
	TU11 & $RegisteredStudentListAction:ViewStudentListAction(exam)$\\
	\hline
	TU12 & $LoginButtonAction:LoginUser()$\\
	\hline
	TU13 & $Help:render$\\
	\hline
	TU14 & $LoginButton:render$\\
	\hline
	TU15 & $LogoutButton:render$\\
	\hline
	TU16 & $NavButton:render$\\
	\hline
	TU17 & $Home:render$\\
	\hline
	TU18 & $App:render$\\
	\hline
	TU19 & $InsertUserForm:render$\\
	\hline
	TU20 & $InsertUser:render$\\
	\hline
	TU21 & $EmptyData:render$\\
	\hline
	TU22 & $DeleteAdministrator:render$\\
	\hline
	TU23 & $Administrators:render$\\
	\hline
	TU24 & $LoadingData:render$\\
	\hline
	TU25 & $LoadingUser:render$\\
	\hline
	TU26 & $NotFound:render$\\
	\hline
	TU27 & $ModifyAcademicYear:render$\\
	\hline
	TU28 & $InsertAcademicYear:render$\\
	\hline
	TU29 & $DeleteAcademicYear:render$\\
	\hline
	TU30 & $AcademicYears:render$\\
	\hline
	TU31 & $ModifyDegreeCourse:render$\\
	\hline
	TU32 & $InsertDegreeCourse:render$\\
	\hline
	TU33 & $DeleteDegreeCourse:render$\\
	\hline
	TU34 & $DegreeCourses:render$\\
	\hline
	TU35 & $ModifyDidacticActivity:render$\\
	\hline
	TU36 & $InsertDidacticActivity:render$\\
	\hline
	TU37 & $DeleteDidacticActivity:render$\\
	\hline
	TU38 & $DidacticActivities:render$\\
	\hline
	TU39 & $InsertExam:render$\\
	\hline
	TU40 & $DeleteProfessor:render$\\
	\hline
	TU41 & $Professors:render$\\
	\hline
	TU42 & $DeleteStudent:render$\\
	\hline
	TU43 & $Students:render$\\
	\hline
	TU44 & $ExamList:render$\\
	\hline
	TU45 & $ExamPage:render$\\
	\hline
	TU46 & $ExamProfessorList:render$\\
	\hline
	TU47 & $RegisteredStudentsList:render$\\
	\hline
	TU48 & $ExamsStudentList:render$\\
	\hline
	TU49 & $SchoolRecords:render$\\
	\hline
	TU50 & $Profile:render$\\
	\hline
	TU51 & $SubNavbar:render$\\
	\hline
	TU52 & $SubNavButton:render$\\
	\hline
	TU53 & $SignUp:render$\\
	\hline
	TU54 & $SignUpForm:render$\\
	\hline
	TU55 & $ViewAcademicYearAction:ReadAcademicData()$\\
	\hline
	TU56 & $ViewDidacticActivitiesActione:ReadDidacticActivities(int,int)$\\
	\hline
	TU57 & $ViewDegreeCoursesAction:ReadDegreeCourses(int)$\\
	\hline
	TU58 & $ViewExamsAction:ReadExams(int,int,int)$\\
	\hline
	TU59 & $ViewProfessor:ReadProfessorList()$\\
	\hline
	TU60 & $ViewAdministrators:ReadAdminList()$\\
	\hline
	TU61 & $ViewStudents:ReadStudentList()$\\
	\hline
	TU62 & $ExamPageAction:ReadExam(int,int,int)$\\
	\hline
	TU63 & $ExamProfessorListActions:ReadExamList(professor)$\\
	\hline
	TU64 & $ViewExamsAction:ReadExamList(student)$\\
	\hline
	TU65 & $ModifyAcademicYearsActions:ModifyAcademicData()$\\
	\hline
	TU66 & $ModifyDidacticActivityAction:ModifyDidacticActivity(int,int,int)$\\
	\hline
	TU67 & $ModifyDegreeCourseAction:ModifyDegreeCourse(int,int)$\\
	\hline
	TU68 & $ModifyExamAction:ModifyExam(int,int,int,int)$\\
	\hline
	TU69 & $InsertUserFormAction:OnSignUpFormSubmit(string,string,string,string,int)$\\
	\hline
	TU70 & $InsertAcademicYearAction:AddAcademicYear(int)$\\
	\hline
	TU71 & $InsertDidacticActivitiesAction:AddDidacticActivity(int,int,int)$\\
	\hline
	TU72 & $InsertDegreeCourseAction:AddDegreeCourse(int,int)$\\
	\hline
	TU73 & $InsertExamAction:AddExam(int,int,int,int)$\\
	\hline
	TU74 & $DeleteAcademicYearAction:RemoveAcademicYear(int)$\\
	\hline
	TU75 & $RemoveDidacticActivities:RemoveDidacticActivities(int,int,int)$\\
	\hline
	TU76 & $DeleteDegreeCoursesAction:removeDegreeCourse(int,int)$\\
	\hline
	TU77 & $DeleteExamAction:removeExam(int,int,int)$\\
	\hline
	TU78 & $DeleteProfessor:RemoveProfessor(admin)$\\
	\hline
\caption[Tracciamento test di unità - metodi]{Tracciamento test di unità - metodi}
\end{longtable}
\clearpage