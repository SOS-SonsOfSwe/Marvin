\pagebreak
\section{Specifica dei test}
Questa sezione definirà i test che verranno implementati ed eseguiti dal team al fine di garantire, in seguito al loro superamento, la creazione di software di alta qualità che soddisfi le richieste ed aspettative del proponente. 
Ogni test viene riconosciuto tramite un identificativo univoco, la cui sintassi è descritta nelle \NdP (§3.2.2) e può assumere uno tra questi stati: \Ts, \Ti, \Ts. 
Dato che il progetto è ancora alla fase iniziale, si presume che questa sezione sarà soggetta a modifiche basate sulle esigenze ed i problemi che si incontreranno durante lo sviluppo.

\subsection{Test di validazione}
Questi test verranno utilizzati durante la fase di collaudo finale, in presenza di committente e proponente, per valutare se il prodotto è conforme a quanto specificato nel contratto.

\newcolumntype{H}{>{\centering\arraybackslash}m{7cm}}
\normalsize
\begin{longtable}{|c|H|c|}
	\hline
	\textbf{Codice test} & \textbf{Descrizione} & \textbf{Stato}\\
	\hline
	\endhead
	TV0F1 & Un utente non autenticato intende registrarsi creando il proprio account. \newline \begin{flushleft}
	Azioni: \newline
	\end{flushleft}
	\begin{enumerate}
		\item Entrare nella pagina di registrazione;
		\item Inserire il nome;
		\item Inserire il cognome;
		\item Inserire la e-mail;
		\item Inserire il codice fiscale;
		\item Inserire il codice univoco;
		\item Inserire la matricola;
		\item Verificare che \emph{metamask}\ped{G} sia attivato correttamente;
		\item Confermare la registrazione;
		\item Verificare che la creazione sia andata a buon fine effettuando un login;	
	\end{enumerate} & \Ts \\
	\hline
	TV2F1.8 & L'utente non autenticato desidera visualizzare un messaggio d'errore nel caso in cui abbia inserito dati per la registrazione in maniera errata. \newline \begin{flushleft}
		Azioni:\newline
			\end{flushleft} 
		\begin{enumerate}
			\item Riempire i campi dato con informazioni già presenti, non valide o nulle;
			\item Verificare che il sistema visualizzi correttamente un messaggio d'errore.
		\end{enumerate} & \Ts \\
	\hline
	TV0F2 & L’utente non autenticato desidera autenticarsi una volta registrato.\newline \begin{flushleft}
		Azioni:\newline
			\end{flushleft} 
		\begin{enumerate}
			\item Verificare che metamask sia attivato correttamente;
			\item Cliccare il pulsante di login;
			\item Verificare che il sistema segnali la corretta autenticazione.
		\end{enumerate} & \Ts \\
	\hline
	TV0F3 & L'utente amministratore o l'utente università desidera aggiungere un anno accademico alla lista degli anni accademici. \newline \begin{flushleft}
		Azioni:\newline
			\end{flushleft} 
		\begin{enumerate}
			\item Verificare di essere correttamente autenticato come amministratore o come università;
			\item Entrare nella pagina per l'inserimento di un anno accademico;
			\item Inserire l'inizio dell'anno accademico;
			\item Inserire la fine dell'anno accademico;
			\item Inserire il nome dell'anno accademico;
			\item Confermare l'aggiunta dell'anno accademico;
			\item Verificare che l'aggiunta sia andata a buon fine cercando l'anno accademico appena aggiunto nella pagina dedicata alla lista degli anni accademici.
		\end{enumerate} & \Ts \\ 
	\hline
	TV2F3.5 & L'utente amministratore o università desidera visualizzare un messaggio d'errore nel caso in cui abbia inserito dati relativi ad un anno accademico in maniera errata. \newline \begin{flushleft}
		Azioni:\newline
			\end{flushleft} 
		\begin{enumerate}
			\item Verificare di essere correttamente autenticato come amministratore o come università;
			\item Riempire i campi dato  relativi con informazioni errate o nulle;
			\item Verificare che il sistema visualizzi correttamente un messaggio d'errore.
		\end{enumerate} & \Ts \\
	\hline
	TV0F4 & L'utente amministratore o l'utente università desidera aggiungere un corso di laurea alla lista dei corsi di laurea. \newline \begin{flushleft}
		Azioni:\newline
	\end{flushleft} 
		\begin{enumerate}
			\item Verificare di essere correttamente autenticato come amministratore o come università;
			\item Entrare nella pagina per l'inserimento di un corso di laurea;
			\item Inserire il codice del corso di laurea;
			\item Inserire il nome del corso di laurea;
			\item Inserire la descrizione del corso di laurea;
			\item Inserire la tipologia del corso di laurea;
			\item Confermare l'aggiunta del corso di laurea;
			\item Verificare che l'aggiunta sia andata a buon fine cercando il corso di laurea appena aggiunto nella pagina dedicata alla lista dei corsi di laurea.
		\end{enumerate} & \Ts \\
	\hline
	TV2F4.7 & L'utente amministratore o università desidera visualizzare un messaggio d'errore nel caso in cui abbia inserito dati relativi ad un corso di laurea in maniera errata. \newline \begin{flushleft}
		Azioni:\newline
	\end{flushleft} 
	\begin{enumerate}
		\item Verificare di essere correttamente autenticato come amministratore o come università;
		\item Riempire i campi dato con informazioni errate o nulle;
		\item Verificare che il sistema visualizzi correttamente un messaggio d'errore.
	\end{enumerate}  & \Ts \\
	\hline
	TV0F5 & L'utente amministratore o l'utente università desidera aggiungere un'attività didattica alla lista delle attività didattiche. \newline \begin{flushleft}
		Azioni:\newline
		\end{flushleft} 
		\begin{enumerate}
			\item Verificare di essere correttamente autenticato come amministratore o come università;
			\item Entrare nella pagina per l'inserimento di un'attività didattica;
			\item Inserire il codice del corso dell'attività didattica;
			\item Inserire il nome dell'attività didattica;
			\item Inserire la descrizione dell'attività didattica;
			\item Inserire il professore associato all'attività didattica;
			\item Inserire il numero di crediti dell'attività didattica;
			\item Inserire il periodo all'attività didattica;
			\item Confermare l'aggiunta dell'attività didattica;
			\item Verificare che l'aggiunta sia andata a buon fine cercando l'attività didattica appena aggiunta nella pagina dedicata alla lista delle attività didattiche.
		\end{enumerate} & \Ts \\
	\hline
	TV2F5.9 &  L'utente amministratore o università desidera visualizzare un messaggio d'errore nel caso in cui abbia inserito dati relativi ad un'attività didattica in maniera errata. \newline \begin{flushleft}
		Azioni:\newline
	\end{flushleft} 
	\begin{enumerate}
		\item Verificare di essere correttamente autenticato come amministratore o come università;
		\item Riempire i campi dato con informazioni errate o nulle;
		\item Verificare che il sistema visualizzi correttamente un messaggio d'errore.
	\end{enumerate}  & \Ts \\
	\hline
	TV0F6 & L'utente amministratore o l'utente università desidera aggiungere un esame alla lista degli esami. \newline \begin{flushleft}
		Azioni:\newline
		\end{flushleft}
		\begin{enumerate}
			\item Verificare di essere correttamente autenticato come amministratore o come università;
			\item Entrare nella pagina per l'inserimento di un esame;
			\item Inserire il codice dell'esame;
			\item Inserire la descrizione dell'esame;
			\item Inserire l'intervallo di prenotazione all'esame;
			\item Inserire la data dell'esame;
			\item Inserire la tipologia dell'esame;
			\item Inserire il luogo dell'esame;
			\item Confermare l'aggiunta dell'esame.
			\item Verificare che l'aggiunta sia andata a buon fine cercando l'esame appena aggiunto nella pagina dedicata alla lista degli esami.
		\end{enumerate} & \Ts \\
	\hline
	TV2F6.9 & L'utente amministratore o università desidera visualizzare un messaggio d'errore nel caso in cui abbia inserito dati relativi ad un esame in maniera errata. \newline \begin{flushleft}
		Azioni:\newline
	\end{flushleft} 
	\begin{enumerate}
		\item Verificare di essere correttamente autenticato come amministratore o come università;
		\item Riempire i campi dato con informazioni errate o nulle;
		\item Verificare che il sistema visualizzi correttamente un messaggio d'errore.
	\end{enumerate} & \Ts \\
	\hline
	TV2F7 & L'utente amministratore o l'utente università desidera modificare un anno accademico nella lista degli anni accademici. \newline \begin{flushleft}
		Azioni:\newline
	\end{flushleft} 
	\begin{enumerate}
		\item Verificare di essere correttamente autenticato come amministratore o come università;
		\item Entrare nella pagina per la modifica di un anno accademico;
		\item Modificare i dati desiderati (modifiche possibili: inizio e fine dell'anno accademico);
		\item Confermare la modifica dell'anno accademico;
		\item Verificare che l'operazione sia andata a buon fine controllando che i dati siano aggiornati secondo le modifiche appena effettuate nella pagina dedicata all'anno accademico.
	\end{enumerate} & \Ts \\ 
	\hline
	TV2F7.5 & L'utente amministratore o università desidera visualizzare un messaggio d'errore nel caso in cui abbia modifica i dati relativi ad un anno accademico in maniera errata. \newline \begin{flushleft}
		Azioni:\newline
	\end{flushleft} 
	\begin{enumerate}
		\item Verificare di essere correttamente autenticato come amministratore o come università;
		\item Modificare i dati con informazioni errate o nulle;
		\item Verificare che il sistema visualizzi correttamente un messaggio d'errore.
	\end{enumerate} & \Ts \\
	\hline
	TV2F8 & L'utente amministratore o l'utente università desidera modificare un corso di laurea nella lista dei corsi di laurea. \newline \begin{flushleft}
		Azioni:\newline
	\end{flushleft} 
		\begin{enumerate}
			\item Verificare di essere correttamente autenticato come amministratore o come università;
			\item Entrare nella pagina per la modifica di un corso di laurea;
			\item Modificare i dati desiderati (modifiche possibili: codice, nome, descrizione, tipologia del corso di laurea);
			\item Confermare la modifica del corso di laurea;
			\item Verificare che l'operazione sia andata a buon fine controllando che i dati siano aggiornati secondo le modifiche appena effettuate nella pagina dedicata al corso di laurea.
		\end{enumerate} & \Ts \\
		\hline
		TV2F8.7 & L'utente amministratore o università desidera visualizzare un messaggio d'errore nel caso in cui abbia modificato i dati relativi ad un corso di laurea in maniera errata. \newline \begin{flushleft}
			Azioni:\newline
		\end{flushleft} 
		\begin{enumerate}
			\item Verificare di essere correttamente autenticato come amministratore o come università;
			\item Modificare i dati con informazioni errate o nulle;
			\item Verificare che il sistema visualizzi correttamente un messaggio d'errore.
		\end{enumerate}  & \Ts \\
		\hline
		TV2F9 & L'utente amministratore o l'utente università desidera aggiungere un'attività didattica alla lista delle attività didattiche. \newline \begin{flushleft}
			Azioni:\newline
		\end{flushleft} 
		\begin{enumerate}
			\item Verificare di essere correttamente autenticato come amministratore o come università;
			\item Entrare nella pagina per la modifica di un'attività didattica;
			\item Modificare i dati desiderati (modifiche possibili: codice, nome, descrizione, professore, crediti, periodo dell'attività didattica);
			\item Confermare la modifica dell'attività didattica;
			\item Verificare che l'operazione sia andata a buon fine controllando che i dati siano aggiornati secondo le modifiche appena effettuate nella pagina dedicata all'attività didattica.
		\end{enumerate} & \Ts \\
		\hline
		TV2F9.9 &  L'utente amministratore o università desidera visualizzare un messaggio d'errore nel caso in cui abbia modificato i dati relativi ad un'attività didattica in maniera errata. \newline \begin{flushleft}
			Azioni:\newline
		\end{flushleft} 
		\begin{enumerate}
			\item Verificare di essere correttamente autenticato come amministratore o come università;
			\item Modificare i dati con informazioni errate o nulle;
			\item Verificare che il sistema visualizzi correttamente un messaggio d'errore.
		\end{enumerate}  & \Ts \\
		\hline
		TV2F10 & L'utente amministratore o l'utente università desidera modificare un esame nella lista degli esami. \newline \begin{flushleft}
			Azioni:\newline
		\end{flushleft}
		\begin{enumerate}
			\item Verificare di essere correttamente autenticato come amministratore o come università;
			\item Entrare nella pagina per la modifica di un esame;
			\item Modificare i dati desiderati (modifiche possibili: codice, descrizione, intervallo di prenotazione, data, tipologia, luogo dell'esame);
			\item Confermare la modifica dell'esame;
			\item Verificare che l'operazione sia andata a buon fine controllando che i dati siano aggiornati secondo le modifiche appena effettuate nella pagina dedicata all'esame.
		\end{enumerate} & \Ts \\
		\hline
		TV2F10.9 & L'utente amministratore o università desidera visualizzare un messaggio d'errore nel caso in cui abbia modificato i dati relativi ad un esame in maniera errata. \newline \begin{flushleft}
			Azioni:\newline
		\end{flushleft} 
		\begin{enumerate}
			\item Verificare di essere correttamente autenticato come amministratore o come università;
			\item Modificare i dati con informazioni errate o nulle;
			\item Verificare che il sistema visualizzi correttamente un messaggio d'errore.
		\end{enumerate} & \Ts \\
		\hline
		TV2F11 & L'utente amministratore o l'utente università desidera eliminare un anno accademico dalla lista degli anni accademici. \newline \begin{flushleft}
			Azioni:\newline
		\end{flushleft}
		\begin{enumerate}
			\item Verificare di essere correttamente autenticato come amministratore o come università;
			\item Entrare nella pagina della lista degli anni accademici;
			\item Cliccare il pulsante per l'eliminazione dell'anno accademico scelto;
			\item Verificare che l'anno accademico sia stato effettivamente cancellato dalla lista.
		\end{enumerate} & \Ts \\
		\hline
		TV2F12 & L'utente amministratore o l'utente università desidera eliminare un corso di laurea dalla lista dei corsi di laurea. \newline \begin{flushleft}
			Azioni:\newline
		\end{flushleft}
		\begin{enumerate}
			\item Verificare di essere correttamente autenticato come amministratore o come università;
			\item Entrare nella pagina della lista dei corsi di laurea;
			\item Cliccare il pulsante per l'eliminazione del corso di laurea scelto;
			\item Verificare che il corso di laurea sia stato effettivamente cancellato dalla lista.
		\end{enumerate} & \Ts \\
		\hline
		TV2F13 & L'utente amministratore o l'utente università desidera eliminare un'attività didattica dalla lista delle attività didattiche. \newline \begin{flushleft}
			Azioni:\newline
		\end{flushleft}
		\begin{enumerate}
			\item Verificare di essere correttamente autenticato come amministratore o come università;
			\item Entrare nella pagina della lista delle attività didattiche;
			\item Cliccare il pulsante per l'eliminazione dell'attività didattica scelta;
			\item Verificare che l'attività didattica sia stata effettivamente cancellata dalla lista.
		\end{enumerate} & \Ts \\
		\hline
		TV2F14 & L'utente amministratore o l'utente università desidera eliminare un esame dalla lista degli esami associati ad un'attività didattica. \newline \begin{flushleft}
			Azioni:\newline
		\end{flushleft}
		\begin{enumerate}
			\item Verificare di essere correttamente autenticato come amministratore o come università;
			\item Entrare nella pagina della lista degli esami associati ad un'attività didattica;
			\item Cliccare il pulsante per l'eliminazione dell'esame scelto;
			\item Verificare che l'esame sia stato effettivamente cancellato dalla lista.
		\end{enumerate} & \Ts \\
		\hline
		TV2F15 & L'utente amministratore o l'utente università desidera inserire un professore associato ad un esame. \newline \begin{flushleft}
			Azioni:\newline
		\end{flushleft}
		\begin{enumerate}
			\item Verificare di essere correttamente autenticato come amministratore o come università;
			\item Entrare nella pagina della lista degli esami;
			\item Entrare nella pagina dell'esame scelto;
			\item Cliccare il pulsante l'inserimento di un professore ad esso associato;
			\item Scegliere il professore;
			\item Confermare l'aggiunta;
			\item Verificare che il professore sia stato effettivamente aggiunto controllando la sezione relativa ai professori nella pagina dell'esame.
		\end{enumerate} & \Ts \\
		\hline
		TV2F16 & L'utente studente desidera iscriversi ad un esame. \newline \begin{flushleft}
			Azioni:\newline
		\end{flushleft}
		\begin{enumerate}
			\item Verificare di essere correttamente autenticato come studente;
			\item Entrare nella pagina della lista degli esami;
			\item Entrare nella pagina dell'esame scelto;
			\item Cliccare il pulsante per l'iscrizione;
			\item Confermare l'iscrizione.
			\item Verificare che l'iscrizione sia andata a buon fine ricercando l'esame a cui ci si è appena iscritti nella pagina relativa a tutti gli esami a cui l'utente ha effettuato l'iscrizione.
		\end{enumerate} & \Ts \\
		\hline
		TV0F17 & L'utente professore desidera visualizzare la lista degli esami a lui associati. \newline \begin{flushleft}
			Azioni:\newline
		\end{flushleft}
		\begin{enumerate}
			\item Verificare di essere correttamente autenticato come professore;
			\item Entrare nella pagina dedicata agli esami associati all'utente;
			\item Verificare che che il sistema visualizzi correttamente la lista di tutti gli esami a cui il professore è stato precedentemente associato.
		\end{enumerate} & \Ts \\
		\hline
		TV0F18 & L'utente professore desidera visualizzare la lista degli studenti iscritti ad un suo esame. \newline \begin{flushleft}
			Azioni:\newline
		\end{flushleft}
		\begin{enumerate}
			\item Verificare di essere correttamente autenticato come professore;
			\item Entrare nella pagina dedicata agli esami associati all'utente;
			\item Cliccare il pulsante per la visualizzazione degli studenti iscritti a fianco all'esame scelto;
			\item Verificare che che il sistema visualizzi correttamente la lista di tutti gli studenti che si sono precedentemente iscritti all'esame.
		\end{enumerate} & \Ts \\
		\hline
		TV0F20 & L'utente studente desidera eliminare l'iscrizione ad un esame. \newline \begin{flushleft}
			Azioni:\newline
		\end{flushleft}
		\begin{enumerate}
			\item Verificare di essere correttamente autenticato come studente;
			\item Entrare nella pagina dedicata alla lista degli esami a cui l'utente è iscritto;
			\item Cliccare il pulsante "Elimina iscrizione" a fianco all'esame da cui ci si vuole disiscrivere;
			\item Verificare che che l'eliminazione sia andata a buon fine controllando che l'esame non sia più presente nella lista.
		\end{enumerate} & \Ts \\
		\hline
		TV0F21 & L'utente studente desidera rifiutare il voto di un esame. \newline \begin{flushleft}
			Azioni:\newline
		\end{flushleft}
		\begin{enumerate}
			\item Verificare di essere correttamente autenticato come studente;
			\item Entrare nella pagina dedicata ai risultati degli esami;
			\item Entrare nella pagina relativa al risultato che si intende rifiutare;
			\item Cliccare il pulsante "Rifiuta voto";
			\item Verificare che il voto sia stato effettivamente rifiutato controllando che esso non sia più presente nella lista dei risultati.
		\end{enumerate} & \Ts \\
		\hline
		TV0F22 & L'utente professore desidera modificare gli esami a lui associati. \newline \begin{flushleft}
			Azioni:\newline
		\end{flushleft}
		\begin{enumerate}
			\item Verificare di essere correttamente autenticato come professore;
			\item Entrare nella pagina dedicata agli esami associati all'utente;
			\item Entrare nella pagina relativa all'esame che si intende modificare;
			\item Modificare i dati desiderati (modifiche possibili: intervallo di prenotazione, data, luogo dell'esame);
			\item Confermare le modifiche;
			\item Verificare che il voto sia stato effettivamente rifiutato controllando che esso non sia più presente nella lista dei risultati.
		\end{enumerate} & \Ts \\
		\hline
		TV2F22.6 & L'utente professore desidera visualizzare un messaggio d'errore nel caso in cui abbia modificato i dati relativi ad un esame a lui associato in maniera errata. \newline \begin{flushleft}
			Azioni:\newline
		\end{flushleft} 
		\begin{enumerate}
			\item Modificare i dati con informazioni errate o nulle;
			\item Verificare che il sistema visualizzi correttamente un messaggio d'errore.
		\end{enumerate} & \Ts \\
		\hline
		TV0F23 & L'utente professore desidera registrare il voto relativo ad un esame di uno studente. \newline \begin{flushleft}
			Azioni:\newline
		\end{flushleft}
		\begin{enumerate}
			\item Verificare di essere correttamente autenticato come professore;
			\item Entrare nella pagina dedicata all'inserimento dei risultati;
			\item Selezionare lo studente al quale si vuole registrare il voto;
			\item Inserire il voto nell'apposito campo;
			\item Confermare le modifiche;
			\item Verificare che il sistema visualizzi un avviso di corretto inserimento del risultato.
		\end{enumerate} & \Ts \\
		\hline
		TV2F23.3 & L'utente professore desidera visualizzare un messaggio d'errore nel caso in cui abbia inserito un voto in maniera errata. \newline \begin{flushleft}
			Azioni:\newline
		\end{flushleft} 
		\begin{enumerate}
			\item Inserire il voto con un formato errato;
			\item Verificare che il sistema visualizzi correttamente un messaggio d'errore.
		\end{enumerate} & \Ts \\
		\hline
		TV0F24 & L'utente amministratore o l'utente università desidera creare ed inserire un nuovo utente nel sistema. \newline \begin{flushleft}
			Azioni:\newline
		\end{flushleft}
		\begin{enumerate}
			\item Verificare di essere correttamente autenticato come amministratore o come università;
			\item Entrare nella pagina dedicata all'inserimento di un nuovo utente;
			\item Selezionare la tipologia di utente;
			\item Inserire la matricola;
			\item Inserire il codice univoco;
			\item Verificare che l'utente sia stato effettivamente inserito ricercandolo nella pagina dedicata alla visualizzazione della lista degli utenti iscritti.
		\end{enumerate} & \Ts \\
		\hline
		TV2F24.6 & L'utente amministratore o l'utente università desidera visualizzare un messaggio d'errore nel caso in cui abbia inserito dati relativi ad un utente in maniera errata. \newline \begin{flushleft}
			Azioni:\newline
		\end{flushleft} 
		\begin{enumerate}
			\item Inserire i dati in maniera errata;
			\item Verificare che il sistema visualizzi correttamente un messaggio d'errore.
		\end{enumerate} & \Ts \\
		\hline
		TV0F25 & L'utente amministratore o l'utente università desidera rimuovere un utente dal sistema. \newline \begin{flushleft}
			Azioni:\newline
		\end{flushleft}
		\begin{enumerate}
			\item Verificare di essere correttamente autenticato come amministratore o come università;
			\item Entrare nella pagina dedicata alla visualizzazione della lista degli utenti iscritti;
			\item Cliccare il pulsante relativo alla rimozione di un utente a fianco all'utente che si vuole rimuovere;
			\item Confermare la rimozione;
			\item Verificare che l'utente sia stato effettivamente rimosso controllando che non sia presente nella pagina dedicata alla visualizzazione della lista degli utenti iscritti.
		\end{enumerate} & \Ts \\
		\hline
			TV0F26 & L'utente studente desidera visualizzare il proprio libretto. \newline \begin{flushleft}
			Azioni:\newline
		\end{flushleft}
		\begin{enumerate}
			\item Verificare di essere correttamente autenticato come studente;
			\item Entrare nella pagina dedicata alla visualizzazione del libretto dell'utente;
			\item Verificare che il sistema visualizzi correttamente tutti gli esami sostenuti o da sostenere da parte dell'utente.
		\end{enumerate} & \Ts \\
		\hline
		TV0F27 & L'utente professore desidera visualizzare la lista degli esami a lui associati. \newline \begin{flushleft}
			Azioni:\newline
		\end{flushleft}
		\begin{enumerate}
			\item Verificare di essere correttamente autenticato come professore;
			\item Entrare nella pagina dedicata alla visualizzazione della lista degli esami associati all'utente;
			\item Verificare che il sistema visualizzi correttamente tutti gli esami precedentemente associati all'utente.
		\end{enumerate} & \Ts \\
		\hline
		TV0F28 & L'utente professore desidera visualizzare la lista degli studenti iscritti ad uno degli esami a lui associati. \newline \begin{flushleft}
			Azioni:\newline
		\end{flushleft}
		\begin{enumerate}
			\item Verificare di essere correttamente autenticato come professore;
			\item Entrare nella pagina dedicata alla visualizzazione della lista degli esami associati all'utente;
			\item Entrare nella pagina dedicata all'esame di cui si vogliono conoscere gli studenti iscritti;
			\item Verificare che il sistema visualizzi correttamente tutti gli studenti precedentemente iscritti all'esame.
		\end{enumerate} & \Ts \\
	\hline
	\caption[Test di validazione]{Test di validazione}
	\label{tabella:TV}
\end{longtable}

	\subsubsection{Tracciamento Test di validazione - Requisiti}
\normalsize
\begin{longtable}{|c|c|}
	\hline
	\textbf{Codice test} & \textbf{Codice requisito} \\
	\hline
	\endhead
	TV0F1 & R0F1\\
	\hline
	TV2F1.8 & R2F1.8\\
	\hline
	TV0F2 & R0F2\\
	\hline
	TV0F3 & R0F3\\
	\hline
	TV2F3.5 & R2F3.5\\
	\hline
	TV0F4 & R0F4\\
	\hline
	TV2F4.7 & R2F4.7\\
	\hline
	TV0F5 & R0F5\\
	\hline
	TV2F5.9 & R2F5.9\\
	\hline
	TV0F6 & R0F6\\
	\hline
	TV2F6.9 & R2F6.9\\
	\hline
	TV2F7 & R2F7\\
	\hline
	TV2F7.5 & R02F7.5\\
	\hline
	TV2F8 & R2F8\\
	\hline
	TV2F8.7 & R2F8.7\\
	\hline
	TV2F9 & R2F9\\
	\hline
	TV2F9.9 & R2F9.9\\
	\hline
	TV2F10 & R02F10\\
	\hline
	TV2F10.9 & R2F10.9\\
	\hline
	TV2F11 & R2F11\\
	\hline
	TV2F12 & R2F12\\
	\hline
	TV2F13 & R2F13\\
	\hline
	TV2F14 & R2F14\\
	\hline
	TV2F15 & R2F15\\
	\hline
	TV2F16 & R2F16\\
	\hline
	TV0F17 & R0F17\\
	\hline
	TV0F18 & R0F18\\
	\hline
	TV0F20 & R0F20\\
	\hline
	TV0F21 & R0F21\\
	\hline
	TV0F22 & R0F22\\
	\hline
	TV2F22.6 & R2F22.6\\
	\hline
	TV0F23 & R0F23\\
	\hline
	TV2F23.3 & R2F23.3\\
	\hline
	TV0F24 & R0F24\\
	\hline
	TV2F24.6 & R2F24.6\\
	\hline
	TV0F25 & R0F25\\
	\hline
	TV0F26 & R0F26\\
	\hline
	\caption[Tracciamento test di validazione - requisiti]{Tracciamento test di validazione - requisiti}
\end{longtable}
\clearpage
	
\subsection{Test di sistema}
Questi test servono per verificare il comportamento dinamico complessivo dell'intero sistema in riferimento ai requisiti dichiarati nel documento \AdR, come attività di controllo interna svolta dal fornitore.

\newcolumntype{H}{>{\centering\arraybackslash}m{7cm}}
\normalsize
\begin{longtable}{|c|H|c|}
	\hline
	\textbf{Codice test} & \textbf{Descrizione} & \textbf{Stato}\\
	\hline
	TS0F1 & Viene verificato che il sistema permetta la creazione di un account permettendo di usufruire delle funzionalità del sistema, limitate dal tipo di account creato.& \Ts \\
	\hline
    TS0F2 & Viene verificato che il sistema permetta ad ogni utente correttamente registrato di autenticarsi automaticamente. & \Ts \\
	\hline
	TS0F3 &Viene verificato che il sistema permetta ad ogni utente amministratore o università di aggiungere un nuovo anno accademico al sistema. & \Ts \\ 
	\hline
	TS0F4 & Viene verificato che il sistema permetta ad ogni utente amministratore o università di aggiungere un nuovo corso di laurea al sistema. & \Ts \\
	\hline
	TS0F5 & Viene verificato che il sistema permetta ad ogni utente amministratore o università di aggiungere una nuova attività didattica al sistema. & \Ts \\
	\hline
	TS0F6 &  Viene verificato che il sistema permetta ad ogni utente amministratore o università di aggiungere un nuovo esame al sistema. & \Ts \\
	\hline
	TS2F7 & Viene verificato che il sistema permetta ad ogni utente amministratore o università di modificare un anno accademico già presente nel sistema. & \Ts \\ 
	\hline
	TS2F8 & Viene verificato che il sistema permetta ad ogni utente amministratore o università di modificare un corso di laurea già presente nel sistema.  & \Ts \\
	\hline
	TS2F9 & Viene verificato che il sistema permetta ad ogni utente amministratore o università di modificare un'attività didattica già presente nel sistema. & \Ts \\
	\hline
	TS2F10 & Viene verificato che il sistema permetta ad ogni utente amministratore o università di modificare un esame già presente nel sistema. & \Ts \\
	\hline
	TS2F11 & Viene verificato che il sistema permetta ad ogni utente amministratore o università di eliminare un anno accademico già presente nel sistema. & \Ts \\
	\hline
	TS2F12 & Viene verificato che il sistema permetta ad ogni utente amministratore o università di eliminare un corso di laurea già presente nel sistema. & \Ts \\
	\hline
	TS2F13 & Viene verificato che il sistema permetta ad ogni utente amministratore o università di eliminare un'attività didattica già presente nel sistema. & \Ts \\
	\hline
	TS2F14 & Viene verificato che il sistema permetta ad ogni utente amministratore o università di eliminare un esame già presente nel sistema. & \Ts \\
	\hline
	TS2F15 &Viene verificato che il sistema permetta ad ogni utente amministratore o università di inserire un nuovo professore associato ad un'attività didattica. & \Ts \\
	\hline
	TS2F16 & Viene verificato che il sistema permetta ad ogni utente studente di iscriversi ad un esame.& \Ts \\
	\hline
	TS0F17 & Viene verificato che il sistema permetta ad ogni utente professore di visualizzare la lista degli esami a lui associati. & \Ts \\
	\hline
	TS0F18 & Viene verificato che il sistema permetta ad ogni utente professore di visualizzare la lista degli studenti iscritti ad uno degli esami a lui associati. & \Ts \\
	\hline
	TS0F19 & Viene verificato che il sistema visualizzi per ogni operazione il costo associato. & \Ts \\
	\hline
	TS0F20 & Viene verificato che il sistema permetta ad ogni utente studente di eliminare l'iscrizione ad un esame. & \Ts \\
	\hline
	TS0F21 & Viene verificato che il sistema permetta ad ogni utente studente di rifiutare il risultato di un esame e che se il sistema accetti automaticamente il voto se esso non è stato rifiutato entro 8 giorni. & \Ts \\
	\hline
	TS0F22 & Viene verificato che il sistema permetta ad ogni utente professore di modificare gli esami a lui associati. & \Ts \\
	\hline
	TS0F23 & Viene verificato che il sistema permetta ad ogni utente professore di registrare il voto relativo ad un esame di uno studente. & \Ts \\
	\hline
	TS0F24 & Viene verificato che il sistema permetta ad ogni utente amministratore o università di creare un nuovo utente nel sistema.& \Ts \\
    \hline
	TS0F25 & Viene verificato che il sistema permetta ad ogni utente amministratore o università di rimuovere un utente dal sistema. & \Ts \\
	\hline
	TS0F26 & Viene verificato che il sistema permetta ad ogni utente studente di visualizzare il proprio libretto. & \Ts \\
	\hline
	TS &  Viene verificato che il sistema offra correttamente tutte le sue funzionalità con la versione 6.0.286 o superiore di Google Chrome, utilizzando metamask versione 4.6 o superiore.  & \Ts \\
	\hline
	TS &   Viene verificato che il sistema offra correttamente tutte le sue funzionalità con la versione 50 o superiore di Mozilla Firefox, utilizzando metamask versione 4.5 o superiore. & \Ts \\
	\hline
	TS &   Viene verificato che il sistema offra correttamente tutte le sue funzionalità con la versione 52 o superiore di Opera, utilizzando metamask versione 3.13.4 o superiore. & \Ts \\
	\hline
	\caption[Test di sistema]{Test di sistema}
\end{longtable}
\clearpage

\subsubsection{Tracciamento Test di sistema - Requisiti}
\normalsize
\begin{longtable}{|c|c|}
	\hline
	\textbf{Codice test} & \textbf{Codice requisito} \\
	\hline
	\endhead
	TS0F1 & R0F1\\
	\hline
	TS0F2 & R0F2\\
	\hline
	TS0F3 & R0F3\\
	\hline
	TS0F4 & R0F4\\
	\hline
	TS0F5 & R0F5\\
	\hline
	TS0F6 & R0F6\\
	\hline
	TS2F7 & R2F7\\
	\hline
	TS2F8 & R2F8\\
	\hline
	TS2F9 & R2F9\\
	\hline
	TS2F10 & R02F10\\
	\hline
	TS2F11 & R2F11\\
	\hline
	TS2F12 & R2F12\\
	\hline
	TS2F13 & R2F13\\
	\hline
	TS2F14 & R2F14\\
	\hline
	TS2F15 & R2F15\\
	\hline
	TS2F16 & R2F16\\
	\hline
	TS0F17 & R0F17\\
	\hline
	TS0F18 & R0F18\\
	\hline
	TS0F19 & R0F19\\
	\hline
	TS0F20 & R0F20\\
	\hline
	TS0F21 & R0F21\\
	\hline
	TS0F22 & R0F22\\
	\hline
	TS0F23 & R0F23\\
	\hline
	TS0F24 & R0F24\\
	\hline
	TS0F25 & R0F25\\
	\hline
	TS0F26 & R0F26\\
	\hline
	TS0F27 & R0F27\\
	\hline
	TS0F28 & R0F28\\
	\hline
	\caption[Tracciamento test di sistema - requisiti]{Tracciamento test di sistema - requisiti}
\end{longtable}
\clearpage

\subsection{Test di integrazione}
Questi test verificano il corretto comportamento di ogni singola componente e le relazioni con il resto del sistema.

\newcolumntype{H}{>{\centering\arraybackslash}m{7cm}}
\normalsize
\begin{longtable}{|c|H|c|}
	\hline
	\textbf{Codice test} & \textbf{Descrizione} & \textbf{Stato}\\
	\hline
	TI1 & Viene verificato che l’applicazione Web gestisca correttamente il front end del prodotto e le sue interazioni con il back end. & \Ts \\
	\hline
	TI2 & Viene verificato che i contratti Solidity si integrino correttamente con la componente web3. & \Ts \\ 
	\hline
	TI3 & Viene verificato che le componenti \emph{action}, \emph{store} e \emph{reducers} di Redux comunichino correttamente tra loro.  & \Ts \\
	\hline
	TI4 & Viene verificato che le componenti visive di React si integrino correttamente con la gestione degli \emph{state} di Redux. & \Ts \\
	\hline
	TI5 & Viene verificato che l'immissione dei dati nel front end corrisponda ad una loro effettiva scrittura nel database del back end. & \Ts \\
	\hline
	TI6 &Viene verificato che l'applicativo web mostri correttamente i dati prelevati dal back end. & \Ts \\ 
	\hline
		\caption[Test di integrazione]{Test di integrazione}
\end{longtable}
\clearpage

\subsection{Test di unità}
Questi test servono per verificare il corretto funzionamento delle singole parti del sistema.

\newcolumntype{H}{>{\centering\arraybackslash}m{7cm}}
\normalsize
\begin{longtable}{|c|H|c|}
	\hline
	\textbf{Codice test} & \textbf{Descrizione} & \textbf{Stato}\\
	\hline
	TU1 & Viene verificato che il metodo \emph{LogoutUser()} faccia uscire l'utente dal sistema e faccia il dispatch dello stato dell'operazione. & \Ts \\
	\hline
	TU2 & Viene verificato che il metodo \emph{OnInsertUserFormSubmit(string,string,int)} inserisca un utente nel sistema e faccia il dispatch dello stato dell'operazione. & \Ts \\
	\hline
	TU3 &Viene verificato che dopo aver aggiunto un utente al sistema, esso sia effettivamente salvato nella blockchain assieme al suo codice univoco e codice fiscale. & \Ts \\ 
	\hline
	TU4 & Viene verificato che, dopo che un utente ha eseguito la registrazione tramite il codice univoco fornito dall'università ed il suo codice fiscale, immettendo i propri dati personali, esso risulti effettivamente registrato con i propri dati salvati. & \Ts \\
	\hline
	TU5 & Viene verificato che l'inserimento di un nuovo anno accademico sia effettivamente salvato con i dati relativi specificati. & \Ts \\
	\hline
	TU6 &  Viene verificato che l'inserimento di un nuovo corso di laurea sia effettivamente salvato con i dati relativi specificati. & \Ts \\
	\hline
	TU7 & Viene verificato che l'inserimento di un nuovo esame sia effettivamente salvato con i dati relativi specificati. & \Ts \\ 
	\hline
	TU8 & Viene verificato che l'associazione di un esame con un professore comporti effettivamente un collegamento tra le due parti.  & \Ts \\
	\hline
	TU9 & Viene verificato che il metodo \emph{RemoveAdmic(admin)} rimuova correttamente le informazioni e faccia il dispatch dello stato dell'operazione. & \Ts \\
	\hline
	TU10 & Viene verificato che il metodo \emph{RemoveStudent(student)} rimuova correttamente le informazioni e faccia il dispatch dello stato dell'operazione. & \Ts \\
	\hline
	TU11 & Viene verificato che il metodo \emph{ViewStudentList(exam)} visualizzi correttamente le informazioni e faccia il dispatch dello stato dell'operazione. & \Ts \\
	\hline
	TU12 & Viene verificato che il metodo \emph{LoginUser()} autentichi correttamente l'utente nel sistema e faccia il dispatch dello stato dell'operazione. & \Ts \\
	\hline
	TU13 & Viene verificato che il metodo \emph{render()} del component \emph{Help} comporti la corretta renderizzazione del componente. & \Ts \\
	\hline
	TU14 & Viene verificato che il metodo \emph{render()} del component \emph{LoginButoon} comporti la corretta renderizzazione del componente. & \Ts \\
	\hline
	TU15 & Viene verificato che il metodo \emph{render()} del component \emph{LogoutButton} comporti la corretta renderizzazione del componente. & \Ts \\
	\hline
	TU16 & Viene verificato che il metodo \emph{render()} del component \emph{NavButton} comporti la corretta renderizzazione del componente. & \Ts \\
	\hline
	TU17 & Viene verificato che il metodo \emph{render()} del component \emph{Home} comporti la corretta renderizzazione del componente. & \Ts \\
	\hline
	TU18 & Viene verificato che il metodo \emph{render()} del component \emph{App} comporti la corretta renderizzazione del componente. & \Ts \\
	\hline
	TU19 & Viene verificato che il metodo \emph{render()} del component \emph{InsertUserForm} comporti la corretta renderizzazione del componente. & \Ts \\
	\hline
	TU20 & Viene verificato che il metodo \emph{render()} del component \emph{InsertUser} comporti la corretta renderizzazione del componente. & \Ts \\
	\hline
	TU21 & Viene verificato che il metodo \emph{render()} del component \emph{EmptyData} comporti la corretta renderizzazione del componente. & \Ts \\
	\hline
	TU22 & Viene verificato che il metodo \emph{render()} del component \emph{DeleteAdministrator} comporti la corretta renderizzazione del componente. & \Ts \\
	\hline
	TU23 & Viene verificato che il metodo \emph{render()} del component \emph{Administrators} comporti la corretta renderizzazione del componente. & \Ts \\
	\hline
	TU24 & Viene verificato che il metodo \emph{render()} del component \emph{LoadingData} comporti la corretta renderizzazione del componente. & \Ts \\
	\hline
	TU25 & Viene verificato che il metodo \emph{render()} del component \emph{LoadingUser} comporti la corretta renderizzazione del componente. & \Ts \\
	\hline
	TU26 & Viene verificato che il metodo \emph{render()} del component \emph{NotFound} comporti la corretta renderizzazione del componente. & \Ts \\
	\hline
	TU27 & Viene verificato che il metodo \emph{render()} del component \emph{ModifyAcademicYear} comporti la corretta renderizzazione del componente. & \Ti \\
	\hline
	TU28 & Viene verificato che il metodo \emph{render()} del component \emph{InsertAcademicYear} comporti la corretta renderizzazione del componente. & \Ts \\
	\hline
	TU29 & Viene verificato che il metodo \emph{render()} del component \emph{DeleteAcademicYear} comporti la corretta renderizzazione del componente. & \Ti \\
	\hline
	TU30 & Viene verificato che il metodo \emph{render()} del component \emph{AcademicYears} comporti la corretta renderizzazione del componente. & \Ts \\
	\hline
	TU31 & Viene verificato che il metodo \emph{render()} del component \emph{ModifyDegreeCourse} comporti la corretta renderizzazione del componente. & \Ti \\
	\hline
	TU32 & Viene verificato che il metodo \emph{render()} del component \emph{InsertDegreeCourse} comporti la corretta renderizzazione del componente. & \Ts \\
	\hline
	TU33 & Viene verificato che il metodo \emph{render()} del component \emph{DeleteDegreeCourse} comporti la corretta renderizzazione del componente. & \Ti \\
	\hline
	TU34 & Viene verificato che il metodo \emph{render()} del component \emph{DegreeCourses} comporti la corretta renderizzazione del componente. & \Ts \\
	\hline
	TU35 & Viene verificato che il metodo \emph{render()} del component \emph{ModifyDidacticActivity} comporti la corretta renderizzazione del componente. & \Ti \\
	\hline
	TU36 & Viene verificato che il metodo \emph{render()} del component \emph{InsertDidacticActivity} comporti la corretta renderizzazione del componente. & \Ts \\
	\hline
	TU37 & Viene verificato che il metodo \emph{render()} del component \emph{DeleteDidacticActivity} comporti la corretta renderizzazione del componente. & \Ti \\
	\hline
	TU38 & Viene verificato che il metodo \emph{render()} del component \emph{DidacticActivities} comporti la corretta renderizzazione del componente. & \Ts \\
	\hline
	TU39 & Viene verificato che il metodo \emph{render()} del component \emph{InsertExam} comporti la corretta renderizzazione del componente. & \Ts \\
	\hline
	TU40 & Viene verificato che il metodo \emph{render()} del component \emph{DeleteProfessor} comporti la corretta renderizzazione del componente. & \Ti \\
	\hline
	TU41 & Viene verificato che il metodo \emph{render()} del component \emph{Professors} comporti la corretta renderizzazione del componente. & \Ts \\
	\hline
	TU42 & Viene verificato che il metodo \emph{render()} del component \emph{DeleteStudent} comporti la corretta renderizzazione del componente. & \Ts \\
	\hline
	TU43 & Viene verificato che il metodo \emph{render()} del component \emph{Students} comporti la corretta renderizzazione del componente. & \Ts \\
	\hline
	TU44 & Viene verificato che il metodo \emph{render()} del component \emph{ExamList} comporti la corretta renderizzazione del componente. & \Ts \\
	\hline
	TU45 & Viene verificato che il metodo \emph{render()} del component \emph{ExamPage} comporti la corretta renderizzazione del componente. & \Ts \\
	\hline
	TU46 & Viene verificato che il metodo \emph{render()} del component \emph{ExamsProfessorList} comporti la corretta renderizzazione del componente. & \Ts \\
	\hline
	TU47 & Viene verificato che il metodo \emph{render()} del component \emph{RegisteredStudentsList} comporti la corretta renderizzazione del componente. & \Ts \\
	\hline
	TU48 & Viene verificato che il metodo \emph{render()} del component \emph{ExamsStudentList} comporti la corretta renderizzazione del componente. & \Ts \\
	\hline
	TU49 & Viene verificato che il metodo \emph{render()} del component \emph{SchoolRecords} comporti la corretta renderizzazione del componente. & \Ts \\
	\hline
	TU50 & Viene verificato che il metodo \emph{render()} del component \emph{Profile} comporti la corretta renderizzazione del componente. & \Ts \\
	\hline
	TU51 & Viene verificato che il metodo \emph{render()} del component \emph{SubNavbar} comporti la corretta renderizzazione del componente. & \Ts \\
	\hline
	TU52 & Viene verificato che il metodo \emph{render()} del component \emph{SubNavButton} comporti la corretta renderizzazione del componente. & \Ts \\
	\hline
	TU53 & Viene verificato che il metodo \emph{render()} del component \emph{SignUp} comporti la corretta renderizzazione del componente. & \Ts \\
	\hline
	TU54 & Viene verificato che il metodo \emph{render()} del component \emph{SignUpForm} comporti la corretta renderizzazione del componente. & \Ts \\
	\hline
	TU55 & Viene verificato che il metodo \emph{ReadAcademicData()} legga correttamente le informazioni e faccia il dispatch dello stato dell'operazione. & \Ts \\
	\hline
	TU56 & Viene verificato che il metodo \emph{ReadDidacticActivities(int,int)} legga correttamente le informazioni e faccia il dispatch dello stato dell'operazione. & \Ts \\
	\hline
	TU57 & Viene verificato che il metodo \emph{ReadDegreeCourses(int)} legga correttamente le informazioni e faccia il dispatch dello stato dell'operazione. & \Ts \\
	\hline
	TU58 & Viene verificato che il metodo \emph{ReadExams(int,int,int)} legga correttamente le informazioni e faccia il dispatch dello stato dell'operazione. & \Ts \\
	\hline
	TU59 & Viene verificato che il metodo \emph{ReadProfessorList()} legga correttamente le informazioni e faccia il dispatch dello stato dell'operazione. & \Ts \\
	\hline
	TU60 & Viene verificato che il metodo \emph{ReadAdminList()} legga correttamente le informazioni e faccia il dispatch dello stato dell'operazione. & \Ts \\
	\hline
	TU61 & Viene verificato che il metodo \emph{ReadStudentsList()} legga correttamente le informazioni e faccia il dispatch dello stato dell'operazione. & \Ts \\
	\hline
	TU62 & Viene verificato che il metodo \emph{ReadExam(int,int,int)} legga correttamente le informazioni e faccia il dispatch dello stato dell'operazione. & \Ts \\
	\hline
	TU63 & Viene verificato che il metodo \emph{ReadExamList(professor)} legga correttamente le informazioni e faccia il dispatch dello stato dell'operazione. & \Ts \\
	\hline
	TU64 & Viene verificato che il metodo \emph{ReadExamList(student)} legga correttamente le informazioni e faccia il dispatch dello stato dell'operazione. & \Ts \\
	\hline
	TU65 & Viene verificato che il metodo \emph{ModifyAcademicData()} modifichi correttamente le informazioni e faccia il dispatch dello stato dell'operazione. & \Ts \\
	\hline
	TU66 & Viene verificato che il metodo \emph{ModifyDidacticActivity(int,int,int)} modifichi correttamente le informazioni e faccia il dispatch dello stato dell'operazione. & \Ts \\
	\hline
	TU67 & Viene verificato che il metodo \emph{ModifyDegreeCourse(int,int)} modifichi correttamente le informazioni e faccia il dispatch dell'operazione. & \Ts \\
	\hline
	TU68 & Viene verificato che il metodo \emph{ModifyExam(int,int,int,int)} modifichi correttamente le informazioni e faccia il dispatch dell'operazione. & \Ts \\
	\hline
	TU69 & Viene verificato che il metodo \emph{OnSignUpFormSubmit(string,string,string,string,int)} inserisca un utente nel sistema registrandone i dati nel sistema e faccia il dispatch dello stato dell'operazione. & \Ts \\
	\hline
	TU70 & Viene verificato che il metodo \emph{AddAcademicYear(int)} aggiunga correttamente le informazioni e faccia il dispatch dello stato dell'operazione. & \Ts \\
	\hline
	TU71 & Viene verificato che il metodo \emph{AddDidacticActivity(int,int,int)} aggiunga correttamente le informazioni e faccia il dispatch dello stato dell'operazione. & \Ts \\
	\hline
	TU72 & Viene verificato che il metodo \emph{AddDegreeCourse(int,int)} aggiunga correttamente le informazioni e faccia il dispatch dello stato dell'operazione. & \Ts \\
	\hline
	TU73 & Viene verificato che il metodo \emph{AddExam(int,int,int,int)} aggiunga correttamente le informazioni e faccia il dispatch dello stato dell'operazione. & \Ts \\
	\hline
	TU74 & Viene verificato che il metodo \emph{RemoveAcademicYear(int)} rimuova correttamente le informazioni e faccia il dispatch dello stato dell'operazione. & \Ts \\
	\hline
	TU75 & Viene verificato che il metodo \emph{RemoveDidacticActivity(int,int,int)} rimuova correttamente le informazioni e faccia il dispatch dello stato dell'operazione. & \Ts \\
	\hline
	TU76 & Viene verificato che il metodo \emph{RemoveDegreeCourse(int,int)} rimuova correttamente le informazioni e faccia il dispatch dello stato dell'operazione. & \Ts \\
	\hline
	TU77 & Viene verificato che il metodo \emph{RemoveExam(int,int,int,int)} rimuova correttamente le informazioni e faccia il dispatch dello stato dell'operazione. & \Ts \\
	\hline
	TU78 & Viene verificato che il metodo \emph{RemoveProfessor(admin)} rimuova correttamente le informazioni e faccia il dispatch dello stato dell'operazione. & \Ts \\
	\hline

	\caption[Test di sistema]{Test di sistema}
\end{longtable}

\subsubsection{Tracciamento Test di unità - Metodi}
\normalsize
\begin{longtable}{|>{\centering\arraybackslash}p{2cm}| p{15cm}|}
	\hline
	\textbf{Codice test} & \textbf{Classe:metodo} \\
	\hline
	\endhead
	TU1 & $InsertUserFormAction:OnInsertUserFormSubmit(string,string,int)$ \\
	\hline
	TU2 & $LogoutButtonAction:LogoutUser()$ \\
	\hline
	TU3 & $Admin:addUser(bytes32 \textunderscore fiscalCode, bytes10 \textunderscore uniCode, uint8 \textunderscore userType)$\newline$UserData:getUsersUniCode(bytes32 \textunderscore fiscalCode)$\newline$UserData:getUsersUserType(bytes32 \textunderscore fiscalCode)$
	\\
	\hline

	TU4 & $UserLogic:signUp(bytes32 \textunderscore fiscalCode, bytes10 \textunderscore uniCode, bytes32 \textunderscore hashData)$\newline$UserData:getRegUsersUniCode(address \textunderscore address) $\newline$UserData:getRegUsersFiscalCode(address \textunderscore address)$\newline$UserData:getRegUsersUserType(address \textunderscore address) $\newline$UserData:getRegUsersBadgeNumber(address \textunderscore address)$\newline$UserData:getRegUsersHashData(address \textunderscore address)$\\
	\hline
	TU5 & $Admin:addNewDegree(bytes10 \textunderscore degreeUniCode, bytes4 \textunderscore year, bytes32 \textunderscore hashData) $\newline$DegreeData:isDegree(bytes10 \textunderscore degreeUniCode) $\newline$DegreeData:getDegreeCourses(bytes10 \textunderscore degreeUniCode)$\\
	\hline
	TU6 &$Admin:addNewExam(bytes10 \textunderscore courseUniCode, bytes10 \textunderscore examUniCode, uint32 \textunderscore examTeacher,$\newline$ bytes32 \textunderscore examHashData) $\newline$ExamData:examExist(bytes10 \textunderscore examUniCode)$\newline$ExamData:getExamTeacher(bytes10 \textunderscore examUniCode) $\newline$ExamData:getAllIdentifiers()$\\
	\hline
	TU7 & $Teacher:myExams()$\\
	\hline
	TU8 &$ Teacher:registerResult(bytes10 \textunderscore examUniCode, uint32 \textunderscore studentBadgeNumber, bytes2 \textunderscore result)$\\
	\hline
	TU9 & $DeleteAdministrator:RemoveAdmin(admin)$\\
	\hline
	TU10 & $DeleteStudent:RemoveStudent(steudent)$\\
	\hline
	TU11 & $RegisteredStudentListAction:ViewStudentListAction(exam)$\\
	\hline
	TU12 & $LoginButtonAction:LoginUser()$\\
	\hline
	TU13 & $Help:render$\\
	\hline
	TU14 & $LoginButton:render$\\
	\hline
	TU15 & $LogoutButton:render$\\
	\hline
	TU16 & $NavButton:render$\\
	\hline
	TU17 & $Home:render$\\
	\hline
	TU18 & $App:render$\\
	\hline
	TU19 & $InsertUserForm:render$\\
	\hline
	TU20 & $InsertUser:render$\\
	\hline
	TU21 & $EmptyData:render$\\
	\hline
	TU22 & $DeleteAdministrator:render$\\
	\hline
	TU23 & $Administrators:render$\\
	\hline
	TU24 & $LoadingData:render$\\
	\hline
	TU25 & $LoadingUser:render$\\
	\hline
	TU26 & $NotFound:render$\\
	\hline
	TU27 & $ModifyAcademicYear:render$\\
	\hline
	TU28 & $InsertAcademicYear:render$\\
	\hline
	TU29 & $DeleteAcademicYear:render$\\
	\hline
	TU30 & $AcademicYears:render$\\
	\hline
	TU31 & $ModifyDegreeCourse:render$\\
	\hline
	TU32 & $InsertDegreeCourse:render$\\
	\hline
	TU33 & $DeleteDegreeCourse:render$\\
	\hline
	TU34 & $DegreeCourses:render$\\
	\hline
	TU35 & $ModifyDidacticActivity:render$\\
	\hline
	TU36 & $InsertDidacticActivity:render$\\
	\hline
	TU37 & $DeleteDidacticActivity:render$\\
	\hline
	TU38 & $DidacticActivities:render$\\
	\hline
	TU39 & $InsertExam:render$\\
	\hline
	TU40 & $DeleteProfessor:render$\\
	\hline
	TU41 & $Professors:render$\\
	\hline
	TU42 & $DeleteStudent:render$\\
	\hline
	TU43 & $Students:render$\\
	\hline
	TU44 & $ExamList:render$\\
	\hline
	TU45 & $ExamPage:render$\\
	\hline
	TU46 & $ExamProfessorList:render$\\
	\hline
	TU47 & $RegisteredStudentsList:render$\\
	\hline
	TU48 & $ExamsStudentList:render$\\
	\hline
	TU49 & $SchoolRecords:render$\\
	\hline
	TU50 & $Profile:render$\\
	\hline
	TU51 & $SubNavbar:render$\\
	\hline
	TU52 & $SubNavButton:render$\\
	\hline
	TU53 & $SignUp:render$\\
	\hline
	TU54 & $SignUpForm:render$\\
	\hline
	TU55 & $ViewAcademicYearAction:ReadAcademicData()$\\
	\hline
	TU56 & $ViewDidacticActivitiesActione:ReadDidacticActivities(int,int)$\\
	\hline
	TU57 & $ViewDegreeCoursesAction:ReadDegreeCourses(int)$\\
	\hline
	TU58 & $ViewExamsAction:ReadExams(int,int,int)$\\
	\hline
	TU59 & $ViewProfessor:ReadProfessorList()$\\
	\hline
	TU60 & $ViewAdministrators:ReadAdminList()$\\
	\hline
	TU61 & $ViewStudents:ReadStudentList()$\\
	\hline
	TU62 & $ExamPageAction:ReadExam(int,int,int)$\\
	\hline
	TU63 & $ExamProfessorListActions:ReadExamList(professor)$\\
	\hline
	TU64 & $ViewExamsAction:ReadExamList(student)$\\
	\hline
	TU65 & $ModifyAcademicYearsActions:ModifyAcademicData()$\\
	\hline
	TU66 & $ModifyDidacticActivityAction:ModifyDidacticActivity(int,int,int)$\\
	\hline
	TU67 & $ModifyDegreeCourseAction:ModifyDegreeCourse(int,int)$\\
	\hline
	TU68 & $ModifyExamAction:ModifyExam(int,int,int,int)$\\
	\hline
	TU69 & $InsertUserFormAction:OnSignUpFormSubmit(string,string,string,string,int)$\\
	\hline
	TU70 & $InsertAcademicYearAction:AddAcademicYear(int)$\\
	\hline
	TU71 & $InsertDidacticActivitiesAction:AddDidacticActivity(int,int,int)$\\
	\hline
	TU72 & $InsertDegreeCourseAction:AddDegreeCourse(int,int)$\\
	\hline
	TU73 & $InsertExamAction:AddExam(int,int,int,int)$\\
	\hline
	TU74 & $DeleteAcademicYearAction:RemoveAcademicYear(int)$\\
	\hline
	TU75 & $RemoveDidacticActivities:RemoveDidacticActivities(int,int,int)$\\
	\hline
	TU76 & $DeleteDegreeCoursesAction:removeDegreeCourse(int,int)$\\
	\hline
	TU77 & $DeleteExamAction:removeExam(int,int,int)$\\
	\hline
	TU78 & $DeleteProfessor:RemoveProfessor(admin)$\\
	\hline
\caption[Tracciamento test di unità - metodi]{Tracciamento test di unità - metodi}
\end{longtable}
\clearpage