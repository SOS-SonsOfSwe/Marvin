\section{Introduction}
This is the user manual of \textbf{Marvin}, a \DH app ran on the EVM, that shapes a subset of \href{www.uniweb.unipd.it}{Uniweb} functionalities. Uniweb is the University of Padua's informative system.
It allows students to keep track of their academic carrier. Professors use Uniweb to see the lists of students that are registered to their exams and, see the exams to which they have been assigned.

More precisely:
\begin{labeling}{alligator}
\item A student will be able to:
\begin{itemize}
\item Accept or refuse a mark;%Si può sostituire con grade
\item Visualize his/her school records;
\item Subscribe to an exam.
\end{itemize}
\item A professor will be able to:
\begin{itemize}
\item Visualize the exams to which he/her has been assigned;
\item See the students registered to his/her exams.
\end{itemize}
\item The university and the administrators will be able to add, eliminate and modify:
\begin{itemize}
\item Academic years;
\item Degree courses;
\item Didactic activities;
\item Exams;
\item Users.
\end{itemize}
\end{labeling}

To use \project{} you need to install \emph{\href{https://metamask.io/}{MetaMask}} to directly execute the \DH app on your browser.

All the users have to be logged through Metamask.