\section{Informazioni generali}
	\begin{itemize}
		\item \textbf{Data riunione}: 2018-03-05;
		\item \textbf{Ora inizio riunione}: 10:30;
		\item \textbf{Ora fine riunione}: 12:00;
		\item \textbf{Luogo di incontro}: 1BC50, Torre Archimede;
		\item \textbf{Oggetto di discussione}: Scelta del \emph{capitolato}\ped{G}, del nome del gruppo e degli strumenti da utilizzare per la comunicazione e per il versionamento;
		\item \textbf{Moderatore}: Giovanni Cavallin;
		\item \textbf{Segretario}: Eleonora Thiella;
		\item \textbf{Partecipanti}: Giovanni Cavallin, Eleonora Thiella, Giovanni Dalla Riva, Stefano Panozzo, Lorenzo Menegon, Andrea Favero e Federico Caldart;
	\end{itemize}

\section{Riassunto della riunione}
	\subsection{Descrizione}
	Durante il corso della riunione si è discusso dei capitolati disponibili ed ogni membro del gruppo ha espresso le proprie preferenze, pronunciando i pregi ed i difetti più significativi. Successivamente, dopo numerose proposte fantasiose, si è passati alla scelta del nome del gruppo. Si sono decisi infine gli strumenti più appropiati per il mantenimento di una comunicazione efficiente ed immediata tra i membri del \emph{team}\ped{G} e quelli per il versionamento e la condivisione di file. 
	
	\subsection{Decisioni prese}
		\begin{itemize}
			\item \textbf{VI1.1}: il capitolato che è stato scelto è il C6, vale a dire Marvin: dimostratore di Uniweb su Ethereum;
			\item \textbf{VI1.2}: dopo una lunga discussione, il nome scelto per il gruppo è stato SonsOfSwe;
			\item \textbf{VI1.3}: per le comunicazioni interne sono stati scelti i seguenti strumenti: Telegram, Trello e Slack. Per il versionamento, invece, si è optato per \emph{GitHub}\ped{G}, mentre per la condivisione di materiale \emph{Google Drive}\ped{G}; 
		\end{itemize}