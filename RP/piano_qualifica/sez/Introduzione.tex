\section{Introduzione}
\subsection{Scopo del documento}
Questo documento ha lo scopo di descrivere gli obiettivi di qualità, di processo e di prodotto da raggiungere nella realizzazione del progetto e le strategie di verifica e validazione adottate per il raggiungimento di tali obiettivi.
\subsection{Scopo del prodotto}
\begin{comment}
L'obiettivo di Marvin è di realizzare un \textit{prototipo}\ped{G} di Uniweb come \textit{ÐApp}\ped{G} che giri su \textit{Ethereum}\ped{G}. Gli attori principali che si rapportano con Marvin sono:
\begin{enumerate}
	\item Università;
	\item Professori;
	\item Studenti.
\end{enumerate}
Il portale deve quindi permettere agli studenti di accedere alle informazioni riguardanti le loro carriere universitarie, di iscriversi agli esami, di accettare o rifiutare voti e di poter vedere il loro libretto universitario.
Ai professori deve invece essere permesso registrare i voti degli studenti.
L'università ogni anno crea una serie di corsi di laurea rivolti a studenti, dove ognuno di essi comprende un elenco di esami disponibili per anno accademico. Ogni esame ha un argomento, un numero di crediti e un professore associato. Gli studenti si iscrivono ad un corso di laurea e tramite il libretto elettronico mantengono traccia ufficiale del progresso.
\end{comment}
Lo scopo del prodotto è quello di realizzare un \emph{prototipo}\ped{G} di Uniweb come una \emph{\DJ App}\ped{G} in esecuzione su \emph{Ethereum}\ped{G}. I cinque attori principali che si rapportano con Marvin sono:
\begin{itemize}
	\item Utente non autenticato; 
	\item Università;
	\item Amministratore;
	\item Professore;
	\item Studente.
\end{itemize} 
Il portale deve quindi permettere agli studenti di accedere alle informazioni riguardanti le loro carriere universitarie, di iscriversi agli esami, di accettare o rifiutare voti e di poter vedere il loro libretto universitario.
Ai professori deve invece essere permesso di registrare i voti degli studenti.
L'università ogni anno crea una serie di corsi di laurea rivolti a studenti, dove ognuno di essi comprende un elenco di esami disponibili per anno accademico. Ogni esame ha un argomento, un numero di crediti e un professore associato. Gli studenti si iscrivono ad un corso di laurea e tramite il libretto elettronico mantengono traccia ufficiale del progresso.
\subsection{Glossario}
Nel documento \G{} i termini tecnici, gli acronimi e le abbreviazioni sono definiti in modo chiaro e conciso, in modo tale da evitare ambiguità e massimizzare la comprensione dei documenti.
\newline I vocaboli presenti in esso saranno posti in corsivo e presenteranno una "G" maiuscola a pedice.
\subsection{Riferimenti}
\subsubsection{Normativi}
\begin{itemize}
	\item
	\textbf{Capitolato d’appalto C6 - Marvin: dimostratore di Uniweb su Ethereum}
	\url{http://www.math.unipd.it/~tullio/IS-1/2017/Progetto/C6.pdf};
	
	\item
	\textbf{\NdP{}};
	 
	\item
	\textbf{ISO/IEC 15504 - SPiCE, Plays-In-Business}
	\url{http://www.plays-in-business.com/isoiec-15504-spice/};
	
	\item
	\textbf{Software Process Improvement and Capability Determination (SPICE) - ISO/IEC 15504}
	\url{https://shahanali.wordpress.com/2011/04/25/software-process-improvement-and-capability-determination-spice-isoiec-15504/};	
	\item
	\textbf{La qualità del software secondo il modello ISO/IEC 9126, Ercole F. Colonese}
	\url{http://www.colonese.it/00-Manuali_Pubblicatii/07-ISO-IEC9126_v2.pdf};
	\item
	\textbf{Lo Standard ISO/IEC 9126 – Software engineering – Product Quality, Anna Rita Fasolino}
	\url{http://www.federica.unina.it/ingegneria/ingegneria-software-ii/isoiec-9126-software-engineering-quality/}.
\end{itemize}

\subsubsection{Informativi}
\begin{itemize}
	\item
	\textbf{\PdP{}};
	\item
	\textbf{Slides del corso di Ingegneria del software - Qualità del prodotto software}
	\url{http://www.math.unipd.it/~tullio/IS-1/2017/Dispense/L13.pdf};
	\item
	\textbf{Slides del corso di Ingegneria del software - Qualità di processo}
	\url{http://www.math.unipd.it/~tullio/IS-1/2017/Dispense/L15.pdf}
	\url{http://www.math.unipd.it/~tullio/IS-1/2017/};
	\item 
	\textbf{Ingegneria del software, decima edizione - Ian Sommerville, capitolo 21};
	\item
	\textbf{SLOC}
	\url{https://it.wikipedia.org/wiki/Source_lines_of_code};
	\item
	\textbf{Ciclo di Deming}
	\url{https://it.wikipedia.org/wiki/Ciclo_di_Deming} (aggiornato al 2018/03/31);
	\item
	\textbf{Deming Cycle: The Wheel of Continuous Improvement}
	\url{https://totalqualitymanagement.wordpress.com/2009/02/25/deming-cycle-the-wheel-of-continuous-improvement/};
	\item
	\textbf{Indice Gulpease}
	\url{https://it.wikipedia.org/wiki/Indice_Gulpease} (aggiornato al 2018/03/31);
	\item
	\textbf{Complessità ciclomatica}
	\url{https://it.wikipedia.org/wiki/Complessità_ciclomatica} (aggiornato al 2018/03/31).
	
\end{itemize}

\subsection{Premessa}
Questo documento sarà soggetto a continue aggiunte e modifiche per tutta la durata del progetto ed è da considerarsi \textit{incrementale}\ped{G} data la natura del suo contenuto. In particolare le sezioni riguardanti la specifica dei test ed il resoconto delle sottoattività di verifica saranno aggiornate in base ai risultati ottenuti nel proseguire del tempo.
