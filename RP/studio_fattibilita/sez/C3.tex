\section{Capitolato C3}
\subsection{Informazioni sul Capitolato}
	\begin{itemize}
		\item \textbf{Nome:}
		DeSpeect;
		\item \textbf{Proponente}:
		Mivoq;
		\item \textbf{Committenti}:
		Prof. Tullio Vardanega, Prof. Riccardo Cardin.
	\end{itemize}

\subsection{Descrizione}
	L'obiettivo di tale capitolato è quello di realizzare un'interfaccia grafica per Speect, una libreria \textit{open-source}\ped{G} per la creazione di sistemi di sintesi vocale che agevoli l’ispezione del suo stato interno durante il funzionamento e la scrittura di test per le sue funzionalità.
	\newline \newline Tale libreria è funzionale per lo sviluppo di front-end e back-end di un \textit{TTS}\ped{G}, una tecnologia che riceve un testo in ingresso e restituisce in uscita un file vocale. Questo sistema di sintesi viene solitamente progettato in due blocchi separati:
	\begin{itemize}
		\item \textbf{Front-end:}
		effettua l'analisi linguistica del testo in ingresso ed estrae da essa una sequenza fonetica dettagliata;
		\item \textbf{Back-end:}
		converte in una forma d'onda la sequenza fonetica, che rappresenta l'intenzione di pronunciare determinati suoni.
	\end{itemize}

\subsection{Dominio Applicativo}
	 Questo tipo di tecnologia si è diffuso rapidamente negli ultimi tempi, infatti la si può trovare per esempio nelle voci guida dei navigatori satellitari, negli annunci dei mezzi di trasporto pubblico, nei centralini telefonici e nei lettori di messaggi.
	 \newline \newline Il contesto in cui opera riguarda perciò tutte quelle \textit{applicazioni}\ped{G} in cui la vista dell'utente per cause di forza maggiore o per limitazioni temporanee (per esempio durante la guida) è privata o impedita.
	 \newline \newline Il prodotto finale dovrà essere concepito come una sorta di debugger per Speect.

\subsection{Dominio Tecnologico}
	Nonostante venga incoraggiato lo sviluppo multipiattaforma, l'applicazione deve essere compatibile con \textit{Linux}\ped{G} e avere licenza open-source. Un requisito fondametale è l'utilizzo di Speect e della versione modificata dal proponente [Mivoq(2014-2017)].
	\newline \newline Per lo sviluppo dell'interfaccia utente il capitolato suggerisce l'utilizzo di:
	\begin{itemize}
		\item Librerie portabili come \textit{\textbf{Gtk}}\ped{G}+ o \textit{\textbf{Qt}}\ped{G};
		\item Programmi come \textit{\textbf{Glade}}\ped{G} o \textit{\textbf{QtCreator}}\ped{G}.
	\end{itemize}
	mentre per quanto riguarda l'automazione della compilazione consiglia \textit{\textbf{CMake}}\ped{G}.
	
\subsection{Aspetti Positivi}
	Gli aspetti positivi salienti riscontrati sono:
	\begin{itemize}
		\item Interesse della maggior parte dei componenti del gruppo nei sistemi di sintesi vocale da testo scritto;
		\item Familiarità con le interfacce grafiche;
		\item Confidenza con QtCreator.
	\end{itemize}
	
\subsection{Potenziali Criticità}
	Le principali criticità constatate sono:
	\begin{itemize}
		\item Attrazione per tale capitolato non condivisa da tutti i componenti del gruppo;
		\item Mancanza di incoraggiamento allo studio e all'apprendimento di nuove tecnologie.
	\end{itemize}

\subsection{Valutazione Finale}
	Nonostante l'interesse da parte della maggioranza dei componenti nei sistemi di sintesi vocale e la familiarità con le tecnologie richieste per lo sviluppo, si è scelto di scartare tale capitolato proprio perchè non fornisce alcuno stimolo per il gruppo all'acquisizione di nuove tecnologie, risultando poco formativo.