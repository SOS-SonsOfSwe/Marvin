%\begin{comment}
\newpage
\section{Processi di Supporto}
\subsection{Processo di documentazione}
Durante lo svolgimento del capitolato si dovrà rendere conto, tramite una documentazione dettagliata, di tutti i processi che saranno coinvolti. 
Per questo motivo, il team suddividerà i documenti in:
\begin{itemize}
	\item \textbf{Documenti interni}
	\newline Tutti quei documenti che saranno visionati da fornitori e committenti;
	\item \textbf{Documenti esterni}
	\newline Tutti quei documenti che saranno visionati anche dai proponenti.
	
\end{itemize}
%	\subsubsection{Processo di garanzia della qualità}
%	\textcolor{red}{Cosa scriviamo qui? Nell'ISO 1995 è al paragrafo 6.3, da verificare nell'ISO aggiornato.}
	
	\subsubsection{Ciclo di vita di un documento}
	Un documento passerà attraverso tre stati:
	\begin{itemize}
		\item \textbf{In lavorazione}:
		si tratta della fase di stesura del documento e non è consultabile;
		\item \textbf{Da verificare}:
		dopo che il documento è stato ultimato, passerà nelle mani del \ver{} (il cui ruolo è descritto nel paragrafo \ref{verificatore}), che dovrà esaminarlo;
		\item \textbf{Approvato}:
		dopo la verifica, il documento dovrà essere approvato definitivamente dal \RdP.
	\end{itemize}
	Ogni documento sarà identificato con un flag alla fine del nome, distanziato con un underscore, in base allo stato in cui si trova. 
	Per il primo si userà \emph{\textunderscore{L}}, per il secondo \emph{\textunderscore{V}}, per il terzo \emph{\textunderscore{A}}.
	
	\subsubsection{Documenti finali ad uso interno}
		\paragraph{Studio di fattibilità (SdF)}
		~\\Lo \SdF{} ha lo scopo di raccogliere le informazioni salienti dei capitolati proposti, esprimendone gli aspetti positivi e 
		le potenziali criticità che sono emerse durante il confronto col gruppo.
		\paragraph{Norme di progetto (NdP)}
		~\\Le \NdP{} contengono le regole che il team utilizzerà durante lo sviluppo del progetto.
		\paragraph{Verbale interno (VI)}
		~\\Il \VI{} servirà al gruppo per documentare le discussioni e le decisioni prese durante le riunioni. 
		La denominazione dovrà essere come segue:\\
		\begin{center}
			\textit{verbale\textunderscore{Tipo del verbale}\textunderscore{Numero del verbale}\textunderscore{Data del verbale}} 
		\end{center}
		dove:
		\begin{itemize}
			\item \textbf{Tipo del verbale}:
			specifica se Interno (I) o Esterno (E);
			\item \textbf{Numero del verbale}:
			numero univoco identificativo del verbale;
			\item \textbf{Data del verbale}:
			identifica la data in cui la riunione si è svolta. Si utilizzerà il formato:
			\begin{center}
				\emph{YYYY-MM-DD}
			\end{center}
		\end{itemize}
		Nella parte introduttiva del verbale verranno specificati:
		\begin{itemize}
			\item Data riunione;
			\item Ora inizio riunione;
			\item Ora fine riunione;
			\item Durata riunione;
			\item Luogo d'incontro;
			\item Oggetto di discussione;
			\item Moderatore;
			\item Segretario;
			\item Partecipanti.
		\end{itemize}

	\subsubsection{Documenti finali ad uso esterno}
	\paragraph{Piano di Progetto (PdP)}
	~\\Il \PdP{} contiene le indicazioni sulle scadenze temporali e fornisce un preventivo dei costi da presentare al proponente. 
	~\\Vengono inoltre individuati i rischi ed analizzate le loro ricorrenze.
	~\\In questo documento vengono fatte emergere le \emph{milestone\ped{G}} legate ai punti critici e viene effettuata una \emph{pianificazione}\ped{G} con l'uso di \emph{diagrammi di Gantt}\ped{G}.
	~\\In questo documento vengono fatte emergere le \emph{milestone\ped{G}} legate ai punti critici e viene effettuata una \emph{pianificazione}\ped{G} con l'uso di diagrammi di \emph{Gantt}\ped{G}.
	\paragraph{Piano di Qualifica (PdQ)} 
	~\\Il \PdQ{} deve fornire ai membri del gruppo tutte le informazioni con cui poter soddisfare gli obiettivi di qualità.
	\paragraph{Analisi dei Requisiti (AdR)}
	~\\L'\AdR{} descrive gli \emph{attori}\ped{G} del sistema, individua i \emph{casi d'uso}\ped{G} a partire dai \emph{requisiti}\ped{G} e fornisce una visione chiara ai \progs{} sul problema da trattare.
	
	\begin{comment}
	\paragraph{Specifica Tecnica (ST)}
	~\\La \ST{} si occupa di dare una descrizione ad alto livello del prodotto, descrivendo pregi e difetti delle sue tecnologie. \textcolor{red}{Da ampliare quando avremo fatto il documento}
	\paragraph{Definizione di Prodotto (DdP)}
	~\\La \DdP{} descrive i dettagli implementativi del prodotto, andando a definire anche le funzioni delle componenti terminali del sistema tramite diagrammi \emph{UML}.
	\end{comment}
	
	\paragraph{Glossario (G)}
	~\\Nel documento \G{} i termini tecnici, gli acronimi e le abbreviazioni sono definiti in modo chiaro e conciso, in modo tale da evitare ambiguità e massimizzare la comprensione dei documenti.
	\paragraph{Manuale Utente (MU)}
	~\\Il \MU{} è un manuale pensato per aiutare l'utente ad utilizzare il prodotto, viene incrementato durante lo sviluppo di quest'ultimo. Deve avere un approccio incentrato sulle funzionalità che il prodotto offre.
	\paragraph{Manuale Sviluppatore (MS)}
	~\\Il \MS{} è un manuale per aiutare lo sviluppatore nella manutenzione e nell'incremento delle funzionalità del prodotto.
	\paragraph{Verbale Esterno (VE)}
	~\\Il \VE{} è un documento in cui si tiene traccia delle discussioni del team con i committenti ed i proponenti. Come struttura ricalca quella del \VI.
	
	\subsubsection{Struttura del documento}
	\paragraph{Prima pagina}
	La prima pagina di ogni documento sarà così strutturata:
	\begin{itemize}
		\item Logo;
		\item Nome del documento;
		\item Nome del gruppo \emph{-} Nome del progetto;
		\item Email del gruppo;
		\item Informazioni sul documento:
		\begin{itemize}
			\item Versione del documento;
			\item Redazione;
			\item \emph{Verifica}\ped{G};
			\item Approvazione;
			\item \emph{Uso}\ped{G};
			\item Distribuzione.
		\end{itemize}
		\item Descrizione del documento.
	\end{itemize}
	\paragraph{Registro delle modifiche}
	~\\In seconda pagina il documento conterrà il registro delle modifiche che traccerà le modifiche apportate al documento. Sarà organizzato in una tabella che conterrà le seguenti colonne:
	\begin{itemize}
		\item Versione:
		indica la versione del documento;
		\item Data:
		indica la data in cui il documento è stato modificato;
		\item Descrizione:
		descrive la modifica effettuata nella relativa versione;
		\item Autore:
		indica il nome della persona che ha effettuato la modifica;
		\item Ruolo:
		indica il ruolo dell'autore.
	\end{itemize}
	\paragraph{Indice}
	~\\Dopo il registro delle modifiche, il documento sarà correlato da un indice di tutte le sezioni. In alcuni documenti, se necessario, sarà aggiunto anche l'indice delle immagini, delle tabelle e dei riferimenti.
	\paragraph{Formattazione generale della pagina}
	~\\Ogni pagina del documento, fatta eccezione per la prima, conterrà una intestazione ed un pié di pagina.
	\\L'intestazione presenterà a sinistra il logo e a destra l'email del gruppo.
	\\Nel pié di pagina saranno presenti a sinistra il nome del documento susseguito dal nome del gruppo e, a destra il numero della pagina.
	%\begin{comment}
	\subsubsection{Norme tipografiche}
	Tutti i documenti dovranno sottostare alle seguenti norme tipografiche ed ortografiche.
	\paragraph{Formati}
	\begin{itemize}
	\item \textbf{Data}: il formato della data seguirà quello esplicato nell'\emph{ISO}\ped{G} \emph{8601:2004}\ped{G}, quindi sarà: 
		\begin{center}
			\emph{YYYY-MM-DD}
		\end{center}
	dove i simboli stanno per:
		\begin{itemize}
			\item YYYY: anno;
			\item MM: mese;
			\item DD: giorno.
		\end{itemize}
	\item \textbf{Orario}: ci si atterrà allo standard europeo delle 24 ore:
		 \begin{center}
		 	\emph{hh:mm}
		 \end{center}
	dove i simboli stanno per:
	 	\begin{itemize}
	 		\item hh: ore;
	 		\item mm: minuti.
	 	\end{itemize}
 	\begin{comment}
		\item \textbf{Nome del documento}: \textcolor{red}{Dovremmo elaborare un comando che, una volta invocato, genera automaticamente il nome del documento e la sua versione}
		\item \textbf{Nome del gruppo}: per riferirsi al nome del gruppo si dovrà
		utilizzare il comando garantendo in questo modo la corretta sintassi;
		\item \textbf{Nome del progetto}: per iferirsi al nome del progetto si dovrà
		utilizzare il comando garantendo in questo modo la corretta sintassi;
		\item \textbf{Link sito del gruppo}: per riferirsi al link del sito del gruppo si dovrà
		utilizzare il comando garantendo in questo modo la corretta sintassi;
		\item \textbf{Email del gruppo}: per riferirsi all'indirizzo email del gruppo si dovrà
		utilizzare il comando garantendo in questo modo la corretta sintassi;
		\item \textbf{Nome del proponente}: per riferirsi al nome del proponente, ovvero del proponente, si dovrà
		utilizzare il comando garantendo in questo modo la corretta
		sintassi.
	\end{comment}
	
	\end{itemize}
	\paragraph{Composizione del testo}
	\begin{itemize}
		\item \textbf{Elenchi puntati}: ogni punto dell'elenco deve terminare con ''\emph{;}''
		tranne l'ultimo che deve terminare con  ''\emph{.}''. La prima parola deve iniziare con unla lettera maiuscola;
		\item \textbf{Glossario}: il pedice ''\ped{G}'' verrà utilizzato in corrispondenza di vocaboli presenti nel \textit{\G}.
	\end{itemize}
	\paragraph{Stili di testo}
	\begin{itemize}
		\item \textbf{Grassetto}: il grassetto deve essere utilizzato per evidenziare parole
		particolarmente importanti negli elenchi puntati o nelle frasi;
		\item \textbf{Corsivo}: il corsivo deve essere utilizzato con i seguenti termini:
		situazioni:
		\begin{itemize}
			\item Ruoli;
			\item Documenti;
			\item Stati del documento;
			\item Citazioni;
			\item Glossario;
			\item Nomi di file.
		\end{itemize}
		\item \textbf{Maiuscolo}: il maiuscolo deve essere utilizzato solamente per gli acronimi.
	\end{itemize}
	\paragraph{Sintassi}
	~\\Per la stesura dei documenti i membri del team adotteranno la terza persona singolare.
	\paragraph{Sigle}
	\begin{itemize}
		\item \textbf{AdR}: Analisi dei Requisiti;
		\item \textbf{PdP}: Piano di Progetto;
		\item \textbf{NdP}: Norme di Progetto;
		\item \textbf{SdF}: Studio di Fattibilità;
		\item \textbf{PdQ}: Piano di Qualifica;
		\item \textbf{LdP}: Lettera di Presentazione;
		\item \textbf{G}: Glossario;
		\item \textbf{TB}: Technology Baseline;
		\item \textbf{PB}: Product Baseline;
		\item \textbf{MU}: Manuale Utente;
		\item \textbf{MS}: Manuale Sviluppatore;
		\item \textbf{RR}: Revisione dei Requisiti;
		\item \textbf{RP}: Revisione di Progettazione;
		\item \textbf{RQ}: Revisione di Qualifica;
		\item \textbf{RA}: Revisione di Accettazione.
	\end{itemize}

\begin{comment}
	\subsubsection{Componenti grafiche}
	\textcolor{red}{Da controllare, quando inizieremo ad usarle}
	\begin{itemize}
		\item Tabelle
		\item Immagini
	\end{itemize}
\end{comment}
	\subsubsection{Nome del file \emph{.pdf}\ped{G}}
	Ogni documento sarà generato come file con estensione .pdf ed avrà un nome che rispetti la seguente convenzione:
	\begin{center}
		\emph{NomeFile\_versione.pdf}
	\end{center}
	Il \emph{NomeFile} seguirà la notazione CamelCase già descritta, invece la \emph{versione} seguirà lo standard descritto qui di seguito.
	
	\subsubsection{Struttura dei file in \LaTeX}
	Lo scheletro dei file necessari per generare un file .pdf di un documento è così strutturato:
	\begin{itemize}
		\item \emph{sos.sty}: contiene tutti i package e le macro che possono trovare utilizzo in tutti i documenti. Contiene  anche anche la macro che genera la copertina della prima pagina e che viene invocata dal file \emph{main.tex};
		\item \emph{main.tex}: Include il template e tutte le sezioni che compongono il documento. Il file non è monolitico perché include le sezioni (che quindi sono dei file \emph{.tex} che hanno vita propria), in questo modo si cerca di puntare all'assenza di conflitti durante modifiche contemporanee a più parti dello stesso documento e, di facilitare la revisione di singole parti del documento;
		\item \emph{comandi.tex}: contiene delle macro specifiche per il documento che si sta creando;
		\item \emph{"img" folder}: contiene il logo del gruppo e le altre immagini che sono incluse nel documento.
	\end{itemize}
	\subsubsection{Versionamento}
	Il \emph{versionamento\ped{G}} permette a ciascun membro del team di condividere il lavoro nello spazio comune e di lavorare su vecchi e nuovi \emph{Configuration Item}\ped{G} senza rischio di sovrascritture accidentali.
	\\Avrà questa forma:
	\begin{center}
		\emph{vX.Y.Z}
	\end{center}
	dove:
	\begin{itemize}
		\item \textbf{X}:
		\begin{itemize}
			\item Inizia da 0;
			\item Viene incrementato dal \RdP{} una volta approvato il documento.
		\end{itemize}
		\item \textbf{Y}:
		\begin{itemize}
			\item Inizia da 0;
			\item Viene incrementato dal \ver{} una volta verificato il documento;
			\item Quando viene incrementato, \emph{Z} viene riportato a 0.
		\end{itemize}
		\item \textbf{Z}:
		\begin{itemize}
			\item Alla creazione parte da 1;
			\item Viene incrementato dal redattore del documento ogni volta che questo è viene modificato.
			\item Quando vengono incrementati X o Y viene riportato a 0.
		\end{itemize}
	\end{itemize}
	\subsubsection{Strumenti}
			\paragraph{\LaTeX}
			~\\Il team ha scelto di usufruire del linguaggio \LaTeX{} per questi motivi:
			\begin{itemize}
				%\item Permette di rendere ogni sezione molto autonoma, mantenendo comunque lo stile coerente;
				\item Permette un versionamento più semplice e compatibile con \emph{Git}\ped{G};
				\item Evita i conflitti che si possono creare usando software differenti (per esempio OpenOffice e Microsoft Word).
			\end{itemize}
		
		%	\paragraph{\textcolor{red}{Da aggiungere qui eventuali altri strumenti che utilizzeremo per altro}}
	
		\subsubsection{Gestione del \emph{repository}\ped{G}}
		Per tenere traccia di tutte le modifiche apportate ai documenti ed al codice che sono stati prodotti, il team ha deciso di utilizzare \emph{GitHub}, creando una \emph{repository}\ped{G} dedicata. Si seguiranno le norme descritte qui di seguito.
			\paragraph{Struttura}
			\begin{itemize}
				\item Il \emph{master}\ped{G} è annidato nella repository \emph{origin}\ped{G} e deve contenere solo il \emph{template}\ped{G} per la scrittura di tutti i documenti;
				\item Ogni \emph{branch}\ped{G} annidato sul master deve corrispondere solo alle \emph{revisioni}\ped{G};
				\item Ogni branch annidato in branch relativi alle revisioni contiene dei documenti che le riguardano;
				\item Ogni membro del gruppo potrà modificare i file di ogni documento o codice, o direttamente sul branch corrispondente oppure creandone uno parallelo a sua discrezione a meno di direttive diverse da parte dei \progs{}.
%				\item \textcolor{red}{Capire come fare la repo quando si dovrà mettere dentro il codice}
			\end{itemize}
			\paragraph{Tipi di file}
			Tutti i file pertinenti verranno caricati nel repository e saranno sottoposti a versionamento. Verranno tuttavia ignorati, quindi inseriti all'interno del file \emph{.gitignore}\ped{G}, tutti quelle quei file generati automaticamente durante le varie \emph{build}\ped{G} e che hanno estensioni indesiderate (ad esempio \emph{.log}, \emph{ .out}). 
			\paragraph{Norme sui \emph{commit}\ped{G}}
			~\\I membri del team dovranno registrare le modifiche apportate alla repository tramite commit. Questo genererà una \emph{pull request}\ped{G} che il responsabile dovrà approvare prima di unire le modifiche al branch di riferimento. %Qualora non si riuscisse a concludere un'attività entro la giornata, si dovrà eseguire un commit alla fine di essa, per garantire una copia offline.
\subsection{Processo di verifica}
Il processo di verifica deve essere continuo, questo serve ad evitare che eventuali errori arrivino fino alla fase di validazione.
Questa attività, che viene svolta in corso d'opera, deve essere:
\begin{itemize}
	\item Tempestiva, cioè il dato deve esserci quando serve;
	\item Accurata, devono essere evitate scorrettezze;
	\item Non intrusiva, non deve interrompere alcuna attività durante la sua esecuzione.
\end{itemize}
Le modalità sfruttate saranno elencate qui di seguito.
%\begin{comment}
	\subsubsection{Analisi}
	\paragraph{Analisi statica}
	~\\Studia il codice sorgente e la documentazione e controlla che essi seguano le norme. Non richiede l'esecuzione del prodotto software in nessuna sua parte, ma si limita all'osservazione. Questo processo può essere visto come una verifica dinamica del comportamento del programma su un insieme finito di casi selezionati nel dominio di tutte le esecuzioni possibili. Ciascun caso di prova, specifica i valori in ingresso e lo stato iniziale del sistema e deve produrre un esito decidibile, verificato rispetto ad un comportamento atteso. 
	\\Si può declinare in due maniere:
	\begin{itemize}
		\item \textbf{\emph{Walkthrough}}\ped{G}
		~\\Questa tecnica di analisi deve essere effettuata nella fase iniziale dello sviluppo del prodotto per trovare eventuali errori che possano essere riscontrati. Non sapendo che tipo di errori cercare, si effettua una lettura a largo spettro. Quando i \vers{} avranno stilato una \emph{checklist}\ped{G} di tutti gli errori più comuni, si passerà alla fase di \emph{Inspection}\ped{G}. Essendo un'analisi continua, la checklist tenderà ad aumentare costantemente di dimensione, rendendo più efficace ed efficiente l'Inspection.
		\item \textbf{Inspection}
		~\\Basandosi sulla checklist redatta durante la Walkthrough, i \vers{} analizzeranno tutto il prodotto cercando, di volta in volta e in modo mirato, tutti gli errori ricorrenti in essa contenuti.
	\end{itemize}
	\paragraph{Analisi dinamica}
	~\\L'analisi dinamica, al contrario di quella statica, richiede l'esecuzione del codice.
	Viene effettuata tramite test ed è coinvolta sia nel processo di verifica che in quello di validazione.
	I test verranno descritti in un capitolo a parte.
	%\begin{comment}
	\subsubsection{Test}
	Il testing è una parte essenziale del processo di verifica: produce una misura della qualità del sistema
	aumentandone il valore, identificandone e rimuovendone i difetti. Il suo inizio non va differito
	al termine delle attività di codifica e le sue esigenze devono essere tenute in conto nella progettazione del sistema. Tutti i test devono essere rieseguibili e devono sempre produrre lo stesso esito. Devono essere eseguiti in condizioni controllate, diventando così deterministici.
	~\\I test si possono dividere nelle categorie qui descritte.
		\paragraph{Test di unità (TU)}
		~\\Il loro obiettivo è quello di verificare la correttezza del codice \emph{as implemented}, ovvero
		puntando a controllare il codice in maniera microscopica.
		\\La responsabilità della loro realizzazione è del \progr{} per le unità più semplici, e di un \ver{} indipendente per le altre.
		\\Le risorse consumate da questo tipo di test sono molto poche perché,  sono coinvolte piccole parti di codice che hanno basso accoppiamento le une con le altre. 
		\paragraph{Test di integrazione (TI)}
		~\\Questi test verificano non solo il corretto comportamento di ogni singolo oggetto, 
		ma anche le relazioni con gli altri componenti dell'applicazione.
		~\\Possono rilevare i seguenti problemi:
		\begin{itemize}
			\item Errori residui nella realizzazione dei componenti;
			\item Modifica delle interfacce o cambiamenti nei requisiti;
			\item Riuso di componenti dal comportamento oscuro o inadatto;
			\item Integrazione con altre applicazioni non bene conosciute.
		\end{itemize}
		Un test di questo tipo consuma molte risorse dato che, la grandezza del codice da controllare è 
		significativa ed il grado di accoppiamento tra le varie parti da testare è massimo. 
		L'utilizzo di un test di questo tipo deve quindi essere ponderato e, in generale, da non preferire a quello di unità.
		\paragraph{Test di sistema (TS)}
		~\\I test di sistema possono essere visti come un'attività interna del fornitore
		per accertare la copertura dei requisiti software.
		\newline{}Questo tipo di test richiede molte risorse e, in generale, non va utilizzato per testare
		unità singole.
		\paragraph{Test di regressione (TR)}
		~\\I test di regressione sono l'insieme dei TU e TI necessari ad accertare che la modifica
		di una parte del prodotto non causi errori nelle altre parti che hanno relazioni con essa.
		\\Un test di questo tipo consuma tante più risorse quanto la parte modificata è accoppiata con le altre, 
		anche perché comporta la ripetizione di test già previsti ed effettuati per ogni parte che non è 
		stata modificata.
		\paragraph{Test di validazione}
		Il test di validazione è un'attività supervisionata dal committente e dal proponente come dimostrazione di conformità del prodotto sulla base 
		di casi di prova specificati o implicati dal contratto.
		Alla validazione segue il rilascio del prodotto con eventuale garanzia e la fine della
		\emph{commessa}\ped{G},	con eventuale manutenzione.
	
	\subsubsection{Strumenti}
	\paragraph{Strumenti per l'analisi statica}
	\begin{itemize}
		\item \emph{\textbf{\emph{TexStudio}}}\ped{G}\footnote{\href{https://www.texstudio.org/}{https://www.texstudio.org/}}
		~\\Questo editor include un correttore ortografico automatico che sarà utilizzato per verificare la corretta sintassi dei manuali in lingua inglese da fornire alla \proponente. Tuttavia tale strumento non è preciso nell'individuazione degli errori più sottili, quindi sarà comunque necessaria un'analisi più approfondita da parte dei \vers;
		\item \textbf{Indice \emph{Gulpease}}\ped{G}
		~\\Per i test di leggibilità il team ricorrà al calcolo dell'indice Gulpease;
		\begin{comment}
		\item 
		\emph{\textbf{\emph{JSHint}}}\ped{G}\footnote{\href{http://jshint.com/}{http://jshint.com/}}
		~\\È uno strumento \emph{OpenSource}\ped{G} funzionale alla rilevazione degli errori e possibili problemi nel codice \emph{JavaScript}\ped{G};
		\item 
		\emph{\textbf{\emph{W3C Markup Validation Service}}}\ped{G}\footnote{\href{https://validator.w3.org/}{https://validator.w3.org/}}
		~\\È uno strumento per la valid
		\item 
		\emph{\textbf{\emph{Draw.io}}}\ped{G}\footnote{\href{https://www.draw.io/}{https://www.draw.io/}}
		\\È un software per la creazione di diagrammi di flusso, di processo, organigrammi, UML, ER e diagrammi di rete.
		azione dei documenti \emph{HTML}\ped{G} e \emph{xHTML}\ped{G};
		\end{comment}
	\end{itemize}
\begin{comment}
	\paragraph{Strumenti per l'analisi dinamica}
	\begin{itemize}
		\item 
		\emph{\textbf{\emph{\textcolor{red}{Karma}}}}\ped{G}\footnote{\href{https://karma-runner.github.io/2.0/index.html}{https://karma-runner.github.io/2.0/index.html}}
		~\\È uno strumento per effettuare test di unità sugli script realizzati, installabile come modulo per \emph{Node.js}\ped{G};
		\item 
		\emph{\textbf{\emph{\textcolor{red}{Mocha}}}}\ped{G}\footnote{\href{https://mochajs.org/}{https://mochajs.org/}}
		~\\È un framework per l'esecuzione dei test asincroni e in serie di Javascript, scritto Node.js.
	\end{itemize}
\end{comment}
