\section{Obiettivi di qualità}
Questa sezione ha l'obiettivo di definire le caratteristiche riguardanti la qualità di prodotto e di processo che dovranno essere perseguite durante lo sviluppo del progetto.
Ogni caratteristica viene valutata da una metrica, una soglia di accettabilità e una possibile soglia di miglioramento che il \emph{team}\ped{G} si prefigge di raggiungere e possibilmente superare; descrizione e formula per il calcolo di tali metriche sono disponibile nel documento \NdP.

\subsection{Qualità di processo}
La qualità di processo influenza direttamente il prodotto finale realizzato. È necessario quindi sviluppare un processo in grado di produrre ciclicamente un prodotto di alta qualità. Per questo motivo si è deciso di stabilire le seguenti caratteristiche da rispettare per tutto lo sviluppo del progetto, contemporaneamente a:
	\begin{itemize}
	\item L'applicazione del \emph{Ciclo di Deming}\ped{G}, o \emph{PDCA}, al fine di perseguire il miglioramento continuo delle attività di processo;
	\item L'adesione allo standard ISO/IEC 15504, denominato \emph{SPICE}\ped{G}, al fine di applicare una valutazione oggettiva sulla maturità dei processi.
	\end{itemize} 

Inoltre, facendo riferimento allo standard ISO/IEC 12207:2008, il team ha identificato i processi più significativi della sezione \emph{Project Processes} (§6.3) e  \emph{Software Implementation Processes} (§7.1) per i quali verranno definiti gli obiettivi ed in seguito le metriche utilizzate per verificarne il raggiungimento.

\subsubsection{Processo di pianificazione di progetto}
Scopo di tale processo è produrre e comunicare un piano di progetto fattibile ed efficace, stabilendo le attività che dovranno essere svolte, con i relativi costi e scadenze, in ogni fase del progetto; tali responsabilità saranno in mano al \Res, il quale dovrà assicurare che:

\begin{itemize}
	\item Costi e scadenze pianificati relativi ad ogni attività siano definiti.
	\item Le attività siano divise in compiti, ognuno dei quali dovrà essere assegnato ad un membro del team.
	\item Gli obiettivi di progetto siano definiti in maniera ponderata, tenendo conto delle risorse a disposizione del team.
\end{itemize}

\paragraph{Compiti assegnati} 
~\\Tale metrica verrà utilizzata per controllare l'assegnazione dei compiti all'interno del team.

\begin{itemize}
	\item \textbf{Soglia di accettabilità e ottimalità}: si ritiene doveroso avere ogni compito in ogni fase assegnato ad un membro del team, dunque accettabilità ed ottimalità coincideranno con un risultato = 100\%.
\end{itemize}

\subsubsection{Processo di controllo e accertamento}
Scopo di tale processo è determinare lo stato del progetto ed assicurare che esso sia in linea con la pianificazione prevista nel \PdP, per quanto riguarda il rispetto di tempi e costi preventivati.
In particolare, per il team è fondamentale rispettare le scadenze ed i budget previsti: andranno effettuati quindi dovuti controlli attraverso un continuo monitoraggio della situazione corrente di ogni attività di progetto. Nel caso in cui venissero rilevati valori non accettabili a seguito del calcolo della Schedule Variance\ped{G} o Cost Variance\ped{G}, sarà compito del team sanarli il prima possibile, compensando con un risparmio (economico o temporale rispettivamente) entro la scadenza finale di consegna: si dovrà prestare particolare attenzione ai costi, per i quali non è ammesso eccedere rispetto alla pianificazione. 
In generale si concretizzano dunque due obiettivi:

\begin{itemize}
	\item Ogni attività dovrà essere portata a termine nella sua completezza entro le scadenze prefissate.
	\item Ogni attività dovrà essere portata a termine nella sua completezza senza eccedere i costi per essa preventivati; si cercherà se possibile di ottenere inoltre un risparmio orario (e quindi economico) in ogni fase, poiché anche un risparmio molto piccolo potrà essere utile a bilanciare eventuali eccedenze nei costi nelle fasi successive.
\end{itemize} 

\paragraph{Schedule Variance}
~\\Tale metrica verrà utilizzata per quanto riguarda il controllo delle scadenze temporali.
\begin{itemize}
	\item \textbf{Soglia di accettabilità}: si è deciso di ritenere accettabile un ritardo massimo del 10\% rispetto a quanto specificato nel \PdP{};
	\item \textbf{Soglia di ottimalità}: si ritiene un miglioramento rispetto all'obiettivo prefissato il caso in cui un lavoro venga portato a termine in anticipo rispetto a quanto specificato nel \PdP{}.
\end{itemize}

\paragraph{Cost Variance}
~\\Tale metrica verrà utilizzata per quanto riguarda il controllo del rispetto dei costi stabiliti.
\begin{itemize}
	\item \textbf{Soglia di accettabilità}: sarà accettabile un risultato $\le$ 0.
	\item \textbf{Soglia di ottimalità}: la soglia di ottimalità verrà raggiunta nel caso in cui i costi saranno inferiori a quanto stabilito nel \PdP{}, quindi un risultato $\textless$ 0.
\end{itemize}

\subsubsection{Processo di misurazione e miglioramento}
Tale processo serve a valutare e migliorare la qualità del lavoro svolto.

\paragraph{SPICE}
~\\Verrà utilizzata come metrica la struttura a 6 livelli che rappresenta la scala di maturità secondo SPICE; la misura di ogni livello sarà effettuata con i 4 livelli N, P, L e F definiti dallo standard.

\begin{itemize}
	\item \textbf{Soglia di accettabilità}: il livello minimo accettabile di maturità della scala in riferimento ai processi è il 2 (Managed); il processo deve cioè fornire i risultati conformi agli standard e ai requisiti iniziali in maniera pianificata e tracciabile;
	\item \textbf{Soglia di ottimalità}: la soglia di ottimalità verrà raggiunta con il livello 4 (Predictable); il processo dovrà cioè essere eseguito in conformità ai principi dell'ingegneria del software e attuato all'interno di limiti ben definiti.
\end{itemize}

Per informazioni più approfondite riguardo allo standard ISO/IEC 15504 o SPICE, si rimanda alla sezione~\nameref{AppA:standardProc} dell'appendice A.

\subsubsection{Processo di gestione dei rischi}
Tale processo verrà utilizzato per individuare, analizzare e monitorare e trattare i rischi in cui si potrebbe incorrere durante l’intera durata del progetto, in maniera continua, fin dalla prima fase del progetto, al fine di riconoscere e saper affrontare potenziali situazioni dannose.
Si individuano i seguenti obiettivi:

\begin{itemize}
	\item Sarà necessario, all'inizio di ogni fase di progetto, analizzare i possibili rischi specifici di tale fase, considerando i possibili nuovi pericoli introdotti dalle precedenti fasi;	
	\item Alla luce dell'esperienza maturata nelle precedenti fasi di progetto, sarà necessario valutare l'entità delle ripercussioni dei problemi riscontrati, al fine di anticiparli e affrontarli efficacemente nel caso in cui si ripresentassero.
	\item Nel caso in cui i rischi possibili siano molti, sarà necessario che il \Res assegni loro diverse priorità, a seconda dell'entità delle possibili ripercussioni.
\end{itemize}

\paragraph{Rischi non preventivati}
~\\Tale metrica verrà utilizzata per valutare l'efficacia della previsione dei rischi da parte del team.

\begin{itemize}
	\item \textbf{Soglia di accettabilità}: verrà considerato come accettabile un valore compreso tra 0 e 3 in una singola fase di progetto.
	\item \textbf{Soglia di ottimalità}: la soglia di ottimalità verrà raggiunta con il valore 0.
\end{itemize}


\subsubsection{Processo di progettazione dell'architettura software/di sistema}
Scopo di tale processo è associare i requisiti del sistema alle singole parti del sistema, implementandole correttamente di conseguenza. L'architettura del sistema dovrà dunque essere sviluppata attraverso una progettazione che consenta la tracciabilità delle relazioni tra componenti e requisiti, nonché tra le componenti stesse; i seguenti obiettivi concretizzano quanto detto:

\begin{itemize}
	\item Ogni componente progettata e sviluppata dovrà essere tracciabile in relazione all'intero sistema e al requisito che soddisfa.	
	\item Ogni parte del sistema dovrà avere il più basso grado possibile di accoppiamento ed alta coesione.
\end{itemize}

\paragraph{SFIN}
~\\Tale metrica sarà utile per verificare il riuso del codice.
\begin{itemize}
	\item \textbf{Soglia di accettabilità}: verrà considerato come accettabile un valore compreso $\ge$ 0.
	\item \textbf{Soglia di ottimalità}: la soglia di ottimalità verrà raggiunta con il valore $\ge$ 2.
\end{itemize}

\paragraph{SFOUT}
~\\Tale metrica sarà utile a calcolare il grado di accoppiamento delle varie componenti software.
\begin{itemize}
	\item \textbf{Soglia di accettabilità}: verrà considerato come accettabile un valore $\le$ 4.
	\item \textbf{Soglia di ottimalità}: la soglia di ottimalità verrà raggiunta con il valore = 0.
\end{itemize}


\subsubsection{Processi di progettazione di dettaglio e codifica del software}
Scopo di tali processi è progettare e creare unità software che implementino e possano essere verificate rispetto ai requisiti, quindi sufficientemente dettagliate da permetterne esecuzione e test.
Il team si impegna a raggiungere i seguenti obiettivi:

	\begin{itemize}
		\item Fornire una progettazione con un grado di dettaglio tale da poter produrre codice testabile.
		\item Produrre tutte le unità software previste dalla progettazione, le quali devono correttamente interagire tra loro e nel sistema, se previsto.
		\item Il codice prodotto dovrà avere bassa complessità, in modo da essere facilmente capibile, correggibile e mantenibile.
	\end{itemize}

\paragraph{Complessità ciclomatica}
\begin{itemize}
	\item \textbf{Accettabilità}: un valore di complessità compreso tra 1 e 15, purchè per valori tra 10 e 15 sia specificato il motivo di tale complessità;
	\item \textbf{Ottimalità}: un valore di complessità compreso tra 1 e 10.
\end{itemize}

\paragraph{CxSLOC - commenti per linee di codice}
\begin{itemize}
	\item \textbf{Accettabilità}: sarà accettato un valore CxSLOC compreso tra 20 e 25;
	\item \textbf{Ottimalità}: sarà dichiarato ottimale un valore CxSLOC compreso tra 25 e 35.
\end{itemize}

\paragraph{Parametri per metodo}
\begin{itemize}
	\item \textbf{Accettabilità}: saranno accettati metodi con un numero di parametri minore o uguale a 10;
	\item \textbf{Ottimalità}: saranno considerati ottimi metodi con un numero di parametri minore o uguale a 5.
\end{itemize}

\paragraph{Linee di codice per metodo}
\begin{itemize}
	\item \textbf{Accettabilità}: saranno accettati metodi con una lunghezza pari o inferiore alle 50 righe;
	\item \textbf{Ottimalità}: saranno considerati ottimi metodi con una lunghezza pari o inferiore alle 30 righe.
\end{itemize}


\subsubsection{Processi di verifica e validazione}
Scopo di tali processi è accertare rispettivamente che l'esecuzione delle attività di processo attuate nel periodo in esame non abbiano introdotto errori e che il prodotto realizzato soddisfi i requisiti. Il team si impegna ad assicurare che:

\begin{itemize}
	\item Siano correttamente applicate le tecniche di Walkthrough prima e di Inspection poi, come specificato nelle \NdP, durante le attività di analisi statica;
	\item I test dinamici effettuati su codice e documenti saranno automatizzati il più possibile;
	\item I test dinamici effettuati sul codice copriranno il maggior numero possibile di casi possibili e di statement.
\end{itemize}

\paragraph{Copertura del codice}
\begin{itemize}
	\item \textbf{Accettabilità}: sarà accettata un numero di statement testati pari al 70\%;
	\item \textbf{Ottimalità}: sarà considerata ottima la capacità di testare almeno il 90\% degli statement.
\end{itemize}

\paragraph{Copertura dei branch}
\begin{itemize}
	\item \textbf{Accettabilità}: sarà accettata un numero di rami testati pari al 75\%;
	\item \textbf{Ottimalità}: sarà considerata ottima la capacità di testare almeno il 95\% dei rami per funzionalità non ancora testate, mentre per codice già testato l'ottimalità sara data dalla capacità di testarne l'80\%.
\end{itemize}


\subsection{Qualità di prodotto}
\subsubsection{Qualità di documento}
Il team si impegna a redigere dei documenti di alta qualità, rispettando le caratteristiche di forma e contenuto descritte di seguito.
\paragraph{Ortografia}
~\\Un documento deve essere prima di tutto privo di errori dal punto di vista grammaticale e ortografico. 
Il primo controllo avverà proprio durante la stesura del documento stesso, tramite il sistema di autocontrollo dell'ambiente  \emph{TexStudio}, per poi essere controllato una seconda volta dal \ver{}.
\begin{itemize}
	\item \textbf{Metrica}: la quantità di errori riscontrata durante la verifica definitiva del documento sarà l'unità di misura presa in considerazione;
	\item \textbf{Soglia di accettabilità}: si è accettata come tollerabile la presenza di massimo 3 errori nella seconda e definitiva verifica da parte del \ver{};
	\item \textbf{Sogia di ottimalità}: la soglia di ottimalità verrà raggiunta nel caso in cui dopo la prima revisione del documento non vengano più riscontrati errori dal \ver{} e dal \RdP{}.
\end{itemize}

\paragraph{Comprensibilità e leggibilità}
~\\Poichè un documento venga considerato leggibile e scorrevole si è deciso di adottare l'\emph{Indice Gulpease}\ped{G} al fine di avere un parametro oggettivo e facilmente misurabile.
\begin{itemize}
	\item \textbf{Metrica}: l'unità di misura utilizzata è l'\emph{Indice Gulpease};
	\item \textbf{Soglia di accettabilità}: verrà considerato come accettabile un valore di 45 sulla scala dell'\emph{Indice Gulpease};
	\item \textbf{Soglia di ottimalità}: la soglia di ottimalità verrà raggiunta nel caso in cui l'\emph{Indice Gulpease} sia maggiore di 60.
\end{itemize}

\paragraph{Correttezza dei contenuti}
~\\Oltre ad essere corretto nella forma, un documento necessita di un contenuto adeguato dal punto di vista argomentativo. Gli \anas{} saranno direttamente responsabili della qualità del contenuto, che poi verrà controllato e corretto dal \ver{}.
Per verificare la correttezza concettuale dei documenti prenderemo in esame i seguenti parametri:
\begin{itemize}
	\item \textbf{Metrica}: la quantità di errori di contenuto riscontrata durante la verifica definitiva del documento sarà l'unità di misura presa in considerazione;
	\item \textbf{Soglia di accettabilità}: si è accettata come tollerabile la presenza di massimo 3 errori nella seconda e definitiva verifica da parte del \ver{};
	\item \textbf{Soglia di ottimalità}: la soglia di ottimalità sarà raggiunta nel caso in cui non si riscontrino errori durante la verifica definitiva del documento.
\end{itemize}

\paragraph{Adesione alle norme interne}
~\\Al fine di ottenere un prodotto coerente ogni documento dovrà essere redatto rispettando strettamente quanto dichiarato nelle \NdP{}.
Qualunque riferimento non attinente o in contrasto a quanto dichiarato verrà considerato un errore.
\begin{itemize}
	\item \textbf{Metrica}: la quantità di errori di adesione alle norme interne riscontrata durante la verifica definitiva del documento sarà l'unità di misura presa in considerazione;
	\item \textbf{Soglia di accettabilità}: si è accettata come tollerabile la presenza di massimo 3 errori nella seconda e definitiva verifica da parte del \ver{};
	\item \textbf{Soglia di ottimalità}: la soglia di ottimalità sarà raggiunta nel caso in cui non si riscontrino errori dopo la prima verifica del documento.
\end{itemize}


\subsubsection{Qualità del prodotto software}
Come detto in precedenza, è impossibile distinguere in maniera netta la qualità di processo dalla qualità del software, in quanto la prima influenza direttamente la seconda; è dunque fondamentale avere alla base una qualità di processo sufficientemente buona per garantire la qualità del prodotto. Nonostante ciò, è necessario stabilire degli obiettivi quantitativi di qualità del software oggettivi e misurabili. A tal fine verrà seguito lo standard ISO/IEC 9126, il quale si sostanzia nei seguenti sei punti:

\paragraph{Funzionalità}
~\\È un requisito funzionale che indica la capacità del software di soddisfare le esigenze esposte dal capitolato ed individuate durante l’ \AdR .
Per valutare la funzionalità del software prenderemo in considerazione i seguenti parametri:
\begin{itemize}
	\item \textbf{Metrica}: la valutazione si baserà sul numero di requisiti soddisfatti;
	\item \textbf{Soglia di accettabilità}: il prodotto verrà valutato come accettabile se tutti i requisiti obbligatori saranno soddisfatti;
	\item \textbf{Soglia di ottimalità}: la soglia di ottimalità sarà raggiunta nel caso in cui siano soddisfatti sia i requisiti obbligatori che tutti i requisiti opzionali.
\end{itemize}

\paragraph{Affidabilità}
~\\È un requisito non funzionale che indica la capacità del software di svolgere correttamente il suo compito mantenendo delle buone prestazioni anche al variare dell’ambiente nel tempo.
Per valutare l'affidabilità del software prenderemo in considerazione i seguenti parametri:
\begin{itemize}
	\item \textbf{Metrica}: la valutazione si baserà sul numero di fallimenti durante la fase di test;
	\item \textbf{Soglia di accettabilità}: il prodotto verrà valutato come accettabile se i test falliti saranno inferiori o uguali al 5\%;
	\item \textbf{Soglia di ottimalità}: la soglia di ottimalità sarà raggiunta nel caso in cui il 100\% dei test avrà dato l'esito desiderato.
\end{itemize}


\paragraph{Efficienza}
~\\È un requisito non funzionale che valuta la capacità di un prodotto software di realizzare le funzioni richieste nel minor tempo possibile e con l’uso minimo di risorse necessarie.
\begin{itemize}
	\item \textbf{Metrica}: la valutazione si baserà sui secondi impiegati dal prodotto per eseguire le richieste dell'utente;
	\item \textbf{Soglia di accettabilità}: la soglia di accettabilità è il periodo tra 0 e 10 secondi;
	\item \textbf{Soglia di ottimalità}: la soglia di ottimalità è 1 secondo.
\end{itemize}

\paragraph{Usabilità}
~\\L'usabilità è un requisito non funzionale che indica la capacità del software di essere capito e usato correttamente da parte dell'utente finale. Dato che il prodotto finale sarà per l'utente un portale web, è impossibile trovare una metrica quantificabile per valutarne l'usabilità: essa dipende da molteplici fattori che coinvolgono anche le capacità dell'utente stesso e gli strumenti a sua disposizione. Verrà dunque valutata in modo oggettivo basandosi sugli standard del web dichiarati dal \emph{W3C}\ped{G} e sugli strumenti che tale organizzazione mette a disposizione, al fine di creare un'interfaccia web il più accessibile possibile.
Prenderemo in considerazione i seguenti parametri:
\begin{itemize}
	\item \textbf{Metrica}: la valutazione si baserà sul numero di errori trovati dagli strumenti del W3C;
	\item \textbf{Soglia di accettabilità}: la soglia di accettabilità è di 2 errori rilevati;
	\item \textbf{Soglia di ottimalità}: il prodotto sarà dichiarato ottimo se saranno rilevati 0 errori.
\end{itemize}

Quanto detto non assicura però una valutazione completa dell'usabilità, la quale è soggettiva; sarà necessario dunque predisporre test specifici per la misurazione, coinvolgendo ad esempio persone esterne al gruppo al fine di stabilire quanto mediamente il software sia capibile. Al momento, tuttavia, il team non è in grado di stabilire con precisione una metrica adatta a misurare questo risultato.

\paragraph{Manutenibilità}
~\\La manutenibilità è un requisito non funzionale che indica la capacità di un prodotto di essere evolvibile nel tempo attraverso correzioni, miglioramenti ed aggiunte.

	\begin{itemize}
	\item \textbf{Metrica}: saranno usate le metriche riguardanti il codice, dato che esso influenza direttamente la manutenibilità del software;
	\item \textbf{Soglia di accettabilità}: la soglia di accettabilità sarà raggiunta se il prodotto raggiungerà tale soglia in tutte le metriche utilizzate per il codice;
	\item \textbf{Soglia di ottimalità}: la soglia di ottimalità sarà raggiunta se il prodotto raggiungerà tale soglia in tutte le metriche utilizzate per il codice.
	\end{itemize}

\paragraph{Portabilità}
~\\La portabilità è un requisito non funzionale che indica la capacità del prodotto di operare in \textit{ambienti}\ped{G} diversi, limitando le necessità di apportare cambiamenti.
\begin{itemize}
	\item \textbf{Metrica}: la valutazione si baserà sul numero di versioni di \emph{browser}\ped{G} e numero di browser stessi su cui il prodotto riesce a venire utilizzato e visualizzato correttamente;
	\item \textbf{Soglia di accettabilità}: la soglia di accettabilità sarà raggiunta se il prodotto sarà supportato correttamente, offrendo la totalità delle sue funzionalità, dalla versione 60.3.3112 o successive di \emph{Google Chrome}\ped{G} (\emph{metamask}\ped{G} versione 4.6 o superiore) e dalla versione 50 o superiore di \emph{Mozilla Firefox}\ped{G} (metamask versione 4.5 o superiore);
	\item \textbf{Soglia di ottimalità}: la soglia di ottimalità sarà raggiunta se il prodotto sarà supportato correttamente, offrendo la totalità delle sue funzionalità in aggiunta ai sopra citati, da \emph{Opera}\ped{G} nella versione 52 o superiore (metamask versione 3.13.4 o superiore).
\end{itemize}



