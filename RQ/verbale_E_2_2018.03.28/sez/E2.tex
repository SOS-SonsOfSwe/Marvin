\section{Informazioni generali}
	\begin{itemize}
		\item \textbf{Data riunione}: 2018-03-28;
		\item \textbf{Ora inizio riunione}: 17:30;
		\item \textbf{Ora fine riunione}: 18:30;
		\item \textbf{Luogo di incontro}: Slack;
		\item \textbf{Oggetto di discussione}: si sono definiti alcuni casi d'uso non chiari dalla descrizione del capitolato;
		\item \textbf{Moderatore}: Giovanni Cavallin;
		\item \textbf{Segretario}: Eleonora Thiella;
		\item \textbf{Partecipanti}: Giovanni Cavallin, Eleonora Thiella, Giovanni Dalla Riva, Stefano Panozzo, Andrea Favero, Lorenzo Menegon e Federico Caldart;
	\end{itemize}

\section{Riassunto della riunione}
	\subsection{Descrizione}
	Dopo aver analizzato con molta attenzione il capitolato prodotto dal proponente Red Babel il gruppo ha deciso di contattarlo per discutere di alcuni punti non chiari o non sufficientemente specifici.
	\subsection{Decisioni prese}
		\begin{itemize}
			\item \textbf{VE2.1}: si è deciso in via definitiva di fare distinzione tra gli attori amministratore e università, in quanto è risultato necessario dare all'università il privilegio di gestione degli amministratori;
			\item \textbf{VE2.2}: per quanto riguarda il requisito "L'Università può inserire una lista di professori", il team ha proposto al proponente di intenderla come una lista di matricole associate ad un codice univoco, che saranno consegnati all'utente tramite email. Con questi due codici, l'utente potrà accedere al form di login e quindi procedere alla registrazione dei propri dati. Il proponente ha accettato questa soluzione;
			\item \textbf{VE2.3}: il proponente non ha espresso particolari bisogni riguardo la visione del verbale e ha dichiaratamente specificato che tutte le discussioni e le decisioni da esse scaturite saranno visionate unicamente dai committenti;
			\item \textbf{VE2.4}: il proponente ha consentito a considerare tutti i requisiti analizzati nel capitolato come obbligatori e a ritenere opzionali tutti quelli ricavati da essi dal gruppo;
		\end{itemize}