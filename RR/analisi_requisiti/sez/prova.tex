\newcounter{a}
\newcommand{\taskA}[1]{\refstepcounter{a}\label{#1}}

\newcounter{b}[a]
\newcommand{\taskB}[1]{
	\refstepcounter{b}
	\label{#1}
}


\section{Introduzione}



\taskA{Glossario}
\subsection{UC\ref{Glossario}: Glossario}
Nel documento Glossario i termini tecnici, gli acronimi e le abbreviazioni sono definiti in modo chiaro e conciso, in modo tale da evitare ambiguità e massimizzare la comprensione dei documenti.
\newline I vocaboli presenti in esso saranno posti in corsivo e presenteranno una "G" maiuscola a pedice.

\taskA{Scopo del documento}
\subsection{UC\ref{Scopo del documento}: Scopo del documento}
Questo documento ha lo scopo di descrivere gli attori del sistema, individuare i casi d'uso individuati a partire dai requisiti e fornire una visione chiara ai progettisti sul problema da trattare. I requisiti verrannoo classificatio in questo documento a seguito di una trattazione col proponente.

\taskB{Sottosezione di Scopo del documento}
\subsubsection{UC\ref{Scopo del documento}.\ref{Sottosezione di Scopo del documento}: Sottosezione di Scopo del documento}

\taskA{Scopo del prodotto}
\subsection{UC\ref{Scopo del prodotto}: Scopo del prodotto}
Lo scopo del prodotto è quello di realizzare un prototipo di Uniweb come ÐApp che giri su rete Ethereum. I tre attori principali che si rapportano con Marvin sono:
\begin{itemize}
	\item Università;
	\item Professori;
	\item Studenti.
\end{itemize} 
Il portale deve quindi permettere agli studenti di accedere alle informazioni riguardanti le loro carriere universitarie, di iscriversi agli esami, di accettare o rifiutare voti e di poter vedere il loro libretto universitario.
Ai professori deve invece essere permesso registrare i voti degli studenti.
L'università ogni anno crea una serie di corsi di laurea rivolti a studenti, dove ognuno di essi comprende un elenco di esami disponibili per anno accademico. Ogni esame ha un argomento, un numero di crediti e un professore associato. Gli studenti si iscrivono ad un corso di laurea e tramite il libretto elettronico mantengono traccia ufficiale del progresso.

\taskB{Sottosezione di Scopo del prodotto}
\subsubsection{UC\ref{Scopo del prodotto}.\ref{Sottosezione di Scopo del prodotto}: Sottosezione di Scopo del prodotto}

\subsection{Riferimenti}
\subsubsection{Normativi}
\begin{itemize}
	\item \textcolor{red}\NdP
\end{itemize}

\subsubsection{Informativi}
\begin{itemize}
	\item Capitolato d'appalto C6: Marvin. Reperibile all'indirizzo:\\ 
	\href{http://www.math.unipd.it/~tullio/IS-1/2017/Progetto/C6.pdf}{http://www.math.unipd.it/~tullio/IS-1/2017/Progetto/C6.pdf};
	\item \textcolor{red}\SdF;
\end{itemize}
