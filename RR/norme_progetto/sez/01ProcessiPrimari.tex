\section{Processi primari}
\subsection{Processo di fornitura}

\subsubsection{Studio di Fattibilità}
Si tratta del documento in cui vengono analizzati tutti i capitolati, valutandone pregi e difetti, allo scopo di scegliere quello di maggiore affinità per il gruppo.
Dopo che il \emph{\RdP} avrà riunito il gruppo e discusso con esso di tutti i capitolati, gli \emph{\anas} avranno il compito di stilare lo \emph{\SdF} 
seguendo le considerazioni emerse.\\
Lo \emph{\SdF} sarà organizzato come segue:
\begin{itemize}
	\item \textbf{Informazioni sul capitolato}
	Vengono ricordati il nome del progetto, il proponente e i committenti.
	\item \textbf{Descrizione}
	Si riassume lo scopo del capitolato.
	\item \textbf{Dominio applicativo}
	Si specifica il settore di utilizzo del prodotto finale.
	\item \textbf{Dominio tecnologico}
	Si elencano le tecnologie che dovranno essere utilizzate nello sviluppo del capitolato.
	\item \textbf{Aspetti positivi}
	Si elencano i motivi considerabili vantaggiosi per il gruppo.
	\item \textbf{Aspetti negativi}
	Si elencano le potenziali criticità che il gruppo dovrà tenere in considerazione nella scelta finale.
	\item \textbf{Valutazione finale}
	Si analizzano gli aspetti positivi e le criticità riscontrate e viene motivata l'eventuale approvazione o esclusione.
\end{itemize}

\begin{comment}
\subsubsection{Rapporti con il proponente}
Una volta scelto il capitolato, ci si dovrà confrontare con il proponente come segue:
\begin{itemize}
	\item \textbf{Primo contatto}
	Sfruttando i contatti messi a disposizione nella descrizione del capitolato, si contatta il proponente per verificarne l'effettiva disponibilità.
	\item \textbf{Definizione dei canali di comunicazione}
	Si dovrà concordare un metodo di comunicazione adatto ad entrambe le parti, per un confronto facile ed immediato tra fornitore e proponente.
\end{itemize}
\\
Confronto per le tecnologie da usare.
\end{comment}

\subsection{Processo di sviluppo}
\subsubsection{Attività}
\paragraph{Analisi dei Requisiti}
	Gli \anas, una volta terminato lo \SdF, dovranno stilare l'\AdR, che si dovrà attenere alle seguenti regole:
	\begin{itemize}
	\item \textbf{Classificazione dei Requisiti}
	\textit{Scrivere qui come poi saranno classificati}
	\item \textbf{Classificazione dei casi d'uso}
	\textit{Scrivere qui come poi saranno classificati}
	\end{itemize}
\paragraph{Progettazione}
	I \progs dovranno delinerare i requisiti utili alla documentazione specifica e determinare le linee guida da seguire.
	\subparagraph{UML}
	Le tipologie di diagrammi UML che verranno adoperate per analizzare, descrivere e specificare le scelte progettuali adottate saranno:
	\begin{itemize}
		\item \textbf{Diagrammi di classe}
		\item \textbf{Diagrammi di package\ped{G}}
		\item \textbf{Diagrammi di attività}
		\item \textbf{Diagrammi di sequenza}
	\end{itemize}
	\subparagraph{Requisiti per i progettisti}
	
	\subparagraph{Obiettivi della progettazione}
	\subparagraph{}
	\paragraph{Codifica}

