\section{Processi primari}
\subsection{Processo di fornitura}

\subsubsection{Studio di Fattibilità}
Si tratta del documento in cui vengono analizzati tutti i capitolati, valutandone pregi e difetti, allo scopo di scegliere quello di maggiore affinità per il gruppo.
Dopo che il \emph{\RdP} avrà riunito il gruppo e discusso con esso di tutti i capitolati, gli \emph{\anas} avranno il compito di stilare lo \emph{\SdF} 
seguendo le considerazioni emerse.\\
Lo \emph{\SdF} sarà organizzato come segue:
\begin{itemize}
	\item \textbf{Informazioni sul capitolato}
	Vengono ricordati il nome del progetto, il proponente e i committenti.
	\item \textbf{Descrizione}
	Si riassume lo scopo del capitolato.
	\item \textbf{Dominio applicativo}
	Si specifica il settore di utilizzo del prodotto finale.
	\item \textbf{Dominio tecnologico}
	Si elencano le tecnologie che dovranno essere utilizzate nello sviluppo del capitolato.
	\item \textbf{Aspetti positivi}
	Si elencano i motivi considerabili vantaggiosi per il gruppo.
	\item \textbf{Aspetti negativi}
	Si elencano le potenziali criticità che il gruppo dovrà tenere in considerazione nella scelta finale.
	\item \textbf{Valutazione finale}
	Si analizzano gli aspetti positivi e le criticità riscontrate e viene motivata l'eventuale approvazione o esclusione.
\end{itemize}

\begin{comment}
\subsubsection{Rapporti con il proponente}
Una volta scelto il capitolato, ci si dovrà confrontare con il proponente come segue:
\begin{itemize}
	\item \textbf{Primo contatto}
	Sfruttando i contatti messi a disposizione nella descrizione del capitolato, si contatta il proponente per verificarne l'effettiva disponibilità.
	\item \textbf{Definizione dei canali di comunicazione}
	Si dovrà concordare un metodo di comunicazione adatto ad entrambe le parti, per un confronto facile ed immediato tra fornitore e proponente.
\end{itemize}
\\
Confronto per le tecnologie da usare.
\end{comment}

\subsection{Processo di sviluppo}
\subsubsection{Attività}
\paragraph{Analisi dei Requisiti}
	Gli \anas, una volta terminato lo \SdF, dovranno stilare l'\AdR, che si dovrà attenere alle seguenti regole:
	\begin{itemize}
	\item \textbf{Classificazione dei Requisiti}
	\textit{Scrivere qui come poi saranno classificati}
	\item \textbf{Classificazione dei casi d'uso}
	\textit{Scrivere qui come poi saranno classificati}
	\end{itemize}
\paragraph{Progettazione}
	I \progs dovranno delinerare i requisiti utili alla documentazione specifica e determinare le linee guida da seguire.
	\subparagraph{UML}
	Le tipologie di diagrammi UML che verranno adoperate per analizzare, descrivere e specificare le scelte progettuali adottate saranno:
	\begin{itemize}
		\item \textbf{Diagrammi di classe}
		\item \textbf{Diagrammi di package\ped{G}}
		\item \textbf{Diagrammi di attività}
		\item \textbf{Diagrammi di sequenza}
	\end{itemize}
%	\subparagraph{Requisiti per i progettisti}
	
	\subparagraph{Obiettivi della progettazione}
	La progettazione ha come scopo quello di soddisfare le peculiarità identificate durante l'\AdR.
	Un altro obiettivo è quello di realizzare un prodotto manutenibile, ovvero che abbia una struttura che faciliti i cambiamenti futuri.
	Infine deve realizzare al meglio i requisiti di qualità imposti dal committente.
	\paragraph{Codifica}
	In questa fase i \progrs, seguendo le norme delineate nella progettazione, devono realizzare il passaggio dalla fase di pianificazione all'effettiva realizzazione del prodotto.
	Le norme qui presenti serviranno come strumento per realizzare un codice uniforme e di alta qualità. Inoltre, per mantenerne la manutenibilità, dovrà essere realizzato in inglese.
	I \progrs si dovranno attenere ai seguenti standard di codifica:
	\subparagraph{Nomi}
	\begin{itemize}
		\item Ogni elemento deve avere un nome rappresentativo e pertinente alla funzione da esso svolta;
		\item Si dovrà utilizzare la notazione \emph{CamelCase}, ovvero la concatenazione di più parole, ognuna delle quali con lettera iniziale maiuscola. In caso di metodi e variabili, la prima lettera dovrà essere maiuscola, per le classi maiuscola;
		\item Si potranno utilizzare singole lettere esclusivamente per identificare gli indici dei cicli;
		\item Saranno da evitare notazioni troppo simili tra di loro in significato ed denominazione;
		\item Si dovranno evitare errori di ortografia;
	\end{itemize}
	\subparagraph{Commenti}
	Sarà necessario che il codice contenga dei commenti esplicativi per facilitarne la comprensione. In particolare, si dovranno seguire queste linee guida:
	\begin{itemize}
		\item \textit{lista di linee guida per i commenti}
	\end{itemize}
	\subparagraph{Formattazione}
	\begin{itemize}
		\item Il rientro predefinito dovrà essere di un \emph{Tab} per allineare le sezioni di codice;
		\item La parentesi graffa di apertura sarà alla fine della riga, mentre quella di chiususa a capo;
		\item Prima e dopo ogni operatore dovrà esserci uno spazio;
		\item I blocchi saranno tra loro spaziati per una maggiore comprensione. \textcolor{red}{Da decidere: lasciamo spazio anche tra i blocchi dentro al metodo?}
		\item Il codice sorgente dovrà essere quanto più suddiviso in file, e dovrà essere raggruppato in sottocartelle che dovranno rispecchiare il pattern utilizzato.
		\item \textit{Se ce ne sono altre le inseriamo qui. Da pensare al pattern, }
	\end{itemize}
\subsubsection{Strumenti}
Gli strumenti utilizzati durante la fase dei processi primari sono:
\begin{itemize}
	\item \textbf{TexStudio}
	Il gruppo ha scelto \emph{TexStudio} come editor multipiattaforma per comporre i documenti in \LaTeX. \textcolor{red}{Serve una descrizione migliore?}
\end{itemize}	

