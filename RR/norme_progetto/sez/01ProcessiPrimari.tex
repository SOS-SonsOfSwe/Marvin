\section{Processi primari}
\subsection{Processo di fornitura}

\subsubsection{Studio di Fattibilità}
Si tratta del documento in cui vengono analizzati tutti i capitolati, valutandone pregi e difetti, allo scopo di scegliere quello di maggiore affinità per il gruppo.
Dopo che il \RdP{} avrà riunito il team e discusso con esso di tutti i capitolati, gli \anas{} avranno il compito di stilare lo \SdF{} seguendo le considerazioni emerse.\\
Lo \SdF{} sarà organizzato come segue:
\begin{itemize}
	\item \textbf{Informazioni sul capitolato}:
	Vengono ricordati il nome del `emph{progetto}\ped{G}, il \textit{proponente}\ped{G} e i \textit{committenti}\ped{G};
	\item \textbf{Descrizione}:
	Si riassume lo scopo del capitolato;
	\item \textbf{Dominio applicativo}:
	Si specifica il settore di utilizzo del prodotto finale;
	\item \textbf{Dominio tecnologico}:
	Si elencano le tecnologie che dovranno essere utilizzate nello sviluppo del capitolato;
	\item \textbf{Aspetti positivi}:
	Si elencano i motivi che il gruppo potrebbe considerare vantaggiosi;
	\item \textbf{Aspetti negativi}:
	Si elencano le potenziali criticità che il gruppo dovrà tenere in considerazione nella scelta finale;
	\item \textbf{Valutazione finale}:
	Si analizzano gli aspetti positivi e quelli negativi riscontrati e si motiva l'eventuale approvazione o esclusione.
\end{itemize}

\subsubsection{Rapporti con il proponente}
Una volta scelto il capitolato, si intende instaurare un rapporto quanto più costante e profittevole con il \proponente{} allo scopo di:
\begin{itemize}
	\item Stabilire un accordo in merito allo sviluppo, al mantenimento, al funzionamento e alla consegna del prodotto;
	\item Realizzare un prodotto che soddisfi totalmente i requisiti obbligatori concordati e quanto più possibile quelli desiderabili;
	\item Stimare i costi;
	\item Concordare la qualifica del prodotto.
\end{itemize}

\subsubsection{Documentazione fornita}
Al \proponente{} e ai \committenti{} verranno forniti i seguenti documenti:
\begin{itemize}
	\item \textbf{\AdR}: contiene l'analisi dei casi d'uso e dei requisiti;
	\item \textbf{\PdQ}: contiene l'attività di verifica, di validazione e la garanzia di qualità di processi e prodotto;
	\item \textbf{\ST}: contiene l'attività di progettazione, che misura il conseguimento di una solida base tecnologica, e verrà consegnata nella fase di \RP;
	\item \textbf{\DdP}: contiene l'attività di progettazione, che misura il conseguimento di una solida base architetturale, e verrà consegnata nella fase di \RQ.
\end{itemize}

\subsubsection{Collaudo e consegna del prodotto}
Una volta terminate le fasi di sviluppo, verifica e validazione si effettuerà il collaudo al fine di dimostrare che tutti i requisiti obbligatori e, possibilmente, anche alcuni dei requisiti opzionali siano stati soddisfatti. 
\\In questa fase inoltre si dovrà dimostrare che l'esecuzione di tutti i test di validazione abbia dato un'esito positivo.
\\Il team consegnerà in ultima il prodotto finale su un supporto fisico a \committenti.

\subsection{Processo di sviluppo}
\subsubsection{Attività}
\paragraph{Analisi dei Requisiti}
	~\\Gli \anas, una volta terminato lo \SdF, dovranno stilare l'\AdR, che si dovrà attenere alle seguenti regole:
	\begin{itemize}
	\item \textbf{Classificazione dei Requisiti}:
	\textit{Scrivere qui come poi saranno classificati};
	\item \textbf{Classificazione dei casi d'uso}:
	\textit{Scrivere qui come poi saranno classificati}.
	\end{itemize}
\paragraph{Progettazione}
	~\\I \progs{} dovranno delinerare i requisiti utili alla documentazione specifica e determinare le linee guida da seguire.
	~\\La progettazione ha come scopo quello di soddisfare le peculiarità identificate durante l'\AdR. Un altro obiettivo è quello di realizzare un prodotto \emph{manutenibile}\ped{G}, ovvero che abbia una struttura che faciliti i cambiamenti futuri.
	\\Infine deve realizzare al meglio i requisiti di qualità imposti dal committente.
	\subparagraph{Linee guida per la progettazione}
	Dopo aver completato l'\AdR{} i \progs{} dovranno sottostare alle seguenti linee guida per lo sviluppo dell'architettura logica del sistema:
	\begin{itemize}
		\item Si dovrà puntare ad una progettazione chiara e di immediata comprensione;
		\item Le componenti progettate dovranno essere quanto più riutilizzabili e manutenibili;
		\item La complessità non dovrà mai essere intrattabile;
		\item I \progs{} dovranno rientrare nei costi e nelle risorse disponibili;
		\item I \progs{} dovranno descrivere i \emph{design pattern}\ped{G} che intendono utilizzare per la realizzazione dell'architettura, fornendone una breve descrizione e un diagramma.
	\end{itemize}
	\subparagraph{UML}
	~\\Le tipologie di diagrammi \emph{UML}\ped{G} che verranno adoperate per analizzare, descrivere e specificare le scelte progettuali adottate saranno:
	\begin{itemize}
		\item \textbf{Diagrammi di classe}:
		\item \textbf{Diagrammi di } \emph{\textbf{package}}\ped{G}: documentano le dipendenze tra le classi ed è utile per controllare la complessità strutturale in sistemi medio-grandi;
		\item \textbf{Diagrammi di attività}: modellano un processo e organizzano più entità in un sistema di azioni secondo un determinato flusso. I diagrammi delle attività sono un tipo particolare di \emph{diagramma di stato}\ped{G} che identifica la variazione di stato al verificarsi di alcune condizioni legate ad una o più entità;
		\item \textbf{Diagrammi di sequenza}: descrivono la collaborazione di un gruppo di oggetti che devono implementare collettivamente un comportamento. Sono diagrammi molto semplici, ma che permettono di capire se l'architettura creata viene eseguita.
	\end{itemize}
\begin{comment}
	\subparagraph{Specifica Tecnica (ST)}
	
	\subparagraph{Definizione di Prodotto (DP)}
\end{comment}
	\paragraph{Codifica}
	~\\In questa fase i \progrs, seguendo le norme delineate nella progettazione, devono realizzare il passaggio dalla fase di pianificazione all'effettiva realizzazione del prodotto.
	Le norme qui presenti serviranno come strumento per realizzare un codice uniforme e di alta qualità. Inoltre, per mantenerne la manutenibilità, dovrà essere realizzato in inglese.
	I \progrs{} si dovranno attenere agli standard di codifica qui di seguito elencati.
	\subparagraph{Nomi}
	\begin{itemize}
		\item Ogni elemento deve avere un nome rappresentativo e pertinente alla funzione da esso svolta;
		\item Si dovrà utilizzare la notazione \emph{CamelCase}, ovvero la concatenazione di più parole, ognuna delle quali con lettera iniziale maiuscola. In caso di metodi e variabili la prima lettera dovrà essere minuscola, mentre per le classi maiuscola;
		\item Si potranno utilizzare singole lettere esclusivamente per identificare gli indici dei cicli;
		\item Saranno da evitare notazioni troppo simili tra di loro in significato e denominazione;
		\item Si dovranno evitare errori di ortografia.
	\end{itemize}
	\subparagraph{Commenti}
	~\\Sarà necessario che il codice contenga dei commenti esplicativi per facilitarne la comprensione.
	\newline In particolare, si dovranno seguire queste linee guida:
	
	\begin{comment}
	\begin{itemize}
		\item \textcolor{red}{lista di linee guida per i commenti}
	\end{itemize}
	\end{comment}
	
	\subparagraph{Formattazione}
	\begin{itemize} 
		\item Il rientro predefinito dovrà essere di un \emph{Tab} per allineare le sezioni di codice;
		\item La parentesi graffa di apertura sarà alla fine della riga, mentre quella di chiusura a capo;
		\item Prima e dopo ogni operatore dovrà esserci uno spazio;
		\item I blocchi saranno tra loro spaziati per una maggiore comprensione;
		\item Il codice sorgente dovrà essere quanto più suddiviso in file e dovrà essere raggruppato in sottocartelle che dovranno rispecchiare il pattern utilizzato.
%		\item \textcolor{red}{Se ce ne sono altre le inseriamo qui. Da pensare al pattern, }
	\end{itemize}
\subsubsection{Strumenti}
Gli strumenti utilizzati durante la fase dei processi primari sono:
\begin{itemize}
	\item \textbf{TexStudio}
	\newline Il gruppo ha scelto \emph{TexStudio} come editor multipiattaforma per comporre i documenti in \LaTeX;
	\item \emph{\textbf{\emph{SWEgo}}}\ped{G}\footnote{\href{https://www.swego.it/}{https://www.swego.it/}}
	~\\ Il gruppo ha scelto \emph{SWEgo} come database per la gestione dei casi d'uso, dei requisiti e del loro tracciamento per il documento \AdR{}. Dopo un'attenta analisi preliminare dello strumento il team ha deciso di utilizzarlo nonostante alcuni problemi rilevati nella generazione del codice \LaTeX{}, perché permette una stesura standardizzata e sempre aggiornata di alcuni capitoli importanti del documento, a fronte di correzioni minori e applicabili in maniera seriale e regolamentata;
	\item \textbf{Astah}\ped{G}
	~\\Il gruppo ha scelto l'utilizzo di \emph{Astah} per la generazione dei diagrammi dei casi d'uso dell'\AdR{};
	\item \textbf{Microsoft Excel 365}
	\newline Il gruppo ha scelto di utilizzare \emph{Microsoft Excel 365}\ped{G} per la generazione di grafici e di tabelle da poter inserire all'interno del \PdQ{} e \PdP{} perché di immediata realizzazione e visualizzazione.
	
\end{itemize}	

