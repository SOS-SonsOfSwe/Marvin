% !TeX spellcheck = de_DE
\newpage
\section{Processi di Supporto}
\subsection{Processo di documentazione}
Durante lo svolgimento del capitolato, si dovrà rendere conto, tramite una documentazione dettagliata, di tutti i processi che saranno coinvolti. Per questo motivo, il team suddividerà i documenti in:
\begin{itemize}
	\item \textbf{Documenti interni}
	\item \textbf{Documenti esterni}
	Da specificare la distribuzione a committenti e proponenti.
\end{itemize}
\subsubsection{Processo di garanzia della qualità}
\textcolor{red}{Cosa scriviamo qui? Nell'ISO 1995 è al paragrafo 6.3, da verificare nell'ISO aggiornato.}

\subsubsection{Ciclo di vita di un documento}
Un documento passerà attraverso tre stati:
\begin{itemize}
	\item \textbf{In lavorazione}
	Si tratta della fase di stesura del documento e non è consultabile.
	\item \textbf{Da verificare}
	Dopo che il documento è stato ultimato, passerà nelle mani del \ver, che dovrà esaminarlo.
	\item \textbf{Approvato}
	Dopo la verifica, il documento dovrà essere approvato definitivamente dal \RdP.
\end{itemize}
Ogni documento sarà identificato con un flag alla fine del nome, distanziato con un underscore, in base allo stato in cui si trova. Per il primo, si userà \emph{\textunderscore{L}}, per il secondo \emph{\textunderscore{V}}, per il terzo \emph{\textunderscore{A}}.

\subsubsection{Documenti finali}
Tra i documenti ad uso interno saranno presenti:
	\paragraph{Studio di fattibilità (SdF)}
	Lo \SdF{} ha lo scopo di raccogliere le informazioni salienti dei capitolati proposti, esprimendone gli aspetti positivi e le potenziali criticità che sono emerse durante il confronto col gruppo.
	\paragraph{Norme di progetto (NdP)}
	Le \NdP{} contengono le regole che il team utilizzerà durante lo sviluppo del progetto.
	\paragraph{Verbale interno (VI)}
	I \VI{} serviranno al gruppo per documentare le discussioni e le decisioni prese durante le riunioni. La denominazione dovrà essere come segue:\\
	\begin{center}
		\textit{verbale\textunderscore{Numero del verbale}\textunderscore{Data del verbale}\textunderscore{Tipo del verbale}} 
	\end{center}
	dove:
	\begin{itemize}
		\item \textbf{Numero del verbale}
		Numero univoco identificativo del verbale;
		\item \textbf{Tipo del verbale}
		Specifica se \emph{Interno} o \emph{Esterno}
		\item \textbf{Data del verbale}
		Identifica la data in cui la riunione si è svolta. Si utilizzerà il formato:
		\begin{center}
			\emph{YYYY-MM-DD}
		\end{center}
	\end{itemize}
	Nella parte introduttiva del verbale verranno specificati:
	\begin{itemize}
		\item \textbf{Data riunione:}
		\item \textbf{Ora inizio riunione:}
		\item \textbf{Ora fine riunione:}
		\item \textbf{Durata riunione:}
		\item \textbf{Luogo d'incontro:}
		\item \textbf{Oggetto di discussione:}
		\item \textbf{Moderatore:}
		\item \textbf{Segretario:}
		\item \textbf{Partecipanti:}
	\end{itemize}
	
Tra i documenti ad uso esterno saranno presenti:
\paragraph{Piano di Progetto (PdP)}
Il \PdP{} contiene indicazioni sulle scadenze temporali e fornisce un preventivo dei costi da presentare al proponente. Vengono inoltre individuati i rischi e analizzate le loro occorrenze.
In questo documento vengono fatte emergere le \emph{milestone} legate ai punti critici e viene effettuata una pianificazione con l'uso di diagrammi di Gant.
\paragraph{Piano di Qualifica (PdQ)} 
Il \PdQ{} deve fornire ai membri del gruppo tutte le informazioni con cui poter soddisfare gli obiettivi di qualità.
\paragraph{Analisi dei Requisiti (AdR)}
L'\AdR{} fornisce la lista dei casi d'uso, i diagrammi delle utilità tra utente e sistema sviluppato e tutti i servizi offerti dal prodotto. Lo scopo è quindi quello di dare una visione generale dei requisiti e dei casi d'uso. \textcolor{red}{Da ampliare quando avremo fatto il documento}
\paragraph{Specifica Tecnica (ST)}
La \SP{} si occupa di dare una descrizione ad alto livello del prodotto, descrivendo pregi e difetti delle sue tecnologie. \textcolor{red}{Da ampliare quando avremo fatto il documento}
\paragraph{Definizione di Prodotto (DdP)}
La \DdP{} descrive i dettagli implementativi del prodotto, andando a definire anche le funzioni delle componenti terminali del sistema tramite diagrammi UML.
\paragraph{Glossario (G)}
Nel documento \G{} i termini tecnici, gli acronimi e le abbreviazioni sono definiti in modo chiaro e conciso, in modo tale da evitare ambiguità e massimizzare la comprensione dei documenti.
\paragraph{Manuale Utente (MU)}
Il \MU{} è un manuale per aiutare l'utente nell'utilizzo del prodotto e cresce durante il suo sviluppo. Deve avere un approccio incentrato sulle funzionalità che esso offre.
\paragraph{Manuale Manutentore (MM)}
\textcolor{red}{Dobbiamo vedere se inserirlo, in quanto non dovremmo occuparci di manutenzione.}
\paragraph{Verbale Esterno}
Il \VE{} è un documento in cui si tiene traccia delle discussioni del team con i committenti e proponenti. Come struttura ricalca quella del \VI.

\subsubsection{Struttura del documento}
\paragraph{Prima pagina}
La prima pagina di ogni documento sarà così strutturata:
\begin{itemize}
	\item Logo
	\item Nome del documento
	\item Nome del gruppo \emph{-} Nome del progetto
	\item Email del gruppo
	\item Informazioni sul documento
	\begin{itemize}
		\item Versione del documento
		\item Redazione
		\item Verifica
		\item Approvazione
		\item Uso
		\item Distribuzione
	\end{itemize}
	\item Descrizione del documento.
\end{itemize}
\paragraph{Diario delle modifiche}
In seconda pagina il documento conterrà il diario delle modifiche, che traccerà le modifiche del documento. Sarà organizzato in una tabella così strutturata:
\textcolor{red}{Capire se serve la descrizione. In caso negativo togliere il grassetto}
\begin{itemize}
	\item \textbf{Versione}
	\item \textbf{Descrizione}
	\item \textbf{Autore}
	\item \textbf{Ruolo}
	\item \textbf{Data}
\end{itemize}
\paragraph{Indice}
Dopo il Diario delle Modifiche, il documento sarà correlato da un Indice di tutte le sezioni. In alcuni documenti, se necessario, sarà aggiunto anche l'indice delle immagini, delle tabelle e dei riferimenti.
\paragraph{Formattazione generale della pagina}
Ogni pagina del documento, fatta eccezione della prima, conterrà una intestazione e un pié di pagina.
\\L'intestazione conterrà a sinistra il logo e a destra la mail del gruppo.
\\Il pié di pagina ci sarà il nome del documento e il nome del gruppo e a destra il nome della pagina sul numero totale.
\subsubsection{Norme tipografiche}
Tutti i documenti dovranno sottostare alle seguenti norme tipografiche e ortografiche.
\paragraph{Formati}
\begin{itemize}
	\item \textbf{Data:} il formato della data seguirà quello esplicato nell'\emph{ISO\ped{G} 8601:2004\ped{G}}, quindi sarà: 
	\begin{center}
		\emph{YYYY-MM-DD}
	\end{center}
	dove i simboli stanno per:
	\begin{itemize}
		\item YYYY: anno;
		\item MM: mese;
		\item DD: giorno:
	\end{itemize}
	\item \textbf{Orario:} ci si atterrà allo standard europeo delle 24 ore:
	 \begin{center}
	 	\emph{hh:mm}
	 \end{center}
 	dove i simboli stanno per:
 	\begin{itemize}
 		\item hh: ore;
 		\item mm: minuti;
 	\end{itemize}
	\item{Nome del documento}
	\begin{itemize}
	\item \textbf{Nome del gruppo}: per riferirsi al nome del gruppo si dovrà
	utilizzare il comando garantendo in questo modo la corretta sintassi;
	\item \textbf{Nome del progetto}: per riferirsi al nome del progetto si dovrà
	utilizzare il comando garantendo in questo modo la corretta sintassi;
	\item \textbf{Link sito del gruppo}: per riferirsi al link del sito del gruppo si dovrà
	utilizzare il comando garantendo in questo modo la corretta sintassi;
	\item \textbf{Email del gruppo}: per riferirsi all'indirizzo email del gruppo si dovrà
	utilizzare il comando garantendo in questo modo la corretta sintassi;
	\item \textbf{Nome del proponente}: per riferirsi al nome del proponente, ovvero del proponente, si dovrà
	utilizzare il comando garantendo in questo modo la corretta
	sintassi.
	\end{itemize}
\end{itemize}
\paragraph{Composizione del testo}
\begin{itemize}
	\item \textbf{Elenchi puntati}: ogni punto dell'elenco deve terminare con il punto e virgola,
	tranne l'ultimo che deve terminare con il punto. La prima parola deve avere la lettera
	maiuscola;
	\item \textbf{Glossario}: il pedice \ped{G} verrà utilizzato in corrispondenza di vocaboli presenti nel \textit{\G}.
\end{itemize}
\paragraph{Stili di testo}
\begin{itemize}
	\item \textbf{Grassetto}: Il grassetto deve essere utilizzato per evidenziare parole
	particolarmente importanti, negli elenchi puntati o nelle frasi; \textcolor{red}{controllare dovunque!}
	\item \textbf{Corsivo}: Il corsivo deve essere utilizzato nelle seguenti
	situazioni: \textcolor{red}{da controllare questi punti}
	\begin{itemize}
		\item Ruoli: ogni riferimento a ruoli di progetto va scritto in corsivo;
		\item Documenti: ogni riferimento a un documento va scritto in corsivo;
		\item Stati del documento: ogni stato del documento va scritto in corsivo;
		\item Citazioni: ogni citazione va scritta in corsivo;
		\item Glossario: ogni parola presente nel glossario, oltre ad avere un pedice, deve
		essere scritta in corsivo.
	\end{itemize}
\end{itemize}
\paragraph{Sigle}
\textcolor{red}{Servono?}
\subsubsection{Composizione email}
Le email dovranno essere utilizzate principalmente per le comunicazioni esterne e ufficiali. Il mittente dovrà essere obbligatoriamente \emph{\Indirizzo}, mentre il destinatario potranno essere il Prof. Tullio Vardanega, il Prof. Riccardo Cardin o i proponenti del progetto. Le email si articoleranno inoltre in:
\begin{itemize}
	\item \textbf{Oggetto}
	Dovrà essere conciso e preciso, che rimandi immediatamente all'argomento trattato nel corpo.
	\item \textbf{Corpo}
	Si dovrà puntare in tutto e per tutto alla sintesi, senza dilungarsi in dettagli inutili.
	\item \textbf{Allegati}
	E' altamente sconsigliato l'invio di allegati direttamente tramite email. Se possibile, e in accordo con il destinatario, si preferirà l'invio tramite link da Google Drive.
\end{itemize} 
\subsubsection{Componenti grafiche}
\textcolor{red}{Da controllare, quando inizieremo ad usarle}
\begin{itemize}
	\item Tabelle
	\item Immagini
\end{itemize}
\subsubsection{Versionamento}
\subsubsection{Strumenti}
\subsection{Processo di verifica}
