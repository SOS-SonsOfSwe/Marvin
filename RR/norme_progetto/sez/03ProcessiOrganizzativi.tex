\section{Processi organizzativi}
\subsection{Processo di coordinamento}

\subsubsection{Comunicazioni}
\subsubsection{Comunicazioni interne}
	Per gestire le comunicazini interne si è riscontrata la necessità di distinguere la modalità di presentazione delle informazioni in:
	\begin{itemize}
	\item\textbf{Comunicazione formale}: questa modalità viene usata per le discussioni ufficiali ed inerenti a gestione e sviluppo del progetto. 
	\item\textbf{Comunicazione informale}: questa modalità viene usata per scambiare informazioni non ufficiali e discutere riguardo le attività di progetto.
	\end{itemize}
	
	Si è deciso di utilizzare i seguenti strumenti:
	\begin{itemize}
	\item\textbf{\textit{Telegram}}\ped{G}, servizio di messaggistica istantanea multipiattaforma.
	\item\textbf{\textit{Slack}}\ped{G}, servizio di comunicazione pensato per facilitare il lavoro di gruppo tramite canali distinti.   	
	\end{itemize}
	
\subsubsection{Comunicazioni esterne}
\begin{itemize}
\item\textbf{E-mail}: E' stata creato l'indirizzo di posta elettronica:
\textbf{sonsofswe.swe@gmail.com}

L'utilizzo di tale casella di posta è affidato unicamente al \textit{responsabile}, per la gestione delle relazioni con committente e proponente del progetto.

Le e-mail dovranno avere la seguente forma:
\begin{itemize}
\item\textbf{Oggetto}: L'oggetto dev'essere chiaro e conciso, per riconoscere e distinguere facilmente tra loro le mail.
\item\textbf{Apertura}: Ogni mail deve iniziare con un aggettivo di circostanza seguito dal nome preceduto dal titolo del destinatario, e terminare con una virgola.
\item\textbf{Corpo}: Il corpo dev'essere breve ma esaustivo; nella prima parte verrà spiegata la ragione per cui si scrive per garantire la corretta comprensione del messaggio. 
\item\textbf{Allegati}: E' possibile l'invio di allegati tramite mail, su richiesta del proponente o del committente.
\item\textbf{Chiusura}: La chiusura deve essere separata dal corpo con il doppio ritorno a capo e prevede un congedo formale e la firma del mittente.
\end{itemize}

\item\textbf{Slack}: Su richiesta del committente verrà utilizzato per lo scambio di informazioni in maniera ufficiosa.
\end{itemize}


