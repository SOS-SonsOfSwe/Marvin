\newpage
\section{Processi organizzativi}
\subsection{Processo di coordinamento}

\subsubsection{Comunicazioni}
\subsubsection{Comunicazioni interne}
	Per gestire le comunicazini interne si è riscontrata la necessità di distinguere la modalità di presentazione delle informazioni in:
	\begin{itemize}
	\item\textbf{Comunicazione formale}: questa modalità viene usata per le discussioni ufficiali ed inerenti a gestione e sviluppo del progetto; 
	\item\textbf{Comunicazione informale}: questa modalità viene usata per scambiare informazioni non ufficiali e discutere riguardo le attività di progetto.
	\end{itemize}
	
	Si è deciso di utilizzare i seguenti strumenti:
	\begin{itemize}
	\item\textbf{\textit{Telegram}}\ped{G}: servizio di messaggistica istantanea multipiattaforma;
	\item\textbf{\textit{Slack}}\ped{G}: servizio di comunicazione pensato per facilitare il lavoro di gruppo tramite canali distinti.   	
	\end{itemize}
	
\subsubsection{Comunicazioni esterne}
\begin{itemize}
\item\textbf{email}: è stato creato l'indirizzo di posta elettronica:
\textbf{\emailgruppo}

L'utilizzo di tale casella di posta è affidato unicamente al \RdP{} per la gestione delle relazioni con committente e proponente del progetto.

Le email dovranno avere la seguente forma:
	\begin{itemize}
	\item\textbf{Oggetto}: l'oggetto dev'essere chiaro e conciso, per riconoscere e distinguere facilmente tra loro le email;
	\item\textbf{Apertura}: ogni email deve iniziare con un aggettivo di circostanza seguito dal nome preceduto dal titolo del destinatario e terminare con una virgola;
	\item\textbf{Corpo}: nel corpo, che deve essere breve ed esaustivo, verranno spiegate le ragioni per cui si sta scrivendo l'email;
	\item\textbf{Allegati}: è possibile l'invio di allegati, su richiesta del proponente o del committente;
	\item\textbf{Chiusura}: la chiusura deve essere separata dal corpo con il doppio ritorno a capo e prevede un congedo formale e la firma del mittente.
	\end{itemize}

\item\textbf{Slack}: verrà utilizzato, su richiesta del committente, per lo scambio di informazioni in maniera ufficiosa.
\end{itemize}

\subsection{Riunioni}
\subsubsection{Obiettivi}
Le riunioni sono di fondamentale importanza per la gestione di un progetto. È indispensabile che i membri del team abbiano
modo di confrontarsi tra loro al fine di risolvere i problemi esistenti e generare nuove idee. Una riunione è inoltre un
ottimo mezzo per creare armonia e consolidare i rapporti all'interno del gruppo. Le riunioni con il proponente/i o committente/i avranno invece lo scopo di condividere gli obiettivi raggiunti e di confrontarsi nel caso in cui si presenti
la necessità di colmare lacune. 
\subsubsection{Riunioni interne}
Per un produttivo svolgimento delle riunioni interne, verranno rispettati i seguenti ruoli:
\begin{itemize}
\item\textbf{Moderatore}: il \RdP{} ha il dovere di convocare il team alle riunioni interne quando necessario per un confronto tra i vari membri;
durante gli incontri è richiesto che tutti i membri siano presenti, salvo eccezioni precedentemente segnalate.
Il \RdP{} avrà quindi l'obbligo di seguire l'ordine del giorno riguardo gli argomenti da affrontare,
\textcolor{red}{da fungere da moderatore QUESTO IL MODERATORE LO FA?}, e di trovare un consenso unanime riguardo le decisioni da prendere.
Per svolgere il prorio compito in maniera adeguata, il moderatore è tenuto ad avere un atteggiamento autorevole, flessibile
e diligente.
\item\textbf{Segretario}\ped{G}: il \emph{Segretario} ha il compito di stendere una prima minuta dell'incontro, controllare che venga seguito punto per punto l'ordine del giorno e redigere la versione finale del \emph{verbale}\ped{G}.
Conclusa una riunione, il \emph{Segretario} avrà inoltre il compito di fornire una copia del verbale ad ogni membro del team,
comprendente un resoconto delle decisioni prese e degli obiettivi prefissati.
\item\textbf{Partecipanti}: i partecipanti sono tenuti a presenziare puntualmente qualora venga convocata una riunione da parte del \RdP{}. Nel caso in cui un membro sia impossibilitato a partecipare è invitato
a comunicarlo tempestivamente al \RdP, al fine di accordarsi su come procedere. È richiesto ai partecipanti di
tenere un comportamento diligente e responsabile per un corretto svolgimento dell'assemblea.
\end{itemize}

\paragraph{Descrizione}
~\\Le riunioni dovranno avere cadenza settimanale, per fare il punto della situazione riguardo gli obettivi prefissati e
le difficoltà incontrate. Il \RdP{}, dovrà informare i vari membri della riunione tramite email e questi saranno tenuti a rispondere confermando la presenza o motivando l'assenza. Una riunione verrà effettivamente considerata
valida solo nel caso in cui siano presenti almeno la metà dei membri convocati. Nel caso contrario, sarà proposta una seconda data all'interno della settimana per ripetere e validare l'incontro.
Le riunioni dovranno concentrarsi su quattro punti cardine:
\begin{itemize}
\item Informare i membri riguardo le novità.
\item \textcolor{red}{mancano gli altri tre punti}
\end{itemize}
\subsubsection{Riunioni esterne}






	