\newpage
\section{Processi organizzativi}
\subsection{Processo di coordinamento}

\subsubsection{Comunicazioni}
\paragraph{Comunicazioni interne}
	Per gestire le comunicazini interne si è riscontrata la necessità di distinguere la modalità di presentazione delle informazioni in:
	\begin{itemize}
	\item\textbf{Comunicazione formale}: questa modalità viene usata per le discussioni ufficiali ed inerenti a gestione e sviluppo del progetto; 
	\item\textbf{Comunicazione informale}: questa modalità viene usata per scambiare informazioni non ufficiali e discutere riguardo le attività di progetto.
	\end{itemize}
	
	Si è deciso di utilizzare i seguenti strumenti:
	\begin{itemize}
	\item\textbf{\textit{Telegram}}\ped{G}.
	\item\textbf{\textit{Slack}}\ped{G}.
	\end{itemize}
	
\paragraph{Comunicazioni esterne}
\begin{itemize}
\item\textbf{Email}: è stato creato l'indirizzo di posta elettronica:
\textbf{\emailgruppo}

L'utilizzo di tale casella di posta è affidato unicamente al \RdP{} per la gestione delle relazioni con committente e proponente del progetto.

Le email dovranno avere la seguente forma:
	\begin{itemize}
	\item\textbf{Oggetto}: l'oggetto dev'essere chiaro e conciso, per riconoscere e distinguere facilmente tra loro le email;
	\item\textbf{Apertura}: ogni email deve iniziare con un aggettivo di circostanza seguito dal nome preceduto dal titolo del destinatario o con un saluto formale e terminare con una virgola;
	\item\textbf{Corpo}: nel corpo, che deve essere breve ed esaustivo, verranno spiegate le ragioni per cui si sta scrivendo l'email;
	\item\textbf{Allegati}: è possibile l'invio di allegati, su richiesta del proponente o del committente;
	\item\textbf{Chiusura}: la chiusura deve essere separata dal corpo con il doppio ritorno a capo e prevede un congedo formale e la firma del mittente.
	\end{itemize}

\item\textbf{Slack}: verrà utilizzato, su richiesta del committente, per lo scambio di informazioni in maniera ufficiosa.
\end{itemize}

\subsubsection{Riunioni}
\paragraph{Obiettivi}
Le riunioni sono di fondamentale importanza per la gestione di un progetto. È indispensabile che i membri del team abbiano
modo di confrontarsi tra loro al fine di risolvere i problemi esistenti e generare nuove idee. Una riunione è inoltre un
ottimo mezzo per creare armonia e consolidare i rapporti all'interno del gruppo. Le riunioni con il proponente/i o committente/i avranno invece lo scopo di condividere gli obiettivi raggiunti e di confrontarsi nel caso in cui si presenti
la necessità di colmare lacune. 
\paragraph{Riunioni interne}
Per un produttivo svolgimento delle riunioni interne, verranno rispettati i seguenti ruoli:
\begin{itemize}
\item\textbf{Moderatore}: il \RdP{} ha il dovere di convocare il team alle riunioni interne quando necessario per un confronto tra i vari membri;
durante gli incontri è richiesto che tutti i membri siano presenti, salvo eccezioni precedentemente segnalate.
Il \RdP{} avrà quindi l'obbligo di seguire l'ordine del giorno riguardo gli argomenti da affrontare,
\textcolor{red}{da fungere da moderatore QUESTO IL MODERATORE LO FA?}, e di trovare un consenso unanime riguardo le decisioni da prendere.
Per svolgere il prorio compito in maniera adeguata, il moderatore è tenuto ad avere un atteggiamento autorevole, flessibile
e diligente.
\item\textbf{Segretario}\ped{G}: il \emph{Segretario} ha il compito di stendere una prima minuta dell'incontro, controllare che venga seguito punto per punto l'ordine del giorno e redigere la versione finale del \emph{verbale}\ped{G}.
Conclusa una riunione, il \emph{Segretario} avrà inoltre il compito di fornire una copia del verbale ad ogni membro del team,
comprendente un resoconto delle decisioni prese e degli obiettivi prefissati.
\item\textbf{Partecipanti}: i partecipanti sono tenuti a presenziare puntualmente qualora venga convocata una riunione da parte del \RdP{}. Nel caso in cui un membro sia impossibilitato a partecipare è invitato
a comunicarlo tempestivamente al \RdP, al fine di accordarsi su come procedere. È richiesto ai partecipanti di
tenere un comportamento diligente e responsabile per un corretto svolgimento dell'assemblea.
\end{itemize}

\subparagraph{Descrizione}
~\\Le riunioni dovranno avere cadenza settimanale, per fare il punto della situazione riguardo gli obettivi prefissati e
le difficoltà incontrate. Il \RdP{}, dovrà informare i vari membri della riunione tramite email e questi saranno tenuti a rispondere confermando la presenza o motivando l'assenza. Una riunione verrà effettivamente considerata
valida solo nel caso in cui siano presenti almeno la metà dei membri convocati. Nel caso contrario, sarà proposta una seconda data all'interno della settimana per ripetere e validare l'incontro.
Le riunioni dovranno concentrarsi su quattro punti cardine:
\begin{itemize}
	\item Informare i membri riguardo le novità.
	\item Valutare il lavoro precedentemente fatto.
	\item Prendere decisioni collettive riguardo eventuali problematiche.
	\item Progettare il percorso per raggiungere gli obiettivi prefissati entro le tempistiche stabilite.
\end{itemize}
Al termine di ogni riunione verrà quindi redatto il corrispondente verbale, ed inviato per mail a tutti i membri del team.

\paragraph{Riunioni esterne}
\subparagraph{Descrizione}
Le riunioni con il proponente hanno la funzione di supportare il gruppo di progetto e di consolidare il rapporto tra entrambe le parti. Durante le riunioni, il gruppo avrà l'impegno di condividere gli obiettivi raggiunti e le eventuali problematiche incontrate. 
Il \RdP{} avrà il dovere di convocare le riunioni quando necessario, anche confrontandosi in base alle esigenze dei membri del team. Il segretario scelto invece avrà il compito di verbalizzare quanto emerso dalla discussione, e le possibili richieste di cambiamento e/o correzione emesse dal proponente.

\subsection{Processo di pianificazione}
\subsubsection{Descrizione}
Lo sviluppo di un progetto prevede la cooperazione di diversi ruoli. Per una comprensione unanime del progetto, ogni membro del team dovrà esercitare obbligatoriamente tutti i ruoli previsti, durante periodi temporali differenti, che saranno di seguito elencati. Nel caso in cui sorga la necessità, un membro potrà ricoprire più di un ruolo contemporaneamente, ma solo nel caso in cui non si tratti di redattore e verificatore. Questo caso in particolare è considerato un controsenso, in quanto la figura del verificatore risulterebbe corrotta nell'analisi dei propri documenti. Inoltre, ogni ruolo avrà incarichi diversi, i quali verranno gestiti ed assegnati dal \RdP{} con l'ausilio di opportuni strumenti, in modo da garantire monitoraggio, controllo e revisione del progetto per tutta la sua durata.
\subsubsection{Ruoli}
\paragraph{Responsabile}
Il \RdP{} è colui che detiene la responsabilità sul lavoro svolto dal team, mantiene i contatti esterni, e presenta al committente il progetto finale. Egli possiede il potere decisionale, e si fa carico dei seguenti oneri:
\begin{itemize}
	\item Pianificazione, coordinamento e controllo delle attività.
	\item Gestione e controllo delle risorse.
	\item Analisi e gestione dei rischi.
	\item Approvazione della documentazione.
	\item Contatti con gli enti esterni.
\end{itemize}
Di conseguenza il \RdP{} ha il compito di assicurarsi che le attività di verifica e validazione vengano svolte in riferimento alle \emph{Norme di progetto}, di redigere l'\emph{{{Organigramma}}\ped{G}}, di garantire che vengano rispettati i ruoli assegnati all'interno del \emph{Piano di progetto}, e di garantire che non vi siano conflitti tra redattori e verificatori. 

\paragraph{Analista}
L'\ana si occupa di capire il problema da affrontare, ascoltando le richieste del committente; è un individuo esperto, in grado di capire il dominio e la complessità del problema. Le sue mansioni principali sono:
\begin{itemize}
	\item Analizzare le qualità e i servizi che dovrà offrire il prodotto finale.
	\item Valutare la fattibilità del progetto, riportando tali valutazioni nel \SdF{}.
	\item Redigere l'\AdR{}, in cui verranno indicati tutti i requisiti del progetto individuati.
\end{itemize}

\paragraph{Amministratore}
L'\adm{} è responsabile della gestione dell'ambiente di lavoro; deve offrire al gruppo più facilitazioni e automazioni possibili al fine di incrementare l'operatività e l'efficienza. I suoi doveri sono:
\begin{itemize}
\item Gestione della documentazione di progetto.
\item Controllo di versioni e configurazioni.
\item Ricerca e gestione di strumenti di supporto che facilitino l'operato del gruppo.
\item Redazione delle \NdP{} e supporto alla redazione del \PdP{}.
\end{itemize}

\paragraph{Progettista}
Il \prog{} ha il compito di gestire la progettazione vera e propria, sfruttando le sue competenze e conoscenze in ambiti tecnico e tecnologico; il suo ruolo è direttamente collegato a quello dell'\ana, in quanto deve capire e spiegare come risolvere i problemi identificati precedentemente dagli \anas. Ha il compito di:
\begin{itemize}
\item Influenzare le scelte tecniche e tecnologiche, per:
\begin{itemize}
\item Orientare il gruppo all'utilizzo di strumenti e servizi il più possibile efficienti. 
\item Effettuare scelte progettuali atte a garantire la manutenibilità e la modularità del prodotto finale.
\end{itemize} 
\item Redigere \ST{}, \DdP{} e la parte programmatica del \PdQ{}.
\end{itemize}

\paragraph{Programmatore}
Il \progr{} avrà il compito di codificare e mantenere il codice prodotto, implementando le soluzioni proposte dal \prog{}: deve perciò avere alte competenze tecniche. I suoi compiti sono:
\begin{itemize}
\item Implementare in maniera rigorosa quanto richiesto dai \progs{}.
\item Implementare componenti aggiuntive allo scopo di creare strumenti di test e verifica del prodotto.
\item Redigere eventuali \MU{} ed \MM{}.
\end{itemize}

\paragraph{Verificatore}
Il \ver{} sarà responsabile della verifica, vale a dire del continuo controllo del prodotto e della documentazione: sarà attivo per tutta la durata del progetto ed agirà sfruttando le proprie capacità di giudizio, esperienza e competenza. I suoi compiti sono:
\begin{itemize}
\item Accertarsi del rispetto delle \NdP{} e della conformità al \PdQ{}.
\item Redigere il \PdQ{}.
\end{itemize}

\subsubsection{Ticketing}
Al fine di permettere una migliore gestione del lavoro interno al gruppo, verrà utilizzato lo strumento \textbf{Asana}, utile a creare ed assegnare \emph{Task}\ped{G}; in questo modo il \RdP  sarà in grado di tenere costantemente sotto controllo l'avanzamento delle attività di progetto. Complementariamente verrà usato \textbf{InstaGantt}, strumento che permette di creare \emph{diagrammi di Gantt}\ped{G} basandosi su ciò che viene dichiarato in Asana.
\paragraph{Procedura di assegnazione}
L'assegnazione di un Task coinciderà con la sua creazione e si basa sulla seguente sequenza di passi:
\begin{itemize}
\item Assegnazione di un titolo al task;
\item Assegnazione del task stesso al/ai membro/i designato/i.
\item Aggiunta di una breve ma concisa descrizione del compito e dei suoi obiettivi finali.
\item Inserimento delle date di inizio e fine.
\item Impostazione dello stato del ticket ad "Aperto".
\end{itemize}

\subparagraph{Possibile stato di un ticket}
Un ticket può essere nei seguenti stati:
\begin{itemize}
\item Aperto;
\item In elaborazione;
\item Sospeso;
\item Completato/Risolto;
\end{itemize}

\subparagraph{\color{red}{aggiungere diagramma uml???}}



\subsection{Processo dell'infrastruttura}

\subsubsection{Ambienti di sviluppo}
\paragraph{Sistemi operativi}
	~\\~\\Ogni componente del gruppo avrà la possibilità di utilizzare il sistema operativo che più gli aggrada, a patto che i servizi offerti siano gli stessi. In particolare, verranno utilizzati:
	\begin{itemize}
		\item Windows 10 Pro x64;
		\item Windows 10 Home x64;
		\item Windows 10 Education x64;
		\item Ubuntu 17.04 x64;
		\item Ubuntu Mate 16.04 x64.
		\item OS X El Capitan versione 10.11.6;
	\end{itemize}
	
\subsubsection{Strumenti}
\paragraph{Telegram} 
	~\\~\\ Telegram è un servizio di messaggistica istantanea multipiattaforma.
	\footnote{\href{https://telegram.org/}{https://telegram.org/}}
\paragraph{Slack}
	~\\~\\ Slack è un servizio di comunicazione multipiattaforma pensato per facilitare il lavoro di gruppo tramite canali distinti. Fornisce inoltre la possibiltà di integrare servizi esterni, come Github e Google Drive.
	\footnote{\href{https://slack.com/}{https://slack.com/}}

\paragraph{Google Drive}
	~\\~\\Google Drive è un servizio di memorizzazione e archiviazione online fornito da Google basato sul \emph{Cloud}\ped{G}. Esso verrà utilizzato dal gruppo per lo scambio di documenti ed informazioni di supporto allo sviluppo del progetto ma non soggetti al versionamento.
	\footnote{\href{https://www.google.com/drive/}{https://www.google.com/drive/}}
	

\paragraph{Fogli Google}
	~\\~\\Fogli Google è un servizio offerto da Google per la creazione di fogli di lavoro online modificabili in contemporanea da chiunque ne abbia accesso. Il gruppo utilizzerà questo strumento per il rendiconto delle ore di lavoro.
	\footnote{\href{https://www.google.it/intl/it/sheets/about/}{https://www.google.it/intl/it/sheets/about/}}
	
\paragraph{Github}
	~\\~\\Github è un' implementazione dello strumento di controllo di versione Git ed offre un servizio di \emph{hosting}\ped{G} per progetti software. Il gruppo utilizzerà un repository comune denominato \textit{Marvin} avente come proprietario il profilo \textit{SOS-SonsOfSwe} , le cui credenziali saranno rese note a tutti i componenti; ognuno potrà inoltre apportare le sue modifiche direttamente dal proprio account personale.
	\footnote{\href{https://github.com/}{https://github.com/}}
	
\paragraph{Asana}
	~\\~\\Asana è un servizio web utile a migliorare la collaborazione all'interno di un gruppo di lavoro, dando la possibilità di gestire i task di ogni componente del team online. 
In esso è possibile:
\begin{itemize}
\item Suddividere il progetto sezioni, in modo da dividere gli ambiti di lavoro;
\item Suddividere le sezioni in task veri e propri;
\item Ipostare per ogni task scadenze diverse; 
\item Impostare dipendenze tra le parti, in modo che un task non possa essere completato se uno o più altri, da cui \emph{dipende}, non sono stati completati in precedenza;
\item Suddividere i task in sotto-task, qualora il compito risulti troppo complesso e necessiti dunque un'ulteriore modularizzazione.
\item Avere un calendario delle scadenze sempre aggiornato;
\item Avere una completa visione della distribuzione del carico di lavoro all'interno del team;
\end{itemize}
Per gruppi di studenti, Asana offre la propria versione premium gratuitamente.
	\footnote{\href{https://asana.com/}{https://asana.com/}}

\paragraph{Instagantt}
	~\\~\\Instagantt è un servizio web fortemente legato ad Asana, strumento con il quale si integra perfettamente, nonostante sia disponibile anche una versione \emph{standalone}\ped{G}. Esso permette di creare diagrammi di Gantt e gestire in maniera semplificata la timeline e la struttura di un progetto.
	\footnote{\href{https://instagantt.com/}{https://instagantt.com/}}


	