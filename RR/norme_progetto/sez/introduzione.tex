\newpage
\section{Introduzione}

\subsection{Scopo del documento}
In questo documento verranno trattate le norme interne alle quali i membri di \gruppo{} dovranno obbligatoriamente sottostare. Ogni membro dovrà visionare il documento e seguirne le regole in esso contenute, per ottenere la massima \emph{efficienza}\ped{G} ed \emph{efficacia}\ped{G}, mantenendo un certo grado di uniformità.
\newline In questo documento verranno esposte le norme riguardanti:
\begin{itemize}
	\item L'identificazione dei ruoli e delle relative mansioni che essi dovranno svolgere;
	\item Le modalità di lavoro durante le fasi di \emph{progetto}\ped{G};
	\item Le interazioni tra i membri del gruppo e con le entità esterne;
	\item L'organizzazione e la cooperazione all'interno del \emph{team}\ped{G};
	\item La modalità e le regole adottate per la stesura dei documenti;
	\item La definizione degli ambienti di sviluppo.
\end{itemize}
\subsection{Scopo del prodotto}
Lo scopo del \emph{capitolato}\ped{G} \emph{Marvin} è di realizzare un \emph{prototipo}\ped{G} di Uniweb come \textit{ÐApp}\ped{G} che giri su \textit{Ethereum}\ped{G}. I tre attori che si rapportano con Marvin sono:
\begin{enumerate}
	\item Università;
	\item Professori;
	\item Studenti.
\end{enumerate}
Il portale deve quindi permettere agli studenti di accedere alle informazioni riguardanti le loro carriere universitarie, di iscriversi a esami, accettare o rifiutare voti e devono poter vedere il loro libretto universitario.
\newline Ai professori deve invece essere permesso di registrare i voti degli studenti.
\newline L'università ogni anno crea una serie di corsi di laurea rivolti a studenti, dove  ognuno di essi comprende un elenco di esami disponibili per anno accademico. Ogni esame ha un argomento, un numero di crediti e un professore associato.
\newline Gli studenti si iscrivono ad un corso di laurea e tramite il libretto mantengono traccia ufficiale del progresso.
\subsection{Glossario}
Nel documento \G{} i termini tecnici, gli acronimi e le abbreviazioni sono definiti in modo chiaro e conciso, in modo tale da evitare ambiguità e massimizzare la comprensione dei documenti.
\newline I vocaboli presenti in esso saranno posti in corsivo e presenteranno una "G" maiuscola a pedice.
\subsection{Riferimenti}
\subsubsection{Normativi}
\begin{itemize}
	\item
	\textbf{Regolamento del progetto didattico}:
	\url{http://www.math.unipd.it/~tullio/IS-1/2017/Dispense/P01.pdf};
	\item
	\textbf{\textit{ISO}}\ped{G} \textbf{31-0}: \url{http://en.wikipedia.org/wiki/ISOwiki/ISO\_31};
	\item
	\textbf{ISO 8601}: \url{http://it.wikipedia.org/wiki/ISO\_8601};
	\item
	\textbf{ISO 12207-1995}: \url{http://www.math.unipd.it/~tullio/IS-1/2009/Approfondimenti/ISO\_12207-1995.pdf}.
\end{itemize}
\subsubsection{Informativi}
\begin{itemize}
%	\item \textbf{\PdPv}; non si possono mettere perché sono dopo le norme
%	\item \textbf{\PdQv};
	\item \textbf{Amministrazione di progetto}: \url{http://www.math.unipd.it/~tullio/IS-1/2014/Dispense/P06.pdf};
	\item \textit{Specifiche \textit{UTF-8}}\ped{G}: \url{http://www.unicode.org/versions/Unicode6.1.0/ch03.pdf};
	\item \textbf{Slides del corso di Ingegneria del Software}:
	\url{http://www.math.unipd.it/~tullio/IS-1/2017/}
\end{itemize}
