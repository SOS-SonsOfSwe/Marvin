\section{Metriche}
\subsection{Metriche per il processo}
\label{AppB:metricheProc}
\subsection{Metriche per il prodotto software}
\label{AppB:metricheProd}
In questa sezione si descrivono le metriche che verranno usate dal gruppo per verificare e garantire la qualità dei prodotti software durante il periodo del progetto. Si sottolinea il fatto che questa sarà solo una prima stesura delle metriche e sarà raffinata nel corso delle varie revisioni, facendo frutto dell'esperienza che verrà acquisita negli intervalli di lavoro tra esse.

\paragraph{Requisiti soddisfatti}
	~\\Tale metrica verrà utilizzata per valutare la funzionalità del software prodotto attraverso una misurazione quantificativa dei requisiti soddisfatti; verranno effettuate due misurazioni differenti, una per i soli requisiti obbligatori e una per tutti.

\begin{description}
\item[Requisiti obbligatori]
	~\\ \begin{displaymath}
		\mbox{ROS}=\frac{\mbox{numero requisiti obbligatori soddisfatti}}{\mbox{numero totale requisiti obbligatori}}
	\end{displaymath}
	
\item[Requisiti obbligatori e facoltativi]
	~\\ \begin{displaymath}
		\mbox{ROFS}=\frac{\mbox{numero requisiti obbligatori soddisfatti} + \mbox{numero requisiti facoltativi soddisfatti}}{\mbox{numero totale requisiti}}
	\end{displaymath}
\end{description}

\subparagraph{Soglie}
\begin{itemize}
\item Accettabilità: il prodotto verrà considerato accettabile quando ROS = 1.
\item Ottimalità: il prodotto verrà considerato ottimale quando ROSF = 1.
\end{itemize}

\paragraph{}
