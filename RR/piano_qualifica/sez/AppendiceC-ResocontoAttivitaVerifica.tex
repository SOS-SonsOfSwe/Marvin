\newpage
\section{Attività di analisi requisiti utente}
\subsection{Resoconto delle attività di verifica}

Dopo aver redatto tutti i documenti presenti nella Revisione dei Requisiti, il team ha svolto le attività di verifica su di essi e sui processi analizzati. I documenti sono stati sottoposti al processo di analisi statica definito nel documento \NdP{}.
Prima è stata utilizzata la tecnica del Walkthrough, segnalando gli errori incontrati tramite una lettura approfondita in un'apposita lista presa in carico dal Verificatore per attuare la correzione del documento. In seguito la stessa lista è stata utilizzata per la tecnica dell'Inspection, che è servita ad individuare la presenza di nuovi errori utilizzando il confronto della lista di quelli commessi in precedenza.
In seguito i documenti sono stati interamente verificati secondo le metriche descritte nell'Appendice~\nameref{AppB:metric}, e sono stati riportati i risultati ottenuti.

\subsection{Verifica dei processi}
\subsubsection{Schedule Variance}
Nella seguente tabella vengono riportati i valori ottenuti calcolando la Schedule Variance sui tempi di stesura di ogni documento rispetto ai tempi prefissati nel \PdP{}:

{
	\renewcommand{\arraystretch}{2}
	\centering
	\begin{tabular}{| L{4cm} | R{0.5cm} | C{3cm} |}
	\hline
	Analisi dei Requisiti v1.0 & 0\% & Accettabile \\
	\hline
	Norme di Progetto v1.0 & +28,57\% & Non accettabile \\
	\hline
	Studio di Fattibilità v1.0 &  -50\% &  Ottimale \\
	\hline
	Piano di Progetto v1.0 &  0\% &  Accettabile\\
	\hline
	Piano di Qualifica v1.0 & 0\% & Accettabile \\
	\hline
	Glossario v1.0 & 0\% & Accettabile\\	
	\hline
	\end{tabular}
	
}

Considerazioni sui risultati ottenuti:

Schedule Variance finale: +1 giorno.
Soglia raggiunta: accettabilità.

\subsubsection{Modello SPICE}
Aspetto la descrizione migliore per descrivere.

\subsubsection{Cost Variance}
Il calcolo della Cost Variance sul \emph{processo di documentazione} ha portato il seguente risultato: 

{
\renewcommand{\arraystretch}{2}
\centering
\begin{tabular}{|c | c | c |}
	\hline
	Processo & CV & Valutazione \\
	\hline
	Documentazione & +7.48\% & Ottimale \\
	\hline
\end{tabular}

}

Cost Variance finale: +0%.
Soglia raggiunta: ottimalità.

\subsection{Verifica dei documenti}
\subsubsection{Schedule Variance}
Nella seguente tabella vengono riportati i valori ottenuti calcolando la Schedule Variance sui tempi di verifica di ogni documento rispetto ai tempi prefissati nel'\textit{Piano di Progetto}:

{
	\renewcommand{\arraystretch}{2}
	\centering
	\begin{tabular}{| c | R{0.5cm} | C{3cm} |}
		\hline
		Analisi dei Requisiti v1.0 & +2 & Accettabile \\
		Norme di Progetto v1.0 & -1 & Ottimale \\
		Studio di Fattibilità v1.0 -1 &  0 &  Ottimale \\
		Piano di Progetto v1.0 &  0 &  Ottimale\\
		Piano di Qualifica v1.0 & -1 & Ottimale \\
		Glossario v1.0 & +1 & Accettabile\\	
		\hline
	\end{tabular}

}

Schedule Variance finale: +1 giorno.
Soglia raggiunta: accettabilità.

\subsubsection{Cost Variance}
Il calcolo della Cost Variance sul \emph{processo di verifica} ha portato il seguente risultato: 

{
	\renewcommand{\arraystretch}{2}
	\centering
	\begin{tabular}{|c | c | c |}
		\hline
		Processo & CV & Valutazione \\
		\hline
		Verifica & 0\% & Accettabile \\
		\hline
	\end{tabular}

}

\subsubsection{Errori ortografici}
Sono stati rilevati errori ortografici all'interno dei documenti analizzati secondo i seguenti parametri:
{
	\renewcommand{\arraystretch}{2}
	\centering
	\begin{tabular}{| c | C{1cm} | C{3cm} |}
			\hline
		Analisi dei Requisiti v1.0 & 1\% & Accettabile \\
			\hline
		Norme di Progetto v1.0 & 2\% & Accettabile\\
			\hline
		Studio di Fattibilità v1.0 & 1\% &  Accettabile \\
			\hline
		Piano di Progetto v1.0 & 1\% & Accettabile \\
			\hline
		Piano di Qualifica v1.0 & 2\% & Accettabile\\
			\hline
		Glossario v1.0 & 1\% & Accettabile\\
			\hline
	\end{tabular}

}

Errori Ortografici calcolati: +2%.
Soglia raggiunta: accettabilità.


\subsubsection{Indice Gulpease}

Tutti i documenti consegnati sono stati sottoposti al calcolo dell'Indice Gulpease per valutarne il grado di leggibilità.
Sono stati rilevati i seguenti indici di leggibilità:

{
	\renewcommand{\arraystretch}{2}
	\centering
	\begin{tabular}{| c | C{3cm} | C{3cm} |}
			\hline
		Analisi dei Requisiti v1.0 &  mancanti & Accettabile \\
			\hline
		Norme di Progetto v1.0 & mancanti & Accettabile\\
			\hline
		Studio di Fattibilità v1.0 & mancanti & Accettabile\\ 
			\hline
		Piano di Progetto v1.0 & mancanti & Accettabile \\
			\hline
		Piano di Qualifica v1.0 & mancanti & Accettabile\\
			\hline
		Glossario v1.0 & mancanti & Accettabile\\
			\hline
	 \end{tabular}
 
}

Indice Gulpease:
Soglia raggiunta:

\subsubsection{Errori concettuali}

Nella seguente tabella vengono riportati i valori ottenuti calcolando la percentuale degli errori concettuali tramite la formula presente nell'apendice in sezione ~\nameref{AppB:ErroriCont}

%Errori concettuali & 0\% & Ottimale   

Errori concettuali calcolati:
Soglia raggiunta:
