\newpage
\section{Attività di Analisi requisiti utente \\\large{resoconto~delle~attività~di~verifica}}
\subsection{Verifica dei processi}
\subsubsection{Schedule Variance}
Nella seguente tabella vengono riportati i valori ottenuti calcolando la Schedule Variance sui tempi di stesura di ogni documento rispetto ai tempi prefissati nel'\textit{Piano di Progetto}:
Analisi dei Requisiti v1.0 & +1 & Accettabile \\
Norme di Progetto v1.0 & -1 & Ottimale \\
Studio di Fattibilità v1.0 &  -2 &  Ottimale \\
Piano di Progetto v1.0 &  0 &  Ottimale\\
Piano di Qualifica v1.0 & 0 & Ottimale \\
Glossario v1.0 & +1 & Accettabile\\	

Considerazioni finali:

	Schedule Variance finale: +1 giorno.
	In generale per il processo di documentazione il team ha completato la stesura 1 giorno in anticipo rispetto a quanto pianificato nel \doc{Piano di Progetto}. Il risultato ottenuto pertanto rientra nella soglia di accettabilità prevista.

\subsubsection{Modello SPICE}
Aspetto la descrizione migliore per descrivere.

\subsubsection{Cost Variance}
Nella seguente tabella vengono riportati i valori ottenuti calcolando la Cost Variance sui tempi di stesura di ogni documento:

	Processo di documentazione & 0\% & Ottimale\\ 	   	

Considerazioni finali: Per il processo di documentazione il team non ha necessitato di ulteriori risorse che potessero aumentare i costi pianificati precedentemente.

\subsection{Verifica dei documenti}
\subsubsection{Schedule Variance}
Nella seguente tabella vengono riportati i valori ottenuti calcolando la Schedule Variance sui tempi di verifica di ogni documento rispetto ai tempi prefissati nel'\textit{Piano di Progetto}:

	Analisi dei Requisiti v1.0 & +2 & Accettabile \\
	Norme di Progetto v1.0 & -1 & Ottimale \\
	Studio di Fattibilità v1.0 -1 &  0 &  Ottimale \\
	Piano di Progetto v1.0 &  0 &  Ottimale\\
	Piano di Qualifica v1.0 & -1 & Ottimale \\
	Glossario v1.0 & +1 & Accettabile\\	

Considerazioni finali: per il processo di verifica, in generale, il team ha ritardato di 0 giorni rispetto a quanto pianificato nel \doc{Piano di Progetto}. Pur essendoci documenti che sono stati verificati in ritardo, il valore generale ottenuto rientra nella soglia di accettabilità prevista.

\subsubsection{Cost Variance}
Nella seguente tabella vengono riportati i valori ottenuti calcolando la Cost Variance sui tempi di verifica di ogni documento rispetto ai tempi prefissati nel'\textit{Piano di Progetto}:

	Processo di verifica & 0\% & Ottimale\\	  

	Considerazioni finali: Per il processo di verifica il \gl{team} non ha necessitato di ulteriori risorse che potessero aumentare i costi pianificati precedentemente.
