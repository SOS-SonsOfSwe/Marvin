\newpage
\section{Attività di Analisi requisiti utente \\\large{resoconto~delle~attività~di~verifica}}
\subsection{Verifica dei processi}
\subsubsection{Schedule Variance}
Nella seguente tabella vengono riportati i valori ottenuti calcolando la Schedule Variance sui tempi di stesura di ogni documento rispetto ai tempi prefissati nel'\textit{Piano di Progetto}:
Analisi dei Requisiti v1.0 & +1 & Accettabile \\
Norme di Progetto v1.0 & -1 & Ottimale \\
Studio di Fattibilità v1.0 &  -2 &  Ottimale \\
Piano di Progetto v1.0 &  0 &  Ottimale\\
Piano di Qualifica v1.0 & 0 & Ottimale \\
Glossario v1.0 & +1 & Accettabile\\	
{Considerazioni finali:}
\begin{description}
	{Schedule Variance finale:} +1 giorno.
	In generale per il processo di documentazione il team ha completato la stesura 1 giorno in anticipo rispetto a quanto pianificato nel \doc{Piano di Progetto}. Il risultato ottenuto pertanto rientra nella soglia di accettabilità prevista.
\end{description}

\subsubsection{Cost Variance}
Nella seguente tabella vengono riportati i valori ottenuti calcolando la Cost Variance sui tempi di stesura di ogni documento:
\begin{tabella}{l!{\VRule}c!{\VRule}c!{\VRule}}
	
	\color{white} \bold{Processo} & \color{white} \bold{CV} &\color{white} \bold{Giudizio} \\
	\endfirsthead
	Processo di documentazione & 0\% & Ottimale\\
	\rowcolor{white}  
	\caption{Esiti della Cost Variance - Attività di Analisi requisiti utente}	   	
\end{tabella}
Considerazioni finali: Per il processo di documentazione il \gl{team} non ha necessitato di ulteriori risorse che potessero aumentare i costi pianificati precedentemente.