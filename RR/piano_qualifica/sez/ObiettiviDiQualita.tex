\section{Obiettivi di qualità}
Questa sezione ha l'obiettivo di definire le caratteristiche riguardanti la qualità di prodotto e di processo che dovranno essere perseguite durante lo sviluppo del progetto.
Ogni caratteristica viene valutata da una metrica, una soglia di accettabilità, ed una possibile soglia di miglioramento che il \emph{team}\ped{G} si prefigge di raggiungere e possibilmente superare.
\subsection{Qualità di processo}
La qualità di processo influenza direttamente il prodotto finale realizzato. É necessario quindi sviluppare un processo in grado di produrre ciclicamente un prodotto di alta qualità. Per questo motivo si è deciso di stabilire le seguenti caratteristiche da rispettare per tutto lo sviluppo del progetto, contemporaneamente all'applicazione dei modelli \emph{Capability Maturity Model}\ped{G}, e \emph{Ciclo di Deming}\ped{G}.
\subsubsection{Pianificazione}
La pianificazione temporale necessita di uno sguardo a ritroso a partire dagli obiettivi prefissati per completare in tempo adeguato il lavoro previsto. Per un \emph{team}\ped{G} è fondamentale rispettare le scadenze previste, e nel caso in cui si verifichi una situazione di possibile ritardo si rischia di violare l'obiettivo di qualità prefissato, e andranno effettuati quindi dovuti controlli.
\begin{itemize}
	\item \textbf{Metrica}: Si è deciso di utilizzare la \emph{\textbf{SV}}\ped{G}. \textcolor{red}{(Sarebbe da aggiungere il link direttamente all'appendice per l'SV)}
	\item \textbf{Soglia di accettabilità}: Si è deciso di ritenere accettabile un ritardo di massimo 3 giorni lavorativi rispetto a quanto specificato nel "\emph{Piano di Progetto}".
	\item \textbf{Soglia di ottimalità}: Si ritiene un miglioramento rispetto all'obiettivo prefissato il caso in cui un lavoro venga portato a termine almeno 2 giorni lavorativi prima del dovuto, in termini di guadagno di tempo complessivo.
\end{itemize}
\subsubsection{Miglioramento}
Al fine di valutare e migliorare la qualità del lavoro svolto è stato assunto il modello Capability Maturity Model (CMM).
\begin{itemize}
	\item \textbf{Metrica}: L'unità di misura utilizzata è la struttura a 5 livelli che rappresenta la scala di Maturità.
	\item \textbf{Soglia di accettabilità}: Il livello minimo di maturità della scala accettabile in riferimento ai processi considerati è il 2 (Repeatable), consistente nell'aver documentazione sufficiente per poter ripetere e migliorare il processo svolto.
	\item \textbf{Soglia di ottimalità}: La soglia di ottimalità verrà raggiunta con il livello 4 (Managed), corrispondente ad un'ottimizzazione tecnologica dei processi in base ai costi. (da cambiare).
\end{itemize}
\subsubsection{Costo}
Per verificare se i costi sono stati rispettati con quanto concordato nel  \emph{''Piano di Progetto''}, è stato deciso di utilizzare la \emph{\textbf{Cost Variance}}\ped{G} (CV).
Qualora un processo non possieda la qualità minima concordata, necessiterà di lavoro aggiuntivo al fine di soddisfare i requisiti richiesti ma alzando il costo complessivo del progetto, che sarà valutato secondo i seguenti parametri:
\begin{itemize}
	\item \textbf{Metrica}: L'unità di misura scelta per valutare l'aumento dei costi stabiliti è la Cost Variance.
	\item \textbf{Soglia di accettabilità}: Sarà accettabile un aumento dei costi superiore a quelli previsti nel \emph{''Piano di Progetto''} di un massimo del 10\%
	\item \textbf{Soglia di ottimalità}: La soglia di ottimalità verrà raggiunta nel caso in cui i costi non aumenteranno rispetto a quanto concordato nel \emph{''Piano di Progetto''}, 
\end{itemize}
\subsection{Qualità di documento}
Il team si impegna a redigere dei documenti di alta qualità, rispettando le carratestiche di forma e contenuto descritte di seguito.
\subsubsection{Ortografia}
Un documento poichè possa essere definito tale, deve essere prima di tutto privo di errori dal punto di vista grammaticale e ortografico. 
Il primo controllo avverà proprio durante la stesura del documento stesso, tramite il sistema di autocontrollo dell'ambiente  \emph{''TexStudio''}, per poi essere controllato dal  \emph{Verificatore}\ped{G}.
\begin{itemize}
	\item \textbf{Metrica}: L'unità di misura considerata è il numero di errori ortografici riscontrati dopo il primo controllo da parte del \emph{Verificatore}.
	\item \textbf{Soglia di accettabilità}: Si è accettata come tollerabile la presenza di massimo un 5\% di errori rispetto alla quantità totale segnalata dopo la prima analisi da parte del \emph{Verificatore}.
	\item \textbf{Sogia di ottimalità}: La soglia di ottimalità verrà raggiunta nel caso in cui dopo la prima revisione del documento non vengano più riscontrati errori dal \emph{Verificatore} e dal \emph{Responsabile}.
\end{itemize}
L'argomento verrà trattato dettagliatamente nella sezione Errori ortografici (ci va il link) in appendice.
\subsubsection{Comprensibilità e leggibilità}
Poichè un documento venga considerato leggibile e scorrevole si è deciso di adottare l'\emph{Indice Gulpease}\ped{G}. 
\begin{itemize}
	\item \textbf{Metrica}: L'unità di misura utilizzata è l'\emph{Indice Gulpease}.
	\item \textbf{Soglia di accettabilità}: Verrà considerato come accettabile un valore di 45 sulla scala dell'\emph{Indice Gulpease}.
	\item \textbf{Soglia di ottimalità}: La soglia di ottimalità verrà raggiunta nel caso in cui l'\emph{Indice Gulpease} sia maggiore di 60.
\end{itemize}
L'argomento verrà trattato dettagliatamente nella sezione Indice Gulpease (ci va il link) in appendice.
\subsubsection{Correttezza dei contenuti}
Oltre che ad essere corretto nella forma, un documento necessita di un contenuto adeguato dal punto di vista argomentativo. Gli \emph{Analisti} saranno direttamente responsabili della qualità del contenuto, che poi verrà controllato e corretto dal \emph{Verificatore}.
Per verificare la correttezza concettuale dei documenti prenderemo in esame i seguenti parametri:
\begin{itemize}
	\item \textbf{Metrica}: La quantità di errori presente dopo la prima verifica del documento sarà l'unità di misura presa in considerazione.
	\item \textbf{Soglia di accettabilità}: La soglia di accettabilità è fissata ad una quantità di errori pari al 5\% rispetto alla precedente verifica del documento.
	\item \textbf{Soglia di ottimalità}: La soglia di ottimalità sarà raggiunta nel caso in cui non si riscontrino errori dopo la prima verifica del documento.
\end{itemize}
\subsubsection{Adesione alle norme interne}
Al fine di ottenere un prodotto coerente ogni documento dovrà essere redatto rispettando strettamente quanto dichiarato nelle \emph{Norme di Progetto}.
Qualunque riferimento non attinente o in contrasto a quanto dichiarato verrà considerato un errore.
\begin{itemize}
	\item \textbf{Metrica}: La quantità di errori presente dopo la prima verifica del documento sarà l'unità di misura presa in considerazione. 
	\item \textbf{Soglia di accettabilità}: La soglia di accettabilità massima è fissata ad una quantità di errori pari al 5\% rispetto alla precedente verifica del documento.
	\item \textbf{Soglia di ottimalità}: La soglia di ottimalità sarà raggiunta nel caso in cui non si riscontrino errori dopo la prima verifica del documento.
\end{itemize}
Per una precisa definizione degli errori in riferimento alle norme interne si veda la sezione Errori di forma (ci va il link) in appendice.
\subsection{Qualità del software}