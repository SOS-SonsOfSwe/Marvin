\section{Obiettivi di qualità}
Questa sezione ha l'obiettivo di definire le caratteristiche riguardanti la qualità di prodotto e di processo che dovranno essere perseguite durante lo sviluppo del progetto.
Ogni caratteristica viene valutata da una metrica, una soglia di accettabilità, ed una possibile soglia di miglioramento che il \emph{team}\ped{G} si prefigge di raggiungere e possibilmente superare.

\subsection{Qualità di processo}
La qualità di processo influenza direttamente il prodotto finale realizzato. É necessario quindi sviluppare un processo in grado di produrre ciclicamente un prodotto di alta qualità. Per questo motivo si è deciso di stabilire le seguenti caratteristiche da rispettare per tutto lo sviluppo del progetto, contemporaneamente a:
\begin{itemize}
\item L'applicazione del \emph{Ciclo di Deming}\ped{G}, o \emph{PDCA}, al fine di perseguire il miglioramente continuo delle attività di processo.
\item L'adesione allo standard ISO/IEC 15504, denominato \emph{SPICE}\ped{G}, al fine di applicare una valutazione oggettiva sulla maturità dei processi.
\end{itemize} 

\subsubsection{Pianificazione}
La pianificazione temporale necessita di uno sguardo a ritroso a partire dagli obiettivi prefissati per completare in tempo adeguato il lavoro previsto. Per un \emph{team}\ped{G} è fondamentale rispettare le scadenze previste, e nel caso in cui si verifichi una situazione di possibile ritardo si rischia di violare l'obiettivo di qualità prefissato, e andranno effettuati quindi dovuti controlli.
\begin{itemize}
	\item \textbf{Metrica}: Si è deciso di utilizzare la \textbf{Schedule Variance}\ped{G}. \textcolor{red}{(Sarebbe da aggiungere il link direttamente all'appendice per l'SV)}
	\item \textbf{Soglia di accettabilità}: Si è deciso di ritenere accettabile un ritardo di massimo 3 giorni lavorativi rispetto a quanto specificato nel "\emph{Piano di Progetto}".
	\item \textbf{Soglia di ottimalità}: Si ritiene un miglioramento rispetto all'obiettivo prefissato il caso in cui un lavoro venga portato a termine almeno 2 giorni lavorativi prima del dovuto, in termini di guadagno di tempo complessivo.
\end{itemize}

\subsubsection{Miglioramento}
Al fine di valutare e migliorare la qualità del lavoro svolto è stato assunto il modello di riferimentoper la valutazione del livello di maturità definito da SPICE.
\begin{itemize}
	\item \textbf{Metrica}: Verrà utilizzata la struttura a 6 livelli che rappresenta la scala di maturità; la misura di ogni livello sarà effettuata con i 4 livelli N,P,L,F definiti dallo standard.
	\item \textbf{Soglia di accettabilità}: Il livello minimo accettabile di maturità della scala in riferimento ai processi è il 2 (Managed); il processo deve cioè fornire i risultati conformi agli standard ed ai requisiti iniziali in maniera pianificata e tracciabile.
	\item \textbf{Soglia di ottimalità}: La soglia di ottimalità verrà raggiunta con il livello 4 (Predictable); il processo dovrà cioè essere eseguito in conformità ai principi dell'ingegneria del software e attuato all'interno di limiti ben definiti.
\end{itemize}
\subsubsection{Costo}
Per verificare se i costi sono stati rispettati con quanto concordato nel  \emph{''Piano di Progetto''}, è stato deciso di utilizzare la \emph{\textbf{Cost Variance}}\ped{G} (CV).
Qualora un processo non possieda la qualità minima concordata, necessiterà di lavoro aggiuntivo al fine di soddisfare i requisiti richiesti ma alzando il costo complessivo del progetto, che sarà valutato secondo i seguenti parametri:
\begin{itemize}
	\item \textbf{Metrica}: L'unità di misura scelta per valutare l'aumento dei costi stabiliti è la Cost Variance.
	\item \textbf{Soglia di accettabilità}: Sarà accettabile un aumento dei costi superiore a quelli previsti nel \emph{''Piano di Progetto''} di un massimo del 10\%
	\item \textbf{Soglia di ottimalità}: La soglia di ottimalità verrà raggiunta nel caso in cui i costi non aumenteranno rispetto a quanto concordato nel \emph{''Piano di Progetto''}, 
\end{itemize}

Per informazioni più approfondite riguardo lo standard ISO/IEC 15504, si rimanda alla sezione~\nameref{AppA:standardProc} dell'appendice A.

\subsection{Qualità di prodotto}
\subsubsection{Qualità di documento}
Il team si impegna a redigere dei documenti di alta qualità, rispettando le carratestiche di forma e contenuto descritte di seguito.
\paragraph{Ortografia}
Un documento poichè possa essere definito tale, deve essere prima di tutto privo di errori dal punto di vista grammaticale e ortografico. 
Il primo controllo avverà proprio durante la stesura del documento stesso, tramite il sistema di autocontrollo dell'ambiente  \emph{''TexStudio''}, per poi essere controllato dal  \emph{Verificatore}\ped{G}.
\begin{itemize}
	\item \textbf{Metrica}: L'unità di misura considerata è il numero di errori ortografici riscontrati dopo il primo controllo da parte del \emph{Verificatore}.
	\item \textbf{Soglia di accettabilità}: Si è accettata come tollerabile la presenza di massimo un 5\% di errori rispetto alla quantità totale segnalata dopo la prima analisi da parte del \emph{Verificatore}.
	\item \textbf{Sogia di ottimalità}: La soglia di ottimalità verrà raggiunta nel caso in cui dopo la prima revisione del documento non vengano più riscontrati errori dal \emph{Verificatore} e dal \emph{Responsabile}.
\end{itemize}
L'argomento verrà trattato dettagliatamente nella sezione Errori ortografici (ci va il link) in appendice.
\paragraph{Comprensibilità e leggibilità}
Poichè un documento venga considerato leggibile e scorrevole si è deciso di adottare l'\emph{Indice Gulpease}\ped{G}, al fine di avere un parametro oggettivo e facilmente misurabile.
\begin{itemize}
	\item \textbf{Metrica}: L'unità di misura utilizzata è l'\emph{Indice Gulpease}.
	\item \textbf{Soglia di accettabilità}: Verrà considerato come accettabile un valore di 45 sulla scala dell'\emph{Indice Gulpease}.
	\item \textbf{Soglia di ottimalità}: La soglia di ottimalità verrà raggiunta nel caso in cui l'\emph{Indice Gulpease} sia maggiore di 60.
\end{itemize}
L'argomento verrà trattato dettagliatamente nella sezione Indice Gulpease (ci va il link) in appendice.
\paragraph{Correttezza dei contenuti}
Oltre che ad essere corretto nella forma, un documento necessita di un contenuto adeguato dal punto di vista argomentativo. Gli \emph{Analisti} saranno direttamente responsabili della qualità del contenuto, che poi verrà controllato e corretto dal \emph{Verificatore}.
Per verificare la correttezza concettuale dei documenti prenderemo in esame i seguenti parametri:
\begin{itemize}
	\item \textbf{Metrica}: La quantità di errori presente dopo la prima verifica del documento sarà l'unità di misura presa in considerazione.
	\item \textbf{Soglia di accettabilità}: La soglia di accettabilità è fissata ad una quantità di errori pari al 5\% rispetto alla precedente verifica del documento.
	\item \textbf{Soglia di ottimalità}: La soglia di ottimalità sarà raggiunta nel caso in cui non si riscontrino errori dopo la prima verifica del documento.
\end{itemize}
\paragraph{Adesione alle norme interne}
Al fine di ottenere un prodotto coerente ogni documento dovrà essere redatto rispettando strettamente quanto dichiarato nelle \emph{Norme di Progetto}.
Qualunque riferimento non attinente o in contrasto a quanto dichiarato verrà considerato un errore.
\begin{itemize}
	\item \textbf{Metrica}: La quantità di errori presente dopo la prima verifica del documento sarà l'unità di misura presa in considerazione. 
	\item \textbf{Soglia di accettabilità}: La soglia di accettabilità massima è fissata ad una quantità di errori pari al 5\% rispetto alla precedente verifica del documento.
	\item \textbf{Soglia di ottimalità}: La soglia di ottimalità sarà raggiunta nel caso in cui non si riscontrino errori dopo la prima verifica del documento.
\end{itemize}
Per una precisa definizione degli errori in riferimento alle norme interne si veda la sezione Errori di forma (ci va il link) in appendice.

\subsubsection{Qualità del software}
Come detto in precedenza, è impossibile distinguere in maniera netta la qualità di processo dalla qualità del software, in quanto la prima influenza direttamente la seconda; è dunque fondamentale avere alla base una qualità di processo sufficientemente buona per garantire la qualità del prodotto. Nonostante ciò, è necessario stabilire degli obiettivi quantitativi di qualità del software oggettivi e misurabili. A tal fine verrà seguito lo standard ISO/IEC 9126, il quale si sostanzia nei sei punti seguenti:

\paragraph{Funzionalità}
È un requisito funzionale che indica la capacità del software di soddisfare le esigenze esposte dal capitolato ed individuate durante l’ \AdR .
Per valutare la funzionalità del software prenderemo in considerazione i seguenti parametri:
\begin{itemize}
	\item \textbf{Metrica}: La valutazione si baserà sul numero di requisiti soddisfatti.
	\item \textbf{Soglia di accettabilità}: Il prodotto verrà valutato come accettabile se tutti i requisiti obbligatori saranno soddisfatti.
	\item \textbf{Soglia di ottimalità}: La soglia di ottimalità sarà raggiunta nel caso in cui siano soddisfatti sia i requisiti obbligatori che tutti i requisiti opzionali.
\end{itemize}

\paragraph{Affidabilità}
È un requisito non funzionale che indica la capacità del software di svolgere correttamente il suo compito, mantenendo delle buone prestazioni anche al variare dell’ambiente nel tempo.
Per valutare l'affidabilità del software prenderemo in considerazione i seguenti parametri:
\begin{itemize}
	\item \textbf{Metrica}: La valutazione si baserà sul numero di fallimenti durante la fase si test.
	\item \textbf{Soglia di accettabilità}: Il prodotto verrà valutato come accettabile se i test falliti saranno inferiori o uguali al 5\%.
	\item \textbf{Soglia di ottimalità}: La soglia di ottimalità sarà raggiunta nel caso in cui il 100\% dei test darà l'esito desiderato.
\end{itemize}

\paragraph{Efficienza \color{red}{Oltre al tempo anche costo di ether???}}
È un requisito non funzionale che valuta la capacità di un prodotto software di realizzare le funzioni richieste nel minor tempo possibile e con l’uso minimo di risorse necessarie.
\begin{itemize}
	\item \textbf{Metrica}: La valutazione si baserà sui secondi impiegati dal prodotto per eseguire le richieste dell'utente.
	\item \textbf{Soglia di accettabilità}: La soglia di accettabilità è il periodo tra 0 e 10 secondi.
	\item \textbf{Soglia di ottimalità}: La soglia di ottimalità è 1 secondo.
\end{itemize}

\paragraph{Usabilità}
L'usabilità è un requisito non funzionale che indica la capacità del software di essere capito e usato correttamente da parte dell'utente finale. Dato che il prodotto finale sarà per l'utente un portale web, è impossibile trovare una metrica quantificabile per valutarne l'usabilità: essa dipende da molteplici fattori che coinvolgono anche le capacità dell'utente stesso e gli strumenti a sua disposizione. Verrà dunque valutata in modo oggettivo basandosi sugli standard del web dichiarati dal \emph{W3C}\ped{G} e sugli strumenti che tale organizzazione mette a disposizione, al fine di creare un'interfaccia web il più accessibile possibile.
Prenderemo in considerazione i seguenti parametri:
\begin{itemize}
	\item \textbf{Metrica}: La valutazione si baserà sul numero di errori trovati dagli strumenti del W3C.
	\item \textbf{Soglia di accettabilità}: La soglia di accettabilità è di 2 errori rilevati.
	\item \textbf{Soglia di ottimalità}: Il prodotto sarà dichiarato ottimo se saranno rilevati 0 errori.
\end{itemize}

Si rimanda alla sezione dell'appendice per maggiori informazioni sulla metrica utilizzata.

Quanto detto non assicura però una valutazione completa dell'usabilità, la quale è soggettiva; sarà necessario dunque predisporre test specifici per la misurazione, coinvolgendo ad esempio persone esterne al gruppo al fine di stabilire quanto mediamente il software sia capibile. Al momento il team non è tuttavia in grado di stabilire con precisione una metrica adatta a misurare questo risultato.

\paragraph{Manutenibilità}
La manutenibilità è un requisito non funzionale che indica la capacità di un prodotto di essere evolvibile nel tempo attraverso correzioni, miglioramenti e aggiunte.

\begin{itemize}
\item \textbf{Metrica}: Saranno usate le metriche riguardanti il codice, dato che esso influenza direttamente la manutenibilità del software.
\item \textbf{Soglia di accettabilità}: La soglia di accettabilità sarà raggiunta se il prodotto raggiungerà tale soglia in tutte le metriche utilizzate per il codice.
\item \textbf{Soglia di ottimalità}: La soglia di ottimalità sarà raggiunta se il prodotto raggiungerà tale soglia in tutte le metriche utilizzate per il codice.
\end{itemize}

Si rimanda alla sezione dell'appendice per maggiori informazioni.

\paragraph{Portabilità}
La portabilità è un requisito non funzionale che indica la capacità del prodotto di operare in \textit{ambienti}\ped{G} diversi, limitando le necessità di apportare cambiamenti.
\begin{itemize}
	\item Metrica: La valutazione si baserà sul numero di versioni di \emph{browser}\ped{G} e numero di browser stessi su cui il prodotto riesce a venire utilizzato e visualizzato correttamente.
	\item Soglia di accettabilità: La soglia di accettabilità sarà raggiunta se il prodotto sarà supportato correttamente, offrendo la totalità delle sue funzionalità, dalla versione aggiornata dei browser \emph{Google Chrome}\ped{G}, \emph{Microsoft Edge}\ped{G}, \emph{Mozilla Firefox}\ped{G}, \emph{Safari}\ped{G} e \emph{Opera}\ped{G} su \emph{dekstop}\ped{G}.
	\item Soglia di ottimalità: La soglia di ottimalità sarà raggiunta se il prodotto sarà supportato correttamente, offrendo la totalità delle sue funzionalità, in aggiunta ai sopra citati, da \emph{Internet Explorer 11}\ped{G} su desktop e da Google Chrome e Safari nelle versioni \emph{mobile}\ped{G} aggiornate.
\end{itemize}

~\\
Per informazioni più approfondite riguardo lo standard ISO/IEC 9126, si rimanda alla sezione~\nameref{AppA:standardProd} dell'appendice A.



