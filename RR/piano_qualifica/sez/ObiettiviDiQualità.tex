\section{Obiettivi di qualità}
Questa sezione ha l'obiettivo di definire le caratteristiche riguardanti la qualità di prodotto e di processo che dovranno essere perseguite durante lo sviluppo del progetto.
Ogni caratteristica viene valutata da una metrica, una soglia di accettabilità, ed una possibile soglia di miglioramento che il team si prefigge di raggiungere e possibilmente superare.
\subsection{Qualità di processo}
La qualità di processo influenza direttamente il prodotto finale realizzato. E' necessario quindi sviluppare un processo in grado di produrre ciclicamente un prodotto di alta qualità. Per questo motivo si è deciso di stabilire le seguenti caratteristiche da rispettare per tutto lo sviluppo del progetto, contemporaneamente all'applicazione dei modelli \textit{Capability Maturity Model}\ped{G}, e \textit{Ciclo di Deming}\ped{G}.
\subsubsection{Pianificazione}
La pianificazione temporale necessita di uno sguardo a ritroso a partire dagli obiettivi prefissati per completare in tempo adeguato il lavoro previsto. Per un \textit{team}\ped{G} è fondamentale rispettare le scadenze previste, e nel caso in cui si verifichi una situazione di possibile ritardo si rischia di violare l'obiettivo di qualità prefissato, e andranno effettuati quindi dovuti controlli.
\begin{itemize}
	\item \textbf{Metrica}: Si è deciso di utilizzare la \textit{\textbf{SV}}\ped{G}; (Sarebbe da aggiungere il link direttamente all'appendice per l'SV)
	\item \textbf{Soglia di accettabilità}: Si è deciso di ritenere accettabile un ritardo di massimo 3 giorni lavorativi rispetto a quanto specificato nel "\textit{Piano di Progetto}".
	\item \textbf{Soglia di miglioramento}: Si ritiene un miglioramento rispetto all'obiettivo prefissato il caso in cui un lavoro venga portato a termine almeno 2 giorni lavorativi prima del dovuto, in termini di guadagno di tempo complessivo.
\end{itemize}
\subsubsection{Miglioramento}

