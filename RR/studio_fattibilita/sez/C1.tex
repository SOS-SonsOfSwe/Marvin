\section{Capitolato C1}
\subsection{Info sul capitolato}
\begin{itemize}
	\item \textbf{Nome:} Ajarvis;
	\item \textbf{\textit{Proponente}}\ped{G}: zero12 srl;
	\item \textbf{\textit{Committenti}}\ped{G}: Prof. Tullio Vardanega, Prof. Riccardo Cardin.
\end{itemize}

\subsection{Descrizione}
L'obiettivo del capitolato è lo sviluppo di un applicativo in grado di ascoltare gli \textit{standup}\ped{G} giornalieri sullo stato di avanzamento dei progetti di zero12 srl, comprendere i dialoghi, analizzarne il contenuto per fornirne un'analisi dello standup estraendo al contesto gli argomenti emersi. Bisogna inoltre realizzare una \textit{dashboard}\ped{G} in grado di rappresentare lo stato di avanzamento del progetto, le tipologie di problematiche riscontrate ed evidenziare gli aspetti comuni ai vari progetti.
\\
Lapplicativo sarà composto da tre parti:
\begin{itemize}
	\item Interfaccia web di registrazione;
	\item \textit{Servizi cloud}\ped{G} per l'analisi dei dati;
	\item Interfaccia web per la reportistica delle analisi realizzate.
\end{itemize}

\subsection{Dominio Applicativo}
Il progetto si inserisce nell'ambito del riconoscimento vocale e del riconoscimento semantico del testo tramite algoritmi di \textit{machine learning}\ped{G}. Il prodotto finale servirà come strumento atto ad aumentare l'efficienza di zero12 migliorando la rappresentazione delle informazioni che emergono durante gli standup giornalieri.

\subsection{Dominio Tecnologico}
\begin{itemize}
	\item \textit{\textbf{Google Cloud Platform}}\ped{G} come infrastruttura di cloud;
	\item \textbf{\textit{Google Cloud Datastore}}\ped{G} o \textbf{\textit{Google SQL}}\ped{G} per la gestione del database;
	\item \textbf{\textit{NodeJS}}\ped{G} per il \textbf{\textit{backend}}\ped{G};
	\item \textbf{\textit{Git}}\ped{G} come sistema di versionamento
	\item Tool di data visualization o \textbf{\textit{HTML5}}\ped{G}, \textbf{\textit{CSS3}}\ped{G} e \textbf{\textit{javascript}}\ped{G} per l'interfaccia di visualizzazione, viene consigliato \textbf{\textit{Bootstrap}}\ped{G} come \textit{framework}\ped{G} \textit{responsive}\ped{G}.
\end{itemize}

\subsection{Aspetti Positivi}
\begin{itemize}
	\item Servizi Google interessanti e probabilmente molto utili in progetti futuri;
	\item Interesse al primo approccio di \textit{machine learning}\ped{G} da parte di quasi tutti i membri del gruppo;
\end{itemize}

\subsection{Potenziali Criticità}
\begin{itemize}
	\item Necessaria una fase di catalogazione del testo molto onerosa in termini di tempo;
	\item A parte i linguaggi di \textit{markup per il web}\ped{G}, nessun membro del gruppo possiede conoscenze nei servizi e nelle tecnologie richieste;
	\item Potenziale grave problema nel caso di registrazione di voci contemporanee.
\end{itemize}

\subsection{Valutazione}
L'interesse del gruppo verso questo capitolato è alto grazie all'interesse generale nell'imparare le tecnologie richieste e soprattutto per il fatto di lavorare su un'applicazione di machine learning.
Il gruppo ha scelto questo capitolato come prima scelta, purtroppo però l'azienda non ha dato disponibilità.