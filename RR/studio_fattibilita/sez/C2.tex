\section{Capitolato C2}
\subsection{Informazioni sul capitolato}
	\begin{itemize}
		\item \textbf{Nome:}
		BlockCV;
		\item \textbf{\textit{Proponente\ped{G}:}}
		Ifin Sistemi;
		\item \textbf{\textit{Committenti\ped{G}:}}
		Tullio Vardanega, Riccardo Cardin.
	\end{itemize}

\subsection{Descrizione}
	Lo scopo del capitolato C2 è quello di creare un sistema distribuito per la pubblicazione dei CV e la ricerca di proposte di lavoro basato su una \textit{permissioned blockchain\ped{G}}. Tale sistema deve gestire le operazione di ricerca di un'occupazione, la creazione, la diffusione e le future modifiche del CV del lavoratore, entrando così a far parte del complesso lavorativo come un valido modello applicabile ad esso. Dei suddetti CV, inoltre, attraverso determinati controlli si assicura che siano sempre aggiornati e con una certificazione di autenticità.
	\newline \newline Le operazioni base che l'utente lavoratore può effettuare all'interno della web application sono:
	\begin{itemize}
		\item Creare, modificare e importare il propio CV;
		\item Effettuare una ricerca tra le offerte di lavoro;
		\item Rispondere ad una determinata offerta;
		\item Condividere, con la possibilità di farlo selettivamente sui propri dati, il proprio CV;
		\item Confermare esperienze lavorative o certificati assegnati da terzi;
		\item Esportare il CV in un formato standard.
	\end{itemize}
	mentre le altre tipologie di utenti possono:
	\begin{itemize}
		\item Inserire annunci di lavoro;
		\item Effettuare una ricerca tra i CV presenti;
		\item Confermare le esperienze lavorative e inserire eventuali commenti o valutazioni;
		\item Aggiungere certificazioni o esperienze lavorative direttamente agli utenti lavoratori;
	\end{itemize}

\subsection{Dominio Applicativo}
	Il dominio applicativo su cui si affaccia BlockCV è quello che riguarda document management in ambito applicativo e di servizio nel settore della gestione
	documentale.

\subsection{Dominio Tecnologico}
	Il software e la documentazione devono essere disponibili su piattaforme pubbliche, come ad esempio GitHub, e licenziati Apache 2.0. La piattaforma blockchain utilizzata deve essere Hyperledger Fabric versione 1.0 o successive, mentre il capitolato consiglia l'uso di linguaggi della piattaforma Java EE, il framework Play o la suite di componenti Vaadin Elements per lo sviluppo dell'interfaccia grafica e MongoDB o Cassandra nel caso in cui sia necessario utilizzare una base di dati persistente esterna.
	
\subsection{Aspetti Positivi}
	Gli aspetti positivi che sono stati riscontrati sono:
	\begin{itemize}
		\item 
	\end{itemize}

\subsection{Potenziali Criticità}
	Le principali criticità incontrate sono:
	\begin{itemize}
		\item 
	\end{itemize}

\subsection{Valutazione Finale}