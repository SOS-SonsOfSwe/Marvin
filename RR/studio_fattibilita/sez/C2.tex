\section{Capitolato C2}
\subsection{Informazioni sul Capitolato}
  \begin{itemize}
    \item \textbf{Nome:}
    BlockCV;
    \item \textbf{Proponente}:
    Ifin Sistemi;
    \item \textbf{Committenti}:
    Prof. Tullio Vardanega, Prof. Riccardo Cardin.
  \end{itemize}

\subsection{Descrizione}
  Lo scopo del capitolato C2 è quello di creare un sistema distribuito per la pubblicazione di Curriculum Vitae (CV) e la ricerca di proposte di lavoro basato su una \textit{permissioned blockchain}\ped{G}. Tale sistema deve gestire le operazioni di ricerca di un'occupazione, la creazione, la diffusione e le future modifiche del CV di un utente lavoratore. Si assicura inoltre che il suddetto documento sia sempre aggiornato e che, attraverso determinati controlli, i dati al suo interno siano autentici e fedeli alla realtà.
  \newline \newline Altre tipologie di utenti possono confermare tali esperienze lavorative e inserire eventuali commenti o valutazioni; essi, inoltre, possono inserire annunci di lavoro, effettuare ricerche tra i CV presenti e aggiungere certificazioni o esperienze lavorative direttamente agli utenti lavoratori.

\subsection{Dominio Applicativo}
  Il dominio applicativo su cui si affaccia BlockCV è quello del Document management (Sistema di gestione dei documenti), una categoria di sistemi software che serve ad organizzare e facilitare la creazione collaborativa di documenti e altri contenuti.

\subsection{Dominio Tecnologico}
  Il software e la documentazione devono essere disponibili su piattaforme pubbliche, come ad esempio \textit{GitHub}{\ped{G}}.
  \newline \newline La piattaforma blockchain utilizzata deve essere:
    \begin{itemize}
      \item \textit{\textbf{Hyperledger Fabric}}\ped{G} versione 1.0 o successive.
    \end{itemize}
 mentre il capitolato consiglia:
     \begin{itemize}
       \item l'uso di linguaggi della piattaforma \textit{\textbf{Java EE}}\ped{G};
       \item il framework \textit{\textbf{Play}}\ped{G} o la suite di componenti \textit{\textbf{Vaadin Elements}}\ped{G} per lo sviluppo dell'interfaccia grafica;
       \item \textit{\textbf{MongoDB}}\ped{G} o \textit{\textbf{Cassandra}}\ped{G} nel caso in cui sia necessario utilizzare una base di dati persistente esterna.
     \end{itemize}
  
\subsection{Aspetti Positivi}
  Gli aspetti positivi riscontrati sono:
  \begin{itemize}
    \item Tecnologia blockchain da utilizzare molto interessante, innovativa e potenzialmente utile in un futuro lavorativo,
    soprattutto vista la sua attualità.
  \end{itemize}

\subsection{Potenziali Criticità}
  I fattori negativi incontrati sono:
  \begin{itemize}
    \item Poco entusiasmo da parte del gruppo nell'applicare questa tecnologia al sistema dei CV;
    \item Linguaggi consigliati e strumenti da utilizzare sconosciuti.
  \end{itemize}
\subsection{Valutazione Finale}
  Malgrado l'attrattiva verso la tecnologia blockchain e l'opportunità offertaci per approfondirla, la sua applicazione al sistema dei CV ha affievolito l'interesse del gruppo; la totale mancanza di conoscenza nelle tecnologie proposte, inoltre, ha allungato considerevolmente i tempi stimati dai membri per la conclusione del progetto.
