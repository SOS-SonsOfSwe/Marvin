\section{Capitolato C3}
\subsection{Informazioni sul capitolato}
	\begin{itemize}
		\item \textbf{Nome:}
		DeSpeect;
		\item \textbf{\textit{Proponente\ped{G}:}}
		MIVOQ;
		\item \textbf{\textit{Committenti\ped{G}:}}
		Tullio Vardanega, Riccardo Cardin.
	\end{itemize}

\subsection{Descrizione}
	L'obiettivo di tale capitolato è quello di realizzare un'interfaccia grafica per Speect [Meraka Institute(2008-2013)], una libreria per la creazione di sistemi di sintesi vocale che agevoli l’ispezione del suo stato interno durante il funzionamento e la scrittura di test per le sue funzionalità.
	\newline \newline Tale sistema deve essere progettato in due blocchi principali:
	\begin{itemize}
		\item \textbf{Frontend:}
		effettua l'analisi linguistica del testo in ingresso ed estrae da essa una sequenza fonetica dettagliata funzionale al backend per la creazione del file vocale;
		\item \textbf{Backend:}
		converte in una forma d'onda tale sequenza fonetica, che rappresenta l'intenzione di pronunciare determinati suoni.
	\end{itemize}
	Oltre alla realizzazione dell'interfaccia grafica è richiesta la documentazione dell'applicazione, che comprende l'analisi dei requisiti e la descrizione tecnica.

\subsection{Dominio Applicativo}
	 Questo tipo di tecnologia si è diffuso rapidamente negli ultimi tempi raggiungendo numerosi ambiti, come per esempio: le voci guida dei navigatori satellitari, gli annunci dei mezzi di trasporto pubblico, i centralini telefonici e i lettori di messaggi.
	 \newline \newline Il contesto in cui opera riguarda perciò tutte quelle applicazioni in cui la vista dell'utente per cause di forza maggiore o per limitazioni temporanee (per esempio durante la guida) è privata o impedita.

\subsection{Dominio Tecnologico}
	Nonostante venga incoraggiato lo sviluppo multipiattaforma, l'applicazione deve essere compatibile con Linux. Un requisito fontametale è l'utilizzo di Speect e della versione modificata dalla proponente [Mivoq(2014-2017)].
	\newline \newline Per lo sviluppo dell'interfaccia utente il capitolato suggerisce l'utilizzo di:
	\begin{itemize}
		\item Librerie portabili come Gtk+ [The GTK+ Team(1998-2017)] o Qt [The Qt Company(1995-2017a)];
		\item Programmi come Glade [The GNOME Project(1998-2017)] o QtCreator [The Qt Company(1995-2017b)].
	\end{itemize}
	mentre per quanto riguarda l'automazione della compilazione consiglia CMake [Kitware Inc.(2000-2017)].
	
\subsection{Aspetti Positivi}
	Gli aspetti positivi salienti riscontrati sono:
	\begin{itemize}
		\item Interesse della maggior parte dei componenti del gruppo nei sistemi di sintesi vocale da testo scritto;
		\item Familiarità con le interfacce grafiche;
		\item Confidenza con Qt.
	\end{itemize}
	
\subsection{Potenziali Criticità}
	Le principali criticità constatate sono:
	\begin{itemize}
		\item Attrazione per tale capitolato non condivisa da tutti i componenti del gruppo;
		\item Mancanza di incoraggiamento allo studio e all'apprendimento di nuove tecnologie.
	\end{itemize}

\subsection{Valutazione Finale}
	Nonostante l'interesse da parte della maggioranza dei componenti nei sistemi di sintesi vocale e la familiarità di essi con le tecnologie richieste per lo sviluppo, si è scelto di scartare tale capitolato proprio perchè non fornisce alcuno stimolo per il gruppo all'acquisizione di nuove tecnilogie, ergo poco formativo.