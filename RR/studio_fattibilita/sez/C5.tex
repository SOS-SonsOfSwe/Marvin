\section{Capitolato C5}
\subsection{Informazioni sul capitolato}
	\begin{itemize}
		\item \textbf{Nome:}
		IronWorks;
		\item \textbf{\textit{Proponente\ped{G}:}}
		Zucchetti;
		\item \textbf{\textit{Committenti\ped{G}:}}
		Tullio Vardanega, Riccardo Cardin.
	\end{itemize}

\subsection{Descrizione}
	Lo scopo di questo capitolato è quello di realizzare un disegnatore di diagrammi di robustezza, ampliando la definizione delle entità persistenti includendo la descrizione dei dati contenuti in modo tale da generare il codice sia delle classi Java che possono ospitarle che dei programmi per scrivere e leggere tali classi in un database relazionale.
	\newline \newline Strutturando il diagramma di robustezza in interfacce, procedure ed entità persistenti è possibile costruire programmi adeguatamente organizzati e soprattutto robusti, cioè resistenti agli errori e ai cambiamenti nel tempo.
	\newline \newline Viene richiesto inoltre di produrre il codice di creazione delle tabelle associate alle entità persistenti e il codice di manutenzione per esse qualora si presenti la neccessità di ridefinire le classi. È obbligatorio infine creare le istruzioni di gestione della transazione e definire l’architettura completa dell’applicazione.
	
\subsection{Dominio Applicativo}
	Il dominio applicativo di tale capitolato riguarda l'ambito della progettazione di robustness diagram; in particolar modo tale prodotto verrà utilizzato all'interno dell'azienda Zucchetti.
	
\subsection{Dominio Tecnologico}
	Per quanto riguarda la parte server, il capitolato C5 richiede che venga utilizzata una tra le seguenti tecnologie proposte:
	\begin{itemize}
		\item Java con server Tomcat;
		\item Javascript con server Node.Js.
	\end{itemize}
	mentre per il lato client il sistema dovrà:
	\begin{itemize}
		\item Essere eseguibile in un browser HTML5;
		\item Utilizzare fogli stile CSS per l’aspetto estetico;
		\item Servirsi di Javascript per la parte attiva.
	\end{itemize}
	
\subsection{Aspetti Positivi}
	Gli aspetti positivi rilevanti che sono stati riscontrati sono:
	\begin{itemize}
		\item Utilità del capitolato per approfondire accuratamente \textit{UML\ped{G}};
		\item Familiarità con le Tecnologie Web;
		\item Stimolo ad apprendere Node.Js, per ora sconosciuta ai membri del gruppo.
	\end{itemize}

\subsection{Potenziali Criticità}
	Le principali criticità incontrate sono:
	\begin{itemize}
		\item Natura troppo accademica del capitolato;
		\item Scarso interesse da parte dei componenti del gruppo.
	\end{itemize}

\subsection{Valutazione Finale}
	Dopo un lungo dibattito, i componenti del gruppo hanno deciso di scartare tale capitolato per un insufficiente interesse e per la sua natura esageratamente accademica, nonostante la familiarità con le Tecnologie Web e l'opportunità di apprendere in modo più specifico l'ambito della progettazione di diagrammi UML. 
	
