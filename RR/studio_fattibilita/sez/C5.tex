\section{Capitolato C5}
\subsection{Informazioni sul capitolato}
	\begin{itemize}
		\item \textbf{Nome:}
		IronWorks;
		\item \textbf{\textit{Proponente}}\ped{G}:
		Zucchetti S.p.A;
		\item \textbf{\textit{Committenti}}\ped{G}:
		Prof. Tullio Vardanega, Prof. Riccardo Cardin.
	\end{itemize}

\subsection{Descrizione}
	Lo scopo di questo \textit{capitolato}\ped{G} è quello di realizzare un disegnatore di diagrammi di robustezza \textit{UML}\ped{G} in modo tale da generare il codice sia delle classi \textit{Java}\ped{G} che dei programmi per scriverle e leggerle in un database relazionale.
	\newline \newline Strutturando adeguatamente il diagramma di robustezza in interfacce, procedure ed entità persistenti è possibile costruire programmi di qualità e particolarmente robusti, cioè resistenti agli errori e ai cambiamenti nel tempo.
	
\subsection{Dominio Applicativo}
	Il dominio applicativo di IronWorks riguarda l'ambito della progettazione di \textit{robustness diagram}\ped{G}; in particolar modo tale prodotto verrà utilizzato all'interno dell'azienda Zucchetti.
	
\subsection{Dominio Tecnologico}
	Per quanto riguarda la parte server, il \textit{capitolato}\ped{G} C5 richiede che venga utilizzata una tra le seguenti tecnologie proposte:
	\begin{itemize}
		\item \textit{\textbf{Java}}\ped{G} con server \textit{\textbf{Tomcat}}\ped{G};
		\item \textit{\textbf{Javascript}}\ped{G} con server \textit{\textbf{Node.Js}}\ped{G}.
	\end{itemize}
	mentre per il lato client il sistema dovrà:
	\begin{itemize}
		\item Essere eseguibile in un browser \textit{\textbf{HTML5}}\ped{G};
		\item Utilizzare fogli stile \textit{\textbf{CSS}}\ped{G} per l’aspetto estetico;
		\item Servirsi di \textit{\textbf{Javascript}}\ped{G} per la parte attiva.
	\end{itemize}
	
\subsection{Aspetti Positivi}
	Gli aspetti positivi rilevanti che sono stati riscontrati sono:
	\begin{itemize}
		\item Utilità del capitolato per approfondire accuratamente \textit{UML\ped{G}};
		\item Familiarità con le tecnologie Web;
		\item Stimolo ad apprendere \textit{Node.Js}\ped{G}, per ora sconosciuta ai membri del gruppo.
	\end{itemize}

\subsection{Potenziali Criticità}
	Le principali criticità incontrate sono:
	\begin{itemize}
		\item Natura troppo accademica del \textit{capitolato}\ped{G};
		\item Scarso interesse da parte del gruppo.
	\end{itemize}

\subsection{Valutazione Finale}
	Dopo un lungo dibattito, i componenti del \textit{team}\ped{G} hanno deciso di scartare tale \textit{capitolato}\ped{G} per un insufficiente interesse e per la sua natura esageratamente accademica, nonostante la familiarità con le tecnologie Web e l'opportunità di apprendere in modo più specifico l'ambito della progettazione dei diagrammi \textit{UML}\ped{G}. 
	
