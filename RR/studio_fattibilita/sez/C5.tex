\section{Capitolato C5}
\subsection{Informazioni sul Capitolato}
	\begin{itemize}
		\item \textbf{Nome:}
		IronWorks;
		\item \textbf{Proponente}:
		Zucchetti S.p.A;
		\item \textbf{Committenti}:
		Prof. Tullio Vardanega, Prof. Riccardo Cardin.
	\end{itemize}

\subsection{Descrizione}
	Lo scopo di questo capitolato è quello di realizzare un disegnatore di diagrammi di robustezza \textit{UML}\ped{G} in modo tale da generare il codice sia delle classi \textit{Java}\ped{G}, che dei programmi per scriverle e leggerle in un \textit{database relazionale}\ped{G}.
	\newline \newline Strutturando adeguatamente il diagramma di robustezza in interfacce, procedure ed entità persistenti è possibile costruire programmi di qualità e particolarmente robusti, cioè resistenti agli errori e ai cambiamenti nel tempo.
	
\subsection{Dominio Applicativo}
	Il dominio applicativo di IronWorks riguarda l'ambito della progettazione di \textit{robustness diagram}\ped{G}; in particolar modo tale prodotto verrà utilizzato all'interno dell'azienda Zucchetti.
	
\subsection{Dominio Tecnologico}
	Per quanto riguarda la parte server, il capitolato C5 richiede che venga utilizzata una tra le seguenti tecnologie proposte:
	\begin{itemize}
		\item \textbf{Java} con server \textit{\textbf{Tomcat}}\ped{G};
		\item \textbf{JavaScript} con server \textbf{Node.Js}.
	\end{itemize}
	mentre per il lato client il sistema dovrà:
	\begin{itemize}
		\item Essere eseguibile in un browser \textbf{HTML5};
		\item Utilizzare fogli stile \textbf{CSS} per l’aspetto estetico;
		\item Servirsi di \textbf{JavaScript} per la parte attiva.
	\end{itemize}
	
\subsection{Aspetti Positivi}
	Gli aspetti positivi rilevanti riscontrati sono:
	\begin{itemize}
		\item Utilità del capitolato per approfondire accuratamente UML;
		\item Familiarità con le tecnologie Web;
		\item Stimolo ad apprendere la tecnologia Node.Js, per ora sconosciuta ai membri del gruppo.
	\end{itemize}

\subsection{Potenziali Criticità}
	Le principali criticità incontrate sono:
	\begin{itemize}
		\item Natura troppo accademica del capitolato;
		\item Scarso interesse da parte del gruppo.
	\end{itemize}

\subsection{Valutazione Finale}
	Dopo un lungo dibattito, i componenti del team hanno deciso di scartare tale capitolato per un insufficiente interesse e per la sua natura esageratamente accademica, nonostante la familiarità con le tecnologie Web e l'opportunità di apprendere in modo più specifico l'ambito della progettazione dei diagrammi UML. 
	
