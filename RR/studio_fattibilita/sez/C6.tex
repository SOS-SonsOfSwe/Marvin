\newpage
\section{Capitolato scelto}
\subsection{Info sul capitolato}
\begin{itemize}
	\item \textbf{Nome:} Marvin;
	\item \textbf{\textit{Proponente}}\ped{G}: Red Babel;
	\item \textbf{\textit{Committenti}}\ped{G}: Prof. Tullio Vardanega, Prof. Riccardo Cardin.
\end{itemize}

\subsection{Descrizione}
L'obiettivo di Marvin è di realizzare un \textit{prototipo}\ped{G} di Uniweb come \textit{ÐApp}\ped{G} che giri su \textit{Ethereum Virtual Machine}\ped{G}. I tre attori che si rapportano con Marvin sono:
\begin{enumerate}
	\item Università;
	\item Professori;
	\item Studenti.
\end{enumerate}
Il portale deve quindi permettere agli studenti di accedere alle informazioni riguardanti le loro carriere universitarie, di iscriversi a esami, accettare o rifiutare voti e devono poter vedere il loro libretto universitario.
Ai professori deve invece essere permesso di registrare i voti degli studenti:
Infine l'università ogni anno crea una serie di corsi di laurea rivolti a studenti. Un corso di laurea comprende un elenco di esami disponibili per anno accademico. Ogni esame ha un argomento, un numero di crediti e un professore associato. Gli studenti si iscrivono ad un corso di laure e tramite il libretto mantengono traccia ufficiale del progresso.

\subsection{Dominio Applicativo}
Il prodotto finale vuole essere una sorta di \textit{PoC}\ped{G} per le tecnologie usate in questo ambito. L'applicazione sarà un "dimostratore" di uniweb, quindi si colloca in un contesto universitario dove gli attori si approcciano al sistema come nell'attuale uniweb. La differenza sta nel backend\ped{G} dove, invece del classico sistema \textit{client}\ped{G}/\textit{server}\ped{G}, troviamo un database distribuito su base \textit{Ethereum}\ped{G}. 

\subsection{Dominio Tecnologico}
\begin{itemize}
	\item \textit{\textbf{Ethereum}\ped{G}}: la piattaforma in cui dovrà girare l'applicazione;
	\item \textit{\textbf{Blockchain}\ped{G}}: la base di dati decentralizzata e distribuita su cui si basa Ethereum;
	\item \textbf{\textit{Javascript}}\ped{G}, \textit{\textbf{React}}\ped{G}, \textit{\textbf{Redux}}\ped{G}, \textbf{\textit{HTML5}}\ped{G}, \textit{\textbf{SCSS}}\ped{G}: 
	per lo sviluppo frontend dell'applicazione;
	\item \textit{\textbf{Solidity}}\ped{G}: per lo sviluppo backend dell'applicazione.
\end{itemize}

\subsection{Aspetti Positivi}
\begin{itemize}
	\item Utilizzo di \textit{React}\ped{G}, una tecnologia molto richiesta nel mondo del lavoro;
	\item Alto interesse del gruppo nel lavorare su \textit{blockchain}\ped{G};
	\item Utilizzo di \textit{blockchain}\ped{G}, una tecnologia potenzialmente utile in futuro.
\end{itemize}

\subsection{Potenziali Criticità}
\begin{itemize}
	\item A parte i \textit{linguaggi di markup}\ped{G} del web, tutte le altre tecnologie non sono conosciute da quasi tutti i membri del gruppo;
	\item Discreta complessità del problema da affrontare;
	\item Le comunicazioni con i proponenti potrebbe risultare difficile per via della loro residenza all'estero.
\end{itemize}

\subsection{Valutazione}
La scelta finale è ricaduta su questo capitolato perchè ha attirato sin da subito l'interesse di tutti i membri del gruppo grazie alle nuove tecnologie come \textit{Ethereum}\ped{G} e \textit{blockchain}\ped{G} e in generale nei sistemi distribuiti.
Inoltre anche la presenza di tecnologie di frequente utilizzo in ambito lavorativo ha inciso sulla scelta di questo capitolato.