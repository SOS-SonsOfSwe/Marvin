\newpage
\section{Capitolato scelto - C6}
\subsection{Informazioni sul Capitolato}
\begin{itemize}
	\item \textbf{Nome:} Marvin;
	\item \textbf{\textit{Proponente}}\ped{G}: Red Babel;
	\item \textbf{\textit{Committenti}}\ped{G}: Prof. Tullio Vardanega, Prof. Riccardo Cardin.
\end{itemize}

\subsection{Descrizione}
L'obiettivo di Marvin è di realizzare un \textit{prototipo}\ped{G} di Uniweb come \textit{ÐApp}\ped{G} che giri su \textit{Ethereum}\ped{G}. I tre attori che si rapportano con Marvin sono:
\begin{enumerate}
	\item Università;
	\item Professori;
	\item Studenti.
\end{enumerate}
Il portale deve quindi permettere agli studenti di accedere alle informazioni riguardanti le loro carriere universitarie, di iscriversi agli esami, di accettare o rifiutare voti e di poter vedere il loro libretto universitario.
Ai professori deve invece essere permesso registrare i voti degli studenti.
L'università ogni anno crea una serie di corsi di laurea rivolti a studenti, dove ognuno di essi comprende un elenco di esami disponibili per anno accademico. Ogni esame ha un argomento, un numero di crediti e un professore associato. Gli studenti si iscrivono ad un corso di laurea e tramite il libretto elettronico mantengono traccia ufficiale del progresso.

\subsection{Dominio Applicativo}
Il prodotto finale vuole essere una sorta di \textit{PoC}\ped{G} per dimostrare la fattibilità di utilizzo di tali tecnologie in quest'ambito. L'applicazione sarà un "dimostratore" di Uniweb, quindi si colloca in un contesto universitario dove gli attori si approcciano al sistema come nell'attuale Uniweb. La differenza sta nel \textit{back-end}\ped{G} dove, invece del classico sistema \textit{client}\ped{G}/\textit{server}\ped{G}, troviamo un database distribuito che sfrutta la piattaforma Ethereum. 

\subsection{Dominio Tecnologico}
Il proponente del \textit{capitolato}\ped{G} C6 propone le seguenti tecnologie:
\begin{itemize}
	\item \textit{\textbf{Ethereum}\ped{G}}: la piattaforma su cui dovrà girare l'applicazione;
	\item \textit{\textbf{Blockchain}\ped{G}}: la base di dati decentralizzata e distribuita su cui si basa Ethereum;
	\item \textbf{\textit{Javascript}}\ped{G}, \textit{\textbf{React}}\ped{G}, \textit{\textbf{Redux}}\ped{G}, \textbf{\textit{HTML5}}\ped{G}, \textit{\textbf{SCSS}}\ped{G}: 
	per lo sviluppo \textit{front-end}\ped{G} dell'applicazione;
	\item \textit{\textbf{Solidity}}\ped{G}: per lo sviluppo back-end dell'applicazione.
\end{itemize}

\subsection{Aspetti Positivi}
Gli aspetti positivi affiorati sono:
\begin{itemize}
	\item Utilizzo di React, una tecnologia molto richiesta nel mondo del lavoro;
	\item Alto interesse del gruppo nel lavorare su blockchain;
	\item Utilizzo di blockchain, una tecnologia potenzialmente utile in futuro.
\end{itemize}

\subsection{Potenziali Criticità}
Gli aspetti negativi sono invece:
\begin{itemize}
	\item A parte i \textit{linguaggi di markup}\ped{G} del web, tutte le altre tecnologie non sono conosciute da quasi tutti i membri del gruppo;
	\item Discreta complessità del problema da affrontare;
	\item Le comunicazioni con i proponenti potrebbe risultare difficile per via della loro residenza all'estero.
\end{itemize}

\subsection{Valutazione Finale}
Dopo una lunga discussione, la scelta finale è ricaduta su questo capitolato perchè ha attirato sin da subito l'interesse di tutti i membri del gruppo grazie alle nuove tecnologie come Ethereum e blockchain.
Inoltre, anche la presenza di tecnologie di frequente utilizzo in ambito lavorativo ha inciso sulla scelta di questo capitolato.