\section{Capitolato C7}
\subsection{Informazioni sul Capitolato}
	\begin{itemize}
		\item \textbf{Nome:}
		OpenAPM;
		\item \textbf{Proponente}:
		Kirey Group;
		\item \textbf{Committenti}:
		Prof. Tullio Vardanega, Prof. Riccardo Cardin.
	\end{itemize}

\subsection{Descrizione}
	Il capitolato C7 prevede lo sviluppo di uno strumento di \textit{APM}\ped{G} open-source, dato che attualmente esistono applicativi professionali di questo tipo solo a pagamento. Scopo generale è fornire uno strumento di supporto e monitoraggio alle applicazioni, per l'analisi delle performance che potrebbero causarne problemi o rallentamenti e quindi per una gestione più semplice di problematiche che potrebbero rivelarsi complesse.
		
\subsection{Dominio Applicativo}
	Il prodotto è rivolto a progettisti, sviluppatori e professionisti in generale operanti nell'ambito \textit{IT}\ped{G}, a supporto del loro lavoro di creazione e \textit{manutenzione}\ped{G} software. 

\subsection{Dominio Tecnologico}
	Il progetto si basa principalmente su:
		\begin{itemize}
			\item \textbf{ElasticSearch}
			\item \textit{\textbf{Lucene}}\ped{G}
			\item \textit{\textbf{Kibana}}\ped{G}
		\end{itemize}

\subsection{Aspetti Positivi}
	Gli aspetti positivi riscontrati sono:
	\begin{itemize}
	\item Applicazione di metodologie di sviluppo importanti da conoscere, cioè \textit{DevOps}\ped{G} e APM.
	\item Utilizzo di tecnologie interessanti che potrebbero rivelarsi utili nel futuro, come ElasticSearch.
	\end{itemize}

\subsection{Potenziali Criticità}
	Le principali criticità incontrate sono:
	\begin{itemize}
		\item Obiettivo generale del progetto poco chiaro e complesso;
		\item Tecnologie sconosciute al team che richiederebbero molto tempo per essere accuratamente studiate.
	\end{itemize}

\subsection{Valutazione Finale}
	Il gruppo ha valutato il capitolato poco interessante dal punto di vista concettuale, considerando anche che propone in maniera poco chiara l'obiettivo finale del prodotto, cosa che non ha permesso di valutare il carico di lavoro associato. Nonostante le tecnologie fossero a detta di tutti interessanti e utili da conoscere, il tempo per apprenderle a pieno è stato valutato troppo lungo. Per questi motivi e poichè non era disponibile, il capitolato in esame è stato scartato.
