\section{Capitolato C7}
\subsection{Informazioni sul capitolato}
	\begin{itemize}
		\item \textbf{Nome:}
		OpenAPM;
		\item \textbf{\textit{Proponente}}\ped{G}:
		Kirey Group;
		\item \textbf{\textit{Committenti}}\ped{G}:
		Tullio Vardanega, Riccardo Cardin.
	\end{itemize}

\subsection{Descrizione}
	Il capitolato C7 prevede sviluppo di uno strumento di \textit{APM}\ped{G} \textit{open-source}\ped{G}, dato che attualmente esistono strumenti di questo tipo solo a pagamento. Scopo generale è fornire uno strumento di supporto all'individuazione e risoluzione di problematiche legate allo sviluppo software, il quale deve tenere il passo con il veloce progredire delle architetture di base.
		
\subsection{Dominio Applicativo}
	Il prodotto è rivolto a progettisti, sviluppatori e professionisti in generale operanti nell'ambito IT, a supporto del loro lavoro di creazione e manutenzione software. 

\subsection{Dominio Tecnologico}
	Il progetto si basa principalmente su \textit{\textbf{ElasticSearch}}, un potente e veloce motore/server di ricerca \textit{open-source}\ped{G} costruito su \textit{\textbf{Lucene}}, \textit{API}\ped{G} \textit{open-source}\ped{G} scritta in \textit{\textbf{Java}}\ped{G}  e \textit{\textbf{Kibana}}, un'interfaccia grafica utile a visualizzare e navigare nei dati memorizzati in \textit{\textbf{Elastic}}.

\subsection{Aspetti Positivi}
	Gli aspetti positivi che sono stati riscontrati sono:
	\begin{itemize}
	\item Applicazione di metodologie di sviluppo importanti da conoscere, cioè \textit{DevOps}\ped{G} e \textit{APM}\ped{G}.
	\item Utilizzo di tecnologie interessanti e che potrebbero rivelarsi utili nel futuro, come \textit{\textbf{ElasticSearch}}.
	\end{itemize}

\subsection{Potenziali Criticità}
	Le principali criticità incontrate sono:
	\begin{itemize}
		\item Obiettivo generale del progetto poco chiaro e complesso.
	\end{itemize}

\subsection{Valutazione Finale}
	Il gruppo ha valutato il capitolato poco interessante dal punto di vista concettuale, considerando anche che propone in maniera poco chiara l'obiettivo finale del prodotto, cosa che non ha permesso di valutare il carico di lavoro associato. Nonostante le tecnologie fossero a detta di tutti interessanti e utili da conoscere, il tempo per apprenderle a pieno è stato valutato troppo lungo. Per questi motivi e poichè non era disponibile, il capitolato in esame è stato scartato.