\section{Capitolato C8}
\subsection{Informazioni sul capitolato}
	\begin{itemize}
		\item \textbf{Nome:}
		 TuTourSelf;
		\item \textbf{\textit{Proponente}}\ped{G}:
		 TuTourSelf;
		\item \textbf{\textit{Committenti}}\ped{G}:
		Prof. Tullio Vardanega, Prof. Riccardo Cardin.
	\end{itemize}

\subsection{Descrizione}
	Lo scopo del capitolato C8 è lo sviluppo di una piattaforma web con l'obiettivo di facilitare ad artisti indipendenti l'organizzazione dei proprio tour, creando una community in cui artisti e locali possano interagire in modo chiaro, rapido e regolamentato. 
	Viena previsto inoltre di dare la possibilità ad utenti esterni (cioè nè artisti nè proprietari di spazi) di fruire delle informazioni nel sistema per conoscere gli eventi di loro interesse. Tutti e tre questi attori hanno inoltre la possibilità di lasciare feedback.
	
\subsection{Dominio Applicativo}
	Il sistema da realizzare trova il proprio dominio applicativo nel mondo della creatività ed è indirizzato a band, musicisti, scrittori che vogliano promuovere
	il proprio libro, stand-up comedians, compagnie teatrali, artisti di strada, live
	performers e pittori alla ricerca di gallerie d’arte.

\subsection{Dominio Tecnologico}
	Oggetto del capitolato è la creazione di un portale e di conseguenza è necessario nel front-end\ped{G} l'utilizzo dei linguaggi del web, cioè \textit{\textbf{HTML}}\ped{G},\textit{\textbf{CSS}}\ped{G},\textit{\textbf{JS}}\ped{G}; per quanto riguarda quest'ultimo è desiderabile l'utilizzo della libreria \textit{\textbf{React}}\ped{G}. Per quanto riguarda il \textit{back-end}\ped{G}, invece, viene lasciata completa libertà di scelta delle tecnologie, purchè aderenti agli standard ed attente alla scalabilità.
	Infine è richiesta la pubblicazione del progetto su un repository\ped{G} pubblico.

\subsection{Aspetti Positivi}
	Gli aspetti positivi che sono stati riscontrati sono:
	\begin{itemize}
	\item Concetto di base lodevole, con potenzialità per avere successo.
	\item Tecnologie \textit{front-end}\ped{G}conosciute e apprezzate.
	\end{itemize}

\subsection{Potenziali Criticità}
	Le principali criticità incontrate sono:
	\begin{itemize}
		\item Poco stimolante in quanto semplice sviluppo di un'interfaccia web.
		\item L'implementazione di funzionalità come live chat e pagamento potrebbe risultare onerosa.
		\item Troppa libertà nella scelta delle tecnologie \textit{back-end}\ped{G}.  
	\end{itemize}

\subsection{Valutazione Finale}
	Il capitolato è risultato per il gruppo concettualmente interessante, tuttavia presentava pochi stimoli e avrebbe impiegato il gruppo nella ricerca delle tecnologie adatte al \textit{back-end}\ped{G}, cosa che avrebbe richiesto troppo tempo vista la poca esperienza sul campo dei componenti.