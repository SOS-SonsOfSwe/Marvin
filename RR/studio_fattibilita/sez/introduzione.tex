\section{Introduzione}
\subsection{Scopo del Documento}
	Nel seguente documento sono riportate le motivazioni che hanno spinto alla scelta del capitolato C6 (Marvin: dimostratore di Uniweb su \textit{EVM}\ped{G}) da parte del gruppo SonsOfSwe. Verranno successivamente descritti i restanti capitolati ed esplicate le motivazioni che hanno portato alla loro esclusione.
	
\subsection{Scopo del Prodotto}
	\begin{comment}
	Il progetto Marvin si pone l'obiettivo di realizzare un sottoinsieme di funzionalità del portale Uniweb come una \textit{ÐApp}\ped{G} (Decentralized Applications, cioè applicazioni  che usano \textit{smart contracts}\ped{G}) in esecuzione su EVM.
	\end{comment}
	Lo scopo del prodotto è quello di realizzare un \emph{prototipo}\ped{G} di Uniweb come una \emph{\DJ App}\ped{G} in esecuzione su \emph{Ethereum}\ped{G}. I cinque attori principali che si rapportano con Marvin sono:
	\begin{itemize}
		\item Utente non autenticato; 
		\item Università;
		\item Amministratore;
		\item Professore;
		\item Studente.
	\end{itemize} 
	Il portale deve quindi permettere agli studenti di accedere alle informazioni riguardanti le loro carriere universitarie, di iscriversi agli esami, di accettare o rifiutare voti e di poter vedere il loro libretto universitario.
	Ai professori deve invece essere permesso di registrare i voti degli studenti.
	L'università ogni anno crea una serie di corsi di laurea rivolti a studenti, dove ognuno di essi comprende un elenco di esami disponibili per anno accademico. Ogni esame ha un argomento, un numero di crediti e un professore associato. Gli studenti si iscrivono ad un corso di laurea e tramite il libretto elettronico mantengono traccia ufficiale del progresso.
	
\subsection{Glossario}
	Nel documento \textit{Glossario\textunderscore v1.0.0} i termini tecnici, gli acronimi e le abbreviazioni sono definiti in modo chiaro e conciso, in modo tale da evitare ambiguità e massimizzare la comprensione dei documenti.
	\newline \newline I vocaboli presenti in esso saranno posti in corsivo e presenteranno una "G" maiuscola a pedice.
	
\subsection{Riferimenti}
	\subsubsection{Normativi}
		\begin{itemize}
			\item \textbf{\textit{NormeDiProgetto\textunderscore v1.0.0}}.
		\end{itemize}
	
	\subsubsection{Informativi}
		\begin{itemize}
			\item \textbf{Capitolato d'appalto C1:}
			Ajarvis: assistente virtuale di cerimonie Agile\\
			\href{http://www.math.unipd.it/~tullio/IS-1/2017/Progetto/C1.pdf}{http://www.math.unipd.it/~tullio/IS-1/2017/Progetto/C1.pdf};
			\item \textbf{Capitolato d'appalto C2:}
			BlockCV: blockchain per gestione di CV certificati\\
			\href{http://www.math.unipd.it/~tullio/IS-1/2017/Progetto/C2.pdf}{http://www.math.unipd.it/~tullio/IS-1/2017/Progetto/C2.pdf};
			\item \textbf{Capitolato d'appalto C3:}
			DeSpeect: interfaccia grafica per Speect\\
			\href{http://www.math.unipd.it/~tullio/IS-1/2017/Progetto/C3.pdf}{http://www.math.unipd.it/~tullio/IS-1/2017/Progetto/C3.pdf};
			\item \textbf{Capitolato d'appalto C4:}
			ECoRe: enterprise content recommendation\\
			\href{http://www.math.unipd.it/~tullio/IS-1/2017/Progetto/C4.pdf}{http://www.math.unipd.it/~tullio/IS-1/2017/Progetto/C4.pdf};
			\item \textbf{Capitolato d'appalto C5:}
			IronWorks: utilità per la costruzione di software robusto\\
			\href{http://www.math.unipd.it/~tullio/IS-1/2017/Progetto/C5.pdf}{http://www.math.unipd.it/~tullio/IS-1/2017/Progetto/C5.pdf};
			\item \textbf{Capitolato d'appalto C6:}
			Marvin: dimostratore di Uniweb su Ethereum\\
			\href{http://www.math.unipd.it/~tullio/IS-1/2017/Progetto/C6.pdf}{http://www.math.unipd.it/~tullio/IS-1/2017/Progetto/C6.pdf};
			\item \textbf{Capitolato d'appalto C7:}
			OpenAPM: cruscotto di Application Performance Management\\
			\href{http://www.math.unipd.it/~tullio/IS-1/2017/Progetto/C7.pdf}{http://www.math.unipd.it/~tullio/IS-1/2017/Progetto/C7.pdf};
			\item \textbf{Capitolato d'appalto C8:}
			TuTourSelf: piattaforma di prenotazioni per artisti in tournee\\
			\href{http://www.math.unipd.it/~tullio/IS-1/2017/Progetto/C8.pdf}{http://www.math.unipd.it/~tullio/IS-1/2017/Progetto/C8.pdf}.
		\end{itemize}