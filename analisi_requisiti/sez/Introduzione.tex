\section{Introduzione}
\subsection{Scopo del documento}
Questo documento ha lo scopo di descrivere gli \emph{attori}\ped{G} del sistema, individuare i \emph{casi d'uso}\ped{G} a partire dai \emph{requisiti}\ped{G} e fornire una visione chiara ai \progs{} sul problema da trattare. I requisiti verranno classificati in questo documento a seguito di una trattazione col proponente.

\subsection{Scopo del prodotto}
Lo scopo del prodotto è quello di realizzare un \emph{prototipo}\ped{G} di Uniweb come una \emph{\DJ App}\ped{G} in esecuzione su \emph{Ethereum}\ped{G}. I cinque attori principali che si rapportano con Marvin sono:
\begin{itemize}
	\item Utente non autenticato; 
	\item Università;
	\item Amministratore;
	\item Professore;
	\item Studente.
\end{itemize} 
Il portale deve quindi permettere agli studenti di accedere alle informazioni riguardanti le loro carriere universitarie, di iscriversi agli esami, di accettare o rifiutare voti e di poter vedere il loro libretto universitario.
Ai professori deve invece essere permesso di registrare i voti degli studenti.
L'università ogni anno crea una serie di corsi di laurea rivolti a studenti, dove ognuno di essi comprende un elenco di esami disponibili per anno accademico. Ogni esame ha un argomento, un numero di crediti e un professore associato. Gli studenti si iscrivono ad un corso di laurea e tramite il libretto elettronico mantengono traccia ufficiale del progresso.

\subsection{Glossario}
Nel documento \G{} i termini tecnici, gli acronimi e le abbreviazioni sono definiti in modo chiaro e conciso, in modo tale da evitare ambiguità e massimizzare la comprensione dei documenti.
\newline I vocaboli presenti in esso saranno posti in corsivo e presenteranno una "G" maiuscola a pedice.
\subsection{Riferimenti}
\subsubsection{Normativi}
\begin{itemize}
	\item \textbf{\NdP{}};
	\item \textbf{Definizione e struttura dell'analisi dei requisiti, dal corso di Vardanega Tullio di Ingegneria del Software, anno 2017/2018}
	\href{http://www.math.unipd.it/~tullio/IS-1/2017/Dispense/L08.pdf}{http://www.math.unipd.it/~tullio/IS-1/2017/Dispense/L08.pdf}
	
\end{itemize}

\subsubsection{Informativi}
\begin{itemize}
	\item \textbf{Capitolato d'appalto C6: Marvin}\\ 
	\href{http://www.math.unipd.it/~tullio/IS-1/2017/Progetto/C6.pdf}{http://www.math.unipd.it/~tullio/IS-1/2017/Progetto/C6.pdf};
	\item \textbf{\SdF};
	\item \textbf{Slides del corso di Ingegneria del Software, anno 2017/2018}\\
	\href{http://www.math.unipd.it/~tullio/IS-1/2017/}{http://www.math.unipd.it/~tullio/IS-1/2017/}
\end{itemize}

\section{Descrizione generale}
	\subsection{Contesto d'uso del prodotto}
	Il prodotto finale vuole essere una sorta di \emph{PoC}\ped{G} per dimostrare la fattibilità di utilizzo di tali tecnologie in quest’ambito. L’applicazione sarà un prototipo di Uniweb, quindi si colloca in un contesto universitario dove gli attori si approcciano al sistema come nell’attuale Uniweb. La differenza sta nel \emph{back-end}\ped{G} dove, invece del classico sistema \emph{client}\ped{G}/\emph{server}\ped{G}, troviamo un database distribuito che sfrutta la piattaforma Ethereum.
	
	\subsection{Funzioni del prodotto}
	Il prodotto permetterà agli studenti di:
	\begin{itemize}
		\item Accettare o rifiutare un voto;
		\item Vedere il proprio libretto;
		\item Iscriversi ad un esame;
	\end{itemize}
	permetterà invece ad un professore di:
	\begin{itemize}
		\item Inserire voti agli studenti;
		\item Vedere gli esami a cui è assegnato;
		\item Vedere gli studenti iscritti ai propri esami;
	\end{itemize}
	e infine permetterà ad amministrazione ed università di aggiungere, eliminare e modificare:
	\begin{itemize}
		\item Anni accademici;
		\item Corsi di laurea;
		\item Attività didattiche;
		\item Esami;
		\item Utenti.
	\end{itemize}
	
	\subsection{Caratteristiche degli utenti}
	Questo prodotto deve risultare accessibile ad un'ampia categoria di utenti senza particolari competenze. L’interfaccia dovrà quindi essere il più chiara ed intuitiva possibile. Verrà fornito anche un \MU{} con tutte le indicazioni necessarie per consentire il corretto utilizzo del prodotto.
	
	\subsection{Piattaforma di esecuzione}
Viene assicurata la completa funzionalità del prodotto nei browser Google Chrome (aggiornato alla versione 60) e Mozilla Firefox (aggiornato alla versione 50).
Sarà quindi possibile utilizzare il prodotto in qualsiasi sistema operativo che rispetti il vincolo sul browser, inoltre la versione di Metamask dovrà essere la 4.7.1.
Non sarà assicurato il corretto funzionamento in piattaforme che non rispettano questi requisiti.
