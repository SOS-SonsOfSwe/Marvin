%generare il pdf con il comando: pdflatex main.tex
\documentclass[a4paper, oneside, openany]{article}
\usepackage{../template/sos}
\newcommand{\Titolo}{Norme di Progetto}

\newcommand{\Gruppo}{SonsOfSwe}

\newcommand{\Redazione}{Caldart Federico, Cavallin Giovanni, Dalla Riva Giovanni, Favero Andrea, Menegon Lorenzo, Panozzo Stefano, Thiella Eleonora}

\newcommand{\ACapoRedazione}{Caldart Federico \newline Cavallin Giovanni \newline Dalla Riva Giovanni \newline Favero Andrea \newline Menegon Lorenzo \newline Panozzo Stefano \newline Thiella Eleonora}

\newcommand{\Verifica}{Caldart Federico}

\newcommand{\Approvazione}{Cavallin Giovanni}

\newcommand{\Distribuzione}{Vardanega Tullio}

\newcommand{\Uso}{interno}

\newcommand{\Data}{3 Marzo 2018}

\newcommand{\NomeProgetto}{Progetto Speect}

\newcommand{\Mail}{sonsofswe.swe@gmail.com}

\newcommand{\DescrizioneDoc}{Questo documento descrive le regole, gli strumenti e le convenzioni adottate dal gruppo SonsOfSwe durante la realizzazione del progetto Marvin.}
\setcounter{secnumdepth}{0}

\begin{document}
\copertina{}
%%%%%%%%%%%%%%%%%%%%%%%%%%%%%%%%%%%%%%%%%%%%%%%%%%%%%%%%%%%%%%%%%%%%%%%%%%%%%%%%%%%%%%%%%%%%%%%%%%%%%%%
%SOMMARIO
\tableofcontents
\newpage
%%%%%%%%%%%%%%%%%%%%%%%%%%%%%%%%%%%%%%%%%%%%%%%%%%%%%%%%%%%%%%%%%%%%%%%%%%%%%%%%%%%%%%%%%%%%%%%%%%%%%%%
%PARAGRAFI
%%%%%%%%%%%%%%%%%%%%%%%%%%%%%%%%%%%%%%%%%%%%%%%%%%%%%%%%%%%%%%%%%%%%%%%%%%%%%%%%%%%%%%%%%%%%%%%%%%%%%%%
\section{A}
\begin{itemize}
\item \textbf{Apache Mahout}: Apache Mahout è un progetto atto a produrre un'implementazione gratuita di algoritmi e applicazioni di machine learning.
\item \textbf{Apache Nutch}: Apache Nutch è un crawler, un software automatizzato che analizza la rete; è open-source, altamente estensibile e scalabile.
\item \textbf{Apache PredictionIO}: Apache PredictionIO è un server open-source creato per sviluppatori e data scientist orientato alle attività di apprendimento automatico.
\item \textbf{Apache Solr}: Apache Solr è una piattaforma di ricerca open-source.
\item \textbf{API}: API è l'acronimo di Application programming interface ed indica un'insieme di procedure, protocolli e strumenti messi a disposizione per lo sviluppo di software, definendo i metodi con i quali interagiscono le varie componenti.
\item \textbf{APM}: APM indica il monitoraggio e la gestione di performance e disponibilità delle applicazioni, al fine di individuare e diagnosticare facilmente problemi complessi che hanno un impatto negativo sul servizio erogato.
\item \textbf{Applicazione}: un'applicazione indica un programma creato per eseguire un determinato insieme di compiti, funzioni o attività a servizio dell'utente che la utilizza.
\item \textbf{Astah}: precedentemente noto come JUDE, è uno strumento gratuito di modellazione UML.
\item \textbf{Attore}: nei diagrammi dei casi d'uso l'attore rappresenta un ruolo coperto da un certo insieme di entità interagenti col sistema (inclusi utenti umani, altri sistemi software, dispositivi hardware e così via). Graficamente è rappresentato da uno stickman.
\end{itemize}


\section{B}
\begin{itemize}
\item \textbf{Back-end}: nell'ingegneria del software, il back-end si occupa di manipolare ed elaborare i dati ricevuti dal front-end.	
\item \textbf{Blockchain}: blockchain è un database distribuito 
\item \textbf{Bootstrap}: bootstrap è un insieme di stumenti open-source utilizzabili nello sviluppo dei siti web.
\item \textbf{Branch}: significa "ramo" e in Git indica un puntatore ad un commit.
\item \textbf{Build}: con il termine build si intende una costruzione o qualcosa di costruito.
\end{itemize}

\section{C}
\begin{itemize}
\item \textbf{Capitolato}: Documento tecnico, in genere allegato ad un contratto di appalto, che vi fa riferimento per definire in quella sede le specifiche tecniche delle opere che andranno ad eseguirsi per effetto del contratto stesso, di cui è solamente parte integrante. 
\item \textbf{Capacità}: Abilità di un'organizzazione, sistema o processo a realizzare un prodotto/servizio in grado di soddisfare i requisiti fissati.
\item \textbf{Caso d'uso}:Tecnica usata nei processi di ingegneria del software per effettuare in maniera esaustiva
e non ambigua, la raccolta dei requisiti al fine di produrre software di qualità. Essa con-
siste nel valutare ogni requisito focalizzandosi sugli attori che interagiscono col sistema,
valutandone le varie interazioni. In UML sono rappresentati dagli Use Case Diagram.
\item \textbf{Cassandra}: Cassandra è un sistema di gestione di database  open-source e gratuito, sviluppato per manipolare una grande mole di dati.
\item \textbf{Client}: in informatica, un client è un computer o un programma (quindi hardware o software) che accede e sfrutta un servizio offerto da un server.
\item \textbf{CMake}: cmake è l'abbreviazione di "cross platform make" ed è un software ideato per automatizzare le fasi di compilazione e test nello sviluppo di un software.
\item \textbf{Commit}: con il termine commit si intende il salvataggio dello stato del progetto su cui si sta lavorando in Git.
\item \textbf{Committente}: Figura che commissiona un lavoro, può essere una persona fisica o una cosiddetta "persona giuridica". Ha il potere decisionale e di spesa relativo alla gestione del lavoro commissionato.
\item \textbf{Configuration Item}: è l'elemento oggetto delle attività di gestione della configurazione, un'attività del processo di sviluppo del software che ha lo scopo di permettere la gestione ed il controllo degli oggetti di sistemi complessi.
\item \textbf{Cloud}: Cloud è l'abbreviazione di \textit{cloud computing} ed indica un modo di distribuire servizi (calcolo, archiviazione ecc) attraverso la rete, utilizzando quindi delle risorse non disponibili fisiacamente, ma accessibili tramite internet.
\item \textbf{CSS3}: CSS è un linguaggio usato nel web per descrivere come vengono visualizzati gli elementi HTML in una pagina web. "3" indica l'ultima versione disponibile.
\end{itemize}


\section{D}
\begin{itemize}
\item \textbf{Dashboard}: dashboard è un termine che sta ad indicare un'interfaccia grafica attraverso la quale è possibile visualizzare dati e/o tenere traccia del loro avanzamento.
\item \textbf{Database relazionale}:
\item \textbf{ÐApp}: ÐApp è l'abbrevazione di applicazione decentralizzata
\item \textbf{Data visualization}: è l'esplorazione visuale/interattiva e la relativa rappresentazione grafica di dati di qualunque dimensione, natura e origine. Permette ai manager di identificare fenomeni e trend che risultano invisibili ad una prima analisi dei dati.
\item \textbf{Design pattern}: è una descrizione o modello logico da applicare per la risoluzione di un problema che può presentarsi in diverse situazioni durante le fasi di progettazione e sviluppo del software.
\item \textbf{DevOps}: DevOps è l'abbreviazione ed unione di "Development and Operations" ed indica una modello di sviluppo in azienda, che mira a creare team multidisciplinari uniti ed in grado di collaborare in maniera efficace nonostante i diversi ambiti d'interesse.
\item \textbf{Diagramma di Gantt}: è uno strumento di supporto alla gestione dei progetti ed è costruito da un asse orizzontale, che rappresenta l'arco temporale totale del progetto suddiviso in fasi incrementali,e da un asse verticale, che rappresenta le mansioni o attività che costituiscono il progetto.
\item \textbf{Diagramma di stato}: è un tipo di diagramma UML che mostra gli stati e gli eventi che innescano una transizione da uno stato all'altro.
\end{itemize}

\section{E}
\begin{itemize}
\item \textbf{Efficacia}: è la misura della capacità di raggiungere gli obiettivi prefissati nei tempi previsti.
\item \textbf{Efficienza}: è la misura per stimare l'abilità di raggiungere gli obiettivi impiegando il minor numero di risorse.
\item \textbf{Elasticsearch}: ElasticSearch è un potente e veloce motore/server di ricerca open-source costruito su Lucene.
\item \textbf{Ethereum}: Ethereum è una piattaforma software open-source basata su Blockchain che permette di sviluppare applicazioni decentralizzate.
\item \textbf{EVM}: EVM è l'abbreviazione di Ethereum Virtual Machine e può essere visto come un grande computer virtuale decentralizzato e si occupa di gestire effettivamente i dati nel database interno e la parte computazionale.
\end{itemize}

\section{F}
\begin{itemize}
\item \textbf{Framework}:
\item \textbf{Front-end}: il front end è la parte del software che si occupa di fornire una semplificazione della logica dell'applicazione attraverso un'interfaccia di facile comprensione da parte dell'utente.
\end{itemize}

\section{G}
\begin{itemize}
\item \textbf{Git}: Git è un sistema di controllo del versionamento del software.
\item \textbf{Glade}: Glade è uno strumento utile a velocizzare e semplificare lo sviluppo di interfacce grafiche con gli strumenti offerti da Gtk+.
\item \textbf{Google Cloud Datastore}: Google Cloud Datastore è un database non SQL sviluppato per avere alte prestazioni in grado di scalare efficacemente per gestire diversi tipi di applicazioni.
\item \textbf{Google SQL}: Google SQL è un sistema di database che rende facile creazione, gestione e manutenzione di database relazionali su Google Platform.
\item \textbf{Google Platform}: Google Platform è un servizio offerto da Google che si sostanzia nel mettere a disposizione delle piattaforme in cui è possibile sviluppare, testare e implementare applicazioni.
\item \textbf{Gtk+}: Gtk+ è un insieme di strumenti mulipiattaforma utili a sviluppare applicazioni grafiche.
\end{itemize}

\section{H}
\begin{itemize}
\item \textbf{Hosting}: Hosting indica un servizio che mette a disposizione una parte della rete per ospitare un'applicazione o un altro servizio al fine di renderlo accessibile attraverso internet.
\item \textbf{Hyperledger Fabric}:	
\item \textbf{HTML5}: HTML5 è un linguaggio di markup con il quale si strutturano le pagine web. "5" indica l'ultima versione disponibile aderente allo standard.
\item \textbf{HTTPS}: abbreviazione di HyperText Transfer Protocol, HTTPS è un protocollo usato per fornire connessioni sicure attraverso una rete.
\end{itemize}

\section{I}
\begin{itemize}
\item \textbf{ISO}: è l'Organizzazione internazionale per la normazione (in inglese International Organization for Standardization) ed è la più importante organizzazione a livello mondiale per la definizione di norme tecniche.
\item \textbf{IT}: IT è l'abbreviazione di Information Technology, cioè qualsiasi attività che faccia uso di elaboratori o strumenti di telecomunicazione per manipolare, gestire e memorizzare dati.
\end{itemize}

\section{J}
\begin{itemize}
\item \textbf{Java}: Java è un linguaggio di programmazione e una piattaforma di elaborazione.
\item \textbf{Java EE}: Java EE è l'abbreviazione di Java Enterprise Edition ed è una raccolta di specifiche per lo sviluppo e la distribuzione di applicazioni aziendali.	
\item \textbf{JavaScript}: Javascript è un linguaggio di programmazione largamente usato nel web per la creazione di contenuti dinamici nel front-end. 
\end{itemize}

\section{K}
\begin{itemize}
\item \textbf{Keycloak}: Keycloack è un sistema open-source usato per la gestione di identità e accesso all'intero di applicazioni software.
\item \textbf{Kibana}: Kibana è un'interfaccia grafica utile a visualizzare e navigare nei dati memorizzati in Elastic.
\end{itemize}

\section{L}
\begin{itemize}
\item \textbf{\LaTeX}: è un linguaggio di markup usato per la preparazione di testi basato sul programma di composizione tipografica \TeX.
\item \textbf{Linguaggi di markup}: vengono definiti linguaggi di markup tutti quei linguaggi che decrivono dati attraverso una formattazione specifica usando dei marcatori; un esempio è HTML5.
\item \textbf{Lucene}: Lucene è una API open-source scritta in Java.
\end{itemize}

\section{M}
\begin{itemize}
\item \textbf{Machine learning}: il machine learning è una branca dell'informatica che si occupa di fornire ad un elaboratore la capacità di "imparare" autonomamente, cioè senza bisogno di venire appositamente programmato per farlo.
\item \textbf{Manutenzione}: con manutenzione si intendono le attività di modifica di un software successive alla sua distribuzione, atte alla correzione di errori o implementazione di funzionalità aggiuntive.
\item \textbf{Manutenibile}: è un importante requisito di progetto di un sistema che definisce la sua capacità di essere facilmente ripristinato qualora sia necessario realizzare un intervento di manutenzione. La manutenibilità è quindi indispensabile al sistema per ottimizzare l'implementazione delle attività manutentive e si misura in tempo e costo del singolo intervento.
\item \textbf{Master}: in Git è il nome del ramo principale, vale a dire un puntatore ad un commit.
\item \textbf{Microsoft Excel 365}: è un programma prodotto da Microsoft dedicato alla produzione ed alla gestione di fogli elettronici. È parte della suite di software di produttività personale Microsoft Office ed è disponibile per i sistemi operativi Windows e Macintosh.
\item \textbf{Milestone}: indica un punto nel tempo associato ad un valore strategico. Ogni milestone di calendario è associata ad uno specifico insieme di baseline e dev'essere: specifica, raggiungibile, misurabile, traducibile in compiti assegnabili e dimostrabile agli stakeholder.
\item \textbf{MongoDB}: MongoDB è un sistema di gestione di database non relazionale, orientato ai documenti.
\end{itemize}

\section{N}
\begin{itemize}
\item \textbf{NodeJS}: NodeJS è un framework opensource utilizzato per creare applicazioni a lato server con Javascript.
\end{itemize}

\section{O}
\begin{itemize}
\item \textbf{Open-source}: Open-source definisce software il cui codice sorgente può essere analizzato, modificato ed esteso, tramite l'applicazione di specifiche licenze d'uso.
\item \textbf{Organigramma}: è la rappresentazione grafica di una struttura organizzativa ed è composta da rettangoli, che rappresentano gli enti, e da linee, che rappresentano le relazioni gerarchiche tra gli enti.
\item \textbf{Origin}: è il nome predefinito che Git da al server da cui si clona.
\end{itemize}

\section{P}
\begin{itemize}
\item \textbf{Package}: è l'insieme dei files che permettono l'installazione ed il corretto funzionamento di una applicazione ed è di solito distribuito come unico file compresso per semplificare e velocizzare le operazioni di distribuzione e trasporto.
\item \textbf{Permissioned blockchain}: permissioned blockchain indica un tipo di blockchain costruito in modo da non essere pubblico, bensì si basa sull'assegnazione di permessi per leggere l'informazione, in modo da limitare l'utenza che può fare transazione sul blockchain o creare nuovi blocchi.
\item \textbf{Pianificazione}: è il sistema operativo, all'interno dell'organizzazione aziendale, attraverso il quale l'azienda definisce i suoi obiettivi, previa analisi della realizzabilità e dei conseguenti vantaggi, e le azioni necessarie atte a conseguirli.
\item \textbf{Play}: Play è un framework open-source avente lo scopo di migliorare la produttività degli sviluppatori, fornendo un consumo di risorse minimale e predicibile.
\item \textbf{PoC}: PoC sta per Proof of Concept ed indica un'incompleta realizzazione di un certo progetto, cioè una bozza, allo scopo di dimostrare l'effettivo potenziale dell'oggetto in esame.
\item \textbf{Processo}: un processo software è un insieme di attività correlate e coese che trasformano dei bisogni (fornitigli in ingresso) in dei prodotti (il loro output).
Per fare ciò queste attività consumano delle risorse.%misto tra sommerville e slide di Tullio
\item \textbf{Prodotto}: un prodotto software è il risultato di un processo software. In particolare è il risultato di almeno un'attività necessariamente effettuata all’interfaccia tra il fornitore ed il cliente ed è generalmente intangibile.
\item \textbf{Progetto}: è l'organizzazione di azioni nel tempo per il perseguimento di uno scopo predefinito attraverso le varie fasi di progettazione da parte di uno o più progettisti. Lo scopo finale è la realizzazione di un bene o servizio il cui ciclo di sviluppo è gestito attraverso tecniche di project management.
\item \textbf{Proponente}: Colui che presenta una proposta, in questo caso il capitolato riguardante il progetto. Nello specifico ci si riferisce a Red Babel.
\item \textbf{Prototipo}: è il modello originale o il primo esemplare di un manufatto, rispetto a una sequenza di eguali o similari realizzazioni successive.
\item \textbf{Pull request}: significa letteralmente "richiesta di pull" e in Git avviene quando un utente ha eseguito la push del proprio lavoro sul fork e devi avvisare i mantenitori.
\end{itemize}

\section{Q}
\begin{itemize}
\item \textbf{Qt}: Qt è una libreria multipiattaforma per lo sviluppo di applicazioni con o senza interfaccia grafica.
\item \textbf{QtCreator}: QtCreator è un ambiente di sviluppo multipiattaforma che fornisce supporto ad implementazione ed esecuzione di applicazioni create sfruttando Qt.
\end{itemize}

\section{R}
\begin{itemize}
\item \textbf{React}: React è una libreria Javascript creata per lo sviluppo di interfacce grafiche.
\item \textbf{Redux}: Redux è una libreria JavaScript open-source avente l'obiettivo di gestire lo stato delle applicazioni web.
\item \textbf{Repository}: repository indica un sistema informativo creato per memorizzare metadata, cioè informazioni riguardo la struttare vera e propria dei dati
\item \textbf{Requisito}: è la descrizione di cosa il sistema deve fare, cioè i servizi che offre e i vincoli sul suo funzionamento. Dal punto di vista del bisogno, il requisito è una condizione necessaria ad un utente per risolvere un problema o raggiungere un obiettivo, mentre dal punto di vista della soluzione è una condizione che dev'essere soddisfatta da un sistema per adempiere ad un obbligo.
\item \textbf{Responsive}: responsive indica una tecnica di web design per la realizzazione di interfacce in grado di adattarsi in maniera automatica al dispositivo su cui vengono visualizzate.
\item \textbf{Revisione}: s'intende un esame per controllare ed eventualmente correggere errori, difetti o imperfezioni.
\item \textbf{Robustness diagram}: è un diagramma di comunicazione/collaborazione UML semplificato che utilizza i simboli grafici.
\end{itemize}

\section{S}
\begin{itemize}
\item \textbf{SCSS}: è un'estensione di CSS che aggiunge potenza ed eleganza alla versione base.
\item \textbf{Segretario}: è un ruolo che ricopre la persona che durante le riunioni del team tiene traccia degli argomenti trattati e delle decisioni prese.
\item \textbf{Server}: Componente o sottosistema informatico di elaborazione e gestione del traffico di informazioni che fornisce, a livello logico e fisico, un qualunque tipo di servizio ad altre componenti (tipicamente chiamate clients, cioè clienti) che ne fanno richiesta attraverso una rete di computer, all’interno di un sistema informatico o anche direttamente in locale su un computer;
\item \textbf{Sistemi di Raccomandazione}: è un sistema che aggrega e rielabora informazioni fornite dall'utente e ne fornisce suggerimenti utili ad esso;
\item \textbf{Slack}: è un software che rientra nella categoria degli strumenti di collaborazione aziendale utilizzato per inviare messaggi in modo istantaneo ai membri del team.
\item \textbf{Smart Contracts}: sono protocolli informatici che facilitano, verificano, o fanno rispettare, la negoziazione o l'esecuzione di un contratto, permettendo talvolta la parziale o la totale esclusione di una clausola contrattuale;
\item \textbf{Solidity}: è un linguaggio di programmazione orientato ai contratti, è usato per scrivere smart contracts ed è molto utilizzato su piattaforme basate su ethereum;
\item \textbf{Stand-alone}: è un'espressione che indica che un oggetto o un software è capace di funzionare da solo o in maniera indipendente da altri oggetti o software, con cui potrebbe altrimenti interagire.
\item \textbf{Standup}: è una riunione di gruppo giornaliera particolarmente usata nella metodologia di sviluppo software Agile;
\item \textbf{SWEgo}: è un tool online dove, in modo rapido e semplificato, è possibile tracciare requisiti e casi d'uso.
\end{itemize}

\section{T}
\begin{itemize}
\item \textbf{Task}: è il compito specifico di un programma applicativo, di una procedura o di una sequenza di istruzioni del sistema operativo di un calcolatore elettronico.
\item \textbf{Team}: una squadra di persone che sviluppano un progetto. Utilizzato pret-
tamente in ambito informatico, si riferisce al gruppo di persone che supporta e sviluppa
un software o un’applicazione all’interno di una società, un’associazione o un’azienda;
\item \textbf{Telegram}: è un servizio di messaggistica istantanea basato su cloud, così da garantire la sincronizzazione istantanea e consente all'utente di poter accedere ai messaggi da diversi dispositivi contemporaneamente, inclusi tablet e computer.
\item \textbf{Template}: è un documento o programma nel quale, come in un foglio semicompilato cartaceo, su una struttura generica o standard esistono spazi temporaneamente "bianchi" da riempire successivamente.
\item \textbf{Tomcat}: Application server sviluppato dalla Apache Software Foundation. Implementa le specifiche JavaServer Pages (JSP) e Servlet, fornendo quindi una piattaforma software per l’esecuzione di applicazione web sviluppate in linguaggio Java;
\item \textbf{TTS}: text-to-speech, sistemi che convertono il testo in parlato.
\end{itemize}

\section{U}
\begin{itemize}
\item \textbf{UML}: linguaggio di modellazione e specifica basato sul paradigma orientato agli oggetti.
\item \textbf{Uso}: per questi documenti può essere interno, se visionato solo dai committenti, o esterno, se visionato anche dai proponenti.
\end{itemize}

\section{V}
\begin{itemize}
\item \textbf{Vaadin Elements}: una serie di elementi utili alle interfacce utente delle applicazioni web.
\item \textbf{Verbale}: è il documento dove sono riportate le informazioni generali, gli argomenti trattati e le decisioni prese durante una riunione del team.
\item \textbf{Verifica}: è il processo che verifica se un sistema soddisfa la sua specifica.
\item \textbf{Versionamento}: è la funzione che dà la possibilità di gestire le varie versioni di uno stesso documento e che tiene traccia delle varie modifiche che vengono effettuate nel tempo.
\end{itemize}



\end{document}
