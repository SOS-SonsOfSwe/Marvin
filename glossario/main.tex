%generare il pdf con il comando: pdflatex main.tex
\documentclass[a4paper, oneside, openany]{article}
\usepackage{../template/sos}
\newcommand{\Titolo}{Norme di Progetto}

\newcommand{\Gruppo}{SonsOfSwe}

\newcommand{\Redazione}{Caldart Federico, Cavallin Giovanni, Dalla Riva Giovanni, Favero Andrea, Menegon Lorenzo, Panozzo Stefano, Thiella Eleonora}

\newcommand{\ACapoRedazione}{Caldart Federico \newline Cavallin Giovanni \newline Dalla Riva Giovanni \newline Favero Andrea \newline Menegon Lorenzo \newline Panozzo Stefano \newline Thiella Eleonora}

\newcommand{\Verifica}{Caldart Federico}

\newcommand{\Approvazione}{Cavallin Giovanni}

\newcommand{\Distribuzione}{Vardanega Tullio}

\newcommand{\Uso}{interno}

\newcommand{\Data}{3 Marzo 2018}

\newcommand{\NomeProgetto}{Progetto Speect}

\newcommand{\Mail}{sonsofswe.swe@gmail.com}

\newcommand{\DescrizioneDoc}{Questo documento descrive le regole, gli strumenti e le convenzioni adottate dal gruppo SonsOfSwe durante la realizzazione del progetto Marvin.}
\setcounter{secnumdepth}{0}

\begin{document}
\copertina{}
%%%%%%%%%%%%%%%%%%%%%%%%%%%%%%%%%%%%%%%%%%%%%%%%%%%%%%%%%%%%%%%%%%%%%%%%%%%%%%%%%%%%%%%%%%%%%%%%%%%%%%%
%SOMMARIO
\tableofcontents
\newpage
%%%%%%%%%%%%%%%%%%%%%%%%%%%%%%%%%%%%%%%%%%%%%%%%%%%%%%%%%%%%%%%%%%%%%%%%%%%%%%%%%%%%%%%%%%%%%%%%%%%%%%%
%PARAGRAFI
%%%%%%%%%%%%%%%%%%%%%%%%%%%%%%%%%%%%%%%%%%%%%%%%%%%%%%%%%%%%%%%%%%%%%%%%%%%%%%%%%%%%%%%%%%%%%%%%%%%%%%%
\section{A}
\begin{itemize}
\item \textbf{Apache Mahout}:
\item \textbf{Apache Nutch}:
\item \textbf{Apache PredictionIO}:
\item \textbf{Apache Solr}:
\item \textbf{API}:
\item \textbf{APM}:
\item \textbf{Applicazione}:
\end{itemize}

\section{B}
\begin{itemize}
\item \textbf{Back-end}: nell'ingegneria del software, il back-end si occupa di manipolare ed elaborare i dati ricevuti dal front-end.	
\item \textbf{Blockchain}: blockchain è un database distribuito 
\item \textbf{Bootstrap}: bootstrap è un insieme di stumenti open-source utilizzabili nello sviluppo dei siti web.  
\end{itemize}

\section{C}
\begin{itemize}
\item \textbf{Capitolato \color{red}{serve davvero?}}: 
\item \textbf{Cassandra}: Cassandra è un sistema di gestione di database  open-source e gratuito, sviluppato per manipolare una grande mole di dati.
\item \textbf{Client}: in informatica, un client è un computer o un programma (quindi hardware o software) che accede e sfrutta un servizio offerto da un server.
\item \textbf{CMake}:
\item \textbf{Committente \color{red}{serve davvero?}}: 
\item \textbf{CSS3}: CSS è un linguaggio usato nel web per descrivere come vengono visualizzati gli elementi HTML in una pagina web. "3" indica l'ultima versione disponibile.
\end{itemize}


\section{D}
\begin{itemize}
\item \textbf{Dashboard}: dashboard è un termine che sta ad indicare un'interfaccia grafica attraverso la quale è possibile visualizzare dati e/o tenere traccia del loro avanzamento.
\item \textbf{Database relazionale}:
\item \textbf{ÐApp}: ÐApp è l'abbrevazione di applicazione decentralizzata
\item \textbf{Data visualization}:
\item \textbf{DevOps}:
\end{itemize}

\section{E}
\begin{itemize}
\item \textbf{Elasticsearch}:
\item \textbf{Ethereum}: 
\item \textbf{EVM}:
\end{itemize}

\section{F}
\begin{itemize}
\item \textbf{Framework}:
\item \textbf{Front-end}:
\end{itemize}

\section{G}
\begin{itemize}
\item \textbf{Git}:
\item \textbf{Glade}:
\item \textbf{Google Cloud Datastore}:
\item \textbf{Google SQL}:
\item \textbf{Google Platform}:
\item \textbf{Gtk+}:
\end{itemize}

\section{H}
\begin{itemize}
\item \textbf{Hyperledger Fabric}:
\item \textbf{HTML5}:
\item \textbf{HTTPS}:
\end{itemize}

\section{I}
\begin{itemize}
\item \textbf{IT}:
\end{itemize}

\section{J}
\begin{itemize}
\item \textbf{Java}:
\item \textbf{Java EE}:
\item \textbf{JavaScript}:
\end{itemize}

\section{K}
\begin{itemize}
\item \textbf{Keycloak}:
\item \textbf{Kibana}:
\end{itemize}

\section{L}
\begin{itemize}
\item \textbf{Linguaggi di markup}:
\item \textbf{Lucene}:
\end{itemize}

\section{M}
\begin{itemize}
\item \textbf{Machine learning}:
\item \textbf{Manutenzione}:
\item \textbf{MongoDB}:
\end{itemize}

\section{N}
\begin{itemize}
\item \textbf{NodeJS}:
\end{itemize}

\section{O}
\begin{itemize}
\item \textbf{Open-source}:
\end{itemize}

\section{P}
\begin{itemize}
\item \textbf{Permissioned blockchain}: una permissioned blockchain è una blockchain che offre un meccanismo di controllo degli accessi in modo che i peer siano autorizzati o rifiutati in base a un valore di controllo.
\item \textbf{Play}: Play è un framework open source, scritto in Java e Scala, che implementa il pattern model-view-controller. Il suo scopo è quello di migliorare la produttività degli sviluppatori usando il paradigma Convention Over Configuration, il caricamento del codice a caldo e la visualizzazione degli errori nel browser.
\item \textbf{PoC}: significa proof of concept, ovvero una dimostrazione pratica dei funzionamenti di base di un applicativo o intero sistema integrandolo all'interno di un ambiente già esistente.
\item \textbf{Proponente}: Colui che presenta una proposta, in questo caso il capitolato riguardante il progetto. 
\item \textbf{Prototipo}: primo esemplare, potenzialmente incompleto, di una serie di realizzazioni successive.
\end{itemize}

\section{Q}
\begin{itemize}
\item \textbf{Qt}: una libreria multipiattaforma per lo sviluppo di programmi con interfaccia grafica tramite l'uso di widget;
\item \textbf{QtCreator}: è un IDE cross-platform fatto su misura per gli sviluppatori Qt.
\end{itemize}

\section{R}
\begin{itemize}
\item \textbf{React}:
\item \textbf{Redux}:
\item \textbf{Repository}:
\item \textbf{Responsive}:
\item \textbf{Robustness diagram}:
\end{itemize}

\section{S}
\begin{itemize}
\item \textbf{SCSS}:
\item \textbf{Server}:
\item \textbf{Sistemi di Raccomandazione}:
\item \textbf{Smart Contracts}:
\item \textbf{Solidity}:
\item \textbf{Standup}:
\end{itemize}

\section{T}
\begin{itemize}
\item \textbf{Team}:
\item \textbf{Tomcat}:
\item \textbf{TTS}
\end{itemize}

\section{U}
\begin{itemize}
\item \textbf{UML}:s
\end{itemize}

\section{V}
\begin{itemize}
\item \textbf{Vaadin Elements}:
\end{itemize}

\section{W}
\begin{itemize}

\end{itemize}

\section{X}
\begin{itemize}

\end{itemize}

\section{Y}

\section{Z}

\end{document}