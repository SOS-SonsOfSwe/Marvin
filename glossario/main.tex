%generare il pdf con il comando: pdflatex main.tex
\documentclass[a4paper, oneside, openany]{article}
\usepackage{../template/sos}
\newcommand{\Titolo}{Norme di Progetto}

\newcommand{\Gruppo}{SonsOfSwe}

\newcommand{\Redazione}{Caldart Federico, Cavallin Giovanni, Dalla Riva Giovanni, Favero Andrea, Menegon Lorenzo, Panozzo Stefano, Thiella Eleonora}

\newcommand{\ACapoRedazione}{Caldart Federico \newline Cavallin Giovanni \newline Dalla Riva Giovanni \newline Favero Andrea \newline Menegon Lorenzo \newline Panozzo Stefano \newline Thiella Eleonora}

\newcommand{\Verifica}{Caldart Federico}

\newcommand{\Approvazione}{Cavallin Giovanni}

\newcommand{\Distribuzione}{Vardanega Tullio}

\newcommand{\Uso}{interno}

\newcommand{\Data}{3 Marzo 2018}

\newcommand{\NomeProgetto}{Progetto Speect}

\newcommand{\Mail}{sonsofswe.swe@gmail.com}

\newcommand{\DescrizioneDoc}{Questo documento descrive le regole, gli strumenti e le convenzioni adottate dal gruppo SonsOfSwe durante la realizzazione del progetto Marvin.}
\setcounter{secnumdepth}{0}

\begin{document}
\copertina{}
%%%%%%%%%%%%%%%%%%%%%%%%%%%%%%%%%%%%%%%%%%%%%%%%%%%%%%%%%%%%%%%%%%%%%%%%%%%%%%%%%%%%%%%%%%%%%%%%%%%%%%%
%SOMMARIO
\tableofcontents
\newpage
%%%%%%%%%%%%%%%%%%%%%%%%%%%%%%%%%%%%%%%%%%%%%%%%%%%%%%%%%%%%%%%%%%%%%%%%%%%%%%%%%%%%%%%%%%%%%%%%%%%%%%%
%PARAGRAFI
%%%%%%%%%%%%%%%%%%%%%%%%%%%%%%%%%%%%%%%%%%%%%%%%%%%%%%%%%%%%%%%%%%%%%%%%%%%%%%%%%%%%%%%%%%%%%%%%%%%%%%%
\section{A}
\begin{itemize}
\item \textbf{Apache Mahout}: Apache Mahout è un progetto atto a produrre un'implementazione gratuita di algoritmi e applicazioni di machine learning.
\item \textbf{Apache Nutch}: Apache Nutch è un crawler, un software automatizzato che analizza la rete; è open-source, altamente estensibile e scalabile.
\item \textbf{Apache PredictionIO}: Apache PredictionIO è un server open-source creato per sviluppatori e data scientist orientato alle attività di apprendimento automatico.
\item \textbf{Apache Solr}: Apache Solr è una piattaforma di ricerca open-source.
\item \textbf{API}: API è l'acronimo di Application programming interface ed indica un'insieme di procedure, protocolli e strumenti messi a disposizione per lo sviluppo di software, definendo i metodi con i quali interagiscono le varie componenti.
\item \textbf{APM}: APM indica il monitoraggio e la gestione di performance e disponibilità delle applicazioni, al fine di individuare e diagnosticare facilmente problemi complessi che hanno un impatto negativo sul servizio erogato.
\item \textbf{Applicazione}: un'applicazione indica un programma creato per eseguire un determinato insieme di compiti, funzioni o attività a servizio dell'utente che la utilizza.
\end{itemize}

\section{B}
\begin{itemize}
\item \textbf{Back-end}: nell'ingegneria del software, il back-end si occupa di manipolare ed elaborare i dati ricevuti dal front-end.	
\item \textbf{Blockchain}: blockchain è un database distribuito 
\item \textbf{Bootstrap}: bootstrap è un insieme di stumenti open-source utilizzabili nello sviluppo dei siti web.  
\end{itemize}

\section{C}
\begin{itemize}
\item \textbf{Capitolato \color{red}{serve davvero?}}: 
\item \textbf{Cassandra}: Cassandra è un sistema di gestione di database  open-source e gratuito, sviluppato per manipolare una grande mole di dati.
\item \textbf{Client}: in informatica, un client è un computer o un programma (quindi hardware o software) che accede e sfrutta un servizio offerto da un server.
\item \textbf{CMake}: cmake è l'abbreviazione di "cross platform make" ed è un software ideato per automatizzare le fasi di compilazione e test nello sviluppo di un software.
\item \textbf{Committente \color{red}{serve davvero?}}: 
\item \textbf{CSS3}: CSS è un linguaggio usato nel web per descrivere come vengono visualizzati gli elementi HTML in una pagina web. "3" indica l'ultima versione disponibile.
\end{itemize}


\section{D}
\begin{itemize}
\item \textbf{Dashboard}: dashboard è un termine che sta ad indicare un'interfaccia grafica attraverso la quale è possibile visualizzare dati e/o tenere traccia del loro avanzamento.
\item \textbf{Database relazionale}:
\item \textbf{ÐApp}: ÐApp è l'abbrevazione di applicazione decentralizzata
\item \textbf{Data visualization}:
\item \textbf{DevOps}: DevOps è l'abbreviazione ed unione di "Development and Operations" ed indica una modello di sviluppo in azienda, che mira a creare team multidisciplinari uniti ed in grado di collaborare in maniera efficace nonostante i diversi ambiti d'interesse.
\end{itemize}

\section{E}
\begin{itemize}
\item \textbf{Elasticsearch}: ElasticSearch è un potente e veloce motore/server di ricerca open-source costruito su Lucene.
\item \textbf{Ethereum}: Ethereum è una piattaforma software open-source basata su Blockchain che permette di sviluppare applicazioni decentralizzate.
\item \textbf{EVM}: EVM è l'abbreviazione di Ethereum Virtual Machine e può essere visto come un grande computer virtuale decentralizzato e si occupa di gestire effettivamente i dati nel database interno e la parte computazionale.
\end{itemize}

\section{F}
\begin{itemize}
\item \textbf{Framework}:
\item \textbf{Front-end}: il front end è la parte del software che si occupa di fornire una semplificazione della logica dell'applicazione attraverso un'interfaccia di facile comprensione da parte dell'utente.
\end{itemize}

\section{G}
\begin{itemize}
\item \textbf{Git}: Git è un sistema di controllo del versionamento del software.
\item \textbf{Glade}: Glade è uno strumento utile a velocizzare e semplificare lo sviluppo di interfacce grafiche con gli strumenti offerti da Gtk+.
\item \textbf{Google Cloud Datastore}: Google Cloud Datastore è un database non SQL sviluppato per avere alte prestazioni in grado di scalare efficacemente per gestire diversi tipi di applicazioni.
\item \textbf{Google SQL}: Google SQL è un sistema di database che rende facile creazione, gestione e manutenzione di database relazionali su Google Platform.
\item \textbf{Google Platform}: Google Platform è un servizio offerto da Google che si sostanzia nel mettere a disposizione delle piattaforme in cui è possibile sviluppare, testare e implementare applicazioni.
\item \textbf{Gtk+}: Gtk+ è un insieme di strumenti mulipiattaforma utili a sviluppare applicazioni grafiche.
\end{itemize}

\section{H}
\begin{itemize}
\item \textbf{Hyperledger Fabric}:	
\item \textbf{HTML5}: HTML5 è un linguaggio di markup con il quale si strutturano le pagine web. "5" indica l'ultima versione disponibile aderente allo standard.
\item \textbf{HTTPS}: abbreviazione di HyperText Transfer Protocol, HTTPS è un protocollo usato per fornire connessioni sicure attraverso una rete.
\end{itemize}

\section{I}
\begin{itemize}
\item \textbf{IT}: IT è l'abbreviazione di Information Technology, cioè qualsiasi attività che faccia uso di elaboratori o strumenti di telecomunicazione per manipolare, gestire e memorizzare dati.
\end{itemize}

\section{J}
\begin{itemize}
\item \textbf{Java}: Java è un linguaggio di programmazione e una piattaforma di elaborazione.
\item \textbf{Java EE}: Java EE è l'abbreviazione di Java Enterprise Edition ed è una raccolta di specifiche per lo sviluppo e la distribuzione di applicazioni aziendali.	
\item \textbf{JavaScript}: Javascript è un linguaggio di programmazione largamente usato nel web per la creazione di contenuti dinamici nel front-end. 
\end{itemize}

\section{K}
\begin{itemize}
\item \textbf{Keycloak}: Keycloack è un sistema open-source usato per la gestione di identità e accesso all'intero di applicazioni software.
\item \textbf{Kibana}: Kibana è un'interfaccia grafica utile a visualizzare e navigare nei dati memorizzati in Elastic.
\end{itemize}

\section{L}
\begin{itemize}
\item \textbf{Linguaggi di markup}: vengono definiti linguaggi di markup tutti quei linguaggi che decrivono dati attraverso una formattazione specifica usando dei marcatori; un esempio è HTML5.
\item \textbf{Lucene}: Lucene è una API open-source scritta in Java.
\end{itemize}

\section{M}
\begin{itemize}
\item \textbf{Machine learning}: il machine learning è una branca dell'informatica che si occupa di fornire ad un elaboratore la capacità di "imparare" autonomamente, cioè senza bisogno di venire appositamente programmato per farlo.
\item \textbf{Manutenzione}: con manutenzione si intendono le attività di modifica di un software successive alla sua distribuzione, atte alla correzione di errori o implementazione di funzionalità aggiuntive.
\item \textbf{MongoDB}: MongoDB è un sistema di gestione di database non relazionale, orientato ai documenti.
\end{itemize}

\section{N}
\begin{itemize}
\item \textbf{NodeJS}: NodeJS è un framework opensource utilizzato per creare applicazioni a lato server con Javascript.
\end{itemize}

\section{O}
\begin{itemize}
\item \textbf{Open-source}: Open-source definisce software il cui codice sorgente può essere analizzato, modificato ed esteso, tramite l'applicazione di specifiche licenze d'uso.
\end{itemize}

\section{P}
\begin{itemize}
\item \textbf{Permissioned blockchain}: permissioned blockchain indica un tipo di blockchain costruito in modo da non essere pubblico, bensì si basa sull'assegnazione di permessi per leggere l'informazione, in modo da limitare l'utenza che può fare transazione sul blockchain o creare nuovi blocchi.
\item \textbf{Play}: Play è un framework open-source avente lo scopo di migliorare la produttività degli sviluppatori, fornendo un consumo di risorse minimale e predicibile.
\item \textbf{PoC}: PoC sta per Proof of Concept ed indica un'incompleta realizzazione di un certo progetto, cioè una bozza, allo scopo di dimostrare l'effettivo potenziale dell'oggetto in esame.
\item \textbf{Proponente  \color{red}{serve davvero?}}:
\item \textbf{Prototipo  \color{red}{serve davvero?}}:
\end{itemize}

\section{Q}
\begin{itemize}
\item \textbf{Qt}: Qt è una libreria multipiattaforma per lo sviluppo di applicazioni con o senza interfaccia grafica.
\item \textbf{QtCreator}: QtCreator è un ambiente di sviluppo multipiattaforma che fornisce supporto ad implementazione ed esecuzione di applicazioni create sfruttando Qt.
\end{itemize}

\section{R}
\begin{itemize}
\item \textbf{React}: React è una libreria Javascript creata per lo sviluppo di interfacce grafiche.
\item \textbf{Redux}: Redux è una libreria JavaScript open-source avente l'obiettivo di gestire lo stato delle applicazioni web.
\item \textbf{Repository}: repository indica un sistema informativo creato per memorizzare metadata, cioè informazioni riguardo la struttare vera e propria dei dati
\item \textbf{Responsive}: responsive indica una tecnica di web design per la realizzazione di interfacce in grado di adattarsi in maniera automatica al dispositivo su cui vengono visualizzate.
\item \textbf{Robustness diagram}:
\end{itemize}

\section{S}
\begin{itemize}
\item \textbf{SCSS}:
\item \textbf{Server}:
\item \textbf{Sistemi di Raccomandazione}:
\item \textbf{Smart Contracts}:
\item \textbf{Solidity}:
\item \textbf{Standup}:
\end{itemize}

\section{T}
\begin{itemize}
\item \textbf{Team}:
\item \textbf{Tomcat}:
\item \textbf{TTS}
\end{itemize}

\section{U}
\begin{itemize}
\item \textbf{UML}:s
\end{itemize}

\section{V}
\begin{itemize}
\item \textbf{Vaadin Elements}:
\end{itemize}



\end{document}