\section{Capitolato C8}
\subsection{Informazioni sul Capitolato}
	\begin{itemize}
		\item \textbf{Nome:}
		 TuTourSelf;
		\item \textbf{Proponente}:
		 TuTourSelf;
		\item \textbf{Committenti}:
		Prof. Tullio Vardanega, Prof. Riccardo Cardin.
	\end{itemize}

\subsection{Descrizione}
	Lo scopo del capitolato C8 è lo sviluppo di una piattaforma web con l'obiettivo di facilitare ad artisti indipendenti l'organizzazione dei propri tour, creando una community in cui artisti e locali possano interagire in modo chiaro, rapido e regolamentato. 
	Viene previsto inoltre di dare la possibilità ad utenti esterni (cioè nè artisti nè gestori) di fruire delle informazioni nel sistema per conoscere gli eventi di loro interesse. Tutti e tre questi attori hanno inoltre la possibilità di lasciare feedback.
	
\subsection{Dominio Applicativo}
	Il sistema da realizzare trova il proprio dominio applicativo nel mondo della creatività ed è indirizzato a band, musicisti, scrittori che vogliano promuovere
	il proprio libro, stand-up comedians, compagnie teatrali, artisti di strada, live
	performers e pittori alla ricerca di gallerie d’arte.

\subsection{Dominio Tecnologico}
	Oggetto del capitolato è la creazione di un portale e necessita di:
		\begin{itemize}
			\item nel front-end l'utilizzo dei linguaggi del web, cioè \textbf{HTML}, \textbf{CSS}, \textbf{JavaScript};
			\item per JavaScript è desiderabile l'utilizzo della libreria \textbf{React};
			\item per quanto riguarda il back-end viene lasciata completa libertà di scelta della tecnologia, purchè aderente agli standard ed attento alla scalabilità.
		\end{itemize}
		Infine è richiesta la pubblicazione del progetto su un \textit{repository}\ped{G} pubblico.

\subsection{Aspetti Positivi}
	Gli aspetti positivi riscontrati sono:
	\begin{itemize}
	\item Concetto di base lodevole, con potenzialità per avere successo.
	\item Tecnologie front-end conosciute e apprezzate.
	\end{itemize}

\subsection{Potenziali Criticità}
	Le principali criticità incontrate sono:
	\begin{itemize}
		\item Poco stimolante in quanto semplice sviluppo di un'interfaccia web.
		\item L'implementazione di funzionalità come live chat e gestione del pagamento potrebbe risultare onerosa.
		\item Troppa libertà nella scelta delle tecnologie back-end.  
	\end{itemize}

\subsection{Valutazione Finale}
	Il capitolato è risultato per il gruppo concettualmente interessante, tuttavia presentava pochi stimoli e avrebbe impiegato il gruppo nella ricerca delle tecnologie adatte al back-end, cosa che probabilmente avrebbe richiesto troppo tempo vista la poca esperienza sul campo dei componenti.