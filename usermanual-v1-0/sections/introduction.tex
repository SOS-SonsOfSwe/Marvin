\section{Introduction}
This is the user manual of \textbf{Marvin}, a \emph{\DH app}\textsubscript{G} ran on the EVM, that shapes a subset of \href{www.uniweb.unipd.it}{Uniweb} functionalities. Uniweb is the University of Padua's informative system.
It allows students to keep track of their academic carrier. Professors use Uniweb to see the lists of students that are registered to their exams and, see the exams to which they have been assigned.

More precisely:
\begin{labeling}{alligator}
	\item A student will be able to:
	\begin{itemize}
		\item Subscribe to an exam;
		\item Confirm a mark;
		\item Visualize his/her booklet.
	\end{itemize}
	\item A professor will be able to:
	\begin{itemize}
		\item Visualize the exams to which he/she has been assigned;
		\item See the students registered to his/her exams;
		\item Assign marks to the students.
	\end{itemize}
	\item The university and the administrators will be able to:
	\begin{itemize}
		\item Create and delete academic years;
		\item Create and delete degree courses;
		\item Create and delete classes;
		\item Create exams;
		\item Create users.
	\end{itemize}
\end{labeling}

To use \project{} you need to install \emph{\href{https://metamask.io/}{MetaMask}}\textsubscript{G} to directly execute the \DH app on your browser.

All the users have to be logged through Metamask.