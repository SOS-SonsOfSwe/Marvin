\section{Informazioni generali}
	\begin{itemize}
		\item \textbf{Data riunione}: 2018-04-28;
		\item \textbf{Ora inizio riunione}: 18:00;
		\item \textbf{Ora fine riunione}: 19:30;
		\item \textbf{Luogo di incontro}: Slack;
		\item \textbf{Oggetto di discussione}: si sono definiti alcuni strumenti da utilizzare nella fase di prototipazione;
		\item \textbf{Moderatore}: Eleonora Thiella;
		\item \textbf{Segretario}: Stefano Panozzo;
		\item \textbf{Partecipanti}: Giovanni Cavallin, Eleonora Thiella, Giovanni Dalla Riva, Stefano Panozzo, Andrea Favero, Lorenzo Menegon e Federico Caldart;
	\end{itemize}

\section{Riassunto della riunione}
	\subsection{Descrizione} Il gruppo si è consultato con i proponenti riguardo agli stumenti da utilizzare durante la fase di prototipazione e ha proposto l'utilizzo di IPFS o Swarm per inserire i dati con una minore importanza all'esterno della blockchain di Ethereum.
	\\I proponenti hanno accolto questa idea innovativa e hanno dato il loro consenso.
	
	\subsection{Decisioni prese}
		\begin{itemize}
			\item \textbf{VE3.1}: si è deciso che si utilizzarà in un secondo momento il protocollo di distribuzione IPFS, oppure in alternativa la piattaforma Swarm.
		\end{itemize}