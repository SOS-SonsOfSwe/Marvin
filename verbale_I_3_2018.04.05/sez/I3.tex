\section{Informazioni generali}
	\begin{itemize}
		\item \textbf{Data riunione}: 2018-04-05;
		\item \textbf{Ora inizio riunione}: 16:30;
		\item \textbf{Ora fine riunione}: 17:00;
		\item \textbf{Luogo di incontro}: Laboratorio P036, Slack;
		\item \textbf{Oggetto di discussione}: Discussione riguardo un requisito non chiaro rilevato durante l'\AdR{}.
		\item \textbf{Moderatore}: Eleonora Thiella;
		\item \textbf{Segretario}: Lorenzo Menegon;
		\item \textbf{Partecipanti}: Giovanni Cavallin, Eleonora Thiella, Giovanni Dalla Riva, Stefano Panozzo e Federico Caldart;
	\end{itemize}

\section{Riassunto della riunione}
	\subsection{Descrizione}
	Durante la stesura dell'\AdR{} si è reso necessario discutere col team dell'interpretazione di un requisito non chiaro tratto dal capitolato. Una volta presa una decisione, è stata comunicata al proponente e confermata.
	\subsection{Decisioni prese}
		\begin{itemize}
			\item \textbf{VI3.1}: la registrazione avverrà secondo questi passaggi:
			\begin{enumerate}
				\item L'università inserirà la matricola, un codice univoco e la tipologia (professore, studente) ad un utente che vorrà registrarsi. Questi dati gli saranno forniti via e-mail;
				\item L'utente che vorrà poi registrarsi al sistema dovrà inserire nel form di registrazione matricola e codice univoco, oltre a tutti gli altri già presenti.
			\end{enumerate}
		\end{itemize}

