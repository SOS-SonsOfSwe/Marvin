\section{Informazioni generali}
	\begin{itemize}
		\item \textbf{Data riunione}: 2018-04-30;
		\item \textbf{Ora inizio riunione}: 15:00;
		\item \textbf{Ora fine riunione}: 16:00;
		\item \textbf{Luogo di incontro}: Laboratorio P036;
		\item \textbf{Oggetto di discussione}: decisione degli strumenti da utilizzare durante la fase di prototipazione e codifica e suddivisione dei ruoli all'interno del team.
		\item \textbf{Moderatore}: Federico Caldart;
		\item \textbf{Segretario}: Lorenzo Menegon;
		\item \textbf{Partecipanti}: Giovanni Cavallin, Eleonora Thiella, Giovanni Dalla Riva, Stefano Panozzo, Lorenzo Menegon, Andrea Favero e Federico Caldart;
	\end{itemize}

\section{Riassunto della riunione}
	\subsection{Descrizione} Durante la riunione, i membri del team hanno deciso di utilizzare per la codifica l'editor \emph{Visual Studio Code}\ped{G}, il framework \emph{Truffle}\ped{G}, il sistema \emph{Ganache}\ped{G}, l'IDE \emph{remixIDE}\ped{G}, il gestore di pacchetti \emph{npm}\ped{G} per \emph{Node.js}\ped{G}, e il protocollo di distribuzione \emph{IPFS}\ped{G} (o in alternativa \emph{Swarm}\ped{G}). Il gruppo ha inoltre suddiviso i compiti per la fase di prototipazione.
	
	\subsection{Decisioni prese}
		\begin{itemize}
			\item \textbf{VI4.1}: si è deciso di utilizzare l'editor Visual Studio Code per la codifica del progetto, dal momento che presenta dei plug-in utili;
			\item \textbf{VI4.2}: si è deciso di utilizzare il sistema Ganache che permette di generare una blockchain su Ethereum in locale;
			\item \textbf{VI4.3}: si è deciso di utilizzare remixIDE per scrivere, complilare e fare test su codice \emph{Solidity}\ped{G};
			\item \textbf{VI4.4}: si è deciso di utilizzare npm come gestore di pacchetti per il framework Node.js;
			\item \textbf{VI4.5}: si è deciso di utilizzare il framework Truffle per compilazione e migrazione verso la blockchain;
			\item \textbf{VI4.6}: si è discusso sul futuro utilizzo di IPFS o Swarm per inserire i dati con una minore importanza all'esterno della blockchain di Ethereum;
			\item \textbf{VI4.7}: si è deciso di assegnare tali compiti ai membri del gruppo:
				\begin{itemize}
					\item Giovanni Cavallin: si occupa del back-end;
					\item Eleonora Thiella: si occupa del front-end;
					\item Giovanni Dalla Riva: si occupa del front-end;
					\item Stefano Panozzo: si occupa del back-end;
					\item Lorenzo Menegon: si occupa del front-end;
					\item Andrea Favero: si occupa del back-end;
					\item Federico Caldart: si occupa del front-end.
				\end{itemize}
		\end{itemize}

